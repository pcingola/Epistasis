\documentclass[12pt,Bold,letterpaper,TexShade]{mcgilletdclass}

\usepackage{amsfonts}
\usepackage{amsmath}
\usepackage{amssymb}
\usepackage{caption}
\usepackage{color}
\usepackage{enumitem}
\usepackage{epsfig}
\usepackage{framed}
\usepackage{geometry}
\usepackage{graphicx}
\usepackage{listings}
\usepackage{subcaption}

%% Lemma, Theorem, etc.
\newtheorem{lemma}{{\bf Lema}}[chapter]
\newtheorem{theorem}{{\bf Teorema}}[chapter]
\newtheorem{corollary}{{\bf Corolario}}[theorem]
\newtheorem{definition}{{\bf Definici\'on}}[chapter]
\newtheorem{propo}{{\bf Proposicion}}[chapter]

%% 
\renewcommand{\vec}[1]{\mbox{$\,${\bf #1}}}
\newcommand{\C}{\mbox{$\mbox{l}\!\!\!\mbox{C}$}}
\newcommand{\N}{\mbox{$\mbox{I}\!\mbox{N}$}}
\newcommand{\R}{\mbox{$\mbox{I}\!\mbox{R}$}}
\newcommand{\Z}{\mbox{$\mbox{Z}\!\!\mbox{Z}$}}
\newcommand{\Rn}{\mbox{$\R^n\,$}}
\newcommand{\Rnxn}{\mbox{$\R^{n \times n}$}}
\newcommand{\Rnxm}{\mbox{$\R^{n \times m}$}}
\newcommand{\Rmxn}{\mbox{$\R^{m \times n}$}}
\newcommand{\Rmxm}{\mbox{$\R^{m \times m}$}}
\newcommand{\inertia}[1]{\mbox{$\cal I$($#1$)}}
\newcommand{\ceil}[1]{\mbox{$\left\lceil #1 \right\rceil$}}
\newcommand{\floor}[1]{\mbox{$\left\lfloor #1 \right\rfloor$}}
\newcommand{\ceilfrac}[2]{\mbox{$\left\lceil\frac{#1}{#2}\right\rceil$}}
\newcommand{\floorfrac}[2]{\mbox{$\left\lfloor\frac{#1}{#2}\right\rfloor$}}
\newcommand{\GEP}{\mbox{(\ref{eq:GEP})}}
\newcommand{\AMB}{\mbox{$A-\mu B$}}
\newcommand{\M}{\mbox{$\cal M$}}
\newcommand{\B}{\mbox{$\cal B$}}
\newcommand{\Q}{\mbox{$\cal Q$}}
\newcommand{\D}[1]{\mbox{${\cal D}_#1$}}
\newcommand{\U}{\mbox{$\cal U$}}
\newcommand{\V}{\mbox{$\cal V$}}
\newcommand{\lead}[2]{\mbox{$#1_{[#2]}$}}
\newcommand{\minor}[2]{\mbox{$\det{\lead{#1}{#2}}$}}
\newcommand{\inner}[3]{\mbox{$<\!\!#1,#2\!\!>_{#3}$}}
\newcommand{\ie}{i.e.\ }
\newcommand{\eg}{e.g.\ }

%-------------------------------------------------------------------------------
% Line spacing
%-------------------------------------------------------------------------------
\newcommand{\BaseDiff}{0}

\newcommand{\GoSingle}{\renewcommand{\baselinestretch}{1}
	\normalfont\tiny\normalsize
}

\newcommand{\GoDouble}{\renewcommand{\baselinestretch}{1.655}
	\renewcommand{\BaseDiff}{0.655}\normalfont\tiny\normalsize
}

%-------------------------------------------------------------------------------
% Tables
%-------------------------------------------------------------------------------

% Define \mytable
% Include one of my tables, in the standard way
%  Parm 1 is the table name
%  Parm 2 is the caption for the "List of tables"
%  Parm 3 is the real caption
\newcommand{\mytable}[3]{
    \begin{table}[htbp]
        \begin{minipage}{\textwidth}
          \begin{center}
	          \TableCaptionOpt{#2 \label{tab:#1}}{#3}
			  \label{tab:#1}
              \input{tables/#1/table.tex}
          \end{center}
        \end{minipage}
    \end{table}
    \normalsize
}

%-------------------------------------------------------------------------------
% Figures
%-------------------------------------------------------------------------------

% Define \fig
% Include one of my figures, in the standard way
%  Parm 1 : File name (no extension)
%  Parm 2 : Label (without 'fig:')
%  Parm 3 : Width
%  Parm 4 : Caption
%  Parm 5 : Figure name (for table of contents)

% \newcommand{\fig}[5]{
% \begin{figure}[ht]
%     \begin{center}
%         \begin{minipage}{#3}
%             \includegraphics[ width={#3} ]{figs/#1}
%             \FigureCaptionOpt{#4}{#5}
%             \label{fig:#2}
%         \end{minipage}
%     \end{center}
% \end{figure}
% }

\newcommand*{\TableCaptionOpt}[2]{
	\caption[#1]{#2}
}

\newcommand*{\FigureCaptionOpt}[2]{
	\caption[#1]{#2}
}

\newcommand{\fig}[5]{
\begin{figure}
  \begin{center}
    \includegraphics[ width={#3} ]{figs/#1}
    \FigureCaptionOpt{#5}{#4}
    \label{fig:#2}
  \end{center}
\end{figure}
}

% Define \figtab
% Include one of my figures (but the figure actually contains a table)
%  Parm 1 : File name (no extension)
%  Parm 2 : Width
%  Parm 3 : Caption
%  Parm 4 : Table name (for table of contents)
\newcommand{\figtab}[5]{
\begin{figure}
  \renewcommand{\figurename}{Table }
  \begin{center}
    \includegraphics[ width={#3} ]{figs/#1}
    \TableCaptionOpt{#5}{#4}
    \label{tab:#2}
  \end{center}
\end{figure}
}

%-------------------------------------------------------------------------------
% Title pages
%-------------------------------------------------------------------------------
\newcommand*{\SetTitle}[1]{\renewcommand*{\Title}{#1}}
\newcommand*{\Title}{No Title Given}

\newcommand*{\SetAuthor}[1]{\renewcommand*{\FullName}{#1}}
\newcommand*{\FullName}{Please Define Your Name}

\newcommand*{\SetThesisType}[1]{\renewcommand*{\ThesisType}{#1}}
\newcommand*{\ThesisType}{THESIS OR DISSERTATION}

\newcommand*{\SetDegreeType}[1]{\renewcommand*{\DegreeType}{#1}}
\newcommand*{\DegreeType}{UNDEFINED DEGREE}

\newcommand*{\SetGradMonth}[1]{\renewcommand*{\GradMonth}{#1}}
\newcommand*{\GradMonth}{UNDEFINED MONTH}

\newcommand*{\SetGradYear}[1]{\renewcommand*{\GradYear}{#1}}
\newcommand*{\GradYear}{UNDEFINED YEAR}

\newcommand*{\SetDepartment}[1]{\renewcommand*{\ETDDepartment}{#1}}
\newcommand*{\ETDDepartment}{UNDEFINED DEPARTMENT}

\newcommand*{\SetChair}[1]{\renewcommand*{\Chair}{#1}}
\newcommand*{\Chair}{UNDEFINED Chair}

\newcommand*{\SetUniversity}[1]{\renewcommand*{\ETDUniversity}{#1}}
\newcommand*{\ETDUniversity}{McGill University}

\newcommand*{\SetUniversityAddr}[1]{\renewcommand*{\ETDUniversityAddr}{#1}}
\newcommand*{\ETDUniversityAddr}{Montreal, Quebec}

\newcommand*{\SetThesisDate}[1]{\renewcommand*{\ETDThesisDate}{#1}}
\newcommand*{\ETDThesisDate}{Date111}

\newcommand*{\SetRequirements}[1]{\renewcommand*{\ETDRequirements}{#1}}
\newcommand*{\ETDRequirements}{Date222}

\newcommand*{\SetCopyright}[1]{\renewcommand*{\ETDCopyright}{#1}}
\newcommand*{\ETDCopyright}{Date223333}

\newenvironment{QZ@Cent}{\centering}{\par}

\renewcommand{\maketitle}{
	\clearpage
	\thispagestyle{empty}
	\begin{QZ@Cent}
		\Title
		\vfill
		\GoSingle
		\normalsize\normalfont
		\FullName\normalsize\normalfont \\*[\BaseDiff\baselineskip]
		\vfill
		\DegreeType\normalsize\normalfont \\*[\BaseDiff\baselineskip]
		\vfill
		\ETDDepartment\normalsize\normalfont \\*[\BaseDiff\baselineskip]
		\vfill
		\ETDUniversity\normalsize\normalfont  \\*[\BaseDiff\baselineskip]
		\ETDUniversityAddr\normalsize\normalfont \\*[\BaseDiff\baselineskip]
		\ETDThesisDate\normalsize\normalfont \\*[\BaseDiff\baselineskip]
		\vfill
		\ETDRequirements\normalsize\normalfont \\*[\BaseDiff\baselineskip]
		\ETDCopyright\normalsize\normalfont \\*[\BaseDiff\baselineskip]
	\end{QZ@Cent}
	\vspace*{0.5in}
	\clearpage
}

%-------------------------------------------------------------------------------
% Acknowledgements
%-------------------------------------------------------------------------------

\newcommand*{\SetAcknowledgeName}[1]{\renewcommand*{\ETDAcknowledgeName}{#1}}
\newcommand*{\ETDAcknowledgeName}{Acknowledgements}

\newcommand*{\SetAcknowledgeText}[1]{\renewcommand*{\ETDAcknowledgeText}{#1}}
\newcommand*{\ETDAcknowledgeText}{Acknowledgements text goes here!}

\newenvironment{simpleenv}[4]{\clearpage}{\clearpage}

\newcommand{\Acknowledge}{
	\begin{simpleenv}{}{}{}{}
		\pagestyle{plain}
		\GoSingle
		\begin{QZ@Cent}
			\bfseries{\ETDAcknowledgeName}
		\end{QZ@Cent}
		\vspace*{0.5in}
		\par
		\GoDouble
		\ETDAcknowledgeText
	\end{simpleenv}
}

%-------------------------------------------------------------------------------
% Abstracts
%-------------------------------------------------------------------------------

\newenvironment{romanPagenumber}[1]
{\setcounter{page}{#1}\renewcommand{\thepage}{\roman{page}}}
{\pagenumbering{arabic}}

\newcommand*{\SetAbstractEnName}[1]{\renewcommand*{\ETDAbstractEnName}{#1}}
\newcommand*{\ETDAbstractEnName}{Abstract}

\newcommand*{\SetAbstractEnText}[1]{\renewcommand*{\ETDAbstractEnText}{#1}}
\newcommand*{\ETDAbstractEnText}{Abstract text goes here!}

\newcommand*{\AbstractEn}{
    \clearpage
    \pagestyle{plain}
    \GoSingle
    \par
    \GoDouble
    \ETDAbstractEnText
}

\newcommand*{\SetAbstractFrName}[1]{\renewcommand*{\ETDAbstractFrName}{#1}}
\newcommand*{\ETDAbstractFrName}{Abr\'{e}g\'{e}}

\newcommand*{\SetAbstractFrText}[1]{\renewcommand*{\ETDAbstractFrText}{#1}}
\newcommand*{\ETDAbstractFrText}{Abstract text goes here!}

\newcommand*{\AbstractFr}{
    \clearpage
    \pagestyle{plain}
    \GoSingle
    %\addcontentsline{toc}{extrachapter}{\ETDAbstractFrName}
    \par
    \GoDouble
    \ETDAbstractFrText
}

%-------------------------------------------------------------------------------
% Bibliography
%-------------------------------------------------------------------------------
\newcommand*{\bibHeading}[1]{
	\renewcommand{\bibname}{#1}
}

%-------------------------------------------------------------------------------
% Table of contents
%-------------------------------------------------------------------------------

\setcounter{tocdepth}{2}
\renewcommand{\contentsname}{Table of contents}

%\renewcommand{\tableofcontents}{
%	\begin{simpleenv}{}{}{}{}
%		\pagestyle{plain}
%		\chapter*{\contentsname}
%		\vspace*{-12pt}%
%		\noindent\phantom{Table}\hfill\par
%		\@starttoc{toc}
%	\end{simpleenv}
%}

%-------------------------------------------------------------------------------
% List of figures
%-------------------------------------------------------------------------------

\newcommand*{\LOFHeading}[1]{
    \renewcommand{\listfigurename}{#1}
}

%-----------------------------------------------------------------------------
% Code linsting options
%-----------------------------------------------------------------------------

\definecolor{dkgreen}{rgb}{0,0.6,0}
\definecolor{gray}{rgb}{0.5,0.5,0.5}
\definecolor{mauve}{rgb}{0.58,0,0.82}

\lstset{frame=tb,
  language=Java,
  aboveskip=3mm,
  belowskip=3mm,
  showstringspaces=false,
  columns=flexible,
  basicstyle={\small\linespread{1.0}\ttfamily,},
  numbers=none,
  numberstyle=\tiny\color{gray},
  keywordstyle=\color{blue},
  commentstyle=\color{dkgreen},
  stringstyle=\color{mauve},
  breaklines=true,
  breakatwhitespace=true,
  numbers=left,
  stepnumber=1,
  firstnumber=1,
  numberfirstline=true
  tabsize=3
}

\renewcommand\lstlistingname{Listing}
\renewcommand\lstlistlistingname{Listings}


\addtolength{\hoffset}{0pt}		% Have you configured your TeX system for proper page alignment? See the McGillETD documentation
\addtolength{\voffset}{0pt}

%-----------------------------------------------------------------------------
% Student info
%-----------------------------------------------------------------------------

\SetTitle{\huge{Computational challenges in genome wide association studies: data processing, variant annotation and epistasis}}
\SetAuthor{Pablo Cingolani}
\SetDegreeType{PhD.}
\SetDepartment{School of Computer Science}
\SetUniversity{McGill University}
\SetUniversityAddr{Montreal,Quebec}
\SetThesisDate{March 2015}
\SetRequirements{A thesis submitted to McGill University in partial fulfillment of the requirements of the degree of Doctor of Philosophy}
\SetCopyright{Pablo Cingolani 2015}

\makeindex[keylist]
\makeindex[abbr]
\listfiles

%-----------------------------------------------------------------------------
% Code linsting options
%-----------------------------------------------------------------------------

\definecolor{dkgreen}{rgb}{0,0.6,0}
\definecolor{gray}{rgb}{0.5,0.5,0.5}
\definecolor{mauve}{rgb}{0.58,0,0.82}

\lstset{frame=tb,
  language=Java,
  aboveskip=3mm,
  belowskip=3mm,
  showstringspaces=false,
  columns=flexible,
  basicstyle={\small\linespread{1.0}\ttfamily,},
  numbers=none,
  numberstyle=\tiny\color{gray},
  keywordstyle=\color{blue},
  commentstyle=\color{dkgreen},
  stringstyle=\color{mauve},
  breaklines=true,
  breakatwhitespace=true,
  numbers=left,
  stepnumber=1,
  firstnumber=1,
  numberfirstline=true
  tabsize=3
}

\renewcommand\lstlistingname{Listing}
\renewcommand\lstlistlistingname{Listings}

%-----------------------------------------------------------------------------
% Document stats here
%-----------------------------------------------------------------------------

\begin{document}

\maketitle

%-----------------------------------------------------------------------------
% Input any special commands below
%-----------------------------------------------------------------------------

% Conditional expression for faster build during "development" cycle
% See http://tex.stackexchange.com/questions/5894/latex-conditional-expression
\newif\ifthesis
%\thesistrue
\thesisfalse

% Final version?
\newif\iffinal
%\finaltrue
\finalfalse

\ifthesis
	% Only show these sections when we are building the 'thesis' version

	\begin{romanPagenumber}{2}
	
	\iffinal

		%-----------------------------------------------------------------------------
		% Acknowledgements
		%-----------------------------------------------------------------------------
		
		\SetAcknowledgeName{\MakeUppercase{Acknowledgements}}
		\SetAcknowledgeText{
		}
		\Acknowledge
	
	\fi

	%-----------------------------------------------------------------------------
	%         English Abstract
	%-----------------------------------------------------------------------------
	
	\SetAbstractEnName{\MakeUppercase{Abstract}}%
	\SetAbstractEnText{ 
Abundant genome sequence information from large cohorts of individuals can now be routinely obtained and this information is poised to ease the identification of genetic variations linked to complex disease. 
In this work, I investigate the computational and statistical challenges involved in the analysis of large genomic datasets and I tackle three different aspects of the analysis, each of them having very different characteristics.
First, in order to analyse large amounts of data from genomic studies we design a programming language, BigDataScript, that simplifies the creation of robust and scalable data analysis pipelines.
Second, we create genomic variant annotation and prioritization methods (SnpEff and SnpSift) that help to calculate putative genetic effects and estimate the genetic impact of variants.
Finally, we address the problem of finding associations between interacting genetic loci and disease by proposing a methodology that combines population-level genetic information with evolutionary information in order to increase the statistical power in epistatic genome wide association studies 
	}
	\AbstractEn%
	
	%-----------------------------------------------------------------------------
	%         French Abstract
	%-----------------------------------------------------------------------------
	
	\SetAbstractFrName{\MakeUppercase{ABR\'{E}G\'{E}}}%
	\SetAbstractFrText{ 
	}
	\AbstractFr%
	
	%-----------------------------------------------------------------------------
	% Tables
	%-----------------------------------------------------------------------------
	
	\TOCHeading{\MakeUppercase{Table of Contents}}
	% \LOTHeading{\MakeUppercase{List of Tables}}
	\LOFHeading{\MakeUppercase{List of Figures and Tables}}
	\tableofcontents 
	%\listoftables 
	\listoffigures
	
	\end{romanPagenumber}

\else
	% Skip all the previous sections
\fi

%-----------------------------------------------------------------------------
% Chapters
%-----------------------------------------------------------------------------

\ifthesis
	% Show all chapters on 'thesis' mode
	%-----------------------------------------------------------------------------
\chapter{Introduction \label{ch:intro}}
%-----------------------------------------------------------------------------

%---
\section{Motivation}
%---

How does your DNA influence your risk of getting a disease? Contrary to popular belief, your future health is not ``hard wired" in your DNA. Only in a few diseases, referred as ``Mendelian diseases", there are well known, almost certain, links between genetic mutations and disease susceptibility. For the majority of what are known as ``complex traits", such as cancer or diabetes, genomic predisposition is subtle and, so far, not fully understood.

With the rapid decrease in the cost of DNA sequencing, the complete genome sequence of large cohorts of individuals can now be routinely obtained. This wealth of sequencing information is expected allow the identification of genetic variations linked to complex traits. In this work, I investigate the analysis of genomic data in relation to complex diseases, which offers a number of important computational and statistical challenges. We tackle several steps necessary for the analysis of sequencing data and identifying the links to disease. Each step, which will correspond to a chapter in my thesis, is characterized by very different problems that need to be addressed.

\begin{itemize}

\item[i] The first step is to analyze large amounts of information generated by sequencers to obtain a set of ``genomic variants" that distinguish each individual. To address these big data processing problems, Chapter 2 shows how we designed a programming language (BigDataScript or BDS), that simplifies the creation robust, scalable data pipelines.

\item[ii] Once genomic variants are obtained, we need to prioritize and filter them to discern which variants should be considered ``important" and which ones are likely to be less relevant. In this process, known as ``functional variant annotation" or simply ``variant annotation", we calculate how the protein product would be affected and add information from relevant genomic databases (such as protein structure, deleteriousness scores or how often the variant is present in a population). We created SnpEff \& SnpSift \cite{cingolani2012program, cingolani2012using} packages that, using optimized algorithms, solve several annotation problems: a) standardize the annotation process, b) calculate putative genetic effects, c) estimate genetic impact, d) add several sources of genetic information, and e) facilitate variants filtering. We applied our methods in two large Genome Wide Association Studies (GWAS) for type II diabetes projects, in order to prioritize variants for statistical analysis. As a result of these studies, novel genes associated with diabetes and glycemic traits were found.
					
\item[iii] Finally, we address the problem of finding associations between ``interacting genetic loci" and disease. One of the main problems in GWAS, known as ``missing heritability", is that most of the phenotypic variance attributed to genetic causes remains unexplained. Since interacting genetic loci have been pointed out as one of the possible causes of missing heritability, finding links between such interactions and disease has great significance in the field. We propose a methodology to increase the statistical power of this type of approaches by combining population-level genetic information with evolutionary information. 

\end{itemize}

In the rest of this introduction we give the background required to understand the material shown in Chapters 2 to 5 while providing motivations for our research. The transformation of raw sequencing data into biological insight in the aetiology of complex disease poses a series of computational, analytical, algorithmic and methodological challenges that we address in the rest of this thesis.

\subsection{Reference genome and genetic variants}

DNA is composed of four basic building blocks, called ``bases'' or ``nucleotides''. These four nucleotides, usually abbreviated $\{A, C, G, T\}$, are Adenine, Cytosine, Guanine, and Thymine. Bases form pairs, either as $A-T$ or $C-G$, that pile-up forming two long polymers, with backbones that run in opposite directions giving rise to a double-helix structure. Arbitrarily, one of the polymers is called the positive strand and the other is called the negative strand.  

The human genome has a total of 3 Giga-base-pairs (Gb), and those bases are divided into 23 chromosomes. We have two copies of each ``autosomal'' chromosomes, one inherited from our mother and one from our father. There are 22 autosomal chromosomes. The longest, being roughly 250 Mega-bases (Mb), is called ``Chromosome 1'' and the shortest, being ~50 Mb is called ``chromosome 22''. We also have two sex chromosomes, called 'X' and 'Y'.

In order to be able to compare different object’s length, we need some reference measure, such as the reference meter. Similarly, in order to compare DNA from different individuals (or samples), we need a ``reference genome". The human reference genome (e.g. GRCh37) does not correspond to the DNA of any particular person, but to a ``mosaic" of thirteen anonymous volunteers from Buffalo, New York \cite{REF}.

When samples are sequenced, the DNA is compared to the ``reference genome". Most of the DNA is the same, but there are differences. These differences, generically known as ``genomic variants" (or ``variants", for short), describe the particular genetic makeup of each individual. There are several different ways a sample can differ from a reference genome. These are known as ``variant types" and can be roughly categorized in the following way:

\begin{description}

\item[Single nucleotide variants (SNV)] or Single nucleotide polymorphism (SNP) are the simplest and more common variants produced by single base difference (e.g. a base in the reference genome, at a given coordinate,  is an ‘A’, whereas the sample is ‘C’). There are several biological mechanisms responsible for this type of variants: i) replication errors, ii) errors introduced by DNA repair mechanism, iii) deamination (a base is changed by hydrolysis which may not be corrected by DNA repair mechanisms), iv) tautomerism (and alteration on the hydrogen bond that results in an incorrect pairing).

\item[Multiple nucleotide polymorphism (MNP)] are differences of more than one base (e.g. reference is ‘ACG’ whereas the sample is ‘TGC’).

\item[Insertions (INS)] refer to a sample having extra base(s) compared to the reference genome (e.g. reference is ‘AT’ and sample is ‘ACT’). Some small insertion are usually attributed to DNA polymerase slipping and replicating the same base/s (this produces a type of insertion known as duplication). Large insertions are can be caused by unequal cross-over event (during meiosis) or transposable elements.

\item[Deletions (DEL)] are the opposite of insertions, the sample has some base(s) removed respect to the reference genome (e.g. reference is ‘ACT’ and sample is ‘AT’). As in the case of insertions, deletions can also be caused by ribosomal slippage, cross-over events during meiosis and transposable elements. 

\item[Mixed variants] can happen as a more complex combinations of combining SNV/MNP + Ins/Del.

\item[Copy number variations (CNVs)] arise when the sample has two or more copies of the same genomic region (e.g. a whole gene that has been duplicated or triplicated) or conversely, when the sample has less copies than the reference genome. Copy number variations can be attributed to problem during homologous recombination events.

\item[Rearrangements] are some complex variants that involve joining different regions (e.g. a translocation between chromosomes). Inversions, a type of rearrangement, result from a whole genomic region being inverted. These types of mutations are often attributed to cross-over events during meiosis.

\end{description}

As humans have two copies of each chromosome, variants could affect zero, one or two of the chromosomes and are called ``homozygous reference", ``heterozygous", and ``homozygous alternative" respectively. Variants are also be classified on how common they are within the population: common, low frequency, or rare (see sections \ref{sec:}). How these types of genetic variants influence traits or risk of disease is a topic of intense research that will be discussed throughout this thesis.

Proteins are composed by chains of amino acids and, as explained by the central dogma of biology,  DNA is the template that instructs cellular machinery how to produce proteins. There are 4 bases in the DNA. There are 20 amino acids, which are the building blocks of all proteins. Each of the twenty amino acids is encoded by a group of three DNA bases called ``codon''. More than one codon can code for the same amino acid (i.e. $4^3=64$ codons $ > 20 $ amino acids) allowing for code redundancy. Additionally, there are codons that mark the end of the protein, these are called ``STOP" codons and signal molecular machinery to end the transcription process. So variations in DNA may sometimes have direct effects on the protein product. We will talk about this in section \ref{sec:} and Chapter 3 where we cover the topic of ``functional annotations".

\subsection{DNA and disease}

It would be fair to say that the Garrod family was fascinated by urine. As a physician at King’s College, Alfred Baring Garrod, discovered gout related abnormalities in uric acid \cite{kennedy2001}. His son, Sir Archibald Garrod, was interested in a condition known as alkaptonuria, in which children are mostly asymptomatic except for producing brown or black urine, but by the age of 30 individuals develop pain in joints of the spine, hips and knees. In 1902, Archibald observed that the family inheritance pattern of alkaptonuria resembled Mendel’s recessive pattern and postulated that a mutation in a metabolic gene was responsible for the disease. Publishing his finding he gave birth to a new field of study known as ``Human biochemical genetics" \cite{REF}.

Diseases having simple inheritance patterns, such as Cystic fibrosis, Phenylketonuria and Huntington's are also known as Mendelian diseases \cite{REF}. The genetic components of several Mendelian diseases have been discovered since the mechanism was first elucidated by Garrod in 1902 and the process has been accelerated in recent years, thanks to the application of DNA sequencing techniques \cite{REF}.

In complex diseases (or complex traits), such as diabetes, cancer or Alzheimer’s, affected individuals cannot be segregated within pedigrees (i.e. no patterns of inheritance can be identified). As opposed to Mendelian diseases the aetiologically of complex traits is complicated due to factors such as: incomplete penetrance (symptoms are not always present in individuals who have the disease-causing mutation), oligogenic inheritance (characterized by more than one gene) and genetic heterogeneity (caused by any of a large number of alleles). This makes  it difficult to pinpoint the genetic variants that increase risk of complex disease.

\subsection{Type II diabetes}

Although this thesis focusses on the development of computational approaches that could be applied to the study of a number of complex diseases, our focus has been on type II diabetes mellitus (T2D), a complex disease first described by the Egyptians in 1500 BCE. Later the Greeks in 230 BCE used the term ``diabetes" meaning ``pass through" (or ``siphon") denoting the constant thirst and frequent urination of the patients. In the 1700s the term ``mellitus" (from honey) was added to denote that the urine was sweet and would ``attracts ants".

Diabetes symptoms include frequent urination, thirst, and constant hunger, high blood sugar (hyperglycemia) and insulin resistance. Long term complication from T2D may include eyesight problems, heart disease, strokes and kidney failure. Type II diabetes, is highly correlated with obesity and disease rate has increased dramatically during the last 50 years. According to the World Health Organisation the prevalence of diabetes is 9\% in adults and an estimated 1.5 millions deaths were caused by diabetes in 2012 \cite{REF}, which is predicted to be the 7th leading cause of death by 2030. The costs associated to treating diabetes patients only in the U.S. are estimated around \$245 billion dollars.

In recent years, over 80 genetic loci related to T2D have been identified \cite{REF}. Nevertheless, the overall effect sizes of these loci account for less than 10\% of the overall disease predisposition \cite{REF}. This poses the question of why, given that so much efforts has been directed at finding the genetic components of this disease, the loci found so far have such modest effects. This lack of large genetic effects do not only arise in T2D but also in almost all complex traits and could be explained by what is known as the ``missing heritability" problem.

\subsection{Missing heritability}

We all know that ``tall parents tend to have tall children", which is an informal way to say that height is a highly heritable trait. It is said that there are 30 cm from the tallest 5\% to the shortest 5\% of the population and genetics are accountable for 80\% to 90\% of this variation, which means that 27cm of variance are assumed to be ``carried" by DNA variants from parents to offspring. Since 2010 the GIANT consortia has been investigating the genetic component of complex traits like height, body mass index (BMI) and waist to hip ratio (WHR). Even though they found many variants associated those traits, their findings only explain 10\% of the phenotypic variance which corresponds to only a few centimeters in height \cite{REF:GIANT}.

In order to calculate heritability, we need to be able to measure it, so we need a formal definition. Heritability is defined as the proportion of phenotypic variance that is attributed to genetic variations. The total phenotypic variation is assumed to be caused by a combination of ``environmental" and genetic variations $Var[P] = Var[G] + Var[E] + 2 Cov[G, E]$ \cite{Emerson}.

The environmental variance $Var[E]$ is the phenotypic variance attributable only to environment, that is the variance for individuals having the same genome $Var[E] = Var[P|G]$. Since cloning humans to calculate this term may be an overkill, we resort to approximate it based on phenotypic differences observed in monozygotic and dizygotic twins.

If the covariance factor $Cov[G, E]$ is assumed to be zero, we can define heritability as $H^2 = \frac{Var[G] }{ Var[P]}$. This is called ``broad sense heritability" because $Var[G]$ takes into account all possible forms of genetic variance: $Var[G] = Var[G_A] + Var[G_D] + Var[G_I]$, where $Var[G_A]$ is the additive variance, $Var[G_D]$ is the variance form dominant alleles, and $Var[G_I]$ is the variance form interacting alleles (epistasis). Non-additive terms are difficult to estimate, so a simpler form of heritability called ``narrow sense heritability" that only takes into account additive variance is defined as $h^2 = \frac{ Var[G_A] }{ Var[P] }$ \cite{zuk2012mystery}.

Focusing on narrow sense heritability, the concept of ``explained heritability" is defined as the part of heritability due to known variants with respect to all phenotypic variation ($\pi_{explained} = h^2_{known} / h^2_{all}$). Similarly, missing heritability is defined as $\pi_{missing} = 1 - \pi_{explained} = 1 - h^2_{known} / h^2_{all}$. When all variants associated with traits are known, then $\pi_{missing} = 0$.

Until recently, it was widely assumed by the research community that the problem of missing heritability lied in finding the appropriate genetic variants to account for the numerator of the equation ($h^2_{known}$) \cite{zuk2012mystery}. However, in a series of theorems published recently, it has been proposed that there is a problem in the way the denominator is estimated \cite{zuk2012mystery}. The authors created a limiting pathway model ($LP(k)$) that accounts for epistasis (gene-gene interactions) in $k$ biological pathways. They showed that a severe inflation of $h^2_{all}$ estimators occurs even for small values of $k$ (e.g. $k \in [2,10]$). As a result, genetic variants estimated to account only for $20\%$ of heritability, could actually account for as much as $80\%$ using an appropriate model \cite{zuk2012mystery}.

Even though this result is encouraging, the problem is now shifted to detecting epistatic interactions, a problem that we analyze in section \ref{sec:} and Chapter 4. In the same work \cite{zuk2012mystery}, the authors show an example of power calculation assuming relatively large genetic effect that would require sequencing roughly $5,000$ individuals to detect links to genetic variants, which is a large but nowadays not uncommon, sample size. Nevertheless other estimates place the sample size requirements as high as  $500,000$ individuals \cite{zuk2012mystery}. Even though this sounds as an extremely large number of samples, it is quickly becoming possible thanks to large technological advances and cost reductions in sequencing and genotyping technologies.

\subsection{Conclusions}

Although some genetic causes of complex traits, such as type II diabetes, have been found, only a small portion of the phenotypic variance can be explained. This might indicate that many risk variants are yet to be discovered. Recent studies on the topic of missing heritability report that these ``difficult to find genetic variants" might be in epistatic interaction (analyzed in section \ref{sec:}) or rare variants (see section \ref{sec:}), analysis of either them requires more complex statistical models and larger sample sizes. In Chapter 4 of this thesis, we focus on methods for finding epistatic interactions related to complex disease and develop computationally tractable algorithms that can process data from sequencing experiments involving large number of samples in a reasonable amount of time.

%---
\section{Identification of genetic variants}
%---

Two of the main milestones in genetics were the discovery of the DNA structure in 1953 \cite{watson1953molecular}, followed by the first draft of the human genome in 2004 \cite{collins2004finishing}. The cost of sequencing the first human reference genome was around \$3 billion (unadjusted US dollars) and it was an endeavor that took around 10 years. Since that time, sequencing technology has evolved substantially so that a human genome can now be sequenced in a three days for a price of less than \$1,000, according to prices estimated by Illumina, one of the main genome sequencer manufacturers.

Having a standard reference sequence facilitates comparisons and analysis. For most well known organisms, ``reference genome" sequences are available and current large scale sequencing projects are extending significantly the number of genomes known, e.g. one project seeks to sequence 10,000 mammalian genomes \cite{REF}, another is targeting all microbes that live within human’s guts \cite{REF}.

The amount of information delivered by sequencing devices is growing much faster than computer speed (Moore's law) and data storage capacity. Having to process huge amounts of sequencing information poses several challenges, a problem informally known as ``data deluge''. In the following sections, we explain how sequencing data is generated and how the huge amount of information delivered by a sequencer can be handled in order to make the problem tractable. Just as a crude example, a leading edge sequencing system is advertized to be capable of delivering 18,000 human genomes at $30x$ coverage per year, yielding over 3.2 PB of information. We want to transform this raw data into knowledge of genomic variants that contribute to disease risk with the ultimate goal to translate these risk variants into biological knowledge that can help to design drugs to treat or prevent disease. As expected, processing huge datasets consisting of thousands of sample is a complex problem. In Chapter 2 we show how mitigate or solve some of these issues, by designing a computer language specially tailored to tackle what are know as ``Big data" problems.

\subsection{Sequencing data}

Different technologies for sequencing machines (or sequencers) exists. In a nutshell, a sequencer detects polymers (or chains) of DNA nucleotides and outputs a string of A, C, G, and Ts. Unfortunately, current technological limitations make it impossible to ``read" a full chromosome as one long DNA sequence. Instead, modern sequencers produce a large number of ``short reads", which range 100 bases to 20 Kilo-bases (Kb) in length, depending on the technology. Since sequencers are unable to read long DNA chains, preparing the DNA for sequencing involves fragmenting it into small pieces. These DNA fragments are a random sub-samples of the original chromosomes. Reading each part of the genome several times allows to increase accuracy and ensure that the sequencer reads as much as possible of the original chromosomes. The coverage of a sequencing experiment is defined as the number of times each base of the genome is read on average. For instance, if the sequencing experiment is designed to produce one billion reads, and each read is 150 bases long, then the total number of bases read is 150Gb. Since the human genome is 3Gb, the coverage is said to be 50.

After sequencing a sample, we have millions of reads but we do not know where these reads originate from in the genome. This is resolved by aligning (also called mapping) reads to the reference genome, which is assumed to be very similar to the genome being sequenced. Once the reads are mapped, we can infer if the sample’s DNA has any differences with respect to the reference genome, a problem is known as ``variant calling''. 

Using current technologies and computational methods for variant calling, detection accuracy varies significantly for different variant types. SNV are by far the most accurately detected. Insertions and deletions, collectively referred as InDels, can be detected less efficiently depending on their sizes. Small InDels consisting of ten bases or less are easier to detect than large InDels consisting of 200 bases or more. The reason being that the most commonly used sequencers reads DNA in stretches roughly 200 bases long. Due to this technological limitations, detection is less reliable for more complex variant types.

Although sequencing costs are dropping fast, it is still relatively expensive to sequence thousands of samples and in some cases it makes sense to focus on specific areas of the genome. A popular experimental setup is to focus on coding regions (exons). A technique called ``exome sequencing" consists of capturing exons using a DNA chip and then sequencing the captured DNA fragments only. Exons are roughly 3\% of the genome, thus this technique reduces sequencing costs significantly, for which it has been widely used by many research groups.

\subsection{Sequence alignment}

Given two sequences $s_1$ and $s_2$ from an alphabet (e.g. $\Sigma = \{A,C,G,T\}$), the alignment problem is to add gap characters (`-') to both sequences, so that a distance, such as Levenshtein distance, $d(s_1,s_2)$ is minimized.

This problem has a well known solution, the Smith-Waterman algorithm \cite{REF}, which is a variation of the global sequence alignment solution from Needleman-Wunsch \cite{REF}. The main problem is that the algorithm is $O(l_1 . l_2)$ where $l_1$ and $l_2$ are the length of the sequences. So, Smith-Waterman algorithm is slow for very long sequences, such as the human genome.

In order to speed up sequence alignments, several heuristic approaches emerged. Most notably, BLAST \cite{altschul1990basic}, which is used for mapping sequences several thousand nucleotides long (i.e. longer than a typical sequencer read) to a reference genome. BLAST uses an index to map parts of the query sequence, called seeds, to the reference genome. Once these seeds have been positioned against the reference, BLAST joins the seeds performing an alignment. Since the alignment is performed only using a small part of the reference, the algorithm is much faster.  

\subsection{Read mapping}

Sequence alignment has an exact algorithm solution and several faster heuristic solutions. But even the fastest solutions are too slow to be used with the millions of reads generated in a typical sequencing experiment. Faster algorithms can be used if we relax our requirements in two ways: i) we allow for sub-optimal results, and ii) instead of requiring information of where each base of the read maps to the reference genome, we just want to know where the first base maps. This relaxed version of the alignment algorithm is called ``read mapping'' and the reduced complexity is enough to speed up the computations significantly. An implicit assumption in this formulation, is that the read will be very similar to the reference and that there will be no big gaps.  Once the mapping is performed, the read is locally aligned, a strategy similar to BLAST algorithm \cite{REF}.

Reformulating the problem this way, allows us to use other methods, such as suffix array \cite{durbin1998biological}. Suffix arrays algorithms are fast, but memory requirements are $O[ n \; log(n) ]$ and this becomes the limiting factor. In order to reduce memory footprint of suffix arrays, Ferragina and Manzini \cite{ferragina2000opportunistic} created a data structure based on the Burrows-Wheeler transform.  This structure, known as an FM-Index, is memory efficient yet fast enough to allow mapping high number of reads.  An FM-index for the human genome can be built in only 1Gb of memory, compared to 12Gb required for an equivalent suffix array \cite{li2010fast}.  Given a genome $G$ and a read $R$, an FM-index search can find the $N_{occ}$ occurrences of $R$ in $G$ in $O(|R| + N_{occ} )$ time, where $|R|$ is the length of $R$ \cite{li2010fast}.

Efficient indexing and heuristic algorithms can decrease mapping time considerably.  Nevertheless, these algorithms are not guaranteed to find an optimal mapping.  Several parameters, such as read length, sequencing error profile, and genome complexity profile can affect performance.  The most commonly used implementation of the FM-index mapping algorithms are BWA \cite{li2010fast, li2010fastlong} and Bowtie \cite{langmead2009ultrafast, langmead2012fast}.  Each of them provide optimized versions for the two most common sequencing types: i) short reads with high accuracy \cite{li2010fast,langmead2009ultrafast} or ii) longer reads with lower accuracy \cite{li2010fastlong, langmead2012fast}.

It is worth noting that the mapping problem appears as a consequence of the technological limitations of sequencers.  Having long, highly accurate reads, the problem becomes much easier to solve.  As an extreme example, having only one read which is as long as a chromosome and has no errors requires no mapping processing.  

\subsection{Mapping quality}

Sequencers not only provide sequence information, but also provide an error estimate for each base \cite{li2011statistical}.  This is often referred as a quality ($Q$) value, which is the probability of an error, measured in negative decibels $Q = -10 \; log_{10}(p)$.

Mapping quality is an estimation of the probability that a read is incorrectly mapped to the reference genome. Mapping algorithms provide estimates of mapping errors. In the MAQ model \cite{li2008mapping}, which is one of the earliest models for calculating mapping quality, three main sources of error are explored: i) the probability that a read does not originate from the reference genome (e.g. sample contamination); ii) the probability that the true position is missed by the algorithm (e.g. mapping error); and iii) the probability that the mapping position is not the true one (e.g. if we have several possible mapping positions). It is assumed that the total error probability can be approximated as $\epsilon \approx max(\epsilon_1,\epsilon_2, \epsilon_3)$.

\subsection{Variant calling}

Once the sequencing reads have been mapped to the reference genome, we can try to find the differences between a sequenced sample and the reference genome.  This is referred as  ``variant calling".  Several factors complicate this task, the two main ones being sequencing errors and mapping errors, described in \ref{sec:mapq}.  Using sequencing and mapping error estimates, a maximum likelihood model can infer when there is a mismatch between a sample and the reference genome \cite{li2008mapping}.  This method works best for differences of a single base (SNV), but it can also work with different degrees of success for short insertions or deletions (InDels) usually consisting of less than 10 bases.

Due to the nature of short reads, this family of methods does not work for structural genomic variants, such as large insertions, deletions, copy number variations, inversions, or translocations.  A different family of algorithms are used to identify structural variants, but their accuracy so far has been low compared to SNV calling algorithm \cite{REF}.

Aligning sequences that contain InDels (gaps) is more difficult than ungapped alignments since finding optimal gap boundary depends on the scoring method being used. This biases variant calling algorithms towards detecting false SNVs near InDels \cite{depristo2011framework}.  An approach to reduce this problem is to look for candidate InDels and perform a local realignment in those regions.  This local re-alignment process reduces significantly the number of false positive SNVs \cite{depristo2011framework}. Another approach to reduce the number of false positive SNVs calls near InDels involves the ``Base Alignment Quality" (BAQ) \cite{li2011improving}, which is the probability of misalignment for each base.  It can be shown that replacing the original base quality with the minimum between base quality and BAQ produces an improvement in SNV calling accuracy.  The BAQ can be calculated using a special type of ``Hidden Markov Model" (HMM) designed for sequence alignment \cite{li2011improving, durbin1998biological}. A more sophisticated option for reducing errors consist of performing a local genome re-assembly on each polymorphic region (e.g. HaplotypeCaller algorithm \cite{REF:web_GATK}).

Finally, the error probabilities inferred by the sequencers are far from perfect.  Once the variants have been called, empirical error probabilities can be easily calculated \cite{mckenna2010genome} by comparing sequenced variants to a set of ``gold standard variants" (i.e. variants that have been extensively validated).  This allows to re-calibrate or re-estimate the error profile of the reads.  This is know as a re-calibration step, and usually improves the number of false positives calls \cite{depristo2011framework}.

%---
\section{Functional annotations of genomic variants}
%---

Once DNA is sequenced, reads are mapped and variants are called as described in previous sections, variants identified are annotated in order to gain biological insight. For instance, we would like to know if it is located in a gene and if so whether the variant could be deleterious to the functionality of the protein encoded by the gene. This is the focus of Chapter 3 of my thesis.

The simplest case of a genetic annotation would be to know whether a genetic variant lies onto a gene or not. This would be trivial to calculate, since it only requires comparing the genomic location of the variant with the genomic location of known genes. However, in a sequencing experiment there are usually millions of variants and hundreds of thousands of genomic ``features'' such as genes, transcripts, exons, introns, splice regions, promoters, etc. The sheer volume of data requires time and memory efficient algorithms and data structures.

\subsection{Functional annotations of coding variant}

Genetic variants that are located within the coding region of a protein-coding gene are called coding variants. Although they form a small subset of all variants, they are the ones whose function can best be predicted.  As explained by the central dogma of biology, genetic information flows from DNA molecules to mRNA molecules, which are used as a template to produce proteins. 

A coding variant that produces a codon change is called ``synonymous" or ``non-synonymous" depending on whether the resulting amino acid remains the same of changes. If the variant is synonymous, we can be reasonably confident that there will be almost no effect in protein function, conversely if a non-synonymous variant creates a new STOP codon, it might be a strong indicator that protein function will be  disrupted thus the variant is deleterious.

Estimating the putative effect of large coding variants (duplications, inversions or fusions) is much more challenging than in the cases of simple variants (SNVs, MNVs and small InDels) since there is still not enough studies to determine what effects large coding variants have in protein expression or function.

\subsection{Non-coding annotations}

For variants in non-coding regions of the genome, annotations are more difficult than in coding regions, mostly due to the fact that not only the location of most non-coding features (such as transcription factor binding sites, chromatin modifications, methylation) are not exactly known, but also non-coding features tend to be tissue specific. How DNA variants affect non-coding features is mostly speculative, and even for the few non-coding feature that have predictive models, these are riddled with false positives.

Assuming that non-coding features, or parts of them, have selective pressure to keep their functionality, conservation scores can be used as a proxy on how ``important" these regions are. Nevertheless, this might not apply for certain classes of non-coding features, such as some transcription factors, where there is evidence of negative selection (meaning that transcription factors binding sites might not only not be conserved, but also change more rapidly than other genomic regions).

\subsection{Conclusions}

In Chapter 3 we show two software packages we designed for efficiently performing functional annotations of sequencing variants. These packages, SnpEff \& SnpSift, allow to annotate, prioritize, filter and manipulate variant annotations as well as combine several public or custom-created databases. It should be noted SnpEff was one of the first annotation packages and has become one of the most widely used annotation software in both research and clinical environments. 

%---
\section{Genome wide association studies}
%---

A genome wide association study aims at identifying genetic variants associated to a particular phenotype. First, the genomes (or exome, depending on the study design) of affected individuals (cases) and healthy individuals (controls) need to be sequenced, variants called, annotated and filtered. Then, the goal is to find variants that exhibit some statistical association with the trait or phenotype of interest, which could be a disease status (e.g. diabetes vs healthy), a biomedical measurement (e.g. cholesterol level), or any measurable characteristic (e.g. height). Since the genome is so large, patterns of mutations that suggest correlation may be encountered by chance, so we need to establish statistical significance in order to distinguish true association from spurious ones. Like most studies, we will focus on SNVs, but most methods can be extended to other genomic variants.

\subsection{Single variant tests and models}

Let's imagine that there is only one variant in the whole genome for the cohort we are analyzing. Since each individual has two sets of chromosomes, the variant can be present in one, both, or neither chromosomes. When a variant is in both chromosomes is said to be ``homozygous'', whereas if present in only one of the chromosomes, it is said to be ``heterozygous". So the number of times a non-reference allele is present in an individual, is $ N_{nr} = \{0, 1,2\}$.

When the trait of interest is binary (e.g healthy vs disease), a cohort can be divided into cases and controls and we can build a 3 by 2 contingency table:

\[
\begin{array}{l|c|c|c|}
	& Homozygous Reference & Heterozygous & Homozygous non-reference\\
	& (N_{variant} = 0) & (N_{nr} = 1) & (N_{nr} = 2) \\
    \hline 
    Cases & N_{ca,ref} & N_{ca,het} & N_{ca,hom} \\ 
    \hline 
    Controls & N_{co,ref} & N_{co,het} & N_{co,hom} \\
    \hline 
\end{array} 
\]

Further assumptions about how many variants are required to increase disease risk can reduce this $3 \times 2$ table to a $2 \times 2$ table. In the ``dominant model'', the effect of a mutated gene dominates over the healthy one, so one variant is enough to increase risk. The opposite, called ``recessive model", is when both chromosomes have to be mutated in order to increase risk \cite{balding2006tutorial, clarke2011basic}. In these models, we can count how many cases and controls have at least one variant (dominant model) or two variants (recessive model). This simplifies the previous table, yielding a $2 \times 2$ contingency table, than can be tested using either a $\chi^2$ test or a Fisher exact test \cite{balding2006tutorial}.

Two other commonly used models, are the ``multiplicative" and the ``additive" models \cite{balding2006tutorial,clarke2011basic}. In these models, a disease risk is assumed to be multiplied (or increased) by a factor $\gamma$ with every variant present. We cannot simplify the contingency table, so we assess significance using a Cochran-Armitrage test \cite{clarke2011basic}.

\subsection{Multiple variant tests}

In a real case scenario there are thousands or millions of variants. We can extend the concept shown in the previous section by performing individual tests for each variant present in the cohort. Multiple testing can be addressed either by performing a correction, such as False Discovery Rate \cite{balding2006tutorial, clarke2011basic}, or using a stricter genome wide significance level. There are $3 \times 10^9$ bases in the genome, but taking into account the correlation between nearby variants (linkage disequilibrium), the genome wide significance level is generally accepted to be $p_{value} \leq 10^{-8}$.

In order to check if the null hypothesis of a significance tests is adequate, a QQ-plot is used (i.e. plotting the $y = -log(p_{value})$ vs $x = -log[ rank(p_{value}) / (N+1) ]$, where $N$ is the total number of variants). Adherence of the p-values to a 45 degree line on most of the range implies few systematic sources of association \cite{balding2006tutorial, clarke2011basic}. If the p-values have a higher slope than the $y=x$ line, there might be ``inflation", possibly due to co-factors, such as population structure (see section \ref{sec:popStruct}). If the inflation is not too high (e.g. less than $5\%$), this bias can be corrected by shifting the p-values towards the 45 degree slope. More sophisticated methods are explained in section \ref{sec:popStruct}.

\subsection{Continuous traits and correcting for co-factors}

Methods analyzed so far are suitable for binary ``traits" or ``phenotypes" (e.g. disease vs. healthy individuals). Statistical methods that link genetic information to traits can also be used on continuous or ``quantitative" traits (e.g. weight, height, cholesterol level, etc.). A linear regression can be used assuming the traits are approximately normally distributed \cite{balding2006tutorial, clarke2011basic}. A significance test ($p_{value}$) for linear models can be calculated using an $F$ statistic, but more sophisticated methods are also available \cite{balding2006tutorial, clarke2011basic}.

Using linear models, it is easy to include known co-factors to correct for biases or inflation. For instance, if it is known that a risk increases with age or that males are more susceptible than females, age and sex can be added to the linear equation in order to correct for these effects \cite{balding2006tutorial,clarke2011basic}. In a similar manner, we can add co-factors to binary traits using logistic regression.

\subsection{Population structure}

It is widely accepted that humans started in Africa and migrated to Europe, then to Asia and later to America \cite{hartl1997principles}. Out of an initial population, a few individuals migrate and colonize a new territory. This implies that the genetic variety of the new colony is significantly reduced, compared to the previous population, since the genetic pool is only a small ``founder population". The ``Out of Africa" hypothesis implies that each new migration produced a reduction in genetic variety, also known as a ``population bottleneck'' \cite{hartl1997principles}.

As we previously mentioned, each individual inherits two chromosome sets, a maternal and a paternal one. In a process known as recombination, a chromosome that is formed by part of the maternal chromosome and part of the paternal one, is inherited to the offspring. As a result of recombination, a child has two sets of chromosomes that are one from each parent and, on average, half of a chromosome from each grandparent. This breaking and shuffling of chromosomes every generation, increases genetic diversity. Nevertheless if variants are located nearby in the chromosome, the chances that they are broken apart by recombination event are smaller than if they are further away from each other. This produces a correlation of close variants or ``linkage disequilibrium" (LD). Nearby highly correlated variants are said to be in the same ``LD-block" \cite{hartl1997principles}. If a population has low genetic variety, the LD-blocks are large. So African population has more variety (smallest LD-blocks) and conversely, European, Asian and Amerindian populations have less variety (larger LD-blocks) \cite{hartl1997principles}.

\subsection{Population as confounding variable }

Imagine that we have a cohort of individuals drawn from two populations ($P_A$ and $P_B$) and that individuals in $P_A$ have much higher risk of diabetes than individuals from $P_B$. Now imagine that individuals from $P_A$ have a variant $v_A$ more often, but $v_A$ is actually neutral and has no health effects whatsoever. If we do not take into account population factors, our study would conclude that $variant_A$ is the cause of diabetes, just because we see $variant_A$ more often in affected individuals. In this case is clear that population structure is being a confounding variable. We could avoid this problem by analyzing each population separately \cite{patterson2006population}, but this would cause a loss of statistical power since we have fewer samples.

A population that is a mixture of two or more populations, is known as an ``admixed population''. For instance the ``African-American'' population is a mixture of, roughly, $80\%$ African and $20\%$ European genomes \cite{hartl1997principles,balding2006tutorial}. This means that analyzing a cohort of African-American individuals, we would get population structure as a confounding variable because of population admixture \cite{hartl1997principles}. Obviously, in this case we cannot analyze each population separately, because each individual in the sample is a mixture of two populations.

The admixed population problem can be studied by performing a correction using the eigen-structure of the sample covariance matrix \cite{patterson2006population}. Samples can be arranged as a matrix $C$ where each row is a sample and each column represents a position in the genome where there is a variant. The numbers $C_{i,j}$ in the matrix indicate whether a sample (row $i$) has a non-reference allele at a genomic position (column $j$). Since the allele can be present in zero, one, or two chromosomes in each individual, the possible values for $C_{i,j}$ are $\{0, 1, 2\}$. The covariance matrix is calculated as $M= \hat{C}^T . \hat{C}$, where $\hat{C}$ is the matrix $C$ corrected to have zero mean columns. Usually, the first two to ten principal components of $M$ are used as factors in linear models (see section \ref{sec:lin}) to correct for population structure \cite{patterson2006population}.

Whether a cohort has any population structure and needs correction or not, can be tested using two methods: a) plotting the projections of the first two principal components and empirically observing the number of clusters in the chart, or b) using a statistic of the eigenvalues of $M$ based on Tracy-Widom's distribution \cite{patterson2006population}.

\subsection{Common and Rare variants}

The ``allele frequency" (AF) is defined as the frequency a variant appears in a population. Variants are usually categorized according to AF into three groups: i) Common variants ($AF \geq 5\%$), ``low frequency" ($1\% < AF < 5\%$), and iii) ``rare variants" ($AF < 1\%$). Common variants originated earlier in the population while rare variants are either relatively recent or selected against.

There are three main models for disease susceptibility  \cite{hartl1997principles, gibson2012rare}:i) the Common-Disease-Common-Variant hypothesis (CDCV) assumes that if disease is common, it must be caused by a common variant; ii) the ``infinitesimal hypothesis" proposes that there are many common variants each having small risk effects; and iii) the Common-Disease-Rare-Variant hypothesis proposes that there exists many rare variants, each one having large risk effects.

\subsection{Rare variants test}

The ``rare variant model'' assumes that multiple rare variants have large effects on a trait. The problem is that, since these variants are rare, huge sample sizes are required for tests to identify statistically significant associations. To overcome this problem, methods known as ``burden tests", collapse several rare variants and perform statistical significance tests on grouped variants \cite{li2008methods}. An example of collapsing technique is to count the number or rare variant in a given window and apply a Fisher exact test, as shown in section \ref{sec:single}. A limitation of some burden tests is that they implicitly assume that all rare variants have the same direction of effect, although rare variants might have no effect, be deleterious, or protective \cite{li2008methods,wu2011rare}.

Several techniques allow weighting rare variants by collapsing them using a kernel matrix. This allows to incorporate other information, such as allele frequency and functional annotations. It can be shown that the statistic induced by kernel weighting functions follows a mixture of $\chi^2$ distributions and there is an efficient way to approximate it \cite{li2008methods,wu2011rare}, avoiding computationally expensive permutations tests.

\subsection{Conclusions}

In this section we introduced the basic concepts and methodologies used in GWAS. Although fairly mature, there is still heavy research and continuous improvement on GWAS statistical methods. Not only it is well known that traditional (i.e. single marker) GWAS methods fail under non-additive models \cite{culverhouse2002perspective}, but also variants so far discovered using these methods do not account for all the expected phenotypic variance attributed to genetic causes (i.e. missing heritability). As other authors pointed out, this might be because we need to look for epistatic variants which are not taken into account using these methods. In the next section, and in Chapter 4, we cover the topic of epistatic GWAS analysis.

%---
\section{Epistasis}
%---

Proteins are the most important part of the cell composing up to 50\% of a cell’s dry weight compared to 3\% of the DNA \cite{REF}. Proteins perform their functions mainly by interacting with other proteins, forming complex pathways that lead to a vast array of cellular functions including catalysis of chemical reactions, cell signaling, and structural conformation of the cell. The 3-dimensional structure of the protein, also called ``tertiary structure", is tailored to bind to other proteins in a specific manner to accomplish a functionality. 

Genome wide association studies focus on single variants or nearby groups of variants. An often cited reason for the lack of discovery of high impact risk factors in complex disease is that these models ignore loci interactions \cite{cordell2009detecting} and recently they have been pointed out as a potential solution for the ``missing heritability" problem \cite{REF}. With interactions being so ubiquitous in cell function, one may wonder why they have been so neglected by GWAS. We should point out that there are several reasons: i) models using interactions are much more complex \cite{REF} and by definition non-linear, ii) information on which proteins interacts with which other proteins is incomplete \cite{REF}, iii) in the cases where there protein-protein interaction information is available, precise interacting sites are unknown \cite{REF}. Taking into account the last two items, we need to explore all possible loci combinations, thus the number of Nth order interactions grows as $O(M^N)$ where $M$ is the number of variants \cite{REF}. This requires exponentially more computational power than single loci models. This also severely reduces statistical power, which translates into requiring larger cohort, thus increasing sample collection and sequencing costs \cite{REF}.

In Chapter 4 we develop a computationally tractable model for analyzing putative interaction of pairs of variants from sequencing experiments involving large case / control cohorts of complex disease. Our model is based on combining multiple sequence alignments using a coevolutionary model in order to perform GWAS analysis of pairs of non-synonymous variants that may interact.

5.1 Detection of interacting sites in proteins using co-evolution

Proteins interactions and interaction loci are expensive to identify reliably experimentally and difficult to predict computationally. Some computational prediction methods are based on the assumption that protein interactions sites are under evolutionary pressure to avoid mutations \cite{marks2012protein}, because such mutations reduce the efficiency or even disrupt pathways. Assuming that evolutionary pressure maintains favorable interaction between loci, compensatory mutations can happen more often than non-compensatory ones. The underlying idea is that fitness is higher for compensatory mutations and higher fitness co-occurring mutations would be fixed in the population. Since several organisms have been sequenced, we can try compare orthologous protein sites occurring in all these organisms and seek evidence of coevolution. It should be noted that this approach can be used to detect interacting sites within two different proteins or two interacting sites within the same protein. Detecting interacting sites within the protein can be valuable for determining protein structure. The most widely used method for inferring co-evolution starts from protein multiple sequence alignments ($\mathcal{M}_{sa}$), and identifies a pair of sites (e.g. one site from each interacting protein) that maximizes mutual information ($MI$) \cite{marks2012protein}. It is known that $MI$ has some limitations  \cite{dunn2008mutual} and is biased due to the fact that the multiple sequence alignment are related by an evolutionary process \cite{dunn2008mutual}. This means that some sequences will be very similar since they are evolutionarily close to each other (e.g. human and chimp), whereas other sequences will be very different (e.g. mouse and coelacanth). 

A proper albeit more complex statistical analysis takes into account MSA’s phylogenetic tree.  Sophisticated coevolutionary models are usually designed with the intent of aiding protein structure predictions, which require to pinpoint the exact loci in each protein. These complex models can take anywhere from minutes to days to run for each pair of proteins, thus making them unfit for GWAS-scale analysis. 

We propose to make use of co-evolutionary information to increase the interaction priors in a GWAS model. Since the goal is to increase GWAS priors instead of pinpointing the exact interaction loci, we can relax coevolutionary methods requirements to design computationally tractable models. In Chapter 4, we introduce an epistatic GWAS approach that while combining coevolutionary and sequencing information it is efficient enough to be applied to GWAS-scale, large cohort, datasets.

\subsection{Epistatic GWAS }

Arguably, the most common model linking binary phenotypes (disease vs. healthy) to genotypes is the logistic regression model which relates log odds probability of disease using multiple regression, $ln(\frac{p}{1-p}) = \hat{\beta}^T \hat{g}$, where $p$ is the probability of disease, $\hat{g}$ are the model’s input variables (usually including genotype, sex, age, population structure, etc.), and $\hat{\beta}$ are the logistic regression coefficients. 

Given a set of genotypes (typically genotype analysis includes ~2,000,000 variants and tens of thousands of samples), the simplest way to look for interactions is an exhaustive search of all combinations. This raises two issues: i) multiple testing, which is often resolved by stringent significance threshold, and ii) computational feasibility, which is solved by efficient algorithms, parallelization, and heuristic approaches to quickly discard uninformative loci combinations.

The definition of epistasis, form a statistical perspective, is a ``departure from a linear model" \cite{cordell2009detecting}. This means that in a logistic regression model the input includes terms with each of the genotypes ($g_i$ and $g_j$), as well as an ``interaction term" $gi . g_j$ \cite{cordell2002epistasis}. Although we mainly talk about interaction between two loci, higher order interactions (three or more loci combinations) can be analyzed, but these models require more parameters and extremely large samples are required to accurately fit them.

Although a comprehensive review is out of the scope of this thesis, it is worth mentioning that several other approaches for epistatic GWAS exist. Here we mention a few (shown in alphabetic order):

\begin{itemize}

\item Allele frequency: In \cite{ackermann2012systematic}, an analysis of imbalanced allele pair frequencies is performed under the assumptions that an implicit test for fitness can be achieved looking for over/under-represented allele pairs in a given population.

\item Bayesian model: In \cite{zhang2007bayesian}, a ``Bayesian partitioning model" is used by providing Dirichlet prior distributions for each partition and computing posterior probabilities using Markov chain Monte Carlo (MCMC) algorithms.  The methodology first test individual makers and picks only the top 10\% to further investigate for epistasis, because it is prohibitive to test all loci.

\item Linkage disequilibrium: Studying LD patterns in a population under two-loci model it was shown \cite{zhao2006test} that interactions creates LD in disease population. The authors show how LD-based p-values can uncover interaction and sometimes (in their simulations) outperform logistic regression tests.

\item Machine learning: From a machine learning point of view, finding interacting variants is simply an optimisation and attribute selection procedure \cite{mckinney2006machine}. Several approaches have emerged to tackle the ``interaction problem" and used a variety of different techniques \cite{koo2013review, mckinney2006machine} , such as neural networks, cellular automata, random forests, multifactor dimensionality reduction, support vector machines, etc.

\end{itemize}

Although all these models have advantages under some assumptions, none of them seems to be a ``clear winner" over the rest \cite{cordell2009detecting}, thus currently there are no de-facto standards in epistatic analysis. In light of this, there is need of different approaches to be explored. In Chapter 5 we combine coevolutionary models and GWAS epistasis of pairs of putatively interacting loci, by using Bayes Factors to combine information. 

%---
\section{Thesis roadmap and Contributions}
%---

The original research presented in this thesis covers topics related to the computational and statistical methodologies related to the analysis of sequencing variants to unveil genetic links to complex disease. Broadly speaking, we address three types of problems: (i) Data processing of large datasets from high throughput biological experiments such as resequencing in the context of a GWAS (Chapter 2); (ii) functional annotations, i.e. calculating variant’s impact at molecular, cellular or even clinical level (Chapter 3); (iii) identification of genetic risk factors for complex disease using models that combine population-level and evolutionary-level data to detect putative epistatic interactions (Chapter 4). It should be pointed out that the chapters are ordered similar to the analysis steps we used when analyzing our data for type II diabetes, starting from raw sequencing data and ending with GWAS analysis. When applicable, background material specific to each chapter is presented in a preface, together with an explanation of how that chapter ties in with the rest of the thesis.

This thesis comprises text and figures of scientific articles which have either been published, submitted for publication, or ready to be submitted (waiting upon data embargo restrictions):

\begin{description}
	
	\item[Chapter 2] For this paper, PC conceptualized the idea and performed the language design and implementation. RS \& MB helped in designing robustness testing procedures. PC, RS \& MB wrote the manuscript. 
	
	
		\begin{itemize}
		\item \textbf{Cingolani, Pablo}, Rob Sladek, and Mathieu Blanchette. ``BigDataScript: a scripting language for data pipelines." Bioinformatics 31.1 (2015): 10-16.
		\end{itemize}
	
	
	\item[Chapter 3] For this paper, PC designed, implemented and tested SnpEff \& SnpSift. RS \& MB suggested several extensions for common research use cases. PC, RS \& MB wrote the manuscript. The manuscript was submitted to Nature Protocols, and the editor suggested for it to be published after the main T2D paper is accepted for publication (see next paragraph).
	
		\begin{itemize}
		\item \textbf{Cingolani, Pablo}, Rob Sladek, and Mathieu Blanchette. ``Genomic variant annotation and prioritization" Ready for submission (waiting on consortia paper submission).
		\end{itemize}
	
	The following studies are T2D (type II diabetes) consortia projects which used SnpEff and SnpSift extensibly, several modules were designed with these projects in mind. This are part of a large consortia involving several institutions:
	
		\begin{itemize}
		
		\item McCarthy M., et al (T2D Genes Consortia). ``Variation in protein-coding sequence and predisposition to type 2 diabetes", Ready for submission.
		
		\item Mahajan, Anubha, et al. ``Identification and Functional Characterization of G6PC2 Coding Variants Influencing Glycemic Traits Define an Effector Transcript at the G6PC2-ABCB11 Locus." PLoS genetics 11.1 (2015): e1004876-e1004876.
		
		\end{itemize}
	
	The original SnpEff and SnpSift publications are provided in the appendices:
	
		\begin{itemize}
		
		\item \textbf{Cingolani, Pablo}, et al. ``A program for annotating and predicting the effects of single nucleotide polymorphisms, SnpEff: SNPs in the genome of Drosophila melanogaster strain w1118; iso-2; iso-3." Fly 6.2 (2012): 80-92.
		
		\item \textbf{Cingolani, Pablo}, et al. ``Using Drosophila melanogaster as a model for genotoxic chemical mutational studies with a new program, SnpSift." Toxicogenomics in non-mammalian species (2012): 92.
		
		\end{itemize}
	
	
	\item[Chapter 4] For this paper, PC designed the methodology under the supervision of MB and RS. PC implemented the algorithms. PC, RS \& MB wrote the manuscript. This work uses data from the T2D consortia, thus it cannot be published until the main T2D paper is accepted for publication (according to T2D data embargo).
	
		\begin{itemize}
		\item \textbf{Cingolani, Pablo}, Rob Sladek, and Mathieu Blanchette. ``A co-evolutionary approach for detecting epistatic interactions in genome-wide association studies" Ready for submission (data embargo restrictions).
		\end{itemize}
	
	
\end{description}

\subsection{Other contributions}

Other scientific articles (grouped by topic) published, submitted for publication, or ready to be submitted, not mentioned in this thesis:

\begin{description}

	\item Epigenetics 

	\begin{itemize}
		\item \textbf{Cingolani, Pablo}, et al. ``Intronic Non-CG DNA hydroxymethylation and alternative mRNA splicing in honey bees." BMC genomics 14.1 (2013): 666.
		\item Senut, Marie-Claude, et al. ``Lead exposure disrupts global DNA methylation in human embryonic stem cells and alters their neuronal differentiation." Toxicological Sciences (2014).
		\item Ruden D., ``Epigenetics as an answer to Darwin’s ‘special difficulty’ Part 2: Natural selection of metastable epialleles in honeybee castes", Submitted.
		\item Arko S, et al. ``Lead exposure induces changes in 5-hydroxymethylcytosine clusters in CpG islands in human embryonic stem cells and umbilical cord blood", Submitted.
		\item Senut, Marie-Claude, et al. ``Epigenetics of early-life lead exposure and effects on brain development." Epigenomics 4.6 (2012): 665-674.
	\end{itemize}	
	
	\item GWAS \& Disease 
	
	\begin{itemize}
		\item Oualkacha, Karim, et al. ``Adjusted sequence kernel association test for rare variants controlling for cryptic and family relatedness." Genetic epidemiology 37.4 (2013): 366-376.
		\item Bongfen, Silayuv E., et al. ``An N-ethyl-N-nitrosourea (ENU)-induced dominant negative mutation in the JAK3 kinase protects against cerebral malaria." PloS one 7.2 (2012): e31012.
		\item Hawn, Thomas R., et al. ``Host-directed therapeutics for tuberculosis: can we harness the host?." Microbiology and Molecular Biology Reviews 77.4 (2013): 608-627.
		\item Meunier, Charles, et al. ``Positional mapping and candidate gene analysis of the mouse Ccs3 locus that regulates differential susceptibility to carcinogen-induced colorectal cancer." PloS one 8.3 (2013): e58733.
		\item Caignard, Grégory, et al. ``Genome-wide mouse mutagenesis reveals CD45-mediated T cell function as critical in protective immunity to HSV-1." PLoS pathogens 9.9 (2013): e1003637.
		\item Bouttier M., et al. ``Genomics analysis reveals elevated LXRα signaling reduces M. tuberculosis viability", Submitted.
		\item Bouttier M., et al. ``Genomic analysis of enhancers engaged in M. tuberculosis-infected macrophages reveals that LXR signaling reduces mycobacterial burden", Submitted.
	\end{itemize}	
	
	\item Other 

	\begin{itemize}
		\item \textbf{Cingolani, Pablo}, and Jesus Alcala-Fdez. ``jFuzzyLogic: a robust and flexible Fuzzy-Logic inference system language implementation." FUZZ-IEEE. 2012.
		\item \textbf{Cingolani, Pablo}, and Jesús Alcalá-Fdez. ``jFuzzyLogic: a java library to design fuzzy logic controllers according to the standard for fuzzy control programming."International Journal of Computational Intelligence Systems 
	\end{itemize}	

\end{description}

	%---
\section{Genomes and genetic variants \label{sec:introRef}}
%---

DNA is composed of four basic building blocks, called ``bases'' or ``nucleotides'' \cite{alberts1995molecular}. These four nucleotides, usually abbreviated $\{A, C, G, T\}$, are Adenine, Cytosine, Guanine, and Thymine. Bases form pairs, either as $A-T$ or $C-G$, that pile-up forming two long polymers, with backbones that run in opposite directions giving rise to a double-helix structure \cite{watson1953molecular}. Arbitrarily, one of the polymers is called the positive strand and the other is called the negative strand. 

Proteins are composed by chains of amino acids and, as explained by the central dogma of biology \cite{alberts1995molecular},  DNA is the template that instructs cellular machinery how to produce proteins. There are 20 amino acids, which are the building blocks of all proteins. Each of the twenty amino acids is encoded by a group of three DNA bases called ``codon'' \cite{crick1961general}. More than one codon can code for the same amino acid (i.e. $4^3=64$ codons $ > 20 $ amino acids) allowing for code redundancy. Additionally, there are codons that mark the end of the protein, these are called ``STOP" and signal molecular machinery to end the translation process \cite{brenner1965genetic}.

Proteins compose up to 50\% of a cell's dry weight compared to 3\% of the DNA \cite{alberts1995molecular}. Proteins perform their functions mainly by interacting with other proteins, forming complex pathways that lead to a vast array of cellular functions including catalysis of chemical reactions, cell signaling, and structural conformation of the cell \cite{alberts1995molecular}. The 3-dimensional structure of the protein, also called ``tertiary structure", is tailored to bind to other proteins in a specific manner to accomplish a specific function. 

The human genome has a total of 3 Giga-base-pairs (Gb), and those bases are divided into 22 ``autosomal'' chromosome pairs (in each pair one chromsome is maternally inherited and the other paternally inherited) and ``sex" chromosomes. The longest of the autosomal chromosomes is roughly 250 Mega-bases (Mb) and the shortest one is ~50 Mb.

In order to compare DNA from different individuals (or samples), we need a ``reference genome". Having a standard reference sequence facilitates comparisons and analysis. For most well studied organisms, ``reference genome" sequences are available and current large scale sequencing projects are extending significantly the number of genomes known, e.g. one project seeks to sequence 10,000 mammalian genomes \cite{haussler2009genome}, another is targeting all microbes that live within the human gut \cite{turnbaugh2007human}. The human reference genome (e.g. GRCh37) does not correspond to the DNA of any particular person, but to a ``mosaic" of the genomes of thirteen anonymous volunteers from Buffalo, New York \cite{schneider2013genome}.

When the genome of an individual is sequenced, the DNA is compared to the ``reference genome". Most of the DNA is the same, but there are differences. These differences, generically known as ``genetic variants" (or ``variants", for short), describe the particular genetic make-up of each individual. There are several different ways a sample can differ from a reference genome. Each variant is the result of a mutations that happened at some point in the evolutionary history of the individual (or that of the reference genome). Variant types can be roughly categorized in the following way:

\begin{description}

	\item[Single nucleotide variants (SNV)] or Single nucleotide polymorphism (SNP) are the simplest and more common variants produced by single base difference (e.g. a base in the reference genome, at a given coordinate,  is an `A', whereas the sample is `C'). Depending on whether the variant was identified in an individual or in a population, it is called a Single Nucleotide Variant (SNV) or Single Nucleotide Polymorphism (SNP). It is estimated that there are roughly $3.6M$ SNPs per individual \cite{10002012integrated}. There are several biological mechanisms responsible for this type of variants: i) replication errors, ii) errors introduced by DNA repair mechanism, iii) deamination (a base is changed by hydrolysis which may not be corrected by DNA repair mechanisms), iv) tautomerism (and alteration on the hydrogen bond that results in an incorrect pairing) \cite{griffiths2005introduction}.

	\item[Multiple nucleotide polymorphism (MNP)] are sequence differences affecting several consecutive nucleotides and are typically treated as a single variant locus if they are in perfect linkage disequilibrium (e.g. reference is ‘ACG' whereas the sample is ‘TGC'). .

	\item[Insertions (INS)] refer to a sample having one or more extra base(s) compared to the reference genome (e.g. the reference sequence is ‘AT' and the sample is ‘ACT'). Short insertions and deletions (indels) of a chromosome region range from 1 to 20 bases in length are reported to be 10 to 30 times less frequent than SNV \cite{10002012integrated}. Small insertions are usually attributed to DNA polymerase slipping and replicating the same bases (this produces a type of insertion known as duplication). Large insertions can be caused by unequal cross-over event (during meiosis) or transposable elements.

	\item[Deletions (DEL)] are the opposite of insertions, the sample has some base(s) removed with respect to the reference genome (e.g. reference is ‘ACT' and sample is ‘AT'). As in the case of insertions, deletions can also be caused by ribosomal slippage, cross-over events during meiosis. Those include large deletions, which can result in the loss of an exon or one or more whole genes \cite{alberts1995molecular}. Short deletions are 10 to 30 times less frequent than SNV \cite{10002012integrated}.

	\item[Copy number variations (CNVs)] arise when the sample has two or more copies of the same genomic region (e.g. a whole gene that has been duplicated or triplicated) or conversely, when the sample has fewer copies than the reference genome. Copy number variations are often attributed to homologous recombination events \cite{alberts1995molecular}.

	\item[Rearrangements] such as inversions and translocations are events that involve two or more genomic breakpoints and a reorganization of genomic segments, possibly resulting in gene fusions or loss of critical regulatory elements. Inversions, a type of rearrangement, result from a whole genomic region being inverted.

\end{description}

\noindent As humans have two copies of each autosome, variants could affect zero, one or two of the chromosomes and are called ``homozygous reference", ``heterozygous", and ``homozygous alternative" respectively. Variants are also classified based on how common they are within the population: common ($\ge 5\%$), low frequency ($\le 5\%$), or rare ($\le 0.1\%$). How these types of genetic variants influence traits or disease risk is a topic of intense research that is discussed throughout this thesis.

\section{DNA and disease risk}

It would be fair to say that the Garrod family was fascinated by urine. As a physician at King's College, Alfred Baring Garrod, discovered gout related abnormalities in uric acid \cite{kennedy2001}. His son, Sir Archibald Garrod, was interested in a condition known as alkaptonuria, in which children are mostly asymptomatic except for producing brown or black urine, but by the age of 30 individuals develop pain in joints of the spine, hips and knees. In 1902, Archibald observed that the family inheritance pattern of alkaptonuria resembled Mendel's recessive pattern and postulated that a mutation in a metabolic gene was responsible for the disease. Publishing his finding he gave birth to a new field of study known as ``Human biochemical genetics" \cite{kennedy2001}.

Diseases having simple inheritance patterns, such as alkaptonuria, cystic fibrosis, phenylketonuria and Huntington's are also known as Mendelian diseases \cite{kennedy2001}. The genetic components of several Mendelian diseases have been discovered since the mechanism was first elucidated by Garrod in 1902 and the process has been accelerated in recent years, thanks to the application of DNA sequencing techniques \cite{bamshad2011exome}.

In complex diseases (or complex traits), such as diabetes or Alzheimer's disease, affected individuals cannot be segregated within pedigrees (i.e. no simple pattern of inheritance can be identified). In contrast to Mendelian diseases the aetiology of complex traits is complicated due to factors such as: incomplete penetrance (symptoms are not always present in individuals who have the disease-causing mutation) and genetic heterogeneity (caused by any of a large number of alleles). This makes it more difficult to pinpoint the genetic variants that increase risk of complex disease as demonstrated by the failure of linkage analysis methods and later on GWAS \cite{botstein2003discovering}.

\subsection{Heritability and Missing heritability}

We all know that ``tall parents tend to have tall children", which is an informal way to say that height is a highly heritable trait. It is said that there are 30 cm from the tallest 5\% to the shortest 5\% of the population and genetics account for 80\% to 90\% of this variation \cite{wood2014defining}, which means that 27cm of variance are assumed to be ``carried" by DNA variants from parents to offspring. Since 2010 the GIANT consortia has been investigating the genetic component of complex traits like height, body mass index (BMI) and waist to hip ratio (WHR). Even though they found many variants associated those traits, their findings only explain 10\% of the phenotypic variance which corresponds to only a few centimeters in height \cite{wood2014defining}.

In order to measure heritability we need a formal definition. Heritability is defined as the proportion of phenotypic variance that is attributed to genetic variations. The total phenotypic variation is assumed to be caused by a combination of ``environmental" and genetic variations $Var[P] = Var[G] + Var[E] + 2 Cov[G, E]$ \cite{zuk2012mystery}
\iffinal
\footnote{Although the referenced paper's notation does not seem absolutely consistent, we quote Emerson \textit{``A foolish consistency is the hobgoblin of little minds"} and proceed...}
\fi
.

The environmental variance $Var[E]$ is the phenotypic variance attributable only to environment, that is the variance for individuals having the same genome $Var[E] = Var[P|G]$. This can be estimated by studying monozygotic and dizygotic twins.

If the covariance factor $Cov[G, E]$ is assumed to be zero, we can define heritability as $H^2 = \frac{Var[G] }{ Var[P]}$. This is called ``broad sense heritability" because $Var[G]$ takes into account all possible forms of genetic variance: $Var[G] = Var[G_A] + Var[G_D] + Var[G_I]$, where $Var[G_A]$ is the additive variance, $Var[G_D]$ is the variance from dominant alleles, and $Var[G_I]$ is the variance from interacting alleles (epistasis). Non-additive terms are difficult to estimate, so a simpler form of heritability called ``narrow sense heritability" that only takes into account additive variance is defined as $h^2 = \frac{ Var[G_A] }{ Var[P] }$ \cite{zuk2012mystery}.

Focusing on narrow sense heritability, the concept of ``explained heritability" is defined as the part of heritability due to known variants with respect to phenotypic variation ($\pi_{explained} = h^2_{known} / h^2_{all}$). Similarly, missing heritability is defined as $\pi_{missing} = 1 \pi_{explained} = 1 h^2_{known} / h^2_{all}$. When all variants associated with traits are known, then $\pi_{missing} = 0$.

Until recently, it was widely assumed by the research community that the problem of missing heritability lay in finding the appropriate genetic variants to account for the numerator of the equation ($h^2_{known}$) \cite{zuk2012mystery}. However, in a series of theorems published recently, it has been proposed that there is a problem in the way the denominator is estimated \cite{zuk2012mystery}. The authors created a limiting pathway model ($LP(k)$) that accounts for epistasis (gene-gene interactions) in $k$ biological pathways. They showed that a severe inflation of $h^2_{all}$ estimators occurs even for small values of $k$ (e.g. $k \in [2,10]$). As a result, genetic variants estimated to account only for $20\%$ of heritability, could actually account for as much as $80\%$ using an appropriate model \cite{zuk2012mystery}.

Even though this result is encouraging, the problem is now shifted to detecting epistatic interactions, a problem that we discuss in section \ref{sec:epi} and Chapter \ref{ch:gwas}. In the same work \cite{zuk2012mystery}, the authors show an example of power calculation assuming relatively large genetic effect that would require sequencing roughly $5,000$ individuals to detect links to genetic variants, which is a large but nowadays not uncommon, sample size. Nevertheless other estimates place the sample size requirements as high as  $500,000$ individuals \cite{zuk2012mystery}. Even though this represents an extremely large number of samples, it is quickly becoming possible thanks to large technological advances and cost reductions in sequencing and genotyping technologies.

\subsection{Conclusions}

Although some genetic causes for complex traits, such as type II diabetes, have been found, only a small portion of the phenotypic variance can be explained. This might indicate that many risk variants are yet to be discovered. Recent studies on the topic of missing heritability suggest that the root of these ``difficult to find genetic variants" might be found in epistatic interactions (analyzed in section \ref{sec:epigwas}) or rare variants (see section \ref{sec:comonrare}). Analysis of either requires more complex statistical models and larger sample sizes with the corresponding increase in computational requirements. In Chapter \ref{ch:gwas} of this thesis, we focus on methods for finding epistatic interactions related to complex disease and develop computationally tractable algorithms that can process data from sequencing experiments involving large number of samples in a reasonable amount of time.

%---
\section{Identification of genetic variants}
%---

Two of the main milestones in genetics were the discovery of the DNA structure in 1953 \cite{watson1953molecular}, followed by the first draft of the human genome in 2004 \cite{collins2004finishing}. The cost of sequencing the first human reference genome was around \$3 billion (unadjusted US dollars) and it was an endeavor that took around 10 years. Since that time, DNA sequencing technology has evolved substantially so that a human genome can now be sequenced in three days for a price of less than \$1,000, according to prices estimated by Illumina, one of the main genome sequencer manufacturers \cite{hayden2015is}.

The amount of information delivered by sequencing devices is growing faster than computer speed (Moore's law) and data storage capacity \cite{schatz2010cloud}. Just as a crude example, a leading edge sequencing system is advertized to be capable of delivering 18,000 human genomes at $30 \times$ coverage per year, yielding over 3.2 PB of information. Having to process huge amounts of sequencing information poses several challenges, a problem informally known as ``data deluge''.
% In this section, we explain how sequencing data is generated and how the huge amount of information delivered by a sequencer can be handled in order to make the problem tractable. 
From this raw data we want to obtain a set of candidate genomic variants that contribute to disease risk with the ultimate goal to translate these risk variants into biological knowledge. As expected, processing huge datasets consisting of thousands of sample is a complex problem. In Chapter \ref{ch:bds} we show how mitigate or solve some of these issues, by designing a computer language specially tailored to tackle what are know as ``Big data" problems.

\subsection{Sequencing data}

DNA sequencing machines are based on different technologies, in a nutshell all these technologies detect a set of polymers (or chains) of DNA nucleotides and outputs a set of strings of A, C, G, and Ts. Unfortunately, current technological limitations make it impossible to ``read" a full chromosome as one long DNA sequence. Instead, modern sequencers produce a large number of ``short reads", which range from 100 bases to 20 Kilo-bases (Kb) in length, depending on the technology \cite{quail2012tale}. Since sequencers are unable to read long DNA chains, preparing the DNA for sequencing involves fragmenting it into small pieces. These DNA fragments are a random sub-samples of the original chromosomes \cite{shendure2008next}. Reading each part of the genome several times allows to increase accuracy and ensure that the sequencer reads as much as possible of the original chromosomes. The coverage of a sequencing experiment is defined as the number of times each base of the genome is read on average \cite{shendure2008next,quail2012tale}. For instance, if the sequencing experiment is designed to produce one billion reads, and each read is 150 bases long, then the total number of bases read is 150Gb. Since the human genome is 3Gb, the coverage is said to be $50 \times$.

After sequencing a sample, we have millions of reads but we do not know where these reads originate from in the genome. This is solved by aligning (also called mapping) reads to the reference genome, which is assumed to be very similar to the genome being sequenced. Once the reads are mapped, we can infer if the sample's DNA has any differences with respect to the reference genome, a problem is known as ``variant calling''. 

Although sequencing costs are dropping fast, it is still expensive to sequence thousands of samples and in some cases it makes sense to focus on specific areas of the genome. A popular experimental setup is to focus on coding regions (exons). A technique called ``exome sequencing" \cite{clark2011performance} consists of capturing exons using a DNA chip and then sequencing the captured DNA fragments only. Exons are roughly 1.2\% of the genome, thus this technique reduces sequencing costs significantly, for which it has been widely used by many research groups although it has the disadvantage of only analysing coding genomic variation.

\subsection{Read mapping}

Once the samples have been sequenced, we have a set of reads from the sequencer. The first step in the analysis is finding the location in the reference genome where each read is supposed to originate from, a process that is complicated by a several factors: i) there are differences between the reference genome and the sample genome, ii) sequencing reads may contain errors, iii) several parts of the reference genome are quite similar making reads from those regions indistinguishable, and iv) a typical sequencing experiment generates millions of reads \cite{shendure2008next}.

\paragraph{Local sequence alignment} We introduce a problem known as \textit{local sequence alignment}: Given two sequences $s_1$ and $s_2$ from an alphabet (e.g. $\Sigma = \{A,C,G,T\}$), the alignment problem is to add gap characters (`-') to both sequences, so that a distance, such as Levenshtein distance, $d(s_1,s_2)$ is minimized. This problem has a well known solution, the Smith-Waterman algorithm \cite{smith1981identification}, which is a variation of the global sequence alignment solution from Needleman-Wunsch \cite{needleman1970general}, having an algorithm complexity $O(l_1 . l_2)$ where $l_1$ and $l_2$ are the length of the sequences. So, Smith-Waterman algorithm is slow since in this case one of the sequences is the entire genome.

In order to speed up sequence alignments, several heuristic approaches emerged. Most notably, BLAST \cite{altschul1990basic}, which could be for mapping sequences to a reference genome. BLAST uses an index of the genome to map parts of the query sequence, called seeds, to the reference genome. Once these seeds have been positioned against the reference, BLAST joins the seeds performing an alignment only using a small part of the reference.

\paragraph{Read mapping} Sequence alignment has an exact algorithm solution and several faster heuristic solutions. But even the fastest solutions are too slow to be used with the millions of reads generated in a typical sequencing experiment. Faster algorithms can be used if we relax our requirements in two ways: i) we allow for sub-optimal results, and ii) instead of requiring the output to be a complete local alignment between a read and the genome, we just want to know the region in the reference genome where the read sequence is from. This relaxed version of the alignment algorithm is called ``read mapping'' and the reduced complexity is enough to speed up the computations significantly. In a nutshell, a read mapping is regarded as correct if it overlaps the true reference genome region where the read originated. Once the mapping is performed, the read is locally aligned, a strategy similar to BLAST algorithm \cite{li2010fast, langmead2009ultrafast}.

Reformulating the alignment problem as a \textit{mapping} problem allows us to use data structures such as suffix trees to index the reference genome. Using suffix trees we can query for a substring (read) \cite{durbin1998biological} of the indexed string in $O(m)$ time, where $m$ is the length of the query. Alternatively, we can use suffix arrays which are a space optimized alternative to suffix trees \cite{durbin1998biological}. An implicit assumption in this solution, is that the read is very similar to the reference and that there are no gaps. Suffix arrays algorithms are fast but, even though they are memory optimized versions of suffix trees, memory requirements are still high ($O[ n \; log(n) ]$, where $n$ is the length of the indexed sequence, in this case the reference genome) and this becomes the limiting factor. In order to reduce the memory footprint of suffix arrays, Ferragina and Manzini \cite{ferragina2000opportunistic} created a data structure based on the Burrows-Wheeler transform.  This structure, known as an FM-Index, is memory efficient yet fast enough to allow mapping high number of reads.  An FM-index for the human genome can be built in only 1Gb of memory, compared to 12Gb required for an equivalent suffix array \cite{li2010fast}.  Given a genome $G$ and a read $R$, an FM-index search can find the $N_{occ}$ exact occurrences of $R$ in $G$ in $O(|R| + N_{occ} )$ time, where $|R|$ is the length of $R$ \cite{li2010fast}. 

We should keep in mind that suffix trees, suffix arrays and FM-indexes are guaranteed to find all matching substring occurrences, nevertheless a sequencing read may not be an exact substring of the reference genome (due to sample's genome differences with the reference genome, read errors, etc.). So, even if efficient indexing and heuristic algorithms can decrease mapping time considerably, these algorithms are not guaranteed to find an optimal mapping. 
Several parameters, such as read length, sequencing error profile, and genome complexity profile can affect performance.  The most commonly used implementation of the FM-index mapping algorithms are BWA \cite{li2010fast, li2010fastlong} and Bowtie \cite{langmead2009ultrafast, langmead2012fast}.  Each provides optimized versions for the two most common sequencing types: i) short reads with high accuracy \cite{li2010fast,langmead2009ultrafast} or ii) longer reads with lower accuracy \cite{li2010fastlong, langmead2012fast}. 
It should also be taken into account that read-mapping algorithms implement heuristics to map reads having differences respect to the reference genome, obviously these heuristics are implementation dependent, thus two mapping algorithms can (and often do) lead to different mappings for the same read set which in turn can lead to different variants being called (see section \ref{sec:varcall}).


\paragraph{Mapping quality\label{sec:mapq}} Sequencers not only provide sequence information, but also provide an error estimate for each base \cite{li2011statistical}.  This is often referred as a quality ($Q$) value, which is the probability of an error, measured in negative decibels $Q = -10 \; log_{10}(\epsilon)$, where $\epsilon$ is the error probability. Mapping quality is an estimate of the probability that a read is incorrectly mapped to the reference genome. 

Mapping algorithms provide estimates of mapping quality. In the MAQ model \cite{li2008mapping}, which is one of the earliest models for calculating mapping quality, three main sources of error are explored: i) the probability that a read does not originate from the reference genome (e.g. sample contamination); ii) the probability that the true position is missed by the algorithm (e.g. mapping error); and iii) the probability that the mapping position is not the true one (e.g. if we have several possible mapping positions). It is assumed that the total error probability can be approximated as $\epsilon \approx max(\epsilon_1,\epsilon_2, \epsilon_3)$.

\subsection{Variant calling \label{sec:varcall}}

Genome-wide variant calling has until recently largely been done using genotyping arrays (for SNVs) or Comparative Genomic Hybridization arrays (for CNVs). The inherent limitations of these technologies, particularly their ability to only assay genotypes at sites that are known in advance to be polymorphic, combined with the declining cost of sequencing, have now made approaches based on high-throughput resequencing the tool of choice for variant calling in clinical studies. 

Once the sequencing reads have been mapped to the reference genome, we can try to find the differences between a sequenced sample and the reference genome. This process is called ``variant calling" \cite{nielsen2011genotype}.  Several factors complicate this task, the two main ones being sequencing errors and mapping errors, described in \ref{sec:mapq}. Based on sequencing data and mapping error estimates, tools such as GATK \cite{mckenna2010genome} and SamTools/BcfTools \cite{li2008mapping} use maximum likelihood models can infer when there is a mismatch between a sample and the reference genome and whether the sample is homozygous or heterozygous for the variant. This method works best for differences of a single base (SNV), but it can also work with different degrees of success for short insertions or deletions (InDels) usually consisting of less than 10 bases. 

Aligning sequences that contain InDels (gaps) is more difficult than ungapped alignments since finding optimal gap boundary depends on the scoring method being used. This biases variant calling algorithms towards detecting false SNVs near InDels \cite{depristo2011framework}.  An approach to reduce this problem is to look for candidate InDels and perform a local realignment in those regions.  This local re-alignment process reduces significantly the number of false positive SNVs \cite{depristo2011framework}. Another approach to reduce the number of false positive SNVs calls near InDels involves the ``Base Alignment Quality" (BAQ) \cite{li2011improving}, which is the probability of misalignment for each base.  It can be shown that replacing the original base quality with the minimum between base quality and BAQ produces an improvement in SNV calling accuracy.  The BAQ can be calculated using a special type of ``Hidden Markov Model" (HMM) designed for sequence alignment \cite{li2011improving, durbin1998biological}. A more sophisticated option for reducing errors consist of performing a local genome re-assembly on each polymorphic region (e.g. HaplotypeCaller algorithm \cite{GATK}).

Finally, one should note that the error probabilities inferred by the sequencers are far from perfect.  Once the variants have been called, empirical error probabilities can be easily calculated \cite{mckenna2010genome} by comparing sequenced variants to a set of ``gold standard variants" (i.e. variants that have been extensively validated).  This allows to re-calibrate or re-estimate the error profile of the reads.  This is know as a re-calibration step, and usually improves the number of false positives calls \cite{depristo2011framework}.

Due to the nature of short reads, this family of methods does not work for structural genomic variants, such as large insertions, deletions, copy number variations, inversions, or translocations.  A different family of algorithms are used to identify structural variants generally making use of pair end reads or split reads, but their accuracy so far has been low compared to SNV calling algorithm \cite{o2013low}.

One of the caveats of current sequencing technologies and computational methods for variant calling, detection accuracy varies significantly for different variant types. SNV are by far the most accurately detected. Insertions and deletions, collectively referred as InDels, can be detected less efficiently depending on their sizes. Small InDels \cite{durbin2010map} consisting of ten bases or less are easier to detect than large InDels consisting of 200 bases or more. The reason being that the most commonly used sequencers reads DNA in stretches roughly 200 bases long. Due to this technological limitations, detection is less reliable for more complex variant types.

%---
\section{Functional annotations of genomic variants \label{sec:funann}}
%---

The development of cost-effective, high-throughput next generation sequencing (NGS) technologies have had a profound impact on our ability to study the effects of individual genetic variants on the pathogenesis and progression of both monogenic and common polygenic diseases. As sequencing costs decrease and throughput increases, it has now become possible to quickly identify a large number of sequence polymorphisms (SNVs, indels, structural) using samples from affected and unaffected subjects and investigate these in epidemiologic studies to identify genomic regions where mutations increase disease risk. However, translating this information into biological or clinical insights is challenging as it is often difficult to determine which specific polymorphisms are the main pathogenetic drivers of disease across a population; and more importantly, how they affect the activity of disease-related molecular pathways in tissues and organism a specific patient. In part, this difficulty results from the large number of genetic variants that are observed in individual genomes (the human population is believed to contain approximately 3.5 million polymorphic sites with minor allele frequency above 5\%) combined with the limited ability of computational approaches to distinguish variants with no impact on genome function (the vast majority) from variants affecting gene function or expression that may be associated with disease risk or drug response (the minority). The development of algorithms for automated variant annotation,which link each variant with information that may help predict its molecular and phenotypic impact, is a critical step towards prioritizing variants that may have a functional impact from those that are harmless or have irrelevant functional effects. In this section we review the key concepts and existing approaches in this important field. In Chapter \ref{ch:snpeff} we introduce an approach to collect relevant information that will help answer questions about genetic variants discovered in next-generation sequencing studies, including: (i) will a given coding variant affect the ability of a protein to carry its functions; (ii) will a given non-coding variant affect the expression or processing of a given gene; and ultimately (iii) will a given coding or non-coding variant have any impact on phenotypes of interest?

Answering these questions is essential for many types of analyses that use large-scale genomics datasets to study quantitative traits and diseases, particularly when only a small number of individuals is studied comprehensively at a genome-wide level. For example, most genome-wide association studies (GWAS) or exome sequencing studies lack the statistical power to identify rare variants or variants with small effects associated with a disease, in part due to the large number of variants assayed. This limitation can be addressed by directing both statistical analysis and subsequent experimental steps to focus on smaller sets of genetic variants that have been prioritized based on external evidence of their putative impact. The common impairment of DNA repair mechanisms and chromatin stability in malignant cells leads to a similar challenge in cancer genomics, where the hundreds or thousands of mutations that distinguish an individual's tumor and germline genomes need to be classified on the basis of their putative phenotypic effects and potential roles in carcinogenesis.

The large number of databases containing potentially helpful information about a given variant make the process of gathering and presenting relevant data challenging, despite excellent tools that already exist to analyze large genomics datasets (including GATK \cite{mckenna2010genome} and Galaxy \cite{goecks2010galaxy}) and visualize the results (such as the UCSC \cite{karolchik2014ucsc} or Ensembl \cite{flicek2012ensembl} genome browsers). Each of these databases uses its own format and is updated asynchronously, which makes it difficult for any analysis to remain up to date. In addition, the lack of comprehensive and computationally efficient models that allow integrative analyses using these resources, makes the task of comprehensive variant annotation overwhelming. By efficiently combining information from tens or hundreds of genome-wide databases, the tools described here are designed to greatly facilitate the process of variant annotation, and make it accessible to groups with limited bioinformatics expertise or resources.

%---
\subsection{Variant types}
%---

Although variant calling is a challenging task and remains an important area of research, many high-quality tools exist for calling SNVs and indels.
We discuss here the problem of annotating the variants identified by some of these tools.
The most common type of variant identified by current technologies and analysis approaches is a single base difference with respect to the reference genome (SNV) followed by multiple base differences (MNP), as well as small insertions and deletions (InDels). Here, we focus on annotating these three types of variants which comprise most of the variants in a typical sequencing experiment. We do not address the annotation of large rearrangements due to the challenges involved in their identification and functional characterization and their relative rarity in the germ line.

\subsection{Types of genetic annotations}

The process of genetic variant annotation consists of the collection, integration, and presentation of experimental and computational evidence that may shed light on the impact of each variant on gene or protein activity and ultimately on disease risk or other phenotypes. Variant annotation has traditionally been divided in two apparently independent but actually interrelated tasks based on the variant's location with respect to known protein-coding genes. Coding variant annotation focuses on variants that are located within coding regions of annotated protein-coding genes and attempts to assess their impact on the function of the encoded protein. In contrast, non-coding variant annotation focuses on variants located outside the coding portion of genes (i.e. in intergenic regions, UTRs, introns, or non-protein-coding genes) and aims to assess their potential impact on transcriptional and post-transcriptional gene regulation. These two categories of variant annotations are not mutually exclusive, as variants located within exons can often have an impact on the gene transcript's processing (splicing). In addition, some transcripts can have both protein-coding and non-coding functions \cite{alberts1995molecular}. Despite the intermingling of the notion of coding and non-coding variants, we will consider each type of annotation separately as assessing their impact requires different sources of data and algorithms.

The ultimate goal of variant annotation is to predict the impact of a sequence variant, although this is an ill-defined term. One the one hand, one may be interested in the molecular impact of a variant on the activity of a protein. On the other, others may be interested in a variant's impact on much higher-level phenotypes such as disease risk. Mutations that are predicted to completely abrogate a gene's activity are called loss-of-function (LOF) mutations. Those that are predicted to have less severe consequences are called moderate or low impact mutations. In practice, a variant will be predicted to cause LOF if it has two properties: (i) its molecular impact is reliably predictable by existing computational approaches (e.g. gain of stop-codon); and (ii) its functional impact, reflected by altered protein activity or expression levels, is expected to be large. Many types of variants, including most non-coding variants, may have a large functional impact but lack predictability, and as a consequence are typically not predicted to be LOF variants.

\subsection{Coding variant annotation}

Coding variants occur in translated exons. When a reliable gene annotation is available, their main impact can be classified by determining their effect on the translated amino acid sequence (if any). A synonymous coding variant (also called silent) does not change the sequence of amino acids encoded by the gene, although it may impact aspects of post-transcriptional regulation such as splicing and translation efficiency and can affect the total protein activity through changes in the amount of translated protein that is made in the cell. In contrast, a non-synonymous coding variant changes one or more amino acids encoded by the gene and can directly alter the protein's activity, localization or stability. Non-synonymous variants include missense substitutions that change a single amino acid, nonsense substitutions that lead to the gain of a stop codon, frame-preserving indels that insert or delete one or more amino acids, and frame-shifting indels that may completely alter the protein's amino acid sequence. Primary annotation and assessment of impact, determines whether a variant falls in any of these categories.

\textbf{Caveats}
	\begin{enumerate}[label=\roman*]
	
	\item \textit{Gene misannotation.} Genomic variants that have a significant effect on a protein's expression or function represent a very small fraction of all variants. Assembly and gene annotation errors or genomic oddities that break classical computational models are also rare, but lead to false positives. This implies that one is likely to find a non-negligible fraction of false-positive high-impact variants among the list of what appear to be the strongest candidates for variants with severe effects. Tools such as SnpEff can anticipate some of the most common causes of misannotation, but the number and diversity of the type of events that can lead to false-positives makes the task very challenging. As a consequence, one should always manually inspect the top candidates to ensure that they have been assigned to the correct genes and transcripts.
	
	\item \textit{Gene isoforms.} In higher eukaryotes, most genes have more than one transcript (or isoform), due to alternative promoters, splicing, or polyadenylation sites. For example, a human gene has an average of 8.8 annotated messenger RNA (mRNA) isoforms and some genes are believed to have over 4,000 isoforms resulting from complex splicing programs. For these genes, a variant may be coding with respect to one mRNA isoform and non-coding with respect to another. There are two frequent approaches to address this situation: (i) annotate a variant using the most severe functional effect predicted for at least one mRNA isoform; or (ii) use only a single canonical transcript per gene to perform primary annotation. 
	
	\item \textit{Variant calling for indels.} Variant annotation relies on knowing the exact genomic coordinates of the variant: this is rarely a problem for isolated SNVs; however, insertions and deletions often cannot be located unambiguously. Consider for example the variant $AA \rightarrow A$. This mutation results in the loss of a single base, but was it the first or second A that was deleted? From the standpoint of the cell, this question is irrelevant and deletion of any A will have the same effect. In contrast, from the standpoint of most variant annotation software, deleting the first A is different from deleting the second. Consider the scenario of a previously annotated transcript where the first A is part of the 5' UTR and the second is the first base of a start codon. If the missing base is assigned to the leftmost position in the motif (as is the current convention), the deletion would be annotated as a low impact 5'UTR variant. However, assigning it to the rightmost A would make it appear (incorrectly) to be a high-impact start-codon deletion. Similar issues may arise when considering conservation scores or transcription factor binding site (TFBS) predictions.
	
		\end{enumerate}

\subsection{Loss of function variants}

True LOF variants are difficult to predict computationally, but specific types of genetic changes will frequently lead to severely impaired protein activity. These include i) stop-gains, also known as nonsense mutations; ii) start-loss mutations which change or remove the transcript's start codon; iii) indels causing frameshifts; iv) large deletions that remove either the first exon or at least 50\% of the protein coding sequence; and v) loss of splice acceptor or donor sites that alter the protein-coding sequence. Variants that introduce premature in-frame stop codons (nonsense mutations and most frameshift indels) are expected to abolish protein function, unless the variant is very near the C-terminus of the coding region \cite{yamaguchi2008distribution} (effectively, downstream of the last functional domain in the protein). Such mutations may have severe consequences in affected cells, tissues or organism, as is seen for mutations that cause monogenic diseases \cite{scheper2007translation}. In addition, a new stop codon that lies upstream of the last exon will likely trigger nonsense mediated decay (NMD), a process that degrades mRNA before protein synthesis occurs \cite{nagy1998rule}. NMD predictions are not exact and many factors can affect mRNA degradation, including the variant's distance from the last exon-exon junction or poly-A tail, and the possibility that transcription may re-initiate downstream of the LOF variant \cite{brogna2009nonsense}.

A variant that leads to the loss of a stop codon, sometimes called a read-through mutation, will result in an elongated protein-coding transcript that terminates at the next in-frame stop codon. While there are no general models that predict how deleterious this may be, such variants can also result in aberrant folding and degradation of the nascent proteins, leading to activation of cellular stress response pathways, in addition to their direct effects on protein activity and expression levels \cite{scheper2007translation}.

The effect of the loss of a start codon depends on the location of a replacement start codon with respect to the translation start site and reading frame of the native protein. If the new start codon maintains the reading frame, the only consequence may be the loss of a few amino acids in the protein transcript; however, in many cases, the new start codon will not be in-frame, thus producing a frame-shifted protein that is later degraded. In addition, the new start codon may lack an appropriate regulatory context (for example, if there is no Kozak sequence nearby or if it disrupts 5' UTR folding) leading to reduced expression of an N-terminally truncated protein. Consequently, losing a start codon is thought to be highly deleterious in most cases, due to the potential that it may reduce both protein production and activity.

\textbf{Caveats}
	\begin{enumerate}[label=\roman*]
	
	\item \textit{Rare amino acids.} Through a process called translational recoding, a UGA ``Stop" codon located in the appropriate mRNA context (determined by both primary mRNA sequence and secondary structure) may be translated to incorporate a selenocysteine amino acid (Sec / U) \cite{alberts1995molecular}. In humans, it is known to occur 100 codons located in mRNAs whose 3' UTR contains a Selenocysteine insertion sequence element (SECIS). Since the translation machinery goes so far to encode these special rare amino acids, the expectation is that mutations at those sites would be highly deleterious. This is supported by evidence that reduced efficiency of selenocysteine incorporation is linked to severe clinical outcomes, such as early onset myopathy  \cite{maiti2009mutation} and progressive cerebral atrophy  \cite{agamy2010mutations}.
	
	\item \textit{False-positives in LOF predictions.} Variants predicted to result in a LOF sometimes actually produce proteins that are partially functional  \cite{macarthur2012systematic}. In fact, an apparently healthy individual is typically heterozygous for around 100 predicted LOF variants, and homozygous for roughly 10, but many of those are unlikely to completely abolish the protein function. Indeed, these variants are enriched toward the 3' end of the gene, where they are likely to be less deleterious. 
	
	\end{enumerate}

\subsection{Variants with low or moderate impact}

Compared to the high impact variants discussed above, where extensive prior biological evidence strongly suggests that a specific type of variant will severely impair protein activity, there are few guidelines that can reliably predict how the majority of nonsynonymous (missense) variants will alter protein function or expression. As a result, the primary annotation performed by SnpEff and most related software packages will broadly categorize missense substitutions and their accompanying amino acid changes (e.g. $K154 \rightarrow L154$) as moderate impact variants. Short indels whose length is a multiple of three are treated similarly, unless they introduce a stop codon, as their effect will usually be localized.

Once missense and frame-preserving indel variants are identified, a more detailed estimation of their impact on protein function can be performed using heuristic and statistical models. The most common approaches are based on sequence conservation, either amongst orthologous or homologous proteins, or protein domains, sometimes adding information of the physio-chemical properties of the reference and variant amino acids (e.g. differences in side chain charge, hydrophobicity, or size). The SIFT algorithm \cite{kumar2009predicting} assesses the degree of selection against specific amino acid changes at a given position of a protein sequence by analyzing the substitution process at that site throughout a collection of predicted homologous proteins identified by PSI-BLAST \cite{altschul1997gapped}. Based on this multiple sequence alignment and the highly conserved regions it contains, SIFT calculates a normalized probability of amino acid replacement (called the SIFT score), which estimates the mutation's effect on protein function. Polyphen \cite{adzhubei2010method}, another commonly used tool, takes the process one step further by searching UniProtKB/Swiss-Prot \cite{uniprot2013update} and the DSSP database of secondary structure assignments \cite{joosten2011series} to determine if the variant is located in a known active site in the protein. In contrast to other methods that categorize each variant individually, VAAST \cite{rope2011using}, a commercially available package, computes scores for groups of variants located within a given gene and ``collapses" them into a single category, a concept similar to burden testing performed for rare variants identified in exome sequencing studies. For human proteins, SnpEff makes use of the Database for Nonsynonymous SNVs' Functional Predictions \cite{liu2011dbnsfp} (dbNSFP), which collects scores produced by several impact assessment algorithms in a single database. Individually, impact assessment methods usually have an estimated accuracy of 60\% to 80\% when compared to manually curated databases of human variants, but predictions from several algorithms can be combined to provide a stringent, but more accurate estimate of impact \cite{choi2012predicting}.

In most cases these algorithms apply best to SNVs since these are common in populations and there is more genomic sequence and experimental data available to refine the statistical methods. However, some recently developed algorithms are capable of assessing variants other than SNVs, including PROVEAN \cite{choi2012predicting}, which extends SIFT to assess the functional impact of indels.

\textbf{Caveats}
	\begin{enumerate}[label=\roman*]
	
	\item \textit{Imprecise models of protein function.} Accurate impact assessment of coding variants remains an open problem and most computational predictions are riddled with both false positives and false negatives. While both missense variants and frame-preserving indels are broadly cataloged as having moderate effects, this is mostly due to lack of a comprehensive model and the extremely complex computations that would be required for an in-depth analysis (such as protein structure predictions). In these cases, proteomic information can be revealing. SnpEff adds annotations from curated proteomic databases, such as NextProt  \cite{lane2012nextprot}, which can help to elucidate if a mutation alters a critical protein amino acid or domain (such as amino acids that are post-translationally modified as part of a signaling cascade or that are form the active site of an enzyme) resulting in a protein may no longer function.
	
	\item \textit{Gain of deleterious function.} Computational variant annotation may eventually be able to fairly accurately predict the molecular impact of a variant in terms of the degree to which it translates in a loss of function for the encoded protein. However, gains of function, including the acquired ability to interact with new partners and disrupt their function, remain vastly more difficult to tackle, although several such variants have been linked to disease \cite{whitcomb1996hereditary}.
	
	\item \textit{Unanticipated effects of synonymous variants.} In most cases, synonymous variants are regarded as non-deleterious (or low impact); however, one needs to seriously consider the possibility that they may have greater functional effects by altering mRNA splicing  \cite{coulombe2009fine} or secondary structure  \cite{sabarinathan2013rnasnp}. Synonymous SNVs may also alter translation efficiency, by changing a frequently used to a rarely used codon and have been linked to changes in protein expression  \cite{sauna2011understanding}.
	
	\end{enumerate}

\subsection{Non-coding variant annotation}

Although coding variants represent less than 2\% of variants in the human genome, they make up the vast majority of confirmed disease-related variants that have been validated at a functional level. This may result from ascertainment bias (since variants in coding regions are straightforward to discover and characterize at a basic level and many studies have largely ignored non-coding variants); or may be explained by the increased complexity of computational approaches and lab assays required to predict and validate the impact of non-coding variants; or by their potentially more subtle impact on gene expression or cell function. Nonetheless, in a compendium of current GWAS studies, roughly 40\% of the variants are intergenic and 30\% intronic. Functional studies of these variants are increasingly emphasizing the importance of non-coding genetic variation at risk loci for complex genetic diseases and traits \cite{hindorff2009potential}.

Functional non-coding regions of the genome encompass a wide variety of regulatory elements contained in DNA and RNA molecules that are involved in transcriptional and post-transcriptional regulation. Cis-regulatory elements include (i) binding sites for DNA-binding proteins such as transcription factors and chromatin remodelers; (ii) binding sites for RNA-binding proteins involved in splicing, mRNA localization, or translational regulation; (iii) micro RNA (miRNA) target sites; and (iv) long non-coding RNA (lncRNA) targets on DNA, RNA and proteins. Non-coding transcripts include well-characterized regulatory RNAs (e.g. miRNA, snoRNA, snRNA, piRNA and lncRNAs) as well as RNAs involved directly in protein synthesis (e.g. tRNA and rRNA).  The annotation and impact assessment of non-coding variants presents a significant challenge for several reasons: (i) reliable technologies to study transcriptional regulatory regions on a genome-wide basis are only just reaching maturity and provide limited resolution of binding sites for individual transcription factors and regulatory RNA molecules; (ii) non-coding functional regions of most genomes remain incompletely mapped as they vary widely among different cell types and cell states (for example, in diseased and healthy tissues); (iii) non-coding regulatory elements often are part of complex transcriptional programs that are time-dependent \cite{mattick2001non}, contain many redundant linkages or reciprocal connections between genes and respond to a wide range of intraand extracellular signals; and (iv) genomic regulatory elements rarely have a strict consensus sequence (for example, compare the position weight matrices used to identify transcription factor or miRNA binding sites with the amino acid triplet code) making the effect of a mutation on gene regulatory programs difficult to predict. As a result, high-quality annotation of non-coding variants relies more heavily on experimental data than is the case for coding variants: since many of these experimental techniques did not study the effects of SNVs on gene regulatory programs, they can only be used to annotate variants and not to predict their effects on gene transcription. In the few cases where the effects of SNVs have been studied (for example, the effects of SNVs that are common in a population and located in genetic loci associated with complex diseases), experimental approaches provide highly accurate functional assessment at a cost of reduced coverage compared to computational approaches.

Large-scale projects such as ENCODE \cite{encode2012integrated} and modENCODE \cite{celniker2009unlocking} have made major steps toward mapping gene transcription and transcriptional regulatory regions in many tissues and cell types, but similar studies in diseased tissues remain at an early stage (for example, the growing collection of disease-related epigenomes from the Epigenome Roadmap \cite{bernstein2010nih}). The base-by-base resolution and number of cell states studied for different types of regulatory elements and non-coding transcripts varies widely among datasets; in part due to the lack of sensitive, comprehensive and high-resolution technologies to study the different molecular species and modes of interaction that can be altered by non-coding variants. Efficient technologies for genome-wide, high-throughput mapping of binding sites for RNA-binding proteins (PAR-CLiP \cite{ascano2012identification}), miRNAs (PAR-CLiP \cite{hafner2012genome} and CLASH \cite{helwak2013mapping}) are starting to be applied on a broad scale as are protocols to map transcription factor binding sites (TFBS) which can improve resolution to a single base (Chip-exo \cite{rhee2012chip}). However, in most cases, DNA and RNA binding sites are only imprecisely located within Chip-Seq peaks that span genomic regions hundreds of base pairs in length, with computational approaches being used to pinpoint the bases most likely mediating the interaction. In the absence of more precise localization data, \textit{de novo} computational prediction of binding sites for DNA and RNA binding proteins remains insufficiently accurate to be of much use in annotating single noncoding variants.

This limitation is particularly critical for functional predictions of putative target sites for microRNAs and other regulatory RNA species. MicroRNAs are short RNA molecules that regulate gene expression post-transcriptionally by binding the messenger RNA of a gene through complementary, usually in the 3' region of the transcript, which leads to mRNA degradation or inhibits translation. Sequence variants that cause the loss or gain of a miRNA target site would lead to dysregulation of the gene, with likely deleterious effects. Although miRNAs are relatively well documented in most model organisms including human, their binding sites are only starting to be mapped experimentally, and computational predictions has very low specificity. Meaningful information regarding the possible role of a variant in disrupting a miRNA target site is starting to emerge \cite{liu2012mirsnp}, although variants that create new miRNA binding sites remain under the radar.

Even if the position of a functional element could be perfectly determined, predicting a variant's impact on chromatin conformation, promoter activity, gene expression, or transcript processing remains challenging. For transcription factors, this involves predicting whether the protein will still be able to recognize its mutated site (and with what affinity), as well as predicting the impact of these changes on gene expression levels. The latter is particularly hard to predict as a result of interactions, competition, and redundancy contained in regulatory networks of transcription factors or RNA binding proteins. As a consequence, computational prediction of the functional impact of non-coding variants remains a very active area of research and there is no broad consensus on the best methodology to use \cite{ward2012interpreting}. One significant exception is the identification of variants affecting canonical splice sites, defined as two bases on the 3' end on the intron (splice site acceptor) and 5' end of the intron (splice site donor). Variants that affect canonical splice sites are easily detected and typically lead to abnormal mRNA processing, involving exon loss or extension that leads to loss of function of the encoded protein.

\subsection{Impact assessment of non-coding variants}

Two broad classes of publicly available genome-wide datasets are commonly combined to assess the functional impact of non-coding genetic variants: (i) computational predictions of sequence conservation and sites involved in molecular interactions such as transcription factor and RBP binding, as well as miRNA-mRNA target interactions; and (ii) experimental genome-wide localization assays for DNA binding proteins, histone modifications, and chromatin accessibility.

\paragraph{Computational sources of evidence:} Interspecies sequence conservation plays a key role in scoring and prioritizing non-coding variants. This is based on the assumption is that sites or regions that have been more conserved across species than expected under a neutral model of evolution are likely to be functional; suggesting that mutations contained in them are likely to be deleterious. In the absence of strong experimental data, sequence conservation measures calculated from whole genome multiple alignments, (for example using PhastCons  \cite{siepel2005evolutionarily}, SciPhy  \cite{garber2009identifying}, PhyloP  \cite{pollard2010detection} , and GERP  \cite{davydov2010identifying}), have been developed to provide a generic indicator of function for non-coding variants. Although high conservation scores generally mean that a genomic region may be functional, the converse is not true and many experimentally proven functional noncoding regions show only modest sequence conservation (for example due to binding site turnover events). Finally, some regulatory regions (e.g. specific elements regulating immune response  \cite{raj2013common}) are under positive selection and may thus show less conservation than surrounding neutral regions. 

In humans, genome-wide computational predictions of transcription factor binding sites based on matching to publicly available position weight matrices are available from variety of sources, including Ensembl \cite{flicek2012ensembl} and Jaspar  \cite{bryne2008jaspar}.  Because of the low information content of most binding affinity profiles, the specificity of the predictions is generally very low. Related approaches exist to predict splicing regulatory regions  \cite{fairbrother2002predictive} and miRNA target sites \cite{ziebarth2011polymirts}, some of which are precomputed for whole genomes and available from the UCSC or Ensembl genome browsers. Recent efforts to determine RNA-binding protein sequence affinities can also be used to identify putative binding sites for these proteins in mRNA  \cite{ray2013compendium}.

\paragraph{Experimental sources of evidence:} To investigate the potential impact of variants on transcriptional regulation, many published experimental data sets produced by large-scale projects such as ENCODE \cite{encode2012integrated}, modENCODE \cite{celniker2009unlocking} and Roadmap Epigenomics \cite{bernstein2010nih}, can be used directly by annotation packages. These include: (i) ChIP-seq or ChIP-exo experiments that identify TFBSs on a genome-wide basis; (ii) DNAseI hypersensitivity or Formaldehyde-Assisted Isolation of Regulatory Elements (FAIRE) assays that identify regions with open chromatin; and (iii) ChIP-seq studies to identify the presence of specific promoter or enhancer-associated histone post-translational modifications, which can be combined to identify active, poised, and inactive enhancers and promoters \cite{ray2013compendium}. Most of these data sets are easily available through Galaxy \cite{goecks2010galaxy} (as tracks from the UCSC Genome Browser) or through SnpEff (as downloadable pre-computed datasets). In parallel with the types of studies described above, expression quantitative trait loci (eQTLs) represent an agnostic way to map putative regulatory regions. An increasing number of such loci are available through the GTEX database  \cite{lonsdale2013genotype}. Experimental data that may support assessment of the impact of variants on post-transcriptional regulation remain sparser, although databases such as doRiNa  \cite{anders2011dorina} or starBase  \cite{yang2011starbase} contain genome-wide datasets obtained by CLIP-Seq and degradome sequencing. To our knowledge, these data have yet to be used in the context of variant annotation studies.

\paragraph{Combining sources of evidence:} Despite the variety of computational and experimental sources of evidence available, impact assessment for non-coding variants remains relatively crude, due to the fact that biological models of gene regulation remain fairly simple. Nonetheless, significant steps forward have been made recently, and two web-based tools, HaploReg  \cite{ward2012haploreg} and RegulomeDb  \cite{boyle2012annotation}, perform SNV and indel impact assessment for variants from dbSNV on the basis of a broad body of computational and experimental evidence. Both use pre-computed scores for variants from dbSnp and therefore cannot be used for rare variants, but they are extremely valuable for exploration by associating the variant of interest with a variant in dbSnp via linkage disequilibrium. 

\textbf{Caveats}
	\begin{enumerate}[label=\roman*]
	
	\item \textit{Sparseness of functional sites within ChIP-seq peaks.} Even if a noncoding variant is located in a region that contains a ChIP-seq peak for a given TF and has all the hallmark signatures of regulatory chromatin, the likelihood that it is deleterious remains low, because most DNA bases contained within a peak are non-functional. 
	
	\item \textit{Gain of function mutations.} While this section has focused on variants causing the loss of a functional regulatory element, genetic variants may also create new or more effective transcription factor binding sites. These are substantially harder to detect as they can occur in regions that show no evidence of function in individuals possessing the reference allele, and show little conservation across species. Furthermore, computational methods to predict gain of affinity for a given TF caused by a variant have insufficient specificity to be of much use on their own. 
	
	\end{enumerate}

%---
\subsection{Clinical effect of variants}
%---

One of the most revealing types of annotation of both coding and noncoding variants reports whether the variant has previously been implicated in a phenotype or disease. Although such information is available for only a small minority of all deleterious variants, their number is growing and should be the first type of annotation one seeks out. Clinical annotations, until recently, have been scattered in a large number of specialized databases of medical conditions with a genetic basis, including the comprehensive, manually curated collection of genetic loci, variants and phenotypes in the Online Mendelian Inheritance in Man database \cite{hamosh2005online} (OMIM, www.omim.org); web pages containing detailed clinical and genetic information about uncommon disorders in the Swedish National Board of Health and Welfare Database for Rare Diseases (www.socialstyrelsen.se/rarediseases) and the peer-reviewed NIH GeneReviews collection \cite{bryne2008jaspar} (www.ncbi.nlm.nih.gov/books/NBK1116); and a curated collection of over 140,000 mutations associated with common and rare genetic disorders in the commercial Human Gene Mutation Database \cite{stenson2003human} (HGMD, www.hgmd.org/). In most cases, these datasets do not use standardized data collection or reporting formats; are designed to primarily provide information to patients and health professionals through a web interface; and rely on heterogeneous criteria to describe disease phenotypes and clinical outcomes; pathological and other clinical laboratory data; as well as the genetic and biologic experiments that have been used to demonstrate disease mechanisms at a molecular or cellular level. These shortcomings are being addressed by initiatives that provide centralized, evidence-based, comprehensive collections of known relationships between human genetic variants and their phenotype that are suitable for computational analysis, such as the NIH effort to aggregate records from OMIM, GeneReviews \cite{pagon1993genereviews} and locus-specific databases in ClinVar \cite{landrum2013clinvar} (www.ncbi.nlm.nih.gov/clinvar). 

Another important application of variant detection and annotation is in the study of cancer genomes, which is occurring increasingly in clinical settings to support treatment decisions for advanced tumors. Annotation of variants detected in tumor sequences can be analyzed for clinical cohorts, using similar techniques as other complex traits, as well as for individual patients, using techniques to identify differences between somatic (tumor) and germline (healthy) tissues. In the latter case, one looks for cancer-associated mutations that distinguish the somatic genome of cancer cells of an individual from the germline genome in order to find the driving mutations that pinpoint the specific mechanisms underlying tumorigenesis or metastasis. Ideally, these mutations can be used to select a treatment for the patient, establish prognosis, or to identify causative mutations that have led to the cancer's progression. In such a setting, given that sequence differences between the cancer and germline genomes are of greater interest than the background genetic changes between the germline and a reference genome, variant calling is performed using specialized algorithms, such as MuTect  \cite{cibulskis2013sensitive} and SomaticSniper  \cite{larson2012somaticsniper}.

One of the main problems in these databases is annotation accuracy. Biological knowledge, as well as molecular and phenotypic evidence supports the identification of certain groups of high impact variants based on simple criteria (such as premature stops, frameshifts, start lost and rare amino acid mutations); however, it is often hard to predict whether non-synonymous variants will have equally large effects on an organism's health. Even when the accepted ``rules of thumb" used in the primary annotation indicate that protein function is impaired, we should consider that these predictions may be based on a small number of model genes and will require appropriate wet-lab validation or confirmatory studies in cohorts. In addition, as more human genomes are sequenced, it is likely that some genetic variants that have been linked to Mendelian diseases will be found in healthy individuals  \cite{riggs2013towards}; and in many cases, may not actually be disease-causing mutations  \cite{bell2011carrier}.

\subsection{Data structures and computational efficiency}

Most computational pipelines for genomic variant annotation and primary impact assessment are relatively efficient and can annotate variants obtained from large resequencing projects involving thousands of samples within a few minutes or hours even using a moderately powered laptop. This is typically achieved through two key optimizations: (i) creation of reference annotation databases and (ii) implementation of efficient search algorithms. Reference database creation refers to the process of creating and storing precomputed genomic data from the reference genome, which can be searched quickly to extract information relevant to each variant. This process needs to be performed only once per reference genome and most annotation tools have pre-computed databases for many organisms available for users to download.

Since these databases are typically quite large, efficient search algorithms are used together with appropriate data structures to optimize the search process. In ANNOVAR \cite{wang2010annovar}, each chromosome is subdivided in a set of intervals of size $k$ and genomic features for a given chromosome are stored in a hash table of size $L/k$, where $L$ is the length of the chromosome. Another approach, used by SnpEff, is to use an ``interval forest", which is a hash of interval trees  \cite{cormen2001introduction} indexed by chromosome. Querying an interval tree requires $O[log(n) + m]$ time, where $n$ is the number of features in the tree and m is the number of features in the result. 

\subsection{Conclusions}

In Chapter \ref{ch:snpeff} we introduce SnpEff \cite{cingolani2012program} \& SnpSift \cite{cingolani2012using}, two approaches we designed for efficiently performing functional annotations of sequencing variants. These packages allow annotating, prioritizing, filtering and manipulating variant annotations as well as combining several public or custom-created databases. It should be noted SnpEff was one of the first annotation packages and has become one of the most widely used annotation software in both research and clinical environments. 

%---
\section{Genome wide association studies}
%---

A genome wide association study aims at identifying genetic variants associated to a particular phenotype. First, the genomes (or exome, depending on the study design) of affected individuals (cases) and healthy individuals (controls) need to be sequenced, variants called, annotated and filtered. Then, the goal is to find variants that exhibit some statistical association with the trait or phenotype of interest, which could be a disease status (e.g. diabetic vs healthy), a biomedical measurement (e.g. cholesterol level), or any measurable characteristic (e.g. height). Since the genome is so large, patterns of mutations that suggest correlation may be encountered by chance, so we need to establish statistical significance in order to distinguish true association from spurious ones. Like most studies, we will focus our discussion on SNVs, but most methods can be extended to other genomic variants.

\subsection{Single variant tests and models \label{sec:single}}

Consider a simple situation where there is only one variant in the whole genome for the cohort we are analysing. Since each individual has two sets of chromosomes, the variant can be present in one, both, or neither chromosomes, so the number of times a non-reference allele is present in an individual, is $ N_{nr} = \{0, 1,2\}$.

When the trait of interest is binary (e.g healthy vs disease), a cohort can be divided into cases and controls and we can build a 3 by 2 contingency table:

\[
\begin{array}{l|c|c|c|}
	& Homozygous Reference & Heterozygous & Homozygous non-reference\\
	& (N_{variant} = 0) & (N_{nr} = 1) & (N_{nr} = 2) \\
    \hline 
    Cases & N_{ca,ref} & N_{ca,het} & N_{ca,hom} \\ 
    \hline 
    Controls & N_{co,ref} & N_{co,het} & N_{co,hom} \\
    \hline 
\end{array} 
\]

Further assumptions about how many variants are required to increase disease risk can reduce this $3 \times 2$ table to a $2 \times 2$ table. In the ``dominant model'', the effect of a mutated gene dominates over the healthy one, so one variant is enough to increase risk. The opposite, called ``recessive model", is when both chromosomes have to be mutated in order to increase risk \cite{balding2006tutorial, clarke2011basic}. In these models, we can count how many cases and controls have at least one variant (dominant model) or two variants (recessive model). This simplifies the previous table, yielding a $2 \times 2$ contingency table, than can be tested using either a $\chi^2$ test or a Fisher exact test \cite{balding2006tutorial}.

Two other commonly used models, are the ``multiplicative" and the ``additive" models \cite{balding2006tutorial,clarke2011basic}. In these models, a disease risk is assumed to be multiplied (or increased) by a factor $\gamma$ with every variant present. We cannot simplify the contingency table, so we assess significance using a Cochran-Armitrage test \cite{clarke2011basic}.

\subsection{Multiple variant tests}

In a real case scenario there are thousands or millions of variants in a typical resequencing or genotyping study. We can extend the concept shown in the previous section by performing individual tests for each variant present in the cohort. Multiple testing can be addressed either by performing a correction, such as False Discovery Rate \cite{balding2006tutorial, clarke2011basic}, or using a stricter genome wide significance level. There are $3 \times 10^9$ bases in the genome, but taking into account the correlation between nearby variants (linkage disequilibrium), the genome wide significance level is generally accepted to be $p_{value} \leq 10^{-8}$.

In order to check if the null hypothesis of a significance tests is adequate, a QQ-plot is used \cite{clarke2011basic} (i.e. plotting the $y = -log(p_{value})$ vs $x = -log[ rank(p_{value}) / (N+1) ]$, where $N$ is the total number of variants). Adherence of the p-values to a 45 degree line on most of the range implies few systematic sources of association \cite{balding2006tutorial, clarke2011basic}. If the p-values have a higher slope than the $y=x$ line, there might be ``inflation", possibly due to co-factors, such as population structure (see section \ref{sec:popStruct}). If the inflation is not too high (e.g. less than $5\%$), this bias can be corrected by shifting the p-values towards the 45 degree slope. More sophisticated methods are explained in section \ref{sec:popStruct}.

\subsection{Continuous traits and correcting for co-factors \label{sec:cofactors}}

The methods described so far are suitable for binary ``traits" or ``phenotypes". Statistical methods that link genetic information to traits can also be used for continuous or ``quantitative" traits (such as weight, height and measurements of cholesterol level). A linear regression can be used assuming the traits are approximately normally distributed \cite{balding2006tutorial, clarke2011basic}. A significance test ($p_{value}$) for linear models can be calculated using an $F$ statistic, but more sophisticated methods are also available \cite{balding2006tutorial, clarke2011basic}.

Using linear models, it is easy to include known co-factors to correct for biases. For instance, if it is known that a risk increases with age or that males are more susceptible than females, age and sex can be added to the linear equation in order to correct for these effects \cite{balding2006tutorial,clarke2011basic}. In a similar manner, we can add co-factors to binary traits using logistic regression.

\subsection{Population structure \label{sec:popStruct}}

It is widely accepted that humans started in Africa and migrated to Europe, then to Asia and later to America \cite{hartl1997principles}. Out of an initial population, a few individuals migrate and colonize a new territory. This implies that the genetic variety of the new colony is significantly reduced, compared to the previous population, since the genetic pool is only a small ``founder population". The ``Out of Africa" hypothesis implies that each new migration produced a reduction in genetic variety, also known as a ``population bottleneck'' \cite{hartl1997principles}.

As we previously mentioned, each individual inherits two chromosome sets, a maternal and a paternal one. Through recombination a chromosome is formed by a crossover combining maternal and paternal chromosomes and then passed down, thus the offspring has two sets of chromosomes, one from each parent. This breaking and shuffling of chromosomes every generation, increases genetic diversity. Nevertheless if variants are located nearby in the chromosome, the chances that they are separated by recombination event are smaller than if they are further away from each other. This produces a correlation of close variants or ``linkage disequilibrium" (LD). Nearby highly correlated variants are said to be in the same ``LD-block" \cite{hartl1997principles}. If a population has low genetic variety, the LD-blocks are large. So African population has more variety (smaller LD-blocks) and conversely, European, Asian and Amerindian populations have less variety (larger LD-blocks) \cite{hartl1997principles}.

\subsection{Population structure as confounding variable }

Imagine that we have a cohort of individuals drawn from two populations ($P_A$ and $P_B$) and that individuals in $P_A$ have much higher risk of diabetes than individuals from $P_B$. Now imagine that individuals from $P_A$ have a variant $v_A$ more often, but $v_A$ is actually neutral and has no health effects whatsoever. If we do not take population factors into account , our study would conclude that $v_A$ is associated with increased susceptibility to diabetes, just because we see $v_A$ more often in affected individuals. In this case it is clear that population structure is a confounding variable. We could avoid this problem by analyzing each population separately \cite{patterson2006population}, but this would cause a loss of statistical power since we have fewer samples.

A population that is a mixture of two or more populations is known as an ``admixed population''. For instance the African-American population is a mixture of, roughly, $80\%$ African and $20\%$ European genomes \cite{hartl1997principles,balding2006tutorial}. In the case that structure is confounding an analysis of an admixed population, such as an African-American cohort, it is not possible to perform a separate analysis of each sub-population simply because each individual in the sample genetic background from both populations \cite{hartl1997principles}.

The admixed population problem can be studied by performing a correction using the eigen-structure of the sample covariance matrix \cite{patterson2006population}. Samples can be arranged as a matrix $C$ where each row is a sample and each column represents a position in the genome where there is a variant. The numbers $C_{i,j}$ in the matrix indicate the number of non-reference alleles in a sample (row) at a genomic position (column $j$). Since the allele can be present in zero, one, or two chromosomes in each individual, the possible values for $C_{i,j}$ are $\{0, 1, 2\}$. The covariance matrix is calculated as $M= \hat{C}^T . \hat{C}$, where $\hat{C}$ is the matrix $C$ corrected to have zero mean columns. Usually, the first two to ten principal components of $M$ are used as factors in linear models (see section \ref{sec:cofactors}) to correct for population structure \cite{patterson2006population}.

Whether a cohort has any population structure and needs correction or not can be tested using two methods: a) plotting the projections of the first two principal components and empirically observing the number of clusters in the chart, or b) using a statistic of the eigenvalues of $M$ based on Tracy-Widom's distribution \cite{patterson2006population}.

\subsection{Common and Rare variants\label{sec:comonrare}}

The ``allele frequency" (AF) is defined as the frequency a variant appears in a population. Variants are usually categorized according to AF into three groups: i) Common variants ($AF \geq 5\%$), ``low frequency" ($1\% < AF < 5\%$), and iii) ``rare variants" ($AF < 1\%$). Common variants originated earlier in the population while rare variants are either relatively recent or selected against.

There are three main models for disease susceptibility  \cite{hartl1997principles, gibson2012rare}:i) the Common-Disease-Common-Variant hypothesis (CDCV) assumes that if disease is common, it must be caused by a common variant; ii) the ``infinitesimal hypothesis" proposes that there are many common variants each having small risk effects; and iii) the Common-Disease-Rare-Variant hypothesis proposes that there exists many rare variants, each one having large risk effects.

\subsection{Rare variants test}

The ``rare variant model'' assumes that multiple rare variants have large effects on a trait. The problem is that, since these variants are rare, huge sample sizes are required for tests to identify statistically significant associations. To overcome this problem, methods known as ``burden tests" collapse groups of rare variants that are hypothesised to have  similar effect on gene or protein activity and perform statistical significance tests on the group \cite{li2008methods}. An example of collapsing technique is to count the number or rare variant in a genomic region surrounding an exon or a gene and apply a Fisher exact test, as shown in section \ref{sec:single}. A limitation of some burden tests is that they implicitly assume that all rare variants have the same direction of effect, although in reality they might have no effect, be deleterious, or protective \cite{li2008methods,wu2011rare}.

Several techniques allow weighting rare variants by collapsing them using a kernel matrix. This allows to incorporate other information, such as allele frequency and functional annotations. It can be shown that the statistic induced by kernel weighting functions follows a mixture of $\chi^2$ distributions and there is an efficient way to approximate it \cite{li2008methods,wu2011rare}, avoiding computationally expensive permutations tests.

	%---
\section{Epistasis \label{sec:epi}}
%---

In this section we introduced the basic concepts and methodologies used in GWAS. Although fairly mature, there is still heavy research and continuous improvement on GWAS statistical methods. Not only it is well known that traditional (i.e. single marker) GWAS methods fail under non-additive models \cite{culverhouse2002perspective}, but also variants so far discovered using these methods do not account for all the expected phenotypic variance attributed to genetic causes (i.e. missing heritability). As other authors pointed out \cite{cordell2009detecting, zuk2012mystery, zuk2014searching}, this might be because we need to look for epistatic variants which are not taken into account using these methods. In the next section, and in Chapter \ref{ch:gwas}, we cover the topic of epistatic GWAS analysis.

\subsection{Historical perspective}
William Bateson first described epistasis in 1907.(2) Like pleiotropy, this concept was developed to explain deviations from Mendelian inheritance \cite{tyler2009shadows}
The term literally means ``standing upon", and Bateson used it to describe characters that were layered on top of other characters thereby masking their expression.  \cite{tyler2009shadows}
The commonly used definition of epistasis--an allele at one locus masks the expression of an allele at another locus--reflects this original definition. \cite{tyler2009shadows}

The term `epistasis' was initially used in the context of Mendelian inheritance; environmental effects are relatively unimportant for Mendelian traits, so Ii individuals can be clearly assigned to one of a limited number of classes according to their phenotype. Here, epistasis was used to describe the situation in which the actions of one locus mask the allelic effects of another locus, in the same way that completely dominant alleles mask the effects of the recessive allele at the same locus. \cite{carlborg2004epistasis}

The term `epistatic' was first used in 1909 by Bateson (1) to describe a masking effect whereby a variant or allele at one locus (denoted at that time as an `allelomorphic pair') prevents the variant at another locus from manifesting its effect.  \cite{cordell2002epistasis}
This was seen as an extension of the concept of dominance. There are, however, some problems with this definition, particularly when applied to binary traits. In human genetics, the phenotype of interest is often qualitative and usually dichotomous, indicating presence or absence of disease. \cite{cordell2002epistasis}
Mathematical models for the joint action of two or more loci usually focus on the penetrance, the probability of developing disease given genotype. \cite{cordell2002epistasis}
Suppose that a predisposing allele is required at both loci in order to exhibit the trait, i.e. one or more copies of both allele A and allele B are required. Then, when the effects of both loci are considered, we obtain the penetrance table shown in Table 2. In this table, the effect of allele A can only be observed when allele B is also present: without the presence of B, the effect of A is not observable. The effect at locus A would appear to be `masked' by that at locus B. \cite{cordell2002epistasis}
This leads to a situation that is not precisely analogous to that described by Bateson (1). In Bateson's (1) definition, it is clear that if factor B is epistatic to factor A, we do not expect factor A to also be epistatic to factor B.  \cite{cordell2002epistasis}
Table 3 is usually assumed to correspond to a situation in which the biological pathways involved in disease influenced by the two loci are at some level separate or independent (5). \cite{cordell2002epistasis}

Epistasis, or interactions between genes, has long been recognized as fundamentally important to understanding the structure and function of genetic pathways and the evolutionary dynamics of complex genetic systems. \cite{phillips2008epistasis}
It has been approximately 100 years since William Bateson invented the term `epistasis' to describe the discrepancy between the prediction of segregation ratios based on the action of individual genes and the actual outcome of a dihybrid cross1 \cite{phillips2008epistasis}
The use of the term epistasis has since expanded to describe nearly any set of complex interactions among genetic loci  \cite{phillips2008epistasis}
Over the years geneticists have used epistasis to describe three distinct things: the functional relationship between genes, the genetic ordering of regulatory pathways and the quantitative differences of allele-specific effects \cite{phillips2008epistasis}
Over the years the disparate needs of geneticists have led to a plethora of differently nuanced meanings for the term epistasis, all of which involve gene interactions at various levels  \cite{phillips2008epistasis}
`Functional epistasis' addresses the molecular interactions that proteins (and other genetic elements) have with one another, whether these interactions consist of proteins that operate within the same pathway or of proteins that directly complex with one another18 \cite{phillips2008epistasis}
`Compositional epistasis' is a new term that is intended to describe the traditional usage of epistasis as the blocking of one allelic effect by an allele at another locus.  \cite{phillips2008epistasis}
`statistical epistasis' is the usage of epistasis that is attributed to Fisher (BOX 1), in which the average deviation of combinations of alleles at different loci is estimated over all other genotypes present within a population.  \cite{phillips2008epistasis}

It should be apparent that the global analysis of geneinteraction patterns bears a striking resemblance to what is now called systems biology \cite{phillips2008epistasis}


\subsection{Definition}
In this review, we provide a historical background to the study of epistatic interaction effects and point out the differences between a number of commonly used definitions of epistasis \cite{cordell2002epistasis}
Sometimes mutations in two genes produce a phenotype that is surprising in light of each mutation's individual effects. This phenomenon, which defines genetic interaction, can reveal functional relationships between genes and pathways. \cite{mani2008defining}
Recent studies have used four mathematically distinct definitions of genetic interaction (here termed Product, Additive, Log, and Min). Whether this choice holds practical consequences has not been clear, because the definitions yield identical results under some condition \cite{mani2008defining}
Here, we show that the choice among alternative definitions can have profound consequences. \cite{mani2008defining}

A quantitative genetic interaction definition has two components: a quantitative phenotypic measure and a neutrality function that predicts the phenotype of an organism carrying two noninteracting mutations. Interaction is then defined by deviation of a double-mutant organism's phenotype from the expected neutral phenotype \cite{mani2008defining}
A double mutant with a more extreme phenotype than expected defines a synergistic (or synthetic) interaction between the corresponding mutations (synthetic lethality, in the extreme case). \cite{mani2008defining}
Alleviating or ``diminishing returns" interactions, in which the double-mutant phenotype is less severe than expected, often result when gene products operate in concert or in series within the same pathway. Alleviating interactions arise, for example, when a mutation in one gene impairs the function of a whole pathway, thereby masking the consequence of mutations in additional members of that pathway. \cite{mani2008defining}
One class of phenotype, fitness, has been central to many large-scale genetic interaction studies. Although fitness was originally measured in terms of population allele frequencies (1, 22, 23), it can also be measured by using growth rates of isogenic microbial cultures. \cite{mani2008defining}
Genetic interaction studies have used different measures of fitness, including: (i) the exponential growth rate of the mutant strain relative to that of wild type (4, 9, 15, 19) (the relative-growthrate measure); (ii) the increase in mutant population relative to wild type in one wild-type generation (the relative-population measure) (6); and (iii) the number of progeny per mutant organism relative to the number of progeny for wild type in one wild-type generation (the relative-progeny measure) (24) \cite{mani2008defining}
Genetic interaction studies have also differed in their choice of neutrality functions, generally using either a multiplicative or a minimum mathematical function. \cite{mani2008defining}
The multiplicative function, which was originally applied to fitness measures defined in terms of allele frequencies, predicts double-mutant fitness to be the product of the corresponding single-mutant fitness values. The multiplicative function can be combined with each of the three fitness measures above to yield three distinct definitions of genetic interaction (4, 6, 15, 19, 24). \cite{mani2008defining}
A fourth (Min) definition of genetic interaction results from the minimum neutrality function, under which noninteracting mutations are expected to yield the fitness of the less-fit single mutant. Each fitness measure above yields an identical set of genetic interactions under this function. A hypothetical example illustrates one rationale for the Min definition: Two single mutations each disrupt a distinct cellular pathway that limits cell growth, such that one of these mutations is substantially more limiting than the other. The double mutant might then be expected to exhibit the phenotype of the most-limiting single mutant.  \cite{mani2008defining}
It has not been clear whether the choice of genetic interaction definition has any practical consequences. To evaluate the impact of definition choice, we applied each of the four definitions in turn to two reference studies. \cite{mani2008defining}
Here, we show that the choice of definition can dramatically alter the resulting set of genetic interactions and the extent to which they correspond to shared gene function.  \cite{mani2008defining}
For a gene pair (x, y), we refer to the fitness of the two single mutants and the double mutant, respectively, as Wx, Wy, and Wxy. \cite{mani2008defining}
The neutrality function E(Wxy), predicting double-mutant fitness for a strain with mutations in noninteracting genes x and y, is defined differently under the Min, Product, Log, and Additive  \cite{mani2008defining}

DATASET: To evaluate the impact of definition choice, we applied each of the four definitions in turn to two reference studies, St. Onge et al. (19) (Study S) and Jasnos and Korona (6) (Study J), both providing quantitative growth-rate measurements of isogenic wild-type and singleand double-mutant cell populations. \cite{mani2008defining}
RESULTS: The Choice of Genetic Interaction Definition Matters: \cite{mani2008defining}
Additive and Log Definitions Demonstrate Different Biases: However, we had observed that interaction strength had a significant positive bias (under all definitions) for pairs involving mutations with extreme fitness effects. \cite{mani2008defining}
Product and Log Definitions Are Equivalent for Deleterious Mutations:  \cite{mani2008defining}
The Product Definition Reveals Functional Relationships Missed by the Min Definition. \cite{mani2008defining}
Genetic Interaction Networks from Min and Product Definitions Differ Greatly. \cite{mani2008defining}

WHICH DEFINITION TO USE?: We examined the distribution of 􏰍, the deviation of the expected double-mutant phenotype from the observed double mutant phenotype, and found the Product and Log definitions to be closest to this ideal in general. Additionally, we showed that the Log and Product definitions are practically equivalent when both single mutants are deleterious. \cite{mani2008defining}


\subsection{Epistasis in quantitative traits}
 In the case of QUANTITATIVE TRAITS, epistasis describes the general situation in which the phenotype of a given genotype cannot be predicted by the sum of its component single-locus effects1 \cite{carlborg2004epistasis}
Epistatic QTL-mapping studies in model organisms have detected many new interactions and have therefore concluded that epistasis makes a large contribution to the genetic regulation of complex traits.  \cite{carlborg2004epistasis}
Complex synthetic interactions. : There is no reason to expect all forms of epistasis to be revealed simply by the absence of a gene, which is certainly an extreme approach to perturbing complex systems. For example, Kroll et al.35 devised a method for looking for interactions that are induced after systematically overexpressing genes. Using this approach, sopko et al.36 found that, when overexpressed in Saccharomyces cerevisiae, about 15\% of a set of 5,280 yeast genes induced a growth defect, with most of the overexpression effects not matching the phenotypes of their corresponding deletions.  \cite{phillips2008epistasis}


\subsection{Epistasis is ubiquitous}
From mutational studies we know that epistasis in the classical sense is ubiquitous because genes interact in hierarchical systems to generate biological function.  \cite{phillips2008epistasis}
From a biological standpoint, there is no a priori reason to expect that traits should be additive. Biology is filled with nonlinearity: The saturation of enzymes with substrate concentration and receptors with ligand concentration yields sigmoid response curves; cooperative binding of proteins gives rise to sharp transitions; the outputs of pathways are constrained by rate-limiting inputs; and genetic networks exhibit bistable states. \cite{zuk2012mystery}
Genetic studies in model organisms have long identified specific instances of interacting genes (17). Important examples include synthetic traits (e.g., 18), which occur only when multiple loci or pathways are all disrupted. \cite{zuk2012mystery}
Studies have begun to reveal that epistasis is pervasive.  \cite{zuk2012mystery}
We assert that epistasis and pleiotropy are not isolated occurrences, but ubiquitous and inherent properties of biomolecular networks. \cite{tyler2009shadows}


\subsection{Epistasis examples: Non-human}
 Extensive work on the control of qualitative genetic variation has highlighted the biological importance of epistasis at a ‘locus-by-locus' level. On the basis of this work, several classic genotype-phenotype patterns that are caused by epistasis such as comb type in chickens, coat colour in various animals, the BOMBAY PHENOTYPE in the ABO blood-group system in humans and kernel colour in wheat \cite{carlborg2004epistasis}
In the case of quantitative genetic variation, several or many genes of largely unknown function combine with environmental influences to control trait variation. This is the case for many complex traits that are of medical relevance in humans or of economic importance in plants and livestock. \cite{carlborg2004epistasis}
A clear example of this can be seen  [in Fig A]  which the dominant allele (I) at the KIT locus, which confers white-coat colour in the pig, is dominant over all alleles at the MC1R locus (E), which confer a darker coat colour. The effects of the various alleles at the E locus can only be determined in individuals with the recessive genotype ii at the I locus. This example was classically termed `dominant epistasis', which gives a segregation ratio of 12:3:1 for white:black:brown, respectively \cite{carlborg2004epistasis}
 Table 1. Example of phenotypes (e.g. hair colour) obtained from different genotypes at two loci interacting epistatically, under Bateson's (1909) definition of epistasis \cite{cordell2002epistasis}
Coat colour variation in mammals has long been is one of the most fruitful examples in the study of the relationship between genotype and phenotype. ... epistasis arises when the effects of alleles at one locus are blocked by the presence of a specific allele at another locus. For example, a cross between agouti and extension (now called the melanocortin 1 receptor or Mc1r) double heterozygotes (AaEa) yields the non-Mendelian segregation ratio of 9:4:3 (instead of 9:3:3:1) \cite{phillips2008epistasis}
In the yeast Saccharomyces cerevisiae, Brem et al. (19) analyzed as quantitative traits the levels of gene transcripts in segregants of a cross between two strains. For each transcript, they found the strongest quantitative trait locus (QTL) in the cross and then, conditional on the genotype at this locus, identified the strongest remaining QTL. In 67\% of cases, these two QTLs demonstrated epistatic interactions. In bacteria, Khan et al. (20) and Chou et al. (21) have recently demonstrated clear epistasis among collections of five mutations that increase growth rate.  \cite{zuk2012mystery}
In mouse and rat, Shao et al. (22) analyzed a panel of chromosome substitution strains, with each strain carrying a different chromosome from a donor strain on a common recipient genetic background. For dozens of quantitative traits, the sum of the effect attributable to the individual donor chromosomes far exceeds (median eightfold) the total effect of the donor genome, indicating strong epistasis.  \cite{zuk2012mystery}
An example in insects is the abnormal-abdomen phenotype in Drosophila mercatorum (DeSalle and Templeton 1986; Hollocher et al. 1992; Hollocher and Templeton 1994). \cite{culverhouse2002perspective}
The study of genetic interaction has become increasingly systematic and large-scale, especially in the yeast Saccharomyces cerevisiae (6, 8-21). \cite{mani2008defining}
Eye color determination in Drosophila provides a classic example. The genes scarlet, brown, and white, play major roles in a simplified model of Drosophila eye pigmentation. Eye pigmentation in Drosophila requires the synthesis and deposition of both drosopterins, red pigments synthesized from GTP, and ommochromes, brown pigments synthesized from tryptophan. A mutation in brown prevents production of the bright red pigment resulting in a fly with brown eyes, and a mutation in scarlet prevents production of the brown pigment resulting in a fly with bright red eyes. In a fly with a mutation in the white gene, neither pigment can be produced, and the fly will have white eyes regardless of the genotype at the brown or scarlet loci. In this example the white gene is epistatic to brown and scarlet. A mutant genotype at the white locus masks the genotypes at the other loci. \cite{tyler2009shadows}


\subsection{Epistasis examples: Human}
Despite considerable efforts, few well-replicated instances of epistasis in common human disease and trait genetics have been discovered thus far. \cite{zuk2012mystery}
The only examples to date involve interactions featuring at least one locus with a large marginal e↵ect, such as HLA.  \cite{zuk2012mystery}
GWAS, in ankylosing spondylitis21 and psoriasis,22 discovered interactions between two di↵erent HLA alleles and ERAP1. (In ankylosing spondylitis, the HLA-B27 allele has an odds ratio of 40.8, and in psoriasis the HLA-C allele has an odds ratio of 4.66.) HLA also plays a role in an interaction e↵ect described in a GWAS of Type 1 diabetes. (In Type 1 diabetes, HLA has a main e↵ect of 5.5, but acts non-additively with the risk of all other alleles considered cumulatively.23). Finally, interaction between RET and EDNRB in Hirschsprung's disease was discovered in a genome-wide linkage study,24 in which RET was strongly associated with disease (log-odds score of 5.6). \cite{zuk2012mystery}
D-allele of the angiotensin I converting enzyme (ACE) gene and the C-allele of the angiotensin II type 1 receptor (AGTR1) gene3. The risk of myocardial infarction is significantly increased by the ACE D-allele in patients who carry that particular AGTR1 allele. \cite{carlborg2004epistasis}
There are numerous cases of epistasis appearing as a statistical feature of association studies of human disease. A few recent examples include coronary artery disease63, diabetes64, bipolar effective disorder65 and autism66. Unfortunately, in only a few cases has the functional basis of these potential interactions been revealed.  \cite{phillips2008epistasis}
One of these cases involves the genetic interactions underlying the autoimmune disease multiple sclerosis. Here, Gregersen et al.67 found evidence that natural selection might be maintaining linkage disequilibrium between the histocompatibility loci HLA-DRB5*0101 (DR2a) and HLA-DRB1*1501 (DR2b) (FIG. 3), which are known to be associated with multiple sclerosis; linkage disequilibrium can be generated by strong epistasis among adjacent loci \cite{phillips2008epistasis}
Indeed, it has been argued that epistatic interactions are a nearly universal component of the architecture of most common traits. Templeton (2000), for instance, has listed a number of phenotypes in which epistasis plays a large role.  \cite{culverhouse2002perspective}
In humans, variation in triglyceride levels can be explained, in part, by two sets of interactions: between ApoB and ApoE in females and between the ApoAI/CIII/AIV complex and low-density lipoprotein receptor in males (Nelson et al. 2001) \cite{culverhouse2002perspective}
Even the seemingly ``simple" Mendelian trait of sickle-cell anemia is revealed to be greatly modified by epistatic interactions. Individuals with sickle-cell anemia who are homozygous for two polymorphisms near the Gg locus (leading to the persistence of fetal hemoglobin) have only mild clinical symptoms \cite{culverhouse2002perspective}
For example, in humans the E4 allele of apolipoprotein epsilon (ApoE) is associated with elevated blood serum cholesterol levels, but only in individuals with the A2A2 genotype at the low density lipoprotein receptor (LDLR) locus.(3) In other words, the contribution of the ApoE allele to cholesterol levels depends on the genotype at the LDLR locus. \cite{tyler2009shadows}


\subsection{Epistasis and networks}
Epistasis-nonlinear genetic interactions between polymorphic loci-is the genetic basis of canalization and speciation, and epistatic interactions can be used to infer genetic networks affecting quantitative traits. \cite{huang2012epistasis}
DATASET: Here, we compared the genetic architecture of three Drosophila life history traits in the sequenced inbred lines of the Drosophila melanogaster Genetic Reference Panel (DGRP) and a large outbred, advanced intercross population derived from 40 DGRP lines (Flyland)\cite{huang2012epistasis}
Surprisingly, none of the SNPs associated with the traits in Flyland replicated in the DGRP and vice versa. However, the majority of these SNPs participated in at least one epistatic interaction in the DGRP.\cite{huang2012epistasis}
Our analysis underscores the importance of epistasis as a principal factor that determines variation for quantitative traits and provides a means to uncover genetic networks affecting these traits. \cite{huang2012epistasis}


\subsection{Epistasis and evolution}
epistasis can have an important influence on a number of evolutionary phenomena, including the genetic divergence between species79, ... the evolution of the structure of genetic systems8 \cite{phillips2008epistasis}
Thus far, these studies81-85 have shown that epistasis can have a strong role in limiting the possible paths that evolution can take, but not in limiting its eventual outcome. \cite{phillips2008epistasis}
linkage can facilitate the maintenance of epistatic interactions (and vice versa)86 and could help to explain how molecular complexity evolves \cite{phillips2008epistasis}
recent analysis of patterns of gene regulation suggest that there can be complex patterns of gene regulation in localized genomic regions8 \cite{phillips2008epistasis}


\subsection{Missing heritability}
IN 2002: Thus, for fixed K, p , and p , maximizing the broad AB heritability (h 2 p V /V ) under the constraint repreIT sented by formula (2) is equivalent to the maximizing of VI. \cite{culverhouse2002perspective}. TABLE 2 and 3: Maxima of heritability using epistasis. \cite{culverhouse2002perspective}. Three-locus models can also give rise to higher relative risks than are possible in corresponding two-locus models. Three-locus penetrance models maximizing heritability at the low end of disease prevalence \cite{culverhouse2002perspective}

missing heritability: overestimation of the denominator happens when epistasis is ignored (phantom) \cite{zuk2012mystery}
phantom heritability could be 62.8\% in Cohn's disease, thus accounting for 80\% of the current missing heritability \cite{zuk2012mystery}
Until recently "The prevailing view among human geneticists appears to be that interactions play at most a minor  part in explaining missing heritability."	 \cite{zuk2012mystery}
But "[they] show that simple and plausible models can give rise to substantial phantom heritability."	 \cite{zuk2012mystery}
...although the pervasiveness of epistasis in experimental organisms suggests that the true heritability h2 of traits may be much lower than current estimates \cite{zuk2012mystery}

Researchers of many complex diseases (including non-insulin-dependent diabetes mellitus, prostate cancer, and schizophrenia) face the conundrum of moderately heritable diseases for which locus-by-locus analyses have not accounted for the predicted genetic variance. The models discussed in the present article provide one possible explanation for this. \cite{culverhouse2002perspective}
These considerations lead us to believe that, in situations in which heritability is moderate to high but in which locus-by-locus analyses do not account for the predicted genetic variance, it is worth pursuing a hypothesis of interacting loci [near the linkage peaks] \cite{culverhouse2002perspective}

\subsection{Detecting Epistasis / interactions}

Whereas most existing epistasis screens explicitly test for a trait, it is also possible to implicitly test for fitness traits by searching for the overor under-representation of allele pairs in a given population.  \cite{ackermann2012systematic}
Such analysis of imbalanced allele pair frequencies of distant loci has not been exploited yet on a genome-wide scale, mostly due to statistical difficulties such as the multiple testing problem. We propose a new approach called Imbalanced Allele Pair frequencies (ImAP) for inferring epistatic interactions that is exclusively based on DNA sequence information. \cite{ackermann2012systematic}
Most gene interaction studies explicitly measure a phenotype such as growth rate or viability [ \cite{ackermann2012systematic}
However, one can also study implicit phenotypes by searching for the overor under-representation of certain allele pairs in a given population. \cite{ackermann2012systematic}
Such allele pairs are examples of Dobzhansky-Mu ̈ller incompatibilities: they establish a fitness bias in favor of individuals inheriting the over-represented allele combination [15]. In their most extreme form such incompatibilities are embryonic lethal. \cite{ackermann2012systematic}
In this context, an implicit phenotype is a trait that is not explicitly measured in the sample but whose regulators can still be inferred from the genotype data. \cite{ackermann2012systematic}
Here, we propose to address this problem by exploiting the additional information gained from studying family trios. We show that by analyzing a sufficiently large number of individuals with known family structure it becomes possible to detect substantially more interactions than what is expected if all markers were independent. \cite{ackermann2012systematic}
Our method, called ``Imbalanced Allele Pair frequencies (ImAP)'' is based on inspecting 3|3 contingency tables that track the frequencies of all possible two-locus allele combinations in heterozygous individuals (assuming a diploid genome). The test that we propose is similar to a x2 test in that it compares the observed frequencies in this table to expected frequencies assuming independence. However, our version corrects the expected frequencies for confounding factors such as family structure or allelic drift [21]. \cite{ackermann2012systematic}
In a population of 2,002 heterozygous mice with known family structure genotyped at 10,168 markers we identify 168 LD block pairs with imbalanced alleles \cite{ackermann2012systematic}


\subsection{Epistasis \& GWAS}
IN 2002 OPINION: for the abandonment of linkage studies in favor of genome scans for association. However, there exists a large class of genetic models for which this approach will fail: purely epistatic models with no additive or dominance variation at any of the susceptibility loci. \cite{culverhouse2002perspective}. Is it reasonable to suppose that an approach that must succeed in identifying fully penetrant Mendelian genes will also succeed for complex diseases?  \cite{culverhouse2002perspective}. The complex relationship between genotype and phenotype, however, may ultimately prove to be inadequately described by simply summing the modest effects from several contributing loci \cite{culverhouse2002perspective}
The main reason that most studies of complex human phenotypes fail to find evidence for epistatic interactions may simply be that commonly used designs and analytic methods inherently minimize or exclude the possibility of epistasis (Frankel and Schork 1996) \cite{culverhouse2002perspective}
The complex relationship between genotype and phenotype, however, may ultimately prove to be inadequately described by simply summing the modest effects from several contributing loci. \cite{culverhouse2002perspective}
We note that the number of tests necessary to evaluate all two-, three-, and four-way interactions, for 30-60 candidate loci, has a range similar to the number of tests suggested for a single genomewide association scan using SNPs (Collins et al. 1999; Kruglyak 1999) \cite{culverhouse2002perspective}
Thus, although searching for two-, three-, four-, or n-way interactions among all the markers in a genome scan would not be practicable, a candidate-locus approach based on a genome scan for linkage may be. \cite{culverhouse2002perspective}

The extent to which epistasis is involved in regulating complex traits is not known, and so we cannot assume that epistasis will be found for every trait in every population. \cite{carlborg2004epistasis}
However, we argue that epistasis has been overlooked for too long and that it now needs to be routinely explored in complex trait studies.  \cite{carlborg2004epistasis}
For complex traits such as diabetes, asthma, hypertension and multiple sclerosis, the search for susceptibility loci has, to date, been less successful than for simple Mendelian disorders. This is probably due to complicating factors such as an increased number of contributing loci and susceptibility alleles, incomplete penetrance, and contributing environmental effects \cite{cordell2002epistasis}
The presence of epistasis is a particular cause for concern, since, if the effect of one locus is altered or masked by effects at another locus, power to detect the first locus is likely to be reduced and elucidation of the joint effects at the two loci will be hindered by their interaction.  \cite{cordell2002epistasis}
Although genetic interactions are hard to detect in humans (see below), several cases involving variants with large marginal effects have been recently reported in Hirschsprung's disease, ankylosing spondylitis, psoriasis, and type I diabetes  \cite{zuk2012mystery}
...geneticists have tested for pairwise epistasis between loci, but have found few significant signals. \cite{zuk2012mystery}
...The reason is that individual interaction effects are expected to be much smaller than linear effects, and the sample size required to detect an effect scales inversely with the square of the effect size. If n loci had equivalent effects, the sample size to detect the n loci would thus scale with $n^2$, whereas the sample size to detect their $n^2$ interactions scales with $n^4$. \cite{zuk2012mystery}
Suppose that we consider two variants with frequency 20\% that contribute to different pathways and increase risk by 1.3-fold (which is a large effect relative to those typically seen in GWAS). The sample size required to detect the variants is 4,900 (with 50\% power and genome-wide significance level of $\alpha = 5 \times 10^{-8}$ in a genome-wide association study with an equal number of cases and controls), whereas the sample size required to detect their pairwise interaction is roughly 450,000 (at 50\% power and an appropriate significance level to account for multiple hypothesis testing). A researcher who studied 100,000 samples would likely discover all of the loci but would find little evidence of epistatic interactions. \cite{zuk2012mystery}
In short, the failure to detect epistasis does not rule out the presence of genetic interactions sufficient to cause substantial phantom heritability \cite{zuk2012mystery}

Cases only. The most straightforward multilocus analysis of cases-only data is a $\chi^2$ test of independent segregation for the loci.  \cite{culverhouse2002perspective}
Case-control. A second approach is a multilocus case-control analysis. One method for doing this would be to compare the distribution of cases among the 3L genotypes, where L is the number of biallelic loci being simultaneously examined, versus the distribution of controls. In this analysis, a sample of N cases and N unrelated controls drawn from a population modeled by table 3 will, again, yield an expected $\chi^2$ statistic 2N. However, the degrees of freedom under the null hypothesis are now 8. \cite{culverhouse2002perspective}

\subsection{Epistasisi GWAS: Power issues}
 We have seen that, if the true genetic model underlying a disease is purely epistatic, with no additive or dominance variation at any of the susceptibility loci, then association methods analyzing one locus at a time will have no power to detect the loci.  \cite{culverhouse2002perspective}
First, we expect that, with a sufficient number of contributing loci, purely epistatic interactions could account for virtually all the variation in affection status for diseases with any prevalence \cite{culverhouse2002perspective}
Of course, there are subclasses of purely epistatic models (providing no marginal evidence for the involvement of any single locus) for which, in addition, no two, three, or L1 loci jointly give evidence of involvement in the disorder. This leads to the concern that even assessment of all two-, three-, and (L1)-way interactions among candidate loci may be insufficient for detection of the contributing loci. \cite{culverhouse2002perspective}
The restriction on maximum heritabilities in these models is most easily seen by examining L-locus models for which no collection of L 􏰂 1 loci shows marginal deviations.  \cite{culverhouse2002perspective}

A small number of recent studies have explored this idea for the genome-level identification of epistatic interactions: if a large number of individuals is genotyped at a large number of genomic positions, it becomes possible to test all allele pairs for overand underrepresentation in that population [18-20]. \cite{ackermann2012systematic}
However, even though some methodological progress has been made [18], previous studies could hardly identify a significant number of interactions. The main obstacle is the humongous number of statistical hypotheses tested when comparing all markers in a genome against all markers. \cite{ackermann2012systematic}

	%---
\section{Coevolution}
%---

In a book published in 1859 entitled \textit{``On the origin of species by means of natural selection"} \cite{darwin1859origin}, Charles Darwin introduced the concept of co-evolution referring to the coordinated changes occurring in pairs of organisms.
In another of his books \textit{``On the various contrivances by which British and foreign orchids are fertilised by insects"}, first published in 1862  \cite{darwin1877various}; Darwin further explored this concept and providing more detailed examples.
By observing the relationship between the size of orchids' corolla and the length of the proboscis of pollinators, Darwin predicted the existence of a new species able to suck from a large spur  which was later confirmed \cite{de2013emerging}.

Co‐evolution originally referred to the coordinated changes occurring in pairs of organisms to improve or refine interactions.
This concept was later extended to pairs of proteins or more generically, any pair of biomolecules which can be within the same organism \cite{de2013emerging}.
The modern use of co-evolution methods in genetics is often attributed to Dobzhansky's \cite{dobzhansky1950genetics} and Elrich's \cite{ehrlich1964butterflies} seminal works that were published in 1950 and 1964 respectively.
In recent years, much effort has been dedicated to research of coordinated sequence changes in proteins (and genes) were co‐evolution could be an important and widespread catalyst of fitness optimization \cite{de2013emerging}.

Distinct allele combinations in co-evolving genes interact to confer different degrees of fitness. 
If this fitness difference is large, selection for alleles could maintain allelic association even between unlinked loci \cite{rohlfs2010detecting}, thus co-evolving genes are expected to maintain their interaction by pressures favouring compensatory mutations \cite{rohlfs2010detecting}.
Under this hypothesis, genetic loci may be invariable due to their functional or structural constraints but these constraints may change subject to mutations in their functional counterpart \cite{fares2006novel}.
In many cases, selective advantages for a specific allele pair could fixate the optimal allele pair in the population \cite{rohlfs2010detecting}.

\paragraph{Co-evolution examples}
In the absence of a clear positive control, identifying gene pairs that is certainly co-evolving are a difficult task \cite{rohlfs2010detecting}.
Here, some well known examples of co-evolution in humans are introduced:

\begin{itemize}

\item HLA ligand and killer-cell immunoglobulin-like receptor (KIR) are two genes located on different chromosomes forming a well established interacting immune-response pair.
Their allele frequencies are highly correlated in human populations as one expects under allele matching selection \cite{single2007global}.

\item A remarkable similarity in the phylogenetic trees of ligands (such as insulin and interleukins) and their corresponding receptors was observed.
This co‐evolution is proposed to be required for maintaining their specific interactions \cite{pazos2001similarity}.

\item Researchers found that ligands and their G-protein coupled receptors have co-evolved so that each subgroup of ligands has a matching subgroup of receptors \cite{goh2000co}.
%Ligand-receptor co-evolution can be detected based on N-terminal and C-terminal phosphoglycerate kinase (PGK) which are covalently linked and form an active site at their interface, therefore, they must are inferred to have co-evolved to preserve enzyme function \cite{goh2000co}.

\item In Hsp90 and GroEL heat-shock proteins, co-evolution was detected in ``almost all" functionally or structurally important site \cite{fares2006novel}.

\item GroESL is involved in the folding of a wide variety of other proteins with the folding activity mediated by the co-chaperonin GroES  \cite{ruiz2013coevolution}.
It was recently shown that different overlapping sets of amino acids co-evolve between GroEL and GroES \cite{ruiz2013coevolution}.

\item Gamete recognition genes ZP3 and ZP19 are highly polymorphic among humans and located on different chromosomes.
Putative interaction was observed between these genes was recently inferred \cite{rohlfs2010detecting}.

\item Helicobacter pylori is the main cause of gastric cancer. 
Host-pathogen interaction accounted for most of the difference in the severity of gastric lesions in the populations analysed. 
For instance African H. pylori ancestry was relatively benign in population of African ancestry but was deleterious in individuals with substantial Amerindian ancestry \cite{kodaman2014human}.
This is in an example of co-evolution modulating disease risk.

\end{itemize}

\subsection{Basic co-evolution inference models}

In this section we review the first methods aimed to uncover co-evolution.
These ``basic methods" serve not only to understand the historical perspective but also they are the basis of more advanced methodologies described in section \ref{sec:coevGlobal}.

\paragraph{Phylogenetic tree similarity}
%Co‐evolution of interacting species, such as symbionts-hosts, predators-prey, and parasites-hosts, is assumed to be manifested by similarities in the phylogenetic trees \cite{de2013emerging}.
Proteins and their interaction partners co-evolve so that divergent changes in one are complemented their interaction partner. 
These changes can be manifested by ``similar evolutionary trees" \cite{goh2000co}.
Thus phylogenetic similarity approaches can successfully be applied for protein-protein co‐evolution assumed to be caused by physical interactions.
These kind of methods have been shown to be capable of identifying interaction partners, such as ligand-receptor pairs \cite{de2013emerging}.

Similarly, evolutionary relationships within protein families can be mined to predict physical interaction specificities \cite{ramani2003exploiting}.
Duplicate genes (paralogs) can diverge in a way such that new binding specificities develop, thus the underlying hypothesis is that interacting proteins exhibit coordinated evolution and tend to have similar phylogenetic trees.
This was first demonstrated in a study of chemokines and their receptors showing phylogenetic tree similarities \cite{goh2000co}.
%Using similarity of phylogenetic trees as a proxy for the co-evolution of interacting proteins \cite{ramani2003exploiting}, a computational method based on matrix alignment can find an optimal alignment between protein family similarity matrices (conceptually equivalent to superimposing  phylogenetic trees from the two protein families) \cite{ramani2003exploiting}.
%One matrix is shuffled using stochastic simulated annealing-based to make the two matrices maximally agree by minimizing the root mean square difference.
%Interactions can be predicted by observing equivalent columns proteins heading in the two matrices.  \cite{ramani2003exploiting}

\paragraph{Correlated mutations}
Although some methods based on phylogenetic tree similarity exists, the majority of co-evolutionary methods focuses on analysis of multiple sequence alignment \cite{rohlfs2010detecting}.
Proteins have evolved to interact or function in specific molecular complexes and the specificity of these interactions is essential for their function. Consequently, residue contacts constrain the protein sequences to some extent \cite{pazos1997correlated}.
In other words, 
%sequences form interacting proteins react as a consequence of adaptation, thus 
it is reasonable to assume that evolution of sequence changes on one of the interacting proteins must be compensated by mutations in the other \cite{pazos1997correlated}.
It should be noted that this relationship between co-evolution and interaction is not symmetrical
%While interaction would involve coevolution, 
since co-evolution does not imply physical interaction \cite{fares2006novel}.
This is emphasized by the fact that co-evolution between clusters of sites not in contact has also been shown \cite{pritchard2000proteins}.

%Identification of genes showing signs of adaptive evolution can be used in determining functional regions in proteins \cite{fares2006novel}.
It has long been suggested that correlations in amino acid changes can be used to infer protein contact, thus helping predict tertiary protein structure \cite{fitch1970improved, morcos2011direct, burger2010disentangling, de2013emerging}.
A large number of genomes and protein sequences have become available in recent years enabling the analysis of co-evolution by means of statistical inference of covariation patterns based on multiple protein sequence alignments \cite{burger2010disentangling, burger2010disentangling}, which has been a fruitful technique for predicting contacting residues in the structure.
This interdependent changes in amino acids was formulated for the first time by the ``covarion model" \cite{fitch1970improved} and applied in multiple sequence alignments of a family of homologue proteins \cite{de2013emerging}.
Statistical methods to find correlated mutations between pairs of proteins can identify putative interaction sites in protein pairs \cite{de2013emerging}, but we should keep in mind that correlated mutations suggesting compensatory changes between residues can be due to several factors different than direct contact, such as physical proximity, catalytic action, binding sites, or even maintaining folding stability.

One of the first attempts of statistical inference of co-evolving loci pairs was performed by Gobel et. al in 1994.
In their seminal paper they point out that \textit{``maintenance of protein function and structure constrains the evolution of amino acid sequences... [sequence alignments] can be exploited to interpret correlated mutations observed in a sequence family as an indication of probable physical contact in three dimensions"} \cite{gobel1994correlated}. 
They  analysed correlations between different positions in a multiple sequence alignment and used such correlations to predict contact maps.
In their study of 11 protein families they compare their results with experimentally validated contact maps determined by crystallography, showing prediction accuracy up to $68\%$.

The promise of developing methods for predicting amino acid contacting pairs from sequence information alone was radically different from and more applicable than traditional docking methods \cite{pazos1997correlated}.
This lead to the development of methods for detecting correlated changes in multiple sequence alignments with the primary objective of using them to detect protein interfaces in interacting molecules \cite{pazos1997correlated}, thus facilitating protein structure prediction.
It was demonstrated that the correlated sequence information was enough to select the right inter-domain docking solution amongst many alternatives.

Correlation and mutual information (MI) have been used to assess co-evolution but they do not take the evolutionary interdependence between protein residues into account \cite{fares2006novel}.
Phylogenetic relationships can inflate these co-evolutionary measures, thus one of main limitations of these methods has been their inability to separate phylogenetic linkage from functional and structural co-evolution \cite{fares2006novel}.
Some methods partially correct these effects but while some studies \cite{gloor2005mutual} claim that these would require alignments of at least $125$ sequences, while other studies \cite{morcos2011direct} suggest that they may require in the order of $1,000$.

\paragraph{Phylogenetic correction}
Mutual information (MI) measures the reduction of uncertainty about one position given information about the other.
When used as a measurement for co-evolution, MI can be confounded by several factors such as structural and functional constraints, and the background sum of contributions from random noise and shared ancestry.
In an attempt to improve MI's signal to noise ratio by eliminating or minimizing the second factor, a model postulated by Dunn et alii \cite{dunn2008mutual} tries to factorize these terms in order to estimate a correction.
They propose that each amino acid position in the MSA has a propensity toward the background MI (related to its entropy and phylogenetic history) and estimate the joint background MI as the product of their propensities.
It follows that a joint background correction term can be approximated as product of the average background MI divided by the average overall MI of all positions in the MSA, which they call the average product correction (APC) \cite{dunn2008mutual}.
They show that APC is a metric than can accurately estimate MI in the absence of structural or functional relationships (i.e. the null model) \cite{dunn2008mutual}.
Finally, by assuming the null model to be normally distributed, a p-value can be inferred using a Z-score.

Another method, CAPS \cite{fares2006novel}, compares transition probability scores from the blocks substitution matrix (BLOSUM) between two sequences at the sites being analysed for interaction.
An alignment-specific BLOSUM matrix is applied depending on the average sequence identity.
Co-evolution between protein sites is estimated by the correlation in the pairwise variability with respect to the mean pairwise variability per site \cite{fares2006novel}.
A limitation of this method arises when sequences are too divergent, since an alignment including highly divergent sequence groups could show unrealistic level of pairwise identity (BLOSUM values are normalized by the time of divergence between sequences to reduce the impact of this).
Another problem common to many MSA-based co-evolutionary methods is that constant amino acid sites, which are very likely to be functionally important, cannot be tested for  \cite{fares2006novel}.

\paragraph{Evolutionary timespan}
What is the appropriate evolutionary time scale required in a multiple sequence alignment in order to perform a co-evolutionary analysis?
Co-evolution is often analysed over very large time frames based on the evolutionary analysis across different species \cite{qian2015recent}.
Nevertheless, genome-wide scans have identified several candidate loci that underlies local adaptations, which seems surprising given the short evolutionary time since the human divergence which is estimated have happened around $50,000$ to $100,000$ years ago when humans migrated out of Africa \cite{qian2015recent}.
In light of this, it may make sense to analyse co-evolution within human populations based on the propositions that multiple genes within a pathway or a functional sub-network may change in the same fitness direction at a same evolutionary rate to achieve a common phenotypic outcome \cite{qian2015recent}.
In a study using data from the 1000 Genomes project \cite{10002012integrated} form East Asians, Europeans, and Africans populations, researchers found that genes having signals of recent positive selection are significantly closer to each other within protein-protein interaction (PPI) networks \cite{qian2015recent}.
The approach was also able to identify known examples such as EGLN1 and EPAS1 (hypoxia-response pathway playing key roles in adaptation to high-altitude) as well as multiple genes in the NRG-ERBB4 (developmental) pathway \cite{qian2015recent}.
This shows that sequences from shorter time spans can also be mined for co-evolution.

\paragraph{MSA quality influences predictions}
Since many co-evolutionary methods rely so heavily on multiple sequence alignments, it should not be surprising to know that the quality of the input alignment may affect the results.
As one example, it is well known that structure-based alignment algorithms may be susceptible to shift error and other systematic errors, thus strong covariation signal can be caused by alignment errors leading to false positive predictions \cite{dickson2010identifying}.
The phylogeny of the sequences also affects performance, since methods work better on large protein families having a wide but homogeneously distributed degree of sequence similarity ranging from distant to similar sequences \cite{de2013emerging}.
In a recent study co-evolutionary methods applied to different alignments of the same protein family gave rise to distinct results, demonstrating that the measurement of co-evolution may greatly depend on the quality of the sequence alignment \cite{dickson2010identifying}.
Even when alignments for the same protein family contained comparable numbers of sequences the number of estimated co-varying positions differed significantly.
The authors of this analysis demonstrated that contact prediction can be improved by removing alignment errors due to several factors such as partial or otherwise erroneous sequences, the presence of paralogous sequences, and improper structure alignment.

\paragraph{Co-Evolution and protein structure}
Protein structure prediction from amino acid sequence is one of the ultimate goals in computational biology \cite{burger2010disentangling}, despite significant efforts the general problem of de novo three-dimensional structure prediction has remained one of the most challenging problems in the field \cite{marks2012protein}.
Unfortunately, \textit{de-novo} protein structure prediction does not scale with longer proteins since the conformational space grows exponentially with the protein length.
Inter-residue contact information can constrain the fold thus significantly reducing the search space.
Since covariation patterns can complement experimental structural biology thus helping to elucidate functional interactions, information of co-evolutionary couplings between residues are often used to compute protein three-dimensional structures from amino acid sequences \cite{marks2012protein}.
It has been observed that information about protein residue contacts, can be used to elucidate the fold of some proteins \cite{jones2012psicov}.
Researchers demonstrated that using co-evolutionary information from multiple sequence alignments greatly helps to deduce which amino acid pairs are close (or in contact) in the three-dimensional structure thus allowing the protein fold to be determined with a reasonable accuracy \cite{marks2012protein}.
It is not surprising that the vast majority of methods for finding protein co-evolution are designed with the specific aim of generating results useful in the context of protein folding.

\paragraph{Protein design}
It has recently been proposed to use co-evolutionary theory in computational methods for protein design.
Significant similarities were found between the amino acid covariation in natural protein sequences and sequences structures optimized by computational protein design methods \cite{ollikainen2013computational}.
Because evolutionary selective pressures on function and structure shaped the sequences to be close to optimal for their structures, natural protein sequences provide an excellent source for computational protein design methods.
%Similarly, computational protein design predicts energetically optimal sequences based on protein structure, so it is expected that highly co-varying amino acids pairs in both designed and natural sequences have co-varied to maintain optimal protein structure.
This study using computational protein design to quantify protein structure constraints from amino acid covariation for 40 diverse protein domains, shows that structural constraints imposed by covariation play a dominant role in protein architecture.
Computational protein design methods could make use of knowledge form natural co-evolution effects \cite{ollikainen2013computational}.

\subsection{Global co-evolution models \label{sec:coevGlobal}}

Imagine a protein sequence of length $L = a_1, a_2, ... , a_n$, amino acid $a_i$ is coupled directly with $a_j$, and $a_j$ to $a_k$, then $a_i$ and $a_k$ will show correlation despite not being directly coupled \cite{weigt2009identification}.
This is an important problem when inferring co-evolution as indirect coupling can make it difficult to recognize the directly co-evolving loci.
%occurring when more than two positions show coordinated substitution patterns.
%Apparent co‐variation between two positions is the consequence of the evolutionary interdependence and these indirect couplings can make it difficult to recognize the directly interdependent positions.

As opposed to models using the independence assumption, a `global' model treats correlated pairs of residues as dependent on each other thereby minimizing effects of transitivity  \cite{marks2012protein}.
Since direct couplings are more reliable predictions of physical interactions, approaches that can distinguish direct from indirect couplings have been an intensive area of study \cite{de2013emerging}.
Global approaches are designed to reach high scores only for amino acid pairs that are likely to be causative of the observed correlations  \cite{marks2012protein}.
In this section we introduce these methods.


\paragraph{Glass spin systems}
Global interaction models are well understood in statistical physics.
A typical example are long-range order observed in spin systems, where the spins only have short-range direct interactions \cite{binney1992theory}.
One of the first global models for co-evolution was proposed by Lapedes \cite{lapedes2012using}, who used a Monte Carlo algorithm to infer the simplest probabilistic distribution able to account for the whole network of co‐variations \cite{de2013emerging}.
He presented a sequence-based probabilistic theory addressing co-operative effects in interacting positions in proteins assuming that a sequence of length $L$ is a global state of an \textit{L-site} spin system of twenty states (for twenty amino acids).
Then he solved the global statistical formalism based on maximizing entropy under constraints which are known to lead to Boltzmann statistics \cite{marks2012protein}.
Finally the conditional mutual information is calculated using this Boltzmann model which leads to the degree of covariation between residues at two positions factoring out contributions by interaction with the rest of the residues \cite{marks2012protein}.
The amount sequence data is a limiting factor when performing inference of Boltzmann distribution parameters, thus it is usually infeasible to use more than first order distributions \cite{lapedes2012using}.
Another limitation is the phylogenetic relatedness of these sequences, which is not addressed in this algorithm and has the potential to decrease accuracy \cite{lapedes2012using}.

\paragraph{Direct coupling analysis}
A similar approach called  direct-coupling analysis (DCA) was also based on spin-glass physics \cite{weigt2009identification}.
In their implementation a generalized message-passing technique is used to massively parallelize the algorithm implementation.
As in in the work of Lapedes \cite{lapedes2012using} an application of the maximum entropy principle yields the Boltzmann distribution which is used to estimate the second order interaction model.
In principle higher correlations of three or more positions can be included, however dataset size (i.e. number of sequences in the MSA) does not allow for inference beyond two-residue model parameters. 
Determining model parameters, which is the most computationally expensive task is achieved by using a two-step procedure: 
i) given a candidate set of model parameters, single and two residue distributions are estimated; 
ii) the summation over all possible protein sequences would require $O(|\Sigma_{AA}|^{N-2} N^2)$ steps (where $\Sigma_{AA}$ is the amino acid alphabet and $|\Sigma_{AA}|$ is the alphabet size), so an approximation is performed using MCMC sampling. 
This last step is the most expensive step and is expected to be very slow for 21-state variables.
The message-passing approach implemented using an efficient heuristic, reduces the computational complexity to $O(|\Sigma_{AA}|^2 N^4)$.
Once all probability distributions are estimated, gradient descent is used to adjust the coupling strengths maximizing the joint probability of the data.
Since the model is convex, it is guaranteed to converge to a single global maximum.
Finally, a quantity called direct information (DI) measures the part of the mutual information of a position pair induced by the direct coupling (intuitively similar to mutual information in a two-variable model).
Even after all optimizations and parallelizations, the method could not be applied to more than $60$ positions in the protein alignment simultaneously.
The authors apply the method to a dataset consisting of over $2,500$ bacterial genes from a two-component signal transduction system,. Their global inference robustly identified residue pairs proximal in space between sensor kinase (SK) and response regulator (RR) proteins as well as homo-interactions in RR proteins \cite{weigt2009identification}.
In their test dataset, the top $10$ candidate interactions identified were shown to be true contacts, furthermore these predictions were then used to calculate an interacting protein complex quite accurately (3 \AA\  RMSD) \cite{weigt2009identification}.

\paragraph{Mean field approximation}
DCA has been shown to yield a large number of correctly predicted contacts based on its ability to disentangle direct and indirect correlations; unfortunately the method is computationally expensive \cite{weigt2009identification}.
A method published by Morcos et alii \cite{morcos2011direct} proposes a ``mean field" approximation to DCA \cite{weigt2009identification}.
They first attempt to mitigate phylogenetic tree biases using a simple sampling correction based on re-weighting  sequences with more than $80\%$ similarity.
In a nutshell, the approximation method also tries to disentangle direct and indirect couplings by inferring a global statistical and least-constrained model which, as discussed before, is achieved using a maximum-entropy principle leading to a Boltzmann distribution of couplings.
The partition function ($Z$) is then approximated by keeping only the linear order term in a Taylor series expansion, thus obtaining the mean-field equations.
This approach is based on small-coupling expansion, thus a Taylor expansion around zero, a technique introduced in disordered Ising spinglass models with binary variables.
A well known result is that the first derivative of the Gibbs potential, the Legendre transform of the free energy $F = - ln(Z)$, equals the average of the coupling term in the Hamiltonian.
This simplifies this average calculation since the joint distribution of all variables becomes factorized over the single sites \cite{morcos2011direct}.
This mfDCA algorithm speeds up the original DCA implementation by $10^3$ to $10^4$ times \cite{morcos2011direct}, and can run on alignments up to $500$ amino acids per row which is an order of magnitude larger the previous version of DCA based on message passing \cite{morcos2011direct, weigt2009identification}.

\paragraph{PSI-COV}
Like other methods, PSI-COV \cite{jones2012psicov} starts from a multiple sequence alignment.
A covariance matrix is calculated by counting how often a given pair of amino acids occurs in a particular pair of positions, summing over all sequences in the MSA.
Since this matrix contains the raw data capturing all residue pair relationships, one can then compute a measure of causative correlations in the global statistical approaches by taking the inverse of the covariance matrix \cite{jones2012psicov, marks2012protein}.
Assuming that this covariance matrix can indeed be inverted, the inverse matrix relates to the degree of direct coupling, a well known fact in statistical theory under the assumption of continuous Gaussian multivariate distributions \cite{marks2012protein}.
Elements significantly different from zero (off-diagonal) indicate pairs of sites which have strong direct coupling and are thus likely to be in direct physical contact \cite{jones2012psicov}.
Unfortunately, the empirical covariance matrices are actually almost always singular simply because it is unlikely that every amino acid is observed at every site.
One of the most powerful techniques to overcome this problem is sparse inverse covariance estimation under Lasso constraints.
The authors claim that the non-zero terms tend to more accurately relate to correct correlations in the true inverse covariance matrix \cite{jones2012psicov}.

\paragraph{Multidimensional mutual information}
In a recent study a simple extension of mutual information was proposed by considering ``additional information channels" corresponding to indirect
amino acid dependencies \cite{clark2014multidimensional}.
This is achieved by defining the information $I(X_1 ; X_3 ; X_2)$ representing an `interaction information' for a channel with two inputs $X_1$ and $X_3$ and a single output $X_2$.
The effect of the indirect input ($X_3$) on the transmission between $X_1$ and $X_2$ can then be marginalized simply by summing mutual information for each possible value $X_3$ weighted by the probability of occurrence \cite{clark2014multidimensional}.
Similarly a four variable model extension can be defined, in which case the marginalization would be done over two variables ($X_3$ and $X_4$).
The authors test and compare their results using a set of $9$ MSAs consisting of less than $400$ sequences each, showing that their simple extension is comparable to other maximum entropy statistical models \cite{clark2014multidimensional}.
Even thought the method is simple, the marginalization sums impose a heavy computational burden requiring long execution times and large memory footprints making the method impractical for sequences longer than 200 residues \cite{clark2014multidimensional}.

\paragraph{Bayesian network model}
Another attempt to disentangle direct from indirect statistical dependencies between residues assumes that the sequences in a MSA are drawn from unknown joint probability distribution \cite{burger2010disentangling}. 
The model considers pairwise conditional dependencies and factorizes the joint probability by a single other position which the residue depends on, using the conditional probabilities as nuisance parameters that are integrated out when calculating the likelihood of the alignment. 
Most notably, the model does not consider only the best way of choosing the dependent position, but rather sums over all possible ways in which dependencies could be chosen.
This sum over all spanning trees is a generalization of Kirchhoff's matrix-tree theorem and can be efficiently computed form the Laplacian of the dependency matrix.

\subsection{Algorithm limitations}

%Residue co‐evolution was originally detected using correlated amino acid changes in pairs of positions represented by two columns of the MSA.
%Under the assumption of interdependent amino acid frequencies or similar patterns of amino acid substitutions it can be assessed by a linear correlation, a method that shows a small but significant capability to recover pairs of positions in physical contact \cite{de2013emerging}.

%MI
Mutual information was one of the first proposed methods used to detect co‐varying positions. 
As opposed to correlation-based methods, mutual information considers the distribution of each amino acid in the different sequences for a position quantifying whether presence of an amino acid one position can be used to predict presence of an amino acid in the other position.
Mutual information does not take into account which amino acids are present, therefore different amino acids are treated just as symbols \cite{de2013emerging}.
MI is an attractive and simple metric because it explicitly measures the dependence of one position on another, but it is limited by factors such as: 
i) positions with higher entropy (variability), tend to have higher MI than positions of lower entropy even though the latter are more constrained and would seem more likely to be co-evolving \cite{dunn2008mutual}; and 
ii) MI arises when alignments do not contain enough sequences to reduce the noise to signal ratio, it was shown that alignments should contain at least 125 sequences to significantly reduce this effect \cite{martin2005using}.

% PHYLO
The influence of the background phylogenetic relationship between sequences in the MSA confounds results and some efforts have tried to address this by removing certain problematic clades from the MSA.
For instance, it has been shown that the effect may be limited to some degree by excluding highly similar sequences (from closely related species) from the alignment \cite{wollenberg2000separation}.
Continuous-time Markov process model for sequence co‐evolution can model this explicitly and some approaches have been implemented for small-scale studies of co‐evolution in small protein families, but computational limitations have hindered their usage in large-scale studies \cite{de2013emerging}.
Other confounding effect is an uneven representation of protein sequence members (e.g. having several small subgroups and one large subgroup) which leads to statistical noise \cite{marks2012protein}.

%Indirect correlations arise because if $A$ correlates with and $B$ are in contact with each other and $B$ and $C$ are in contact as well, there is an observed indirect correlation between $A$ and $C$ \cite{marks2012protein}.
Since amino acids often contact more than one amino acids, transitive effects tend to form a network.
Thus pairs of residues analysed using a simple statistical model (such as correlation or mutual information) may not necessarily be close in space or functionally constrained \cite{marks2012protein}.
Algorithms to overcome this limitation exists, but they are based in global probabilistic models which require parameter estimation of complex distributions, such as the Bolzmann distribution, as well as marginalizing over all indirect variables.
This makes global models computational prohibitively for all but very small datasets and impossible to apply to genome wide scale analysis.

Usually co-evolutionary methods are tested with high quality MSAs containing large number of sequences varying from $5L$ to $25L$ (where $L$ is  sequence length).
Such large number of homogeneous sequences are rarely available and when they are, they usually correspond to well studied proteins and might already have a crystallized structure, thus analysis of amino acids in contact are not needed to infer the 3-D structure.
Often, investigators study less well-characterized proteins having MSA of less than $L$ sequences, and low alignment quality due to the presence of many gaps, in which case, existing methods are of limited value \cite{clark2014multidimensional}.

Finally it should be mentioned that results from different models usually do not agree, even for complex global models.
In a recent study, a comparison of several methods shows that while all methods detected similar numbers of co-varying pairs 
%(when taking into account residues separated by $\le 8$ \AA\ in reference X-ray structures)
, there is less than $65\%$ overlap between the top scoring prediction from methods based on different principles \cite{clark2014multidimensional}.

	%---
\section{Thesis roadmap and Contributions}
%---

The original research presented in this thesis covers topics related to the computational and statistical methodologies related to the analysis of sequencing variants to unveil genetic links to complex disease. Broadly speaking, we address three types of problems: (i) Data processing of large datasets from high throughput biological experiments such as resequencing in the context of a GWAS (Chapter \ref{ch:bds}); (ii) functional annotations, i.e. calculating variant's impact at the molecular, cellular or even clinical level (Chapter \ref{ch:snpeff}); (iii) identification of genetic risk factors for complex disease using models that combine population-level and evolutionary-level data to detect putative epistatic interactions (Chapter \ref{ch:gwas}). When applicable, background material specific to each chapter is presented in a preface, together with an explanation of how that chapter ties in with the rest of the thesis.

This thesis comprises text and figures of articles that have either been published, submitted for publication, or ready to be submitted (waiting upon data embargo restrictions):
\\

\begin{description}
	
	\item[Chapter \ref{ch:bds}] ~ 
	
		\begin{enumerate}
			\item \textbf{P. Cingolani}, R. Sladek, and M. Blanchette. ``BigDataScript: a scripting language for data pipelines." Bioinformatics 31.1 (2015): 10-16.
		\end{enumerate}

		For this paper, PC conceptualized the idea and performed the language design and implementation. RS \& MB helped in designing robustness testing procedures. PC, RS \& MB wrote the manuscript.
		\\
	
	\item[Chapter \ref{ch:snpeff}] ~
	
		\begin{enumerate}[resume]
			\item \textbf{P. Cingolani}, A. Platts, M. Coon, T. Nguyen, L. Wang, S.J. Land, X. Lu, D.M. Ruden, et al. ``A program for annotating and predicting the effects of single nucleotide polymorphisms, snpeff: Snps in the genome of drosophila melanogaster strain $w^{1118}; iso-2; iso-3$". Fly, 6(2), 2012.
		\end{enumerate}

		For this paper, PC conceptualized the idea, implemented the program and performed testing.
		AP contributed several feature ideas, software testing and suggested improvements.
		XL, DR, SL, LW, TN, MC, LW performed mutagenesis and sequencing experiments.
		XL and DR performed the biological interpretation of the data.
		All authors contributed to the manuscript.
		\\

		SnpEff's accompanying publication (SnpSift):
	
		\begin{enumerate}[resume]		
			\item \textbf{P. Cingolani}, V. M. Patel, M. Coon, T. Nguyen, S. Land, D. M. Ruden, and X. Lu.`` Using drosophila melanogaster as a model for genotoxic chemical mutational studies with a new program, snpsift". Toxicogenomics in non-mammalian species, page 92, 2012.
		\end{enumerate}
		
		~ \\

		We used SnpEff \& SnpSift and developed a number of new functionalities in the context of two collaborative GWAS projects on type II diabetes:

		~ \\
			
		\begin{enumerate}[resume]
		
			\item M. McCarthy, T2D Genes Consortia. ``Variation in protein-coding sequence and predisposition to type 2 diabetes", Ready for submission.
			
			\item A. Mahajan, X. Sim, H. Ng, A. Manning, M. Rivas, H. Heather, A. Locke, N. Grarup, H. K. Im, \textbf{P. Cingolani}, et al. ``Identification and Functional Characterization of G6PC2 Coding Variants Influencing Glycemic Traits Define an Effector Transcript at the G6PC2-ABCB11 Locus." PLoS genetics 11.1 (2015): e1004876-e1004876.
		
		\end{enumerate}
		~ \\
	
	\item[Chapter \ref{ch:gwas}] ~
	
		\begin{enumerate}[resume]
		\item \textbf{P. Cingolani}, R. Sladek, and M. Blanchette. ``A co-evolutionary approach for detecting epistatic interactions in genome-wide association studies". Ready for submission (data embargo restrictions).
		\end{enumerate}
	
		For this paper, PC designed the methodology under the supervision of MB and RS. PC implemented the algorithms. PC, RS \& MB wrote the manuscript. This work uses data from the T2D consortia, thus it cannot be published until the main T2D paper is accepted for publication (according to T2D data embargo).
		\\
	
	\item[Other contributions] ~	\linebreak
		During my thesis I have co-authored several other scientific articles (grouped by topic) published, submitted for publication, or ready to be submitted, not mentioned in this thesis:
		\\

	\item[Epigenetics] ~

		\begin{enumerate}[resume]
			\item \textbf{P. Cingolani}, X. Cao, R. Khetani, C.C. Chen, M. Coon, A. Bollig-Fischer, S. Land, Y. Huang, M. Hudson, M. Garfinkel, and others. ``Intronic Non-CG DNA hydroxymethylation and alternative mRNA splicing in honey bees." BMC genomics 14.1 (2013): 666.
			\item M. Senut, A. Sen, \textbf{P. Cingolani}, A. Shaik, S. Land, Susan J and D. M. Ruden. ``Lead exposure disrupts global DNA methylation in human embryonic stem cells and alters their neuronal differentiation." Toxicological Sciences (2014).
			\item D. M. Ruden, \textbf{P. Cingolani}, A. Sen, W. Qu, L. Wang, M. Senut, M. Garfinkel, V. Sollars, X. Lu, ``Epigenetics as an answer to Darwin's 'special difficulty' Part 2: Natural selection of metastable epialleles in honeybee castes", Frontiers in Genetics (2015).
			\item M. Senut, A. Sen, \textbf{P. Cingolani}, A. Shaik, S. Land, Susan J and D. M. Ruden. ``Lead exposure induces changes in 5-hydroxymethylcytosine clusters in CpG islands in human embryonic stem cells and umbilical cord blood", Submitted to `Epigenomics.
			\item M. Senut, \textbf{P. Cingolani}, A. Sen, Arko, A. Kruger, A. Shaik, H. Hirsch, S. Suhr, D. Ruden. ``Epigenetics of early-life lead exposure and effects on brain development." Epigenomics 4.6 (2012): 665-674.
		\end{enumerate}
		~ \\
	
	\item[GWAS \& Disease] ~
	
		\begin{enumerate}[resume]
			\item K. Oualkacha, Z. Dastani, R. Li, \textbf{P. Cingolani}, T. Spector, C. Hammond, J. Richards, A. Ciampi, C. Greenwood. ``Adjusted sequence kernel association test for rare variants controlling for cryptic and family relatedness." Genetic epidemiology 37.4 (2013): 366-376.
			\item S. Bongfen, I. Rodrigue-Gervais, J. Berghout, S. Torre, \textbf{P. Cingolani}, S. Wiltshire, G. Leiva-Torres, L. Letourneau, R. Sladek, M. Blanchette, and others. ``An N-ethyl-N-nitrosourea (ENU)-induced dominant negative mutation in the JAK3 kinase protects against cerebral malaria." PloS one 7.2 (2012): e31012.
			\item C. Meunier, L. Van Der Kraak, C. Turbide, N. Groulx, I. Labouba, Ingrid, \textbf{P. Cingolani}, M. Blanchette, G. Yeretssian, A. Mes-Masson, M. Saleh, and others. ``Positional mapping and candidate gene analysis of the mouse Ccs3 locus that regulates differential susceptibility to carcinogen-induced colorectal cancer." PloS one 8.3 (2013): e58733.
			\item G. Caignard, G. Leiva-Torres, M. Leney-Greene, B. Charbonneau, A. Dumaine, N. Fodil-Cornu, M. Pyzik, \textbf{P. Cingolani}, J. Schwartzentruber, J. Dupaul-Chicoine, and others. ``Genome-wide mouse mutagenesis reveals CD45-mediated T cell function as critical in protective immunity to HSV-1." PLoS pathogens 9.9 (2013): e1003637.
			\item M. Bouttier, D. Laperriere, M. Babak Memari, M. Verway, E. Mitchell, \textbf{P. Cingolani}, T. Wang, M. Behr, R. Sladek, M. Blanchette, S. Mader and J. White. ``Genomics analysis reveals elevated LXRα signaling reduces M. tuberculosis viability", Submitted to Journal of Clinical Investigation.
			\item M. Bouttier, D. Laperriere, M. Babak Memari, M. Verway, E. Mitchell, \textbf{P. Cingolani}, T. Wang, M. Behr, R. Sladek, M. Blanchette, S. Mader and J. White. ``Genomic analysis of enhancers engaged in M. tuberculosis-infected macrophages reveals that LXR signaling reduces mycobacterial burden", Submitted to PLOS Pathogens.
		\end{enumerate}	
		~ \\
	
	\item[Fuzzy logic] ~

		\begin{enumerate}[resume]
			\item \textbf{P. Cingolani} and Jesus Alcala-Fdez. ``jFuzzyLogic: a robust and flexible Fuzzy-Logic inference system language implementation." FUZZ-IEEE. 2012.
			\item \textbf{P. Cingolani} and Jesus Alcala-Fdez. ``jFuzzyLogic: a java library to design fuzzy logic controllers according to the standard for fuzzy control programming." International Journal of Computational Intelligence Systems (2013), vol 6, pages 65-75.
		\end{enumerate}	

\end{description}

	
%-----------------------------------------------------------------------------
\chapter{BigDataScript: A scripting language for data pipelines \label{ch:bds}}
%-----------------------------------------------------------------------------

%---
\section{Preface}
%---

The overall goal in this thesis is to find genetic loci related to complex disease. In order to have enough statistical power to find these risk loci, we need to sequence thousands of cases and controls (i.e. patients and healthy individuals). Obviously the first step is to find all these patients, obtain patient’s consent, take samples and keep track of clinically relevant variables (such as age, sex, BMI, and glycemic traits). Just by the sheer number of patients involved, it’s easy to see that the logistics are challenging, to say the least. 

Once the sequencing of each patient’s DNA is performed, we need to process the raw sequencing information by performing what is known as “primary sequencing analysis”, which involves mapping reads to the reference genome, calling variants, as well as performing several types of quality controls. The term “primary analysis” makes it sound as if this step is simple,but it is not.  Managing such volume of information is a huge task that requires large computational resources, and coordinating the process involved at every stage of the analysis is not trivial, even if the jobs are relatively easy to parallelize. 

As an example of the complexity and data volumes involved in these analysis pipelines, mapping the raw reads to the reference genome (i.e. the first stage of the primary analysis) for our T2D sequencing data is estimated to take over 12,000 CPU hours, that is over 32 CPU/years, under the most optimistic assumptions. At this magnitude hardware and failures become a significant issue since the probability of one or more nodes malfunction, while the data is being processed, is quite high.

We designed and implemented a simple script-like programming language called BigDataScript (BDS), with a clean and minimalist syntax to develop and manage pipeline execution and provide robustness to various types of software and hardware failures as well as portability.  This programming language specifically tailored for data processing pipelines, improves abstraction from hardware resources and assists with robustness. Hardware abstraction allows BDS pipelines to run without modification on a wide range of computer architectures, from a small laptop to multi-core servers, server farms, clusters, clouds or even whole datacenters. BDS achieves robustness by incorporating the concepts of absolute serialization and lazy processing, thus allowing pipelines to recover from errors. By abstracting pipeline concepts at programming language level, BDS simplifies implementation, execution and management of complex bioinformatics pipelines, resulting in reduced development and debugging cycles as well as cleaner code. BDS was used to create data analysis pipelines required for our research, including the ones described throughout this thesis, and is currently used by other research groups and  sequencing facilities in both academic and private environments.

The rest of the chapter is published in: Cingolani, Pablo, Rob Sladek, and Mathieu Blanchette. ``BigDataScript: a scripting language for data pipelines." Bioinformatics 31.1 (2015): 10-16.


	
%-----------------------------------------------------------------------------
\chapter{SnpEff: Genomic variant annotation and prioritization\label{ch:snpeff}}
%-----------------------------------------------------------------------------

%---
\section{Preface}
%---

As this thesis is focused on extracting biological insight from sequencing data, in this chapter we examine algorithms we created for calculating ``functional annotations" of genomic variants. In essence, functional variant annotations are bits of biological knowledge that allow us to make prioritize variants that are assumed to be more relevant to the phenotypic trait under study and to filter out variants assumed irrelevant. The spectrum of functional annotations for a genomic variant is wide and may involve information on which genes are affected by the variant, how the protein product is affected, how conserved is the genomic region the variant lies onto, and which clinically relevant information is associated with the loci; just to mention a few typical use cases.

When trying to find variants that affect risk of complex disease, statistical power is paramount. We need to be able to ``separate wheat from chaff". In our context this means two different but closely related tasks: i) performing functional annotations, and ii) using that information for prioritizing variants (and filtering out the ones we suspect are not related to the particular trait under study). Failing to efficiently filter out irrelevant variants would reduce our statistical power as more statistical tests are calculated, thus would decrease our chances of finding the associations we are looking for. In order to efficiently annotate and filter variants, we created two software packages called SnpEff and SnpSift that deal with the annotation and filtering aspects respectively.

%---
\section{Epilogue?}
%---

At the beginning of my Ph.D., functional annotation of genomic variants was an unsolved problem with many research labs creating in-house custom solutions that oftentimes were inefficient and lacking of rigorous testing. As a consequence, shortly after SnpEff \& SnpSift were released they quickly became widely adopted by the research community as well as many private organizations. Currently SnpEff \& SnpSift has over 250 downloads per week (as reported by  SourceForge, where the tools are hosted). So far SnpEff \& SnpSift  have been cited over 400 times.

\subsection{Data structures for annotations}

A very simple approach used by ANNOVAR \cite{wang2010annovar} is to create an index by dividing each chromosome into $N$ bins of equal size. All genomic features are stored in a hash table indexed by chromosome name and bin number. This approach has running time of $O(n)$ where $n$ is the number of features, but it can be easily tuned by creating small bins, at the cost of increased memory requirements.

Another approach \cite{cingolani2012program} is to use an ``interval forest'', which is a hash of ``interval trees'' indexed by chromosome. Each interval tree is composed of nodes. Each node has five elements i) a center point, ii) a pointer to a node having all intervals to the left of the center, iii) a pointer to a node having all intervals to the right of the center, iv) all intervals overlapping the center point sorted by start position, and v) all interval overlapping the center point sorted by end position. Querying an interval tree requires $O[log(n) + m]$ time, where $n$ is the number of features in the tree and $m$ is the number of features in the result. Having a hash of trees optimizes the search by reducing the number of intervals per tree.


%---
\section{Background}
%---

The development of cost-effective, high-throughput next generation sequencing (NGS) technologies is poised to have a profound impact on our ability to study the effects of individual genetic variants on the pathogenesis and progression of both monogenic and common polygenic diseases. As sequencing costs decrease and throughput increases, it has now become possible to quickly identify a large number of sequence polymorphisms (SNVs, indels, structural) using samples from affected and unaffected subjects and investigate these in epidemiologic studies to identify genomic regions where mutations increase disease risk. However, translating this information into biological or clinical insights is challenging as it is often difficult to determine which specific polymorphisms are the main pathogenetic drivers of disease across a population; and more importantly, how they affect the activity of disease-related molecular pathways in tissues and organism a specific patient. In part, this difficulty results from the large number of genetic variants that are observed in individual genomes (the human population is believed to contain approximately 3.5 million polymorphic sites with minor allele frequency above 5\%) combined with the limited ability of computational approaches to distinguish variants with no impact on genome function (the vast majority) from variants affecting gene function or expression that may be associated with disease risk or drug response (the minority). The development of algorithms for automated variant annotation,which link each variant with information that may help predict its molecular and phenotypic impact, is a critical step towards prioritizing variants that may have a functional impact from those that are harmless or have irrelevant functional effects. The goal of this protocol is to collect relevant information that will help answer questions about genetic variants discovered in next-generation sequencing studies, including: (i) will a given coding variant affect the ability of a protein to carry its functions; (ii) will a given non-coding variant affect the expression or processing of a given gene; and ultimately (iii) will a given coding or noncoding variant have any impact on phenotypes of interest?

Answering these questions is essential for many types of analyses that use large-scale genomics datasets to study quantitative traits and diseases, particularly when only a small number of individuals is studied comprehensively at a genome-wide level. For example, most genome-wide association studies (GWAS) or exome sequencing studies lack the statistical power to identify rare variants or variants with small effects associated with a disease, in part due to the large number of variants assayed. This limitation can be addressed by directing subsequent experimental steps to focus on smaller sets of genetic variants that have been prioritized based on external evidence of their putative impact. The common impairment of DNA repair mechanisms and chromatin stability in malignant cells leads to a similar challenge in cancer genomics, where the hundreds or thousands of mutations that distinguish an individual’s tumor and germline genomes need to be classified on the basis of their putative phenotypic effects and potential roles in carcinogenesis.

The large number of databases containing potentially helpful information about a given variant make the process of gathering and presenting relevant data challenging, despite excellent tools that already exist to analyze large genomics datasets (including GATK \cite{mckenna2010genome} and Galaxy \cite{goecks2010galaxy}) and visualize the results (such as the UCSC \cite{karolchik2014ucsc} or Ensembl \cite{flicek2012ensembl} genome browsers). Each of these databases uses its own format and is updated asynchronously, which makes it difficult for any analysis to remain up to date. In addition, the lack of comprehensive and computationally efficient models that allow integrative analyses using these resources, makes the task of comprehensive variant annotation overwhelming. By efficiently combining information from tens or hundreds of genome-wide databases, the tools and protocol described here are designed to greatly facilitate the process of variant annotation, and make it accessible to groups with limited bioinformatics expertise or resources.

In this protocol, we describe an approach to variant annotation that automatically collects, integrates, and presents a wide body of publicly available evidence of functional impact of a given set of genomic variants. The pipeline, based on the SnpEff package \cite{cingolani2012program}, is easy to execute and efficiently extracts a comprehensive set of variant annotations that can be used to prioritize downstream clinical or functional studies. SnpEff is used in many large genome centers and supports variant annotation for thousands of species, although the extent and quality of annotations extracted for a set of variants depends on the amount of publicly available genomics data for that species. In the case of whole-genome or exome sequences from human DNA samples, SnpEff extracts metadata (annotations) for each variant from relevant sources including gene annotation data identifying transcribed and translated regions; estimates of the frequency with which each variant occurs in different populations (from the 1000 Genomes project and the Exome Sequencing project \cite{10002012integrated}); and data that describes the function of regulatory elements that may be altered by the variant (obtained from the ENCODE project \cite{encode2012integrated} and Epigenome Roadmap \cite{bernstein2010nih}). SnpEff allows flexible and efficient querying of these annotations and is sufficiently fast to analyze very large sets of variants on a small computer (see Box 1 for computational and algorithmic considerations). It is also able to detect and be robust to a variety of gene annotation inconsistencies that would otherwise trigger false-positive high-impact variant predictions. In addition to SnpEff, an number of other annotation packages have been developed (including ANNOVAR \cite{wang2010annovar}, the Ensembl Variant Effect Predictor \cite{mclaren2010deriving}, GEMINI \cite{paila2013gemini} and VAT \cite{habegger2012vat}), which differ in terms of their functionality, ease-of-use, computational efficiency, and robustness.

%---
\section{Genetic variants}
%---

For most species, genetic variants are identified by comparing genome sequences from an individual organism to a reference (haploid) genome (see Box 2). A genetic variant, at a specific location in the genome (genetic locus) can be represented by agenotype which describes the difference between the DNA sequence present on each chromosome in the individual’s diploid genome and the reference genome. At a specific polymorphic locus, an individual can thus be homozygous for the reference allele, heterozygous, or homozygous for the non-reference allele.

There are as many types of genetic variants as there are types of mutations, and their frequency and breadth of impact on the genome vary tremendously \cite{ng2009targeted}. The most common type of variant identified by current technologies and analysis approaches is a single base difference with respect to the reference genome. Depending on whether the variant was identified in an individual or in a population, it is called a Single Nucleotide Variant (SNV) or Single Nucleotide Polymorphism (SNP). Sequence differences affecting several consecutive nucleotides are called a multiple nucleotide polymorphism (MNP) and are typically treated as a single variant locus if they are in perfect linkage disequilibrium. Short insertions and deletions (indels) of a chromosome region range from 1 to 20 bases in length are approximately 30 times less frequent than SNVs1but may have profound effects on protein activity by altering the translation reading frame or deleting a protein domain. Genomic variants involving larger regions are more rare and more difficult to infer using short-read NGS technologies. Those include large deletions, which can result in the loss of an exon or one or more whole genes, and insertions, which often originate from transposable elements or tandem duplications. Large genome variants that cause the number of copies of a particular genomic region to be polymorphic in a population are called a Copy Number Variants (CNV). Genomic rearrangements such as inversions and translocations are events that involve two or more genomic breakpoints and a reorganization of genomic segments, possibly resulting in gene fusions or loss of critical regulatory elements. This protocol does not address the annotation or rearrangements due to the challenges involved in their identification and functional characterization and their relative rarity in the germ line.

The process of inferring variants present in an individual’s genome from sequencing data is called variant calling and is based on sophisticated algorithms that have been reviewed elsewhere \cite{nielsen2011genotype}. Genome-wide variant calling has until recently largely been done using genotyping arrays (for SNVs) or Comparative Genomic Hybridization arrays (for CNVs). The inherent limitations of these technologies, particularly their ability to only assay genotypes at sites that are known in advance to be polymorphic, combined with the declining cost of sequencing, have now made approaches based on high-throughput resequencing the tool of choice for variant calling in clinical studies. Although this is a challenging task and remains an important area of research, many high-quality tools exist for calling SNVs and indels (such as GATK2 and SamTools \cite{li2009sequence}), as well as detecting CNVs (such as PennCNV \cite{wang2007penncnv} and CNVhap \cite{coin2010cnvhap}), and structural variants (e.g. VariationHunter \cite{hormozdiari2011simultaneous}). The output of these tools – variant calls – are stored using the standardized format called the Variant Call Format \cite{danecek2011variant} (VCF; see Box 3). Their calling accuracy depends on the type of variants, their frequency in the population of patients or tumor cells being studied, and the quantity (coverage) and quality of the sequencing data. As the length, accuracy and coverage of sequencing reads increases, variant calling will become easier and more accurate. Therefore, we discuss here the problem of annotating the variants identified by some of these tools, and refer the reader to the review by Nielsenet al.14 to learn more about the process of variant calling itself.

\subsection{Types of genetic annotations}

The process of genetic variant annotation consists of the collection, integration, and presentation of experimental and computational evidence that may shed light on the impact of each variant on gene or protein activity and ultimately on disease risk or other phenotypes. Variant annotation has traditionally been divided in two apparently independent but actually interrelated tasks based on the variant’s location with respect to known protein-coding genes (see Table 1 for a list of commonly used variant annotations).Coding variant annotation focuses on variants that are located within coding regions of annotated protein-coding genes and attempts to assess their impact on the function of the encoded protein. In contrast,non-coding variant annotation focuses on variants located outside the coding portion of genes (i.e. in intergenic regions, UTRs, introns, or non-protein-coding genes) and aims to assess their potential impact on transcriptional and post-transcriptional gene regulation. These two categories of variant annotations are not mutually exclusive, as variants located within exons can often have an impact on the gene transcript’s processing (splicing). In addition, some transcripts can have both protein-coding and non-coding functions. Despite the intermingling of the notion of coding and noncoding variants, we will consider each type of annotation separately as assessing their impact requires different sources of data and algorithms.

The ultimate goal of variant annotation is to predict the impact of a sequence variant, although this is an ill-defined term. One the one hand, one may be interested in the molecular impact of a variant on the activity of a protein. On the other, others may be interested in a variant’s impact on much higher-level phenotypes such as disease risk. Mutations that are predicted to completely abrogate a gene’s activity are calledloss-of-function (LOF) mutations; while mutations that are tentatively predicted to have less severe consequences are called moderate or low impact mutations.In practice, a variant will be predicted to cause LOF if it has two properties: (i) its molecular impact is reliably predictable by existing computational approaches (e.g. gain of stop-codon); and (ii) its functional impact, reflected by altered protein activity or expression levels, is expected to be large. Many types of variants, including most non-coding variants, may have a large functional impact but lack predictability, and as a consequence are typically not predicted to be LOF variants.

\subsection{Coding variant annotation}

Coding variants occur in a translated exon. When a reliable gene annotation is available, their main impact can be classified by determining their effect on the translated amino acid sequence (if any). A synonymous coding variant (also called silent) does not change the sequence of amino acids encoded by the gene, although it may impact aspects of post-transcriptional regulation such as splicing and translation efficiency and can affect the total protein activity through changes in the amount of translated protein that is made in the cell. In contrast, a non-synonymous coding variant changes one or more amino acids encoded by the gene and can directly alter the protein’s activity, localization or stability. Non-synonymous variants include missense substitutions that change a single amino acid, nonsense substitutions that lead to the gain of a stop codon,frame-preserving indels that insert or delete one or more amino acids, and frame-shifting indels that may completely alter the protein’s amino acid sequence. Primary annotation and assessment of impact, which performed directly by SnpEff, determines whether a variant falls in any of these categories.

\textbf{Caveats}
	\begin{enumerate}[label=\roman*]
	
	\item \textit{Gene misannotation.} Genomic variants that have a significant effect on a protein’s expression or function represent a very small fraction of all variants. Assembly and gene annotation errors or genomic oddities that break classical computational models are also rare, but often overestimate the variant’s impact. This implies that one is likely to find a non-negligible fraction of false-positive high-impact variants among the list of what appear to be the strongest candidates for variants with severe effects. Tools such as SnpEff can anticipate some of the most common causes of misannotation, but the number and diversity of the type of events that can lead to false-positives makes the task very challenging. As a consequence, one should always manually inspect the top candidates to ensure that they have been assigned to the correct genes and transcripts.
	
	\item \textit{Gene isoforms.} In higher eukaryotes, most genes have more than one transcript (or isoform), due to alternative promoters, splicing, or polyadenylation sites. For example, a human gene has an average of 8.8 annotated messenger RNA (mRNA) isoforms and some genes are believed to have over 4,000 isoforms resulting from complex splicing programs. For these genes, a variant may be coding with respect to one mRNA isoform and non-coding with respect to another. There are two frequent approaches to address this situation: (i) annotate a variant using the most severe functional effect predicted for at least one mRNA isoform; or (ii) use only a single canonical transcript per gene to perform primary annotation. 
	
	\item \textit{Variant calling for indels.} Variant annotation relies on knowing the exact genomic coordinates of the variant: this is rarely a problem for isolated SNVs; however, insertions and deletions often cannot be located unambiguously. Consider for example the variant $AA \rightarrow A$. This mutation results in the loss of a single base, but was it the first or second A that was deleted? From the standpoint of the cell, this question is irrelevant and deletion of any A will have the same effect. In contrast, from the standpoint of most variant annotation software, deleting the first A is different from deleting the second. Consider the scenario of a previously annotated transcript where the first A is part of the 5' UTR and the second is the first base of a start codon. If the missing base is assigned to the leftmost position in the motif (as is the current convention), the deletion would be annotated as a low impact 5'UTR variant. However, assigning it to the rightmost A would make it appear (incorrectly) to be a high-impact start-codon deletion. Similar issues may arise when considering conservation scores or transcription factor binding site (TFBS) predictions.
	
		\end{enumerate}

\subsection{Loss of function variants}

True LOF variants are difficult to predict computationally, but specific types of genetic changes will frequently lead to severely impaired protein activity. These include (i) stop-gains (nonsense mutations) and start-loss; (ii) indels causing frameshifts; (iii) large deletions that remove either the first exon or at least 50\% of the protein coding sequence; and (iv) loss of splice acceptor or donor sites that alter the protein-coding sequence. Variants that introduce premature in-frame stop codons (nonsense mutations and most frameshift indels) are expected to abolish protein function, unless the variant is very near the C-terminus of the coding region \cite{yamaguchi2008distribution} (effectively, downstream of the last functional domain in the protein). This may cause severe consequences in affected cells, tissues or organism, as is seen for mutations that cause monogenic diseases \cite{scheper2007translation}. In addition, a new stop codon that lies upstream of the last exon will likely trigger nonsense mediated decay (NMD), a process that degrades mRNA before protein synthesis occurs \cite{nagy1998rule}. NMD predictions are not exact and many factors can affect mRNA degradation, including the variant’s distance from the last exon-exon junction or poly-A tail, and the possibility that transcription may re-initiate downstream of the LOF variant \cite{brogna2009nonsense}.

A variant that leads to the loss of a stop codon, sometimes called aread-through mutation, will result in an elongated protein-coding transcript that terminates at the next in-frame stop codon. While there are no general models that predict how deleterious this may be, variants that elongate the reading frame can also result in aberrant folding and degradation of the nascent proteins, leading to activation of cellular stress response pathways in addition to their direct effects on protein activity and expression levels21.

The effect of the loss of a start codon depends on the location of a replacement start codon with respect to the translation start site and reading frame of the native protein. If the new start codon maintains the reading frame, the only consequence may be the loss of a few amino acids in the protein transcript; however, in many cases, the new start codon will not be in-frame, thus producing a frame-shifted protein that is later degraded. In addition, the new start codon may lack an appropriate regulatory context (for example, if there is no Kozak sequence nearby or if it disrupts 5’ UTR folding) leading to reduced expression of an N-terminally truncated protein. Consequently, losing a start codon is thought to be highly deleterious in most cases, due to the potential that it may reduce both protein production and activity.

\textbf{Caveats}
	\begin{enumerate}[label=\roman*]
	
	\item \textit{Rare amino acids.} Through a process called translational recoding, a UGA ``Stop" codon located in the appropriate mRNA context (determined by both primary mRNA sequence and secondary structure) may be translated to incorporate a selenocysteine amino acid (Sec / U). In humans, this occurs at approximately 100 codons located in mRNAs whose 3’ UTR contains a Selenocysteine insertion sequence element (SECIS). Since the translation machinery goes so far to encode these special rare amino acids, the expectation is that mutations at those sites would be highly deleterious. This is supported by evidence that reduced efficiency of selenocysteine incorporation is linked to severe clinical outcomes, such as early onset myopathy  \cite{maiti2009mutation} and progressive cerebral atrophy  \cite{agamy2010mutations}.
	
	\item \textit{False-positives in LOF predictions.} Variants predicted to result in a LOF sometimes actually produce proteins that are partially functional  \cite{macarthur2012systematic}. In fact, an apparently healthy individual is typically heterozygous for around 100 predicted LOF variants, and homozygous for roughly 10 variants, but many of those are unlikely to completely abolish the protein function. Indeed, these variants are enriched toward the 3’ end of the gene, where they are likely to be less deleterious. 
	
	\end{enumerate}

\subsection{Variants with low or moderate impact}

Compared to the high impact variants discussed above, where extensive prior biological evidence strongly suggests that a specific type of variant will severely impair protein activity, there are few guidelines that can reliably predict how the majority of nonsynonymous (missense) variants will alter protein function or expression. As a result, the primary annotation performed by SnpEff and most related software packages will broadly categorize missense substitutions and their accompanying amino acid changes (e.g. K154->L154) as moderate impact variants. Short indels whose length is a multiple of three are treated similarly, unless they introduce a stop codon, as their effect will usually be localized.

Once missense and frame-preserving indel variants are identified, a more detailed estimation of their impact on protein function can be performed using heuristic and statistical models. The most common approaches are based on conservation, either amongst orthologous or homologous proteins, or protein domains, sometimes adding information of the physio-chemical properties of the reference and variant amino acids (e.g. differences in side chain charge, hydrophobicity, or size). The SIFT algorithm \cite{kumar2009predicting} assesses the degree of selection against specific amino acid changes at a given position of a protein sequence by analyzing the substitution process at that site throughout a collection of predicted homologous proteins identified by PSI-BLAST \cite{altschul1997gapped}. Based on this multiple sequence alignment and the highly conserved regions it contains, SIFT calculates a normalized probability of amino acid replacement (called the SIFT score), which estimates the mutation’s effect on protein function.Polyphen \cite{adzhubei2010method}, another commonly used tool, takes the process one step further by searching UniProtKB/Swiss-Prot \cite{uniprot2013update} and the DSSP database of secondary structure assignments \cite{joosten2011series} to determine if the variant is located in a known active site in the protein.In contrast to other methods that categorize each variant individually, VAAST \cite{rope2011using}, a commercially available package, computes scores for groups of variants located within a given gene and ``collapses" them into a single category, a concept similar to burden testing performed for rare variants identified in exome sequencing studies.For human proteins, SnpEff makes use of the Database for Nonsynonymous SNVs’ Functional Predictions \cite{liu2011dbnsfp} (dbNSFP), which collects scores produced by several impact assessment algorithms in a single database. Individually, impact assessment methods usually have an estimated accuracy of 60\% to 80\%, but predictions from several algorithms can be combined to provide a stringent, but more accurate estimate of impact \cite{choi2012predicting}.

In most cases these algorithms apply best to SNVs since these are common in populations and there is more genomic sequence andexperimental data available to refine the statistical methods. However, some recently developed algorithms are capable of assessing variants other than SNVs, including PROVEAN \cite{choi2012predicting}, which extends SIFT to assess the functional impact of indels.

\textbf{Caveats}
	\begin{enumerate}[label=\roman*]
	
	\item \textit{Imprecise models of protein function.} Accurate impact assessment of coding variants remains an open problem and most computational predictions are riddled with both false positives and false negatives. While both missense variants and frame-preserving indels are broadly cataloged as having moderate effects, this is mostly due to lack of a comprehensive model and the extremely complex computations that would be required for an in-depth analysis (such as protein structure predictions). In these cases, proteomic information can be revealing. SnpEff adds annotations from curated proteomic databases, such as NextProt  \cite{lane2012nextprot}, which can help to elucidate if a mutation alters a critical protein amino acid or domain (such as amino acids that are post-translationally modified as part of a signaling cascade or that are form the active site of an enzyme) resulting in a protein may no longer function.
	
	\item \textit{Gain of deleterious function.} Computational variant annotation may eventually be able to fairly accurately predict the molecular impact of a variant in terms of the degree to which it translates in a loss of function for the encoded protein. However, gains of function, including the acquired ability to interact with new partners and disrupt their function, remain vastly more difficult to tackle, although a several such variants have been linked to disease, such as hereditary pancreatitis  \cite{whitcomb1996hereditary}.
	
	\item \textit{Unanticipated effects of synonymous variants.} In most cases, synonymous variants are regarded as non-deleterious (or low impact); however, one needs to seriously consider the possibility that they may have greater functional effects by altering mRNA splicing  \cite{coulombe2009fine} or secondary structure  \cite{sabarinathan2013rnasnp}. Synonymous SNVs may also alter translation efficiency, by changing a frequently used to a rarely used codon and have been linked to changes in protein expression  \cite{sauna2011understanding}.
	
	\end{enumerate}

\subsection{Non-coding variant annotation}

Although coding variants represent less than 2\% of variants in the human genome, they make up the vast majority of confirmed disease-related variants that have been validated at a functional level. This may result from ascertainment bias (since variants in coding regions are straightforward to discover and characterize at a basic level and many studies have largely ignored non-coding variants); or may be explained by the increased complexity of computational approaches and lab assays required to predict and validate the impact of non-coding variants; or by their potentially more subtle impact on gene expression or cell function. Nonetheless, in a compendium of current GWAS studies, roughly 40\% of the variants are intergenic and 30\% intronic and functional studies of these variants are increasingly emphasizing the importance of non-coding genetic variation at risk loci for complex genetic diseases and traits \cite{hindorff2009potential}.

Functional non-coding regions of the genome encompass a wide variety of regulatory elements contained in DNA and RNA molecules that are involved in transcriptional and post-transcriptional regulation. Cis-regulatory elements include (i) binding sites for DNA-binding proteins such as transcription factors and chromatin remodelers; (ii) binding sites for RNA-binding proteins involved in splicing, mRNA localization, or translational regulation; (iii) micro RNA (miRNA) target sites; and (iv) long non-coding RNA (lncRNA) targets on DNA, RNA and proteins. Non-coding transcripts include well-characterized regulatory RNAs (e.g. miRNA, snoRNA, snRNA, piRNA and lncRNAs) as well as RNAs involved directly in protein synthesis (e.g. tRNA and rRNA).  The annotation and impact assessment of non-coding variants presents a significant challenge for several reasons: (i) reliable technologies to study transcriptional regulatory regions on a genome-wide basis are only just reaching maturity and provide limited resolution of binding sites for individual transcription factors and regulatory RNA molecules; (ii) non-coding functional regions of most genomes remain incompletely mapped as they vary widely among different cell types and cell states (for example, in diseased and healthy tissues); (iii) non-coding regulatory elements often are part of complex transcriptional programs that are time-dependent, contain many redundant linkages or reciprocal connections between genes and respond to a wide range of intra- and extracellular signals; and (iv) genomic regulatory elements rarely have a strict consensus sequence (for example, compare the position weight matrices used to identify transcription factor or miRNA binding sites with the amino acid triplet code) making the effect of a mutation on gene regulatory programs difficult to predict. As a result, high-quality annotation of non-coding variants relies more heavily on experimental data than is the case for coding variants: since many of these experimental techniques did not study the effects of SNVs on gene regulatory programs, they can only be used to annotate variants and not to predict their effects on gene transcription. In the few cases where the effects of SNVs have been studied (for example, the effects of SNVs that are common in a population and located in genetic loci associated with complex diseases), experimental approaches provide highly accurate functional assessment at a cost of reduced coverage compared to computational approaches.

Large-scale projects such as ENCODE7and modENCODE \cite{celniker2009unlocking} have made major steps toward mapping gene transcription and transcriptional regulatory regions in many tissues and cell types, but similar studies in diseased tissues remain at an early stage (for example, the growing collection of disease-related epigenomes from the Epigenome Roadmap8). The base-by-base resolution and number of cell states studied for different types of regulatory elements and non-coding transcripts varies widely among datasets; in part due to the lack of sensitive, comprehensive and high-resolution technologies to study the different molecular species and modes of interaction that can be altered by non-coding variants. Efficient technologies for genome-wide, high-throughput mapping of binding sites for RNA-binding proteins (PAR-CLiP \cite{ascano2012identification}), miRNAs (PAR-CLiP \cite{hafner2012genome} and CLASH \cite{helwak2013mapping}) are starting to be applied on a broad scale as are protocols to map transcription factor binding sites (TFBS) which can improve resolution to a single base (Chip-exo \cite{rhee2012chip}). However, in most cases, DNA and RNA binding sites are only imprecisely located within Chip-Seq peaks that span genomic regions hundreds of base pairs in length, with computational approaches being used to pinpoint the bases most likely mediating the interaction. In the absence of more precise localization data,de novo computational prediction of binding sites for DNA and RNA binding proteins remains insufficiently accurate to be of much use in annotating single noncoding variants.

This limitation is particularly critical for functional predictions of putative target sites for microRNAs and other regulatory RNA species. MicroRNAs are short RNA molecules that regulate gene expression post-transcriptionally by binding the messenger RNA of a gene through complementary, usually in the 3’ region of the transcript, which leads to mRNA degradation or inhibits translation. Sequence variants that cause the loss or gain of a miRNA target site would lead to dysregulation of the gene, with likely deleterious effects. Although miRNAs are relatively well documented in most model organisms including human, their binding sites are only starting to be mapped experimentally, and computational predictions has very low specificity. Meaningful information regarding the possible role of a variant in disrupting a miRNA target site is starting to emerge \cite{liu2012mirsnp}, although variants that create new miRNA binding sites remain under the radar.

Even if the position of a functional element could be perfectly determined, predicting a variant’s impact on chromatin conformation, promoter activity, gene expression, or transcript processing remains challenging. For transcription factors, this involves predicting whether the protein will still be able to recognize its mutated site (and with what affinity), as well as predicting the impact of these changes on gene expression levels. The latter is particularly hard to predict as a result of interactions, competition, and redundancy contained in regulatory networks of transcription factors or RNA binding proteins. As a consequence, computational prediction of the functional impact of non-coding variants remains a very active area of research and there is no broad consensus on the best methodology to use \cite{ward2012interpreting}. One significant exception is the identification of variants affecting canonical splice sites, defined as two bases on the 3’ end on the intron (splice site acceptor) and 5’ end of the intron (splice site donor). Variants that affect canonical splice sites are easily detected and typically lead to abnormal mRNA processing, involving exon loss or extension that leads to loss of function of the encoded protein.

\subsection{Impact assessment of non-coding variants}

Two broad classes of publicly available genome-wide datasets are commonly combined to assess the functional impact of non-coding genetic variants: (i) computational predictions of sequence conservation and sites involved in molecular interactions such as transcription factor and RBP binding, as well as miRNA-mRNA target interactions; and (ii) experimental genome-wide localization assays for DNA binding proteins, histone modifications, and chromatin accessibility.

\paragraph{Computational sources of evidence:} Interspecies sequence conservation plays a key role in scoring and prioritizing non-coding variants. This is based on the assumption is that sites or regions that have been more conserved across species than expected under a neutral model of evolution are likely to be functional; suggesting that mutations contained in them are likely to be deleterious. In the absence of strong experimental data, sequence conservation measures calculated from whole genome multiple alignments, (for example using PhastCons  \cite{siepel2005evolutionarily}, SciPhy  \cite{garber2009identifying}, PhyloP  \cite{pollard2010detection} , and GERP  \cite{davydov2010identifying}), have been developed to provide a generic indicator of function for non-coding variants. Although high conservation scores generally mean that a genomic region may be functional, the converse is not true and many experimentally proven functional noncoding regions show only modest sequence conservation (for example due to binding site turnover events). Finally, some regulatory regions (e.g. specific elements regulating immune response  \cite{raj2013common}) are under positive selection and may thus show less conservation than surrounding neutral regions. 

In human, genome-wide computational predictions of transcription factor binding sites based on matching to publically available position weight matrices are available from variety of sources, including Ensembl 5 and Jaspar  \cite{bryne2008jaspar}.  Because of the low information content of most binding affinity profiles, the specificity of the predictions is generally very low. Related approaches exist to predict splicing regulatory regions  \cite{fairbrother2002predictive} and miRNA target sites \cite{ziebarth2011polymirts}, some of which are precomputed for whole genomes and available from the UCSC or Ensembl genome browsers. Recent efforts to determine RNA-binding protein sequence affinities can also be used to identify putative binding sites for these proteins in mRNA  \cite{ray2013compendium}.

\paragraph{Experimental sources of evidence:} To investigate the potential impact of variants on transcriptional regulation, many published experimental data sets produced by large-scale projects such as ENCODE 7, modENCODE 42 and Roadmap Epigenomics 8, can be used directly by annotation packages. These include: (i) ChIP-seq or ChIP-exo experiments that identify TFBSs on a genome-wide basis; (ii) DNAseI hypersensitivity or Formaldehyde-Assisted Isolation of Regulatory Elements (FAIRE) assays that identify regions with open chromatin; and (iii) ChIP-seq studies to identify the presence of specific promoter or enhancer-associated histone post-translational modifications, which can be combined to identify active, poised, and inactive enhancers and promoters 57. Most of these data sets are easily available through Galaxy 3 (as tracks from the UCSC Genome Browser) or through SnpEff (as downloadable pre-computed datasets). In parallel with the types of studies described above, expression quantitative trait loci (eQTLs) represent an agnostic way to map putative regulatory regions. An increasing number of such loci are available through the GTEX database  \cite{lonsdale2013genotype}. Experimental data that may support assessment of the impact of variants on post-transcriptional regulation remain sparser, although databases such as doRiNa  \cite{anders2011dorina} or starBase  \cite{yang2011starbase} contain genome-wide datasets obtained by CLIP-Seq and degradome sequencing. To our knowledge, these data have yet to be used in the context of variant annotation studies.

\paragraph{Combining sources of evidence:} Despite the variety of computational and experimental sources of evidence available, impact assessment for non-coding variants remains relatively crude, due to the fact that biological models of gene regulation remain fairly simple. Nonetheless, significant steps forward have been made recently, and two web-based tools, HaploReg  \cite{ward2012haploreg} and RegulomeDb  \cite{boyle2012annotation}, perform SNV and indel impact assessment for variants from dbSNV on the basis of a broad body of computational and experimental evidence. Both use pre-computed scores for variants from dbSnp and therefore cannot be used for rare variants, but they are extremely valuable for exploration by associating the variant of interest with a variant in dbSnp via linkage disequilibrium. 

\textbf{Caveats}
	\begin{enumerate}[label=\roman*]
	
	\item \textit{Sparseness of functional sites within ChIP-seq peaks.} Even if a noncoding variant is located in a region that contains a ChIP-seq peak for a given TF and has all the hallmark signatures of regulatory chromatin, the likelihood that it is deleterious remains low, because most DNA bases contained within a peak are non-functional. 
	
	\item \textit{Gain of function mutations.} While this section, has focused on variants causing the loss of a functional regulatory element, genetic variants may also create new or more effective transcription factor binding sites. These are substantially harder to detect as they can occur in regions that show no evidence of function in individuals possessing the reference allele, and show little conservation across species. Furthermore, computational methods to predict gain of affinity for a given TF caused by a variant have insufficient specificity to be of much use on their own. 
	
	\end{enumerate}

%---
\section{ Clinical effect of variants}
%---

One of the most revealing types of annotation of both coding and noncoding variants reports whether the variant has previously been implicated in a phenotype or disease. Although such information is available for only a small minority of all deleterious variants, their number is growing and should be the first type of annotation one seeks out. Clinical annotations, until recently, have been scattered in a large number of specialized databases of medical conditions with a genetic basis, including the comprehensive, manually curated collection of genetic loci, variants and phenotypes in the Online Mendelian Inheritance in Man database (OMIM, http://www.omim.org); web pages containing detailed clinical and genetic information about uncommon disorders in the Swedish National Board of Health and Welfare Database for Rare Diseases (http://www.socialstyrelsen.se/rarediseases) and the peer-reviewed NIH GeneReviews collection (54://www.ncbi.nlm.nih.gov/books/NBK1116); and a curated collection of over 140,000 mutations associated with common and rare genetic disorders in the commercial Human Gene Mutation Database (HGMD, http://www.hgmd.org/). In most cases, these datasets do not use standardized data collection or reporting formats; are designed to primarily provide information to patients and health professionals through a web interface; and rely on heterogeneous criteria to describe disease phenotypes and clinical outcomes; pathological and other clinical laboratory data; as well as the genetic and biologic experiments that have been used to demonstrate disease mechanisms at a molecular or cellular level. These shortcomings are being addressed by initiatives that provide centralized, evidence-based, comprehensive collections of known relationships between human genetic variants and their phenotype that are suitable for computational analysis, such as the NIH effort to aggregate records from OMIM, GeneReviews and locus-specific databases in ClinVar (http://www.ncbi.nlm.nih.gov/clinvar). 

Another important application of variant detection and annotation is in the study of cancer genomes, which is occurring increasingly in clinical settings to support treatment decisions for advanced tumors. Annotation of variants detected in tumor sequences can be analyzed for clinical cohorts, using similar techniques as other complex traits, as well as for individual patients, using techniques to identify differences between somatic (tumor) and germline (healthy) tissues. In the latter case, one looks for cancer-associated mutations that distinguish the somatic genome of cancer cells of an individual from the germline genome in order to find the driving mutations that pinpoint the specific mechanisms underlying tumorigenesis or metastasis. Ideally, these mutations can be used to select a treatment for the patient, establish prognosis, or to identify causative mutations that have led to the cancer’s progression. In such a setting, given that sequence differences between the cancer and germline genomes are of greater interest than the background genetic changes between the germline and a reference genome, variant calling is performed using specialized algorithms, such as MuTect  \cite{cibulskis2013sensitive} and SomaticSniper  \cite{larson2012somaticsniper}.

Once variants are called, variant annotation focuses on somatic variants that are not present in the germline genome, which is the new ``reference genome". Although SnpEff was originally developed for the study of germline genomes, it also contains modules that allow the annotation of cancer mutations. A seemingly simple approach would be to create a new reference genome using the individual’s germline genome, and then annotate somatic mutations by comparison to this new reference. Unfortunately, this approach would be laborious and computationally expensive, so a preferred solution is to compare each genome to the reference human genome, and reconcile shared differences by creating a germline genome ``on the fly" only for those regions that require it (i.e. variants that are shared by the germline and cancer genome are disregarded). This optimization reduces the processing time from hours to only a few seconds, making it viable for analysis of hundreds of samples simultaneously.

\textbf{Caveats}
	\begin{enumerate}[label=\roman*]

	\item \textit{Annotation accuracy.} Biological knowledge, as well as molecular and phenotypic evidence supports the identification of certain groups of high impact variants based on simple criteria (such as premature stops, frameshifts, start lost and rare amino acid mutations); however, it is often hard to predict whether non-synonymous variants will have equally large effects on an organism's health. Even when the accepted ``rules of thumb" used in the primary annotation indicate that protein function is impaired, we should consider that these predictions may be based on a small number of model genes and will require appropriate wet-lab validation or confirmatory studies in cohorts. In addition, as more human genomes are sequenced, it is likely that some genetic variants that have been linked to Mendelian diseases will be found in healthy individuals  \cite{riggs2013towards}; and in many cases, may not actually be disease-causing mutations  \cite{bell2011carrier}.
	
	\end{enumerate}

\textbf{BOX 1: Data structures and computational efficiency}
\begin{framed}
Most of the computational pipelines for genomic variant annotation and primary impact assessment are relatively efficient and can annotate variants obtained from large resequencing projects involving thousands of samples within a few minutes or hours even using a moderately powered laptop. This is typically achieved through two key optimizations: (i) creation of reference annotation databases and (ii) implementation of efficient search algorithms. Reference database creation refers to the process of creating and storing precomputed genomic data from the reference genome, which can be searched quickly to extract information relevant to each variant. This process needs to be performed only once per reference genome and most annotation tools have pre-computed databases for many organisms available for users to download (for instance, SnpEff currently offers databases for over 25,000 organisms). Since these databases are typically quite large, efficient search algorithms are used together with appropriate data structures to optimize the search process. In ANNOVAR 9, each chromosome is subdivided in a set of intervals of size k and genomic features for a given chromosome are stored in a hash table of size L/k, where L is the length of the chromosome. Another approach, used by SnpEff, is to use an ``interval forest", which is a hash of interval trees  \cite{cormen2001introduction} indexed by chromosome. Querying an interval tree requires O[log(n) + m] time, where n is the number of features in the tree and m is the number of features in the result. Both approaches are extremely efficient.
\end{framed}

\textbf{BOX 2: Reference genomes and gene annotations}
\begin{framed}
A reference genome is the standard against which every genome within a species is compared. For example, the latest assembly of the human reference genome from NCBI is hg19, which is equivalent to ENSEMBL’s GRCh37. A reference genome is not static; and new and improved assemblies are released frequently for newly assembled genomes and less frequently for more mature genomes such as human or mouse. Because genomic coordinates may change from one assembly to the next, it is critical that all analyses of genomic variants in a given project be done with respect to the same version of the reference genome (ideally the most recent), and also that the version of reference genome used in an analysis be reported in publications. Mapping reads to one reference genome, while performing variant annotations using another reference genome is a common mistake that leads to disastrous results.

Gene annotations consist of the genomic coordinates of every known isoform for all genes in the genome. This includes the position of the start and end of transcription, splice sites, and in the case of protein-coding genes, the start and stop codons. In contrast to genome assemblies, there are typically several different sources of gene annotations for each genome; and these sources are often not consistent with each other. For example, human genome annotations include the RefSeq genes  \cite{pruitt2007ncbi}, a set of well characterized and highly curated genes, as well as the UCSC  \cite{hsu2006ucsc} and ENSEMBL  \cite{curwen2004ensembl} gene annotations, which have higher coverage but are less stringently curated. Gene annotation revisions are frequent and may be made as part of a specifically identified ``genome version" or on a continuing basis. Whereas ENSEMBL releases new sub-versions of their genome annotation several times a year (e.g. GRCh37.72 is human reference version 37, subversion 72), this is not the case for most other genome annotation providers, where changes in genome annotation happen asynchronously and unannounced. For those, it is important to report the full transcript ID (e.g. NC\_XXXX.S), which will identify the annotation subversion, because future transcript subversions might correct genomic annotation errors. In most cases, coordinates from different genome annotation versions based on the same reference genome assembly will be the same, but it is not uncommon to find a gene in a different position, or even a different chromosome.

Despite the huge amount of work being done to curate gene annotations in reference genomes, some errors and oddities persist; including non-phased start positions, proteins without start or stop codons, incomplete transcripts, 4-base codons, 1-base introns, etc. Indeed, the predicted amino acid sequences for about 10\% of the transcripts in the RefSeq database do not exactly match the corresponding protein. These cases may be due to errors in the reference genome assembly or genome annotation, but may also reflect rare post-transcriptional regulatory events such as RNA editing. Because they confuse most variant annotation pipelines, these cases can easily lead to an overestimation of the potential impact of a coding variant. SnpEff can identify the most common of these sources of errors and flag suspicious cases for the user to review.
\end{framed}

\textbf{BOX 3: Approaches to standardization}
\begin{framed}
Bioinformatic standards make it possible to create programs that interoperate and share complex datasets. In the absence of a ``one size fits all" format to describe genes, proteins and genetic variants, different file formats are used for different purposes, each one having their strengths and weaknesses. A crucial part of bioinformatics analysis is to use the right file format appropriately, so here we introduce file formats and standards used most commonly used variant annotations. Many bioinformatics formats are text based and can be read using a text editor or as a spreadsheet, which are convenient ways to identify problems or debug analysis strategies.

VCF (Variant Call Format): This format, introduced by the 1000 Genomes project, provides a standard for describing genetic variants. Each line in a VCF file (record) represents a genomic location (a variant) and is described by metadata in ``fields" separated by tabs. A VCF file record contains eight mandatory fields: i) chromosome name (CHROM), ii) position (POS), iii) variant name (ID), iv) reference allele (REF), v) alternative allele (ALT), vi) variant call quality which is an error probability estimation (QUAL), vii) filter pass or filter fail parameters (FILTER), and viii) a generic container for information (INFO). The INFO field is used to add additional metadata in a semi-structured way and is the field where annotations can be added. Here is an example of a few lines of a VCF file:

\scriptsize{
	\begin{verbatim}
	#CHROM POS     ID        REF    ALT     QUAL FILTER INFO                              
	20     14370   rs6054257 G      A       29   PASS   NS=3;DP=14;AF=0.5;DB;H2           
	20     17330   .         T      A       3    q10    NS=3;DP=11;AF=0.017               
	20     1110696 rs6040355 A      G,T     67   PASS   NS=2;DP=10;AF=0.333,0.667;AA=T;DB 
	20     1230237 .         T      .       47   PASS   NS=3;DP=13;AA=T                   
	20     1234567 microsat1 GTC    G,GTCT  50   PASS   NS=3;DP=9;AA=G                    
	\end{verbatim}
}

\normalsize{
Missing data is indicated by a period. The mandatory fields may be followed by optional genotype fields - one column per sample sequenced or genotyped - that describe whether the genetic variant was observed in that sample.

In the VCF standard, the REF field (column 4) identifies the base present in the reference genome and is independent of the samples studied in a particular experiment. This is an important distinction for cancer samples, where it is common to compare somatic to germline sequences using a ``virtual reference genome" (the germline genome). For compatibility with other software that uses VCF files, this ``virtual reference" should not be placed in the REF field. Somatic and germline samples are treated just as two samples added as two genotype fields (to columns after column 9).

BED (Browser Extensible Data) Files: This simple format is often used to describe intervals or regions in the genome. Annotation software allows users define custom intervals using BED formatted files, since they can be easily created and manipulated in spreadsheets or text documents. The BED format recognizes one interval per line: the line can include metadata separated by spaces or tabs that describes: i) the chromosome name, ii) the start position as a zero-based coordinate, iii) the end position as a one-based coordinate. These mandatory fields can be followed by other labels or scores (see https://genome.ucsc.edu/FAQ/FAQformat.html\#format1 for details). The unusual coordinate choice of zero- / one-based in the start and end position is a source of many headaches and plenty of confusion amongst researchers, but provides more flexibility in describing genomic regions. For example, it is possible to define a zero length interval by providing the same coordinate in both fields (which is not possible in many other formats).

HGVS nomenclature: A standard used to describe variants occurring in a protein, DNA or RNA molecules, which has emerged as a standard in translational research. As a simple example of this format ``p.Arg22Ser" describes a variant changing amino acid number 22 from Arginine to Serine. The standard is quickly evolving in an attempt to make it comprehensive for all types of variants observed in sequencing studies. 

Variant annotation and sequence ontology. Recent efforts have been made to standardize the output format of variant annotation tools. The Human Genome Variation Society (HGVS) created a format to describe protein, RNA and DNA mutations that has emerged as a standard in translation research. As a simple example of this format ``p.Arg22Ser" describes a variant that changes amino acid number 22 from Arginine to Serine. This format is supported by most annotation software packages, at least to some degree. Another format gaining momentum is the ``Sequence Ontology", which is an ontology of sequence changes and their putative effects (e.g. ``stop\_gained" or ``missense", see Table 1). These two standards help to provide a unified vocabulary for annotations that helps to avoid nomenclature problems and artificial barriers between different software programs. Hopefully, standards for annotations can propagate into the VCF specification, as this will make it easier to transfer datasets between different annotation and analysis packages, change processing pipelines, or even use multiple packages providing complementary functionality. All these tasks are now difficult due to lack of standards for describing the results of large-scale genotyping and next generation sequencing studies. 
}
\end{framed}

%---
\section{Materials}
%---

In this protocol we show how to analyse genomic variants using the SnpEff pipeline.

\textbf{Computer hardware:} The materials required for this protocol are: a computer running a Unix operating system (Linux, OS.X), with at least 4GB of RAM, at least 1Gb of free disk space, Java version 1.6 or higher installed, and a reasonably fast internet connection. Users of Windows computers can install CygWin, a free Linux-like environment for Windows, although the precise commands listed in the protocol may need to adapted.

\textbf{Software:} We use the SnpEff annotation program and its companion tool SnpSift. These programs can perform annotation, primary impact assessment and variants filtering, as well as many other tasks beyond the scope of this protocol. We highly recommend reading their comprehensive documentation available at http://snpeff.sourceforge.net/ .

Before starting the protocol, it is necessary to download and install SnpEff. To do this, open a Unix, Linux, OS.X or Cygwin shell (terminal) and execute the following commands:

\begin{lstlisting}[language=bash]
cd
curl -v \
    -L http://sourceforge.net/projects/snpeff/files/snpEff_latest_core.zip \
    > snpEff_latest_core.zip unzip snpEff_latest_core.zip
\end{lstlisting}

Notes:
\begin{enumerate}

	\item SnpEff \& SnpSift annotation software used in this protocol are under very active development
	and some command line option may change in the future. The tutorials were developed and tested
	using SnpEff version 3.3. In order to keep up with the latest versions we recommend looking at
	the online documentation for this protocol (http://snpeff.sourceforge.net/protocol.html).
	
	\item The standard installation is to add the package in the \texttt{\$HOME/snpEff} directory (where \texttt{\$HOME} is your home directory). To install SnpEff elsewhere, update the \texttt{data\_dir} parameter in your \texttt{snpEff.config} file, as described in SnpEff's documentation.
\end{enumerate}

Once SnpEff is installed, we will enter the following commands to download the pre-built human database (GRCh37.75) that will be used to annotate our data.

\begin{lstlisting}[language=bash]
cd snpEff
java -jar snpEff.jar download -v GRCh37.75
\end{lstlisting}

A list of pre-built databases for a set of 20,000 other species is available by running the following command: \texttt{java -jar snpEff.jar databases}.

\textbf{Dataset:} In this genomic annotation example, we use a simulated dataset to show how to find genetic variants of a Mendelian recessive disease, Cystic fibrosis, caused by a high impact coding variant, a nonsense mutation in CFTR gene (G542* or, in HGVS notation, p.Gly542*). The data files come from the publicly available "CEPH\_1463" dataset, sequenced by Complete Genomics [http://www.completegenomics.com/public-data/], and contains sequencing information for a family consisting of 4 grandparents, 2 parents and 11 siblings (see Figure \ref{fig:snpeff2}). Although these are healthy individuals, we artificially introduced a known Cystic fibrosis mutation on three siblings (cases) in a manner that was consistent with the underlying haplotype structure. We now download and un-compress the example data used in this protocol, which, for reasons of space and time, is limited to only chromosomes 7 and 17:

\begin{lstlisting}[language=bash]
curl -v -L http://sourceforge.net/projects/snpeff/files/protocols.zip \
              > protocols.zip unzip protocols.zip
\end{lstlisting}

%---
\section{Procedure I: Analysis of coding variants}
%---

The goal in this example is to use SnpEff to find a mutation causing a Mendelian recessive trait. This will be done using a dataset of variant calls for chromosome 7 from a pedigree of 17 healthy individuals, sequenced by Complete Genomics, in which a coding variant causing cystic fibrosis was artificially introduced in three siblings (see Materials). For the purpose of this example, we assume that we do not know the causative variant, but that we know that we are dealing with a Mendelian recessive disorder, where the three siblings are affected (cases), but the 14 parents and grandparents are not (controls).

Genomic variants are usually provided in a VCF file containing variant information of all the samples; storing the variant data in a single VCF file is the standard practice, not only because variant calling algorithms have better accuracy when run on all samples simultaneously, but also because it is much easier to annotate, manipulate and compare individuals when the data is stored and transferred together. A caveat of this approach is that VCF files can become very large when performing experiments with thousands of samples (from several Gigabytes to Terabytes in size). In the following protocol, SnpEff will add annotation fields to each variant record in the input VCF file. We will then use SnpSift, a filtering program to extract the most significant variants having annotations meeting certain criteria.

\paragraph{Step 1. Primary variant annotation and quality control} Our first step is to annotate each of the ~700,000 variants contained in the VCF file. By default, SnpEff adds primary annotations and basic impact assessment for coding and non-coding variants as described above. SnpEff has several command line options that can be used in this annotation stage and which are described in detail in the online manual (http://snpeff.sourceforge.net/SnpEff\_manual.html). In this example, we annotate: i) loss of function and nonsense mediated decay predictions (by adding the \texttt{-lof} command line option); ii) protein domain annotations from the curated NextProt database (option \texttt{-nextProt}); as well as iii)
putative transcription factor binding sites from the ENSEMBL 'Regulatory Build' and Jaspar database (option \texttt{-motif}). We also instruct the program to: i) use HGVS notation for amino acid changes (option \texttt{-hgvs}); and ii) to create a web page summarizing the annotation results in \texttt{ex1.html} (option \texttt{-stats}):

\begin{lstlisting}[language=bash]
java -Xmx4g -jar snpEff.jar \
       -v -lof -motif -hgvs -nextProt \
       GRCh37.75 protocols/ex1.vcf \
       > protocols/ex1.eff.vcf
\end{lstlisting}

SnpEff produces three output files (i) the HTML file containing summary statistics about the variants and their annotations; (ii) an annotated VCF file; and (iii) a text file summarizing the number of variant types per gene. Creation of the summary files can be de-activated to speed up the program (for example, when the application is used together with Galaxy). By default, the statistics file ``ex1.html" is a standard HTML file that can be opened in any web browser to view quality control (QC) metrics. It can also be created in comma-separated values format (CSV) to be used by downstream processing programs as part of an automated pipeline. In our example, the summary file (Figure \ref{fig:snpeff1}) contains basic quality control statistics calculated from the variant file: for our data, the Ts/Ts ratio is close to 2.0 (Figure \ref{fig:snpeff1}c) and missense / silent ratio is around 1.0 (Figure \ref{fig:snpeff1}d), both of which are expected for human data (but these numbers may differ for other species). Large deviations from the expected values for the organism being sequenced might indicate problems with either the sequencing or variant calling pipelines. The summary file also contains QC information for the gene annotation used as input. In this example, 829 warnings (Figure \ref{fig:snpeff1}a) were identified as a result of possible genomic annotation errors or small inconsistencies identified in the reference genome so we have to be careful analyzing those genes/transcripts. Other summary statistics are available, such as variant types (Figure \ref{fig:snpeff1}e), variants effects (Figure \ref{fig:snpeff1}d and \ref{fig:snpeff1}g), and primary impacts (Figure \ref{fig:snpeff1}b and \ref{fig:snpeff1}g).

\begin{figure}
    \centering
    \includegraphics[width=14cm]{snpeff_1.png}
    \caption{Summary file produced by SpnEff. a) Overall summary showing input file, number of variants, command line, errors and warnings; b) number of impact assessed in each category; c) Transitions vs transversions; d) number of missense, nonsense and silent SNPs; e) number of variants by type (SNV, MNP, insertions, deletions and mixed); f) base change matrix; g) number of effects in each category.}
    \label{fig:snpeff1}
\end{figure}

\paragraph{Step 2. Counting variants in case and control subjects} In the first step of our protocol, SnpEff created a VCF file with half million annotated variants. Rather than scanning each annotation manually, we will use the SnpSift program to create a filter that will identify a small subset of variants with interesting functional properties. Since the VCF files used in most sequencing studies are even larger than the one in this example, our overall approach is to start by creating a filter using a very restrictive set of criteria. If no relevant variant is found using this stringent filter, we will relax the criteria to include variants with lower predicted impact.

In our example, since the pedigree is consistent with a Mendelian recessive disease, so we will first use SnpEff to find high impact variants that are homozygous in cases and either absent or heterozygous in controls. This provides a very strong genetic argument to select the promising variants and will be used as the first step in our filter. To do this, we will identify the case and control samples by providing SnpEff with pedigree information using a ``TFAM" file (a standard file format used to describe pedigrees). In our example, the TFAM file (\texttt{pedigree.tfam}) identifies the three cases (NA12879, NA12885, NA12886), and lists the other family members as controls. The \texttt{caseControl} command instructs the SnpSift program to count the number homozygous non-reference, heterozygous and allele count (number of non-reference alleles in each DNA sample) for both cases and controls groups (running time: $~60$ minutes):

\begin{lstlisting}[language=bash]
java -Xmx1g -jar SnpSift.jar \
       caseControl \
       -v -tfam protocols/pedigree.tfam protocols/ex1.eff.vcf \
       > protocols/ex1.eff.cc.vcf
\end{lstlisting}

This analysis creates an output VCF file (\texttt{ex1.eff.cc.vcf}) by adding new information to the \texttt{INFO} field for each variant: this includes information such as \texttt{Cases=1,1,3} and \texttt{Controls=8,6,22}, which correspond to the number of homozygous non-reference, heterozygous and total allele counts in cases and controls for each variant. The program also calculates basic statistics for each variant based on the allele frequencies in the two groups using different models, which can be useful as a starting point for more in-depth statistical analysis.

\paragraph{Step 3. Filtering variants} We can use the \texttt{SnpSift filter} command to reduce the number of candidate loci base on alleles in cases and controls. SnpSift \texttt{filter} allows users to create powerful filters that select variants using Boolean expressions containing data from the VCF fields. The expression we use to filter the VCF file \texttt{ex1.eff.vcf} is developed based on the expectation that all the three cases and none of the controls to be homozygous for the mutation. This is expressed using the following filter: \texttt{(Cases[0] = 3) \& (Controls[0] = 0)}. The full command line is:

\begin{lstlisting}[language=bash]
cat protocols/ex1.eff.cc.vcf \
    | java -jar SnpSift.jar filter \
         "(Cases[0] = 3) & (Controls[0] = 0)" \
    > protocols/ex1.filtered.hom.vcf
\end{lstlisting}

The filtered output file, \texttt{filtered.hom\_cases.vcf}, contains over 400 variants satisfying our criteria. This is still too large to analyze by hand, so can we can add another filter to see if any of these variants is expected to have a high impact. To identify variants where any of these impacts is classified as either \textit{HIGH} or \textit{MODERATE}, we add the condition \texttt{EFF[*].IMPACT = 'HIGH') | (EFF[*].IMPACT = 'MODERATE')}. The new filtering commands become:

\begin{lstlisting}[language=bash]
cat protocols/ex1.eff.cc.vcf \
    | java -jar SnpSift.jar filter \
            "(Cases[0] = 3) & (Controls[0] = 0) & ((EFF[*].IMPACT = 'HIGH') | (EFF[* ].IMPACT = 'MODERATE'))" \
    > protocols/ex1.filtered.vcf
\end{lstlisting}

After filtering, only two variants satisfy our criteria (Figure \ref{fig:snpefffilteredvars}), one of them is a \textit{STOP\_GAINED} loss of function variant, whereas the other one is a \textit{NON\_SYNONYMOUS} amino acid change. The first one is a known Cystic fibrosis variant.

\begin{figure}
    \centering
    \includegraphics[width=14cm]{snpeff_table_variants_filtered.png}
    \caption{Candidate variants remaining after filtering.}
    \label{fig:snpefffilteredvars}
\end{figure}

A chart showing how the variant propagates across the pedigree structure (Figure \ref{fig:snpeff2}) can be created as follows:

\begin{lstlisting}[language=bash]
java -jar SnpSift.jar pedShow \
       protocols/pedigree.tfam protocols/ex1.filtered.vcf protocols/chart
\end{lstlisting}

\begin{figure}
    \centering
    \includegraphics[width=8cm]{snpeff_2.png}
    \caption{Pedigree used for Procedure 1. A pedigree, created using the "SnpEff pedShow" command, shows heterozygous individuals in green, and homozygous alternative in red. It shows how the cystic fibrosis variant gets propagated from heterozygous unaffected grandparents and parents to homozygous affected offspring.}
    \label{fig:snpeff2}
\end{figure}

\paragraph{Step 4. Using clinical databases.} So far, since the purpose of the example was to show how annotations and filtering are performed to uncover new variants, we assumed that the causative variant was not known. In reality the variant is known and databases, such as ClinVar, have this information in convenient VCF format that can be used for annotations. We can annotate using ClinVar by using the following command:

\begin{lstlisting}[language=bash]
java -Xmx1g -jar SnpSift.jar annotate \
       -v protocols/db/clinvar_00-latest.vcf protocols/ex1.eff.cc.vcf \
       > protocols/ex1.eff.cc.clinvar.vcf
\end{lstlisting}

Our variant of interest is then annotated as \textit{``Cystic Fibrosis"} (to find the variant, we filter for variants having ClinVar annotation \textit{``CLNDBN"} that are in CFTR gene and have a \textit{STOP\_CODON} annotation):

\begin{lstlisting}[language=bash]
cat protocols/ex1.eff.cc.clinvar.vcf | java -jar SnpSift.jar filter \
     "(exists CLNDBN) & (EFF[*].EFFECT = 'STOP_GAINED') & (EFF[*].GENE = 'CF TR')"
\end{lstlisting}

\paragraph{Software Integration (Optional)} Sequence analysis software is often run in high performance computers combining several programs into processing pipelines. Annotations and impact assessment software needs to provide integration points with other analysis steps of the pipeline. In the following paragraphs we describe how to integrate SnpEff with two programs commonly used in sequencing analysis pipelines: i) Genome Analysis toolkit (GATK 2), a command-line driven software; and ii) Galaxy 3, a web based software.

\paragraph{GATK} The Genome Analysis Toolkit 2 is one of the most popular programs for bioinformatics pipelines. Annotations can be easily integrated into GATK using SnpEff and GATK’s VariantAnnotator module. Here we show how to annotate a file using SnpEff and GATK, as an alternative way of performing step 1. You should perform this step only if your processing pipeline is based on GATK: compared to running SnpEff from the command line, the results obtained when using GATK will only contain the highest impact annotation for each variant. This was a conscious trade-off made by the designers of GATK, partly because most biologists do this implicitly when reading a list of variants, but also to improve the readability and reduce the size of the annotation results.

The method requires two steps: i) Annotating a VCF file using SnpEff and ii) using GATK’s VariantAnnotator to incorporate those annotations into the final VCF file. When using SnpEff for GATK compatibility, we must use the \texttt{-o gatk} command line option:

\begin{lstlisting}[language=bash]
java -Xmx4g -jar snpEff.jar -v \
       -o gatk GRCh37.75 protocols/ex1.vcf \
       > protocols/ex1.eff.gatk.vcf
\end{lstlisting}

Next, we process these variants using GATK. For this step to work correctly, we need to make sure that our data files are compatible with the requirements GATK places on reference genomes (see GATK’s documentation for more details): (i) in the fasta file, chromosomes are expected to be sorted in karyotypic order; (ii) a genome fasta-index file must be available; and (iii) a dictionary file must be pre-computed. Assuming these requirements are satisfied, we can run the following command, which will produce a GATK annotated file (\texttt{ex1.gatk.vcf}):

\begin{lstlisting}[language=bash]
java -Xmx4g -jar $HOME/tools/gatk/GenomeAnalysisTK.jar \
       -T VariantAnnotator \
       -R $HOME/genomes/GRCh37.75.fa \
       -A SnpEff \
       --variant protocols/ex1.vcf \
       --snpEffFile protocols/ex1.eff.gatk.vcf \
       -L protocols/ex1.vcf \
       -o protocols/ex1.gatk.vcf
\end{lstlisting}

Note: We assumed GATK is installed in \texttt{\$HOME/tools/gatk/} and the reference genome is contained in \texttt{\$HOME/genomes/GRCh37.75.fa}. These file locations should be adapted to the actual path in your computer.

\paragraph{Galaxy} Another popular tool in bioinformatics is Galaxy 3, which allows pipelines to be created in a web environment using graphical interface, making it flexible and straightforward to use. SnpEff provides Galaxy modules (see http://snpeff.sourceforge.net/SnpEff\_manual.html\#galaxy for details). Once these modules are installed, we can run our sample annotation pipeline in Galaxy (Figure \ref{fig:snpeff3}). A step-by-step tutorial can be found at http://snpeff.sourceforge.net/protocol

\begin{figure}
    \centering
    \includegraphics[width=14cm]{snpeff_3.png}
    \caption{Use of Galaxy combined with SnpEff. (a) Galaxy Protocol for example 1. Galaxy automatically creates a pipeline showing the processing steps, which can be modified and later used to process new datasets; (b) Results after the last step was executed by Galaxy server.}
    \label{fig:snpeff3}
\end{figure}

%---
\section{Procedure II: Analysis of non-coding variants}
%---

This example shows how to perform basic annotation of non-coding variants. It is based on a short list of 20 non-coding that were identified by sequencing a 700 kb region surrounding the gene T-box transcription factor (TBX5) in 260 patients with congenital heart disease 67. TBX5 is a transcription factor that plays a well-established dosage-dependent role in heart and limb development. Coding mutations in TBX5 have been frequently identified in patients with Holt–Oram syndrome, which is associated with abnormal hand, forearm and cardiac development.

\paragraph{Step 1. Annotating variants} We will perform non-coding variant annotation using SnpEff following a similar approach to Procedure I. In this case, we construct a command line that instructs SnpEff to include motif information (\texttt{-motif}) and putative transcription factor binding sites (TFBS) identified in the ENSEMBL Regulatory Build and the Jaspar database:

\begin{lstlisting}[language=bash]
java -Xmx4g -jar snpEff.jar -v \
       -motif GRCh37.75 protocols/ex2.vcf \
       > protocols/ex2.eff.basic.vcf
\end{lstlisting}

\paragraph{Step 2. Adding custom regulatory information} A quick scan through the results shows that most variants are catalogued as \textit{INTERGENIC}, and none of them is associated with a known TFBS. This is not surprising since TFBS are small and also because regulatory elements involved in cardiac or limb development may not be widely active in commonly studied adult tissues. In this case, basic annotations did not provide additional information that can be used to narrow down the list of candidate SNVs. To solve this, the authors examined data from other sources, including ChIP-seq data for H3K4me1 (a post-translationally modified histone protein found in transcriptionally active genome regions, including enhancers and promoters). Data produced from ChIP-Seq analysis are frequently published in BED, BigBed or similar formats, which can be used directly by SnpEff by adding the ``-interval" command line option. This command line option can be used to add annotations using ChIP-Seq experiments from the ENCODE and Epigenome Roadmap projects: since multiple ``-interval" options are allowed in each command line, it is a simple way to combine several annotations:

\begin{lstlisting}[language=bash]
java -Xmx4g -jar snpEff.jar -v \
       -motif -interval protocols/ex2_regulatory.bed GRCh37.75 \
       protocols/ex2.vcf \
       > protocols/ex2.eff.vcf
\end{lstlisting}

In the output VCF file, variants intersecting genomic regions from the \texttt{-interval} command line option are annotated as \texttt{CUSTOM[ex2\_regulatory]} : the name in brackets identifies the file name provided to distinguish multiple annotation files.

\paragraph{Step 3. Adding conservation information} In order to refine our search, we can also look for variants in highly conserved non-coding bases. SnpEff natively supports PhastCons scores, but can also add annotations based on any other user-defined score provided as a Wig or VCF file. The command line for annotating using the PhastCons score is:

\begin{lstlisting}[language=bash]
java -Xmx1g -jar SnpSift.jar \
      phastCons -v protocols/phastcons protocols/ex2.eff.vcf \
      > protocols/ex2.eff.cons.vcf
\end{lstlisting}

Now we can filter our results looking for a highly conserved SNP in the regulatory region. We do this by using a \texttt{SnpSift filter} command and the appropriate Boolean expression:

\begin{lstlisting}[language=bash]
cat protocols/ex2.eff.cons.vcf \
    | java -jar SnpSift.jar filter \
            "(EFF[*].EFFECT = 'CUSTOM[ex2_regulatory]') & (exists PhastCons) & (PhastC ons > 0.9)" \
    > protocols/ex2.filtered.vcf
\end{lstlisting}

SnpSift \texttt{filter} supports a flexible syntax to create Boolean expressions using the annotation data that provides a versatile way to prioritize shorter lists of SNPs for subsequent validation. This syntax is described in detail in the online manual (http://snpeff.sourceforge.net/SnpSift.html\#filter). In this example, our filter results in only two candidate SNPs, one of which was extensively validated in the original study and is assumed to be causative 67.

The principles illustrated in our example for a small set of SNVs can be applied to millions of variants from whole genome sequencing experiments. Similarly, although we filtered the SNVs using \texttt{custom} ChIP-seq data that provided in the original study 67, regulatory information from public Encode or Epigenome Roadmap datasets could be used in a first line investigation before generating our own Chip- seq or RNA-seq data using disease-relevant cells and tissues.

\section{Troubleshooting}

Complete documentation for SnpEff, together with troubleshooting instructions are available at http://snpeff.sourceforge.net/SnpEff\_manual.html.

\section{Acknowledgements}

We thank Louis Létourneau and Adrian Platts, for their valuable suggestions for improving SnpEff. This work was funded by the National Institute of Diabetes \& Digestive \& Kidney diseases (NIDDK 5-U01- DK085545-02) and the Canadian Institutes for Health Research (CIHR MOP-102703).

	
%-----------------------------------------------------------------------------
\chapter{Epistatic GWAS analysis\label{ch:gwas}}
%-----------------------------------------------------------------------------

%---
\section{Preface}
%---


In recent years, over 80 genetic loci related to T2D have been identified \cite{morris2012large, consortium2014genome}. Nevertheless, the overall effect sizes of these loci account for less than 10\% of the overall disease predisposition \cite{manolio2009finding}. This poses the question of why, given that so much efforts has been directed at finding the genetic components of this disease, the loci found so far have such modest effects. This lack of large genetic effects, known as the ``missing heritability" problem, does not only arise in T2D but also in almost all complex traits. In recent studies about missing heritability \cite{zuk2012mystery, zuk2014searching} it was proposed that this effect might be partly explained by taking into account epistasis (i.e. gene interactions).

In this chapter, we propose a novel framework that takes into account putative epistatic interactions into genome wide association studies (GWAS). 

%We detect putative interactions by means of a co-evolutionary model that we develop based on Markov evolutionary theory. We fit the co-evolutionary model's parameters using protein structural data (PDB) an genome wide multiple sequence alignments (UCSC's 100-way). Once this model is parametrized, we perform genome wide association using of pairs of variants  from a sequencing study by incorporating the co-evolutionary model as priors in a Bayesian Framework.

Although this thesis focusses on the development of computational approaches that could be applied to the study of a number of complex diseases, our focus has been on type II diabetes mellitus (T2D), a complex disease first described by the Egyptians in 1500 BCE. Later the Greeks in 230 BCE used the term ``diabetes" meaning ``pass through" (or ``siphon") denoting the constant thirst and frequent urination of the patients. In the 1700s the term ``mellitus" (from honey) was added to denote that the urine was sweet and would ``attracts ants".

Diabetes symptoms include frequent urination, thirst, and constant hunger, high blood sugar (hyperglycemia) and insulin resistance. Long term complication from T2D may include eyesight problems, heart disease, strokes and kidney failure. Type II diabetes, is highly correlated with obesity and disease rate has increased dramatically during the last 50 years. According to the World Health Organisation the prevalence of diabetes is 9\% in adults and an estimated 1.5 millions deaths were caused by diabetes in 2012 \cite{guariguata2014global}, which is predicted to be the 7th leading cause of death by 2030. The costs associated to treating diabetes patients only in the U.S. are estimated around \$245 billion dollars.

The rest of the chapter is published in: \textbf{P. Cingolani}, R. Sladek, M. Blanchette, ``A co-evolutionary approach for detecting epistatic interactions in genome-wide association studies"

%---
\section{Abstract}
%---

\paragraph{Motivation} Epistasis, broadly defined as genetic interactions, is one of the likely causes why genome-wide association studies (GWAS) account for a small portion of heritable disease risk. Due to their high complexity, reduced statistical power and sometimes prohibitive computational requirements, epistatic GWAS have rarely been performed. 

\paragraph{Results} In this paper, we propose a novel methodology for analysing putative epistatic interactions by combining multiple genome alignments and sequencing information. Using protein structures for individual and co-crystallized complexes information and genome wide multiple species alignment we create a co-evolutionary model that allows the calculation of the posterior probability of physical interaction between residues given evolutionary data. These probabilities are then used as the interaction priors for an epistatic GWAS analysis as basis for genome wide Bayesian framework. 

\paragraph{Results} Our optimized algorithms can be applied to genome wide scale sequencing studies for tens of thousands of samples, that typically yield millions of variants. We applied our approach to a large type II diabetes (T2D) case-control cohort and inferred a number of putative interactions associated with increased risk of developing T2D. 

\paragraph{Availability} Our code is publicly available at \texttt{github.com/pcingola/Epistasis}

%---
\section{Introduction}
%---

Genetic studies aim to discover how a phenotype of interest, such as disease risk or height, is affected by in individual's genetic background. Genome wide association studies (GWAS) are powerful techniques aimed at finding statistical associations between a phenotype and genetic variants \cite{clarke2011basic}. Although several genetic variants related to different phenotypes have been found, variants discovered in GWAS so far can only explain a small part for the phenotypic heritability. For instance, all genetic variants associated to height collectively account for few centimetres in the offspring's height \cite{wood2014defining}. Similarly the known variants related to type 2 diabetes risk collectively explain only 5\% to 10\% of the overall variance in disease predisposition \cite{morris2012large, consortium2014genome}. This problem is known as ``missing heritability" \cite{manolio2009finding} and recent theories suggest that genetic interactions (epistasis) might play an important role in it \cite{zuk2012mystery, zuk2014searching}.

The foundations for epistasis \cite{gao2010classification}, have been proposed almost a hundred years ago by Bateson (1909) and Fisher (1918). It was the latter who coined the term to denote a ``statistical deviation of multilocus genotype values from an additive linear model for the value of a phenotype" \cite{gao2010classification}. There is evidence of such interactions being involved in complex diseases. For instance an interaction between BACE1 and APOE4 having a significant association with Alzheimer's disease has consistently been replicated in different studies \cite{combarros2009epistasis}. Many types of situations can lead to epistatic interactions. Among them, perhaps the most common involved pairs of variants that encode amino acids whose physical interactions is regulated for their function.

One of the main problems in finding association between interactions and disease is that out of the whole set of molecular interactions (the interactome) only a small part of it has been characterized \cite{venkatesan2009empirical}. Interacting proteins can be identified experimentally through several types of approaches (yeast two hybrid, protein fragment complementation assay, glutathione-s-transferase, affinity purification coupled to mass spectrometry, tandem affinity purification, etc. \cite{shoemaker2007deciphering}) and large databases of protein-protein interactions are now available for human \cite{stark2006biogrid, shoemaker2007deciphering}. In almost all cases, these methods identify the presence of an interaction between proteins but do not discern the exact residues mediating such interactions. Furthermore, it is estimated that up to 80\% of the human protein-protein interactions remains unknown \cite{venkatesan2009empirical}.

These issues can be partially addressed using computational predictions of either pairs of interacting proteins or interacting residues \cite{shoemaker2007decipheringP2}. A type of approaches that has been gaining popularity recently is one that makes use of the plethora of genomic sequences available for species other than human in order to discover evolutionary evidence of selective pressure on pairs of residues to identify interacting sites and interfaces \cite{marks2012protein}. Interacting residues and their neighbours may then be subject to compensating epistasis, where a mutation at a residue in one protein may be compensated by another mutation at a residue in the second protein \cite{pazos1997correlated}.For example assuming that evolutionary pressure acts on both interaction sites simultaneously, co-occurring compensatory mutations can become fixed in the population with higher probability than non-compensatory ones. In light of this hypothesis, one can use statistical methods on multiple sequence alignments of proteins from different organisms to find coevolving sites. This types of approaches has been used to identify coevolving sites both within a protein (e.g. N-terminal and C-terminal domains in PKG protein \cite{goh2000co}, GroES-L chaperoning system \cite{ruiz2013coevolution}, $\alpha$ and $\beta$ haemoglobin subunits \cite{pazos1997correlated}), and between interacting proteins (e.g. G-protein coupled receptors and protein ligands \cite{goh2000co}).

Many methods exist to find putative interaction loci, both within and across proteins, based on evolutionary evidence (see \cite{de2013emerging} for a review). One of the simplest methods for inferring co-evolution uses mutual information between two loci \cite{marks2012protein} in a multiple sequence alignment. However, methods based on correlation or mutual information are known to have systematic biases due to the fact that they ignore phylogenetic relationships \cite{de2013emerging}, or sequence heterogeneity problems \cite{weigt2009identification}. More sophisticated methods, such as DCA \cite{morcos2011direct}, PSICOV \cite{jones2012psicov} or mdMI \cite{clark2014multidimensional} try to overcome these biases, however they are usually not suitable for GWAS-scale analysis for two main reasons. First, they require multiple alignments of a very large number of sequences (ranging from $400$ to $25L$, where $L$ is the length of the protein \cite{clark2014multidimensional}), and such depth is not usually available at whole genome scale. Second, they are computationally demanding (e.g. running for minutes or even days for each interacting pair of proteins being considered), making them unsuitable for analyses involving millions of variants spanning over thousands of proteins. Furthermore, a recent study shows that overall agreement between methods is not high (65\% or less) and predictive power is quite low (only 6\% of the ``top scoring pairs" are real interactions) \cite{clark2014multidimensional}.

Applying epistatic interaction models to GWAS studies is a challenging problem for several reasons: i) interaction models are by definition non-linear \cite{gao2010classification}; ii) analyzing all order $N$ variant combinations requires great computational power and efficient algorithms because the number tests grows exponentially with $N$ \cite{phillips2008epistasis}; iii) multiple hypothesis testing correction can render association tests underpowered for all but very large cohorts \cite{gao2010classification, phillips2008epistasis}; and iv) there is no consensus of what genetic interaction means, which is reflected in the difficulty to find a unified model \cite{phillips2008epistasis,mani2008defining}. For all these reasons and due to the lack of sequencing cohorts large enough to detect these interactions, the application of epistatic models to sequencing studies has not been widespread. Furthermore, there is no clear consensus on the required sample size to detect epistatic interactions. Depending on phenotypic effect size and variant's allele frequency some estimates assume in the order of 10,000 to 500,000 cases \cite{jostins2013using} to be required. Such cohorts are now becoming feasible due to improvements and cost reductions in sequencing technology.

Approaches for epistatic GWAS do exist and they apply a wide array of methodologies. In \cite{zhao2006test}, the authors infer epistatic probabilities by noting that interactions create linkage disequilibrium patterns in the disease population. A Bayesian framework is applied in \cite{zhang2007bayesian} taking into account several risk models, using Dirichlet priors the distribution for each model can be solved analytically, then the combined model's posterior distribution is calculated using an MCMC sampling technique. In \cite{ackermann2012systematic}, the authors look for over / under-represented allele pairs in a given population by performing an analysis of imbalanced allele pair frequencies. Finally, finding interacting variants can be viewed as an attribute selection problem, thus many machine learning methodologies have been proposed \cite{mckinney2006machine}. While all algorithms have relative advantages, there is no standard in epistatic analysis, we believe that we can create better methods by combining other sources of biological information, such as evolutionary evidence.

In this work we propose an approach to prioritize pairs of variants identified in case/control cohorts by combining genome wide association with epistatic interaction models. In a nutshell, our method uses recently computed 100-way vertebrate genome alignments \cite{blanchette2004aligning} to calculate interaction posterior probabilities for any given pair of residues in human proteins. This is achieved by contrasting the likelihood of the observed pair of alignment columns under a joint substitution model that factors in dependencies between interacting sites, and a null model of independent evolution.  These posterior probabilities are then used as priors to modulate the evidence of epistatic interaction derived from GWAS data. Our implementation is sufficiently efficient to be applied to GWAS-scale datasets of tens of thousands of samples. Finally we apply this methods to a cohort of $\sim 13,000$ individuals in a case-control study of type II diabetes (for study details, see \cite{mccarthy2015T2D}) and identify suggestive associations of putatively epistatic interactions.

%---
\section{Methods \label{sec:gwasMeth}}
%---

Our epistatic GWAS analysis pipeline involves three key steps, as shown in Figure \ref{fig:gwaspipeline}. First, we learn a co-evolutionary substitution rate matrix for pairs of amino acids that are in contact in proteins. Second, we analyze a GWAS data set to identify pairs of non-synonymous SNPs that show (possibly weak) evidence of epistasis. Third, for each pair of SNP identified in step 2, we measure the evidence of co-evolution of the pair of encoded amino acids, and combine it with the GWAS evidence by adding up the corresponding Bayes factors.

\fig{gwas_epistasis_pipeline}{gwaspipeline}{14cm}{Complete pipeline example}{Complete pipeline example}

\subsection{Substitution model for pairs of interacting amino acids \label{sec:gwasQ2}}

In this section, we describe how we estimate two substitution rate matrices. The first is the usual $20 \times 20$ substitution rate matrix $Q$ describing the evolution of individual amino acids. The second, $Q_2$, is a $400 \times 400$ substitution rate matrix for pairs of interacting residues. 

We used the 100-way vertebrate multiple sequence alignment and accompanying phylogenetic tree $T$ available from the UCSC Genome Browser \cite{karolchik2014ucsc}. This alignment includes the DNA sequences of 100 species whose genome is completely or nearly completely sequenced, with 12 primates, 44 non-primates eutherians, 5 marsupials, 14 birds, 6 reptiles, 16 ray-finned fish and 8 lobe-fined fish.
%6 Afrotheria, 14 Avians, 14 Euarchontoglires, 16 Fish, 25 Laurasiatheria, 5 Mammalians, 12 Primates and 8 Sarcopterygii. 
The multiple alignment is performed using ``multiz" algorithm \cite{blanchette2004aligning,kielbasa2011adaptive}.

From the $\sim 21,000$ human protein structures (resolution less than $3$ \AA) available in Protein Data Bank, we extracted a set of $\sim770,000$ pairs of ``within protein interactions" residues, defined as pairs of residues from the same protein where at least one pair of atoms is within $3 \AA$  or less. Similarly, from the set of $\sim5,700$ models of co-crystallized complexes in PDB, we extracted a set of $\sim12,000$ pairs of ``protein-protein interacting" residues, defined as amino acids from different proteins that satisfy the same distance criterion.

To derive rate matrix $Q$, we consider the complete set of $n \sim 22 \times 10^6M$ protein coding sites present in the alignment, irrespective of the presence or absence of contacts. $Q$ is obtained following classical sequence evolution theory (\cite{yang2006computational, felsenstein2004inferring}). First, for each pair of species  $s_i$ and $s_j$, we obtain $c_i(a)$ defined as the count of amino acid $a$ in species $s_i$, and $c_{i,j}(a,b)$ defined as the number of sites that have had a transition from amino acid $a$ in $s_i$ to $b$ in $s_j$. Stationary probability of amino acid $a$ in genome $s_i$ is then defined as $\pi_i = c_i(a)/n$. Assuming a time reversible model, we get the frequency of change from $a$ to $b$: $f_{i,j}(a,b) = f_{j,i}(a,b) = (c_{i,j}(a,b) + c_{j,i}(a,b))/(2n)$. Let $P_{i,j}$ be the amino acid transition probability matrix from $s_i$ to $s_j$, i.e. $P_{i,j}(a,b)$ is the probability that species $s_j$ has amino acid $b$ given that species $s_i$ has amino acid $a$. Then $P_{i,j}$ is obtained through the relation $f_{i,j}(a,b) = \pi_i(a) \cdot P_{i,j}(a,b)$, or $P_{i,j}(a,b) =   f_{i,j}(a,b) /  \pi_i(a)$.  Let $t_{i,j}$ be the total branch length between $s_i$ and $s_j$ (obtained from the phylogenetic tree). Assuming time reversibility, we have $P_{i,j} = e^{Q \cdot t_{i,j}}$, and thus $Q=log[ P_{i,j} / t_{i,j} ]$ \cite{yang2006computational}. Taking into account the estimation error, the equation becomes $\hat{Q}(t_i+t_j) = Q = log[ P_{i,j} / t_{i,j} ] + \epsilon_{i,j}$, where $\epsilon_{i,j}$ is an error matrix. Under the assumption that the mean error is zero, we can approximate the rate matrix by the calculating an average of all estimates:

\begin{eqnarray*}
	\hat{Q} & = & \frac{1}{N(N-1)/2} \sum_{i < j} \hat{Q}_(t_i+t_j) \\
	            & =  & \frac{2}{N(N-1)} \sum_{i<j} \frac{1}{t_i+t_j} log[ \hat{P}_(t_i+t_j) ]
\end{eqnarray*}

The much larger substitution matrix $Q_2$ describes the substitution rate from any pair of amino acids $(a,b)$ to any other pair $(c,d)$. It is derived similarly to $Q$, but considering only pairs of amino acids from the set of within protein interacting pairs of amino acids. We only take into account amino acids pairs within the same chain, that are separated by 20 amino acids or more. 

\subsection{Calculating likelihood of individual and pairs of alignment columns}

Given a substitution rate matrix $Q$, the likelihood $L_1(i)$ of an alignment column $i$ assigning an amino acid to each leaf in the tree $T$ is calculated using the well known Felsenstein algorithm \cite{felsenstein2004inferring}. This is achieved in time $O(N \cdot |\Sigma|^2)$, where $|\Sigma|=20$. Given matrix $Q_2$, the same algorithm can be used to compute the likelihood $L_2(i,j)$ of a pair of alignment columns $(i,j)$, but now in time $O(N \cdot |\Sigma|^4)$. 

A test for co-evolution of two positions $i$, $j$ of the same or different proteins is obtained using the likelihood ratio under the two models: 

\begin{eqnarray*}
	LR[{M_{SA}}(i,j)] = \frac{P(i, j | M_{SA}, Q_2)}{P(a_i | M_{SA}, Q) \cdot P(a_j | M_{SA}, Q)}
\end{eqnarray*}

where the denominator assumes that the amino acids $i$ and $j$ evolve independently. Similarly, the log-likelihood is defined as

\begin{eqnarray}
	LL[M_{SA}(i,j)] = log \left[ \frac{P(i, j | M_{SA}, Q_2)}{P(a_i | M_{SA}, Q) \cdot P(a_j | M_{SA}, Q)} \right]
\end{eqnarray}


%Given a pair of columns in the multiple sequence alignment $M_{sa}$, the corresponding phylegentic tree $\mathcal{T}$, we calculate the likelihood ratio by propagating the probabilities from the leaves up the root of the phylogenetic tree using Felsestein's algorithm \cite{felsenstein2004inferring} using transition matrices $Q$ and $Q_2$ for the null and alternative models respectively.

Because the calculations described in this section will need to be performed on a very large number of pairs of sites, optimizations we are required to ensure manageable running time. First, pre-calculation of  matrix exponentials $P(t) = e^{Qt}$ is necessary for all values of $t$ corresponding to individual branch lengths. Another optimization (``constant-tree caching") is used to cache likelihood values for subtrees of the phylogenetic tree where all nodes have the same amino acid values. This optimization results in speed-up only if the phylogenetic tree remains constant throughout the genome, which is the case in our model.

\subsection{GWAS model}

Consider a GWAS with $N_S$  samples (individuals) and $N_V$ variants, we use the standard notation for phenotypes and code them as $d_s=1$ when individual $s$ is affected by disease and $d_s=0$ if it is "healthy". Let $\bar{d} = [d_1, ..., d_{N_s}]$ be a phenotype vector and $g_{s,i} \in \{0,1,2\}$ a genomic variant for sample $s$ at locus $i$. A logistic model of disease risk \cite{balding2006tutorial} is

\begin{eqnarray*}
    p_{s,i} & = & P( d_s | g_{s,i} ) \\
    & = & \phi( \theta_0 + \theta_1 g_{s,i} + \theta_2 c_{s,1} + \theta_4 c_{s,2} + ... ) \\
    & = & \frac{1}{1 + e^{\theta_0 + \theta_1 g_{s,i} + \theta_2 c_{s,1} + \theta_4 c_{s,2} + ...}} \\
    & = & \phi( \bar{\theta}^T \bar{g}_{s,i})
\end{eqnarray*}

, where $\phi(\cdot)$ is the sigmoid function, $c_{s,1}, c_{s,2}, ... $ are covariates for each individual $s$ (these covariates usually include sex, age and eigenvalues from population structure analysis \cite{price2006principal}), $\bar{g}_{s,i} = [ 1, g_{s,i} , c_{s,1}, c_{s,2}, ... , c_{s,N_C} ]$, and $\bar{\theta} = [\theta_1, \theta_2, ..., \theta_m] $. The parameter estimates $\bar{\theta}$ are obtained by solving the maximum likelihood equation

\begin{eqnarray*}
    L( \bar{\theta} ) & = & \prod_{s=1}^{N_S}{ P( d_s | \bar{\theta}, g_{s,i} ) } \\
    & = & \prod_{s=1}^{N_S}{ p_{s,i}^{d_s} (1-p_{s,i})^{1-d_s} } \\
    & = & \prod_{s=1}^{N_S}{ \phi( \bar{\theta}^T \bar{g}_{s,i})^{d_s} (1-\phi( \bar{\theta}^T \bar{g}_{s,i}))^{1-d_s} }
\end{eqnarray*}

% where $p_{s,i} = P( d_s | \bar{\theta}, g_{s,i} )$ is the probability of individual $s$ disease outcome, given a genomic variant.

Using this model, we have two hypotheses: i) the null hypothesis, $H_0$, assumes that genotype does not influence disease probability (i.e. $\theta_1 = 0$). ii) the alternate hypothesis, $H_1$, assumes that the genotype does influence disease probability (i.e. $\theta_1 \neq 0$). We can compare these two hypotheses using a likelihood ratio test. We define

\begin{eqnarray} \label{eq:gwasLogLikLogReg}
	LR & = & \frac{L( \bar{\theta}_{alt} | H_1 ) }{ L( \bar{\theta}_{null} | H_0 ) }\\
	LL & = & log \left[ LR \right] = log \left[ \frac{L( \bar{\theta}_{alt} | H_1 ) }{ L( \bar{\theta}_{null} | H_0 ) } \right]
\end{eqnarray}

, where $\bar{\theta}_{null}$ and $\bar{\theta}_{alt}$ are the maximum likelihood estimates for null and alternate model respectively. According to Wilk's theorem \cite{wilks1938large}, the log likelihood ratio has a $\chi^2_1$ distribution under the null hypothesis, so we can easily calculate a p-value.

Next, we extend the logistic model to accommodate interacting loci. For an individual (sample $s$), we model interactions between two genetic loci $i$ and $j$, having genotypes $g_{s,i}$ and $g_{s,j}$, by extending the logistic model

\begin{eqnarray} \label{eq:gwasLogRegH1}
    P( d_s | g_{s,i},g_{s,j}, H_1) & = & \phi[ \theta_0 + \theta_1 g_{s,i} + \theta_2 g_{s,j} + \theta_3 (g_{s,i} g_{s,j}) \\
    & & ... + \theta_4 c_{s,1} + ... + \theta_m c_{s,N_{cov}} ] \\
    & = & \phi( \bar{\theta}^T \bar{g}_{s,i,j}) )
\end{eqnarray}

where $\bar{g}_{s,i,j} =  [1, g_{s,i}, g_{s,j}, ( g_{s,i} g_{s,j}), c_{s,1}, c_{s,2}, ..., c_{s,N_{cov}} ]^T$. An implicit assumption in this equation is that $g_{s,i}$ and $g_{s,j}$ are not correlated (e.g. they are not located in the same LD-Block). This can be enforced either by using haplotype structure information (e.g. from HapMap) or by limiting the application of the model to variants either in different chromosomes or sufficiently distant (say $> 1Mb$). The null hypothesis $H_0$ assumes that variants act independently

\begin{eqnarray} \label{eq:gwasLogRegH0}
    P( d_s | g_{s,i},g_{s,j}, H_0) & = & \phi[ \theta_0' + \theta_1 g_{s,i} + \theta_2 g_{s,j} + \theta_3 c_{s,1} + ... ] \\
   & = & \phi( \bar{\theta}^T \bar{g}_{s,i,j}' )
\end{eqnarray}

where $\bar{g}_{s,i,j}' =  [1, g_{s,i}, g_{s,j}, c_{s,1} , c_{s,2}, ..., c_{s,N_{cov}} ]^T$.

We investigated several algorithms for logistic regression parameter fitting. The fastest convergence is obtained using Iterative Reweighted Least Squares (IRWLS \cite{daubechies2010iteratively}) and Broyden-Fletcher-Goldfarb-Shanno algorithm (BFGS \cite{broyden1970convergence}) with some code optimizations. In most cases, IRWLS converges faster, so it was selected as the default implementation in our analysis.

Another way to compare the null hypothesis to the alternative hypothesis, is using a Bayesian formulation \cite{kass1995bayes, wakefield2009bayes}

\begin{eqnarray*}
	P(H_1 | \mathcal{D}) & = & \frac{ P( \mathcal{D} | H_1) P(H_1) }{ P(\mathcal{D}) } = \frac{ \int{ P(\mathcal{D} | \theta , H_1) P( \theta | H_1)  d\theta } }{ P(\mathcal{D}) }  \\
	\Rightarrow  \frac{ P(H_1 | D)  }{ P(H_0 | D)  } & = & \frac{ \int{ P(\mathcal{D} | \theta , H_1) P( \theta | H_1)  d\theta } }{\int{ P(\mathcal{D} | \theta , H_0 ) P( \theta | H_0)  d\theta } } \frac{ P(H_1) }{ P(H_0)  }  
	=  B_F \frac{ P(H_1) }{ P(H_0)  }
\end{eqnarray*}

\noindent where $B_F$, the ratio of the two integrals, is the Bayes factor. Using a Bayesian formulation has two main advantages: i) the hypothesis are automatically corrected for model complexity since Bayes factor asymptotically converge to Bayesian Information Criteria (BIC) \cite{kass1995bayes}, and ii) we can compare non-nested models. The Bayes factor for the epistatic model becomes:

\begin{eqnarray}\label{eq:bf2}
	B_F = \frac
	{ \int{ \prod_{s=1}^{N_S}{ \phi( \bar{\theta}^T \bar{g}_s)^{d_s} [ 1-\phi( \bar{\theta}^T \bar{g}_s) ]^{1-d_s} } P( \bar{\theta} | H_1)  d\theta } }
	{ \iint{ \prod_{s=1}^{N_S}{ 
	\phi( \bar{\theta}^T \bar{g}_{s,i}) 
	\phi( \bar{\theta'}^T \bar{g}_{s,j} ) )^{d_s} 
	[ 1-\phi( \bar{\theta}^T \bar{g}_{s,i}) \phi( \bar{\theta'}^T \bar{g}_{s,j} ) ]^{1-d_s} } 
	P( \bar{\theta} | H_0) 
	P( \bar{\theta}' | H_0)  
	d\theta d\theta' } }
\end{eqnarray}

Calculating Bayes factors is challenging and most of the times there are no closed form equations. Calculating the integrals using numerical algorithms is possible, but  imposes a significant computational burden thus making it impractical for large datasets, such as GWAS data, even using large computing clusters. We can approximate the integrals using Laplace's method  \cite{kass1995bayes}. If $g(x)$ has a maximum at $x_0$, it can be shown that

\begin{eqnarray*}
	\int{e^{-\lambda g(x)} h(x) dx} & \simeq & h(x_0) e^{\lambda g(x_0)} \sqrt{\frac{2 \pi}{\lambda g''(x_0)}} \\
\end{eqnarray*}

The multivariate case, for $\bar{x} \in \Re^d$, is analogous: we just need a Hessian matrix instead of a second derivate of $g(\cdot)$

\begin{eqnarray}\label{eq:laplace}
	\int{e^{\lambda g(\bar{x})} h(\bar{x}) d\bar{x}} & \simeq & h(\bar{x}_0) e^{\lambda g(\bar{x}_0)} 
	\left( \frac{2 \pi}{\lambda} \right)^{d/2} \left[ \frac{\partial^2 g(\bar{x}) }{\partial \bar{x} \partial \bar{x}^T} \right] ^{-1/2}
\end{eqnarray}

Using equation \ref{eq:laplace} we can try to approximate the complex integrals in equation \ref{eq:bf2} by the transformation $L(\bar{\theta}) = e^{\ell(\bar{\theta})}$, where $\ell(\cdot)$ is the log-likelihood of the data. So, we can use Laplace approximation by using Eq.\ref{eq:laplace}, at the point of the maximum likelihood. In order to do so, we need to calculate the Hessian matrix in Eq.\ref{eq:laplace}. Fortunately, for logistic models, we can make a few simplifications. Considering that $L(\bar{\theta}) = \prod_{s=1}^{N_S}{ \phi( \bar{\theta}^T \bar{g}_s)^{d_s} [ 1-\phi( \bar{\theta}^T \bar{g}_s) ]^{1-d_i} }$, it can be shown that for genotype terms

\begin{eqnarray*}
	\frac{ \partial^2 \ell(\bar{\theta}) }{ \partial\theta_i \partial\theta_j } 
	= \sum_s{ g_{s,i} g_{s,j} p_s (1-p_s) } 
\end{eqnarray*}

Using analogous derivation for the covariates, we can find an analytic form of the Hessian, which completes the Laplace approximation formula.

Calculating Bayes factors involves using prior parameter distributions. In order to estimate these distributions, we run the logistic regression fitting analysis and plot the parameter distributions for different levels of significance. As expected most parameters have unimodal distribution, except for $\theta_3$, which has a multimodal distribution (Figure \ref{S3}). For all parameters, except $\theta_3$, we use a normal distribution centred at the mean and variance set to one ($\sigma=1)$ even though most times the variance is much smaller. This is done to avoid penalizing outliers too heavily and to have smooth derivatives near the maximum likelihood estimates. For $\theta_3$, which has a multimodal distribution, we fit a mixture model parameters using an EM algorithm, as shown in Supplementary Figure / Table \ref{S3??}.


\paragraph{Computational and statistical issues} It is easy to see that the computational burden for the detection of pairs of interacting genetic loci affecting disease risk is significantly larger than in a standard (single variant) GWAS study. A priori all pairs of variants should be analyzed, thus significantly increasing the number of statistical tests. This also reduces the statistical power since the required p-value significance level would be orders of magnitude smaller. A na\:ive approach would estimate that if a typical genetic sequencing study has $10^6$ variants, a GWAS on epistatic variants would square that number of statistical tests, thus p-values required for significance would be in the order of $0.05 / (10^6)^2 = 5 \cdot 10^{-14}$. 

Fortunately these numbers can be reduced significantly. First, in this study, we only concentrate on non-synonymous coding variants. Second, as required by our co-evolutionary model, only variants overlapping a multiple sequence alignment are taken into account (when several multiple sequence alignments overlapped a region, the alignment with the longest number of proteins was selected).Third, if two variants $g_i$ and $g_j$ are such that the interaction term $(g_{s,i} g_{s,j})$ is zero in all samples, which usually happens for pairs of rare variants, then $B_F = 1$. Fourth, if the variants and the epistatic term $[g_{s,i}, g_{s,j}, g_{s,i} g_{s,j}]$ are linearly dependent, the logistic regression result will be meaningless, so we can safely skip such variant pairs. Fourth, if one of the variants has high allele frequency respect to the other, all non-zero epistatic terms may lie in the same positions as non-zero genotypes from the low frequency variant, causing logistic regression estimates to artificially inflate the coefficients of the low frequency variant and the epistatic term thus creating an artificially high association (low p-value). So we filter out these variant pairs as well. Finally, we filter out all variants having Hardy-Weinberg p-value of less than $10^{-6}$, since these variants also artificially inflate the logistic regression coefficients.  Once the results are obtained, we can focus on interactions by further filtering results and keeping variant pairs whose alternative logistic model (see equation \ref{eq:gwasLogRegH1}) has small absolute values for $\theta_1$ and $\theta_2$ while having large absolute values for $\theta_3$, specifically we keep results if $|\theta_3| > K ( |\theta_1| + |\theta_2| )$ (based on empirical data, we set $K=3$). 

\subsection{Putting it all together}

In summary, we first calculate the transitions matrices for the Markov models ($Q$ and $Q_2$) based on observations from protein structures (PDB) and multiple sequence alignments (UCSC's 100-way). We analyze variants from genome sequencing data first by filtering only for non-synonymous variants, then analyzing all possible pairs of variants and filtering out those that are unsuitable for further analysis (e.g. in linear dependence, deviation from Hardy-Weinberg equilibrium having p-value less than $10^{-6}$, etc.). From the pairs of variants that pass filtering, we fit two logistic regression models (null and alternative hypothesis), then calculate a p-value using the log-likelihood ratio, and keeping pairs of variants having p-values below a predefined threshold ($10^{-6}$). These pairs of variants are then analyzed under our co-evolutionary model, we find the corresponding columns in the multiple sequence alignment and calculate the likelihoods for the null and alternative models by means of Felsenstein's algorithm (using matrices $Q$ and $Q_2$ in respectively). Finally, likelihoods from co-evolutionary model and likelihoods from logistic regression models are incorporated into a Bayes Factor equation, which is calculated using Laplace's approximation.

%---
\section{Results}
%---

Our approach, which is summarized in Figure ??, involves three main components. First we estimate evolutionary substitution rates for individual amino acids in a protein as well as for pairs of amino acids (either from the same protein or not) that are physically interacting. Given a set of multiple sequence alignment of protein sequences, these evolutionary models can be used to calculate the likelihood of interaction between any two given amino acids, without the need for any structural information. Second, a statistical test for epistasis is developed to identify pairs of non-synonymous SNPs that show (often weak) evidence of interaction in the way they associate to a given trait. Finally, information from the co-evolution component is combined with that from the epistasis component to give more power to the epistasis test.
 
\subsection{Co-evolutionary substitution models}

The approach described in Methods was used to obtain substitution rate matrix $Q$ for individual amino acids and $Q_2$ for pairs of physically interacting residues within the same protein. Unsurprisingly, $Q$ (or more precisely a transition matrix $P(t)$ obtained from $Q$) is very similar to well known transitions matrixes such as PAM \cite{} (Supplementary Figure \ref{f:S1} and Table \ref{tab:S1}).

%Estimating $Q_2$ requires information about amino acids that are known to be ``interacting". A pair of amino acids is considered to be ``interacting" if any pair of atoms (one from each amino acid) has a distance of $3$ \AA or less \cite{burger2010disentangling}. 

The structure of $Q_2$, which describes substitution rates between one pair of interacting amino acids to another, is richer (Supplementary Figure \ref{fig:S2} and supplementary file \ref{fileQ2_hat.txt}). Of particular interest are the pairs of pairs of amino acids for which the ratio $R(ab,cd) = Q_2 (ab, cd) / ( Q(a,c) \cdot Q(b,d) )$ is large. Those substitution pairs are the ones that are most strongly indicative of an interaction. Figure \ref{fig:q_q2_compare} shows that the number of pairs for which $R$ deviates significantly from 1 is quite large, arguing that interacting sites have co-evolutionary rates that differ from the bulk of non-interacting sites.

For example, the case with the highest rate ratio is \texttt{[V.I -> W.W]}' (i.e. amino acid 'V' switched to 'W' in the one sequence, and amino acid 'I' changed to 'W' in the other). In fact, the top 10 pairs are all transitions to 'W-W' amino acid pairs. This makes sense considering that (i) individual amino acid substitution rates to tryptophan are generally very low, but that (ii) tryptophan pairs are well known $\beta$-hairpin stabilizers and are considered as a paradigm for designing stable $\beta$-hairpins \cite{santiveri2010tryptophan}.

Another type of pair transitions with large ratio is the double transitions to a pair of phenylalanine amino acids from a pairs of hydrophobic amino acids (Lysine, Asparagine, Glutamine, Arginine, Aspartic acid and Glutamic acid). Phenylalanine-Phenylalanine interaction pairs are assumed to conform $\pi-\pi$ interactions which are predicted and experimentally observed to be energetically favourable \cite{hunter1991pi}.

\subsection{Co-evolutionary model validation}

We first assessed the ability of our co-evolutionary model to detect interacting sites located within the same protein by computing the likelihood ratio of the evolutionary history of a candidate pair of sites under an co-evolutionary model ($Q_2$) versus under independence ($Q$). Although such pairs of sites are unlikely to exhibit evidence of epistasis in GWAS studies (due to linkage), accurate prediction of interacting sites in a given protein are useful for many other purposes, such as protein structure prediction and prediction of the impact of individual mutations.  Figure \ref{f:} shows that interacting sites tend to have higher likelihood ratio scores than non-interacting ones (Mann-Whitney p-value $< 2.2 \times 10^{-16}$. Although the likelihood ratio score it itself cannot perfectly discriminate between the two classes, only 25.9\% of non-interacting pairs have a likelihood ratio above the median likelihood ratio of interacting pairs. 

\fig{gwas_figure_2}{gwasf2}{12cm}{Histogram of log-likelihood values of pairs of amino acids in contact (red) and not in contact (blue) for amino acids within the protein (PDB). Log-odds of contacting vs non-contacting pairs (black) and smoothed log-odds (dotted grey).}{Histogram of log-likelihood values of pairs of amino acids in contact vs not in contact (within protein).}

To confirm that an evolutionary model estimated based on pairs of interacting sites from the same protein is useful at predicting pairs of interacting sites between proteins, we repeated the same type of analysis on $\sim 3,000$ pairs of interacting ($< 3$ \AA) and $\sim 3,000$ pairs of non-interacting ($>30$ \AA) residues from distinct proteins, obtained from co-crystal structures in PDB (see Methods). As seen on Figure \ref{fig:}, the two classes of sites have substantially different likelihood ratio distributions (Mann-Whitney one sided test: $p-value < 2.2 \times 10^{-16}$), although slightly less so than for sites from the same protein. Only 29\% of non-interacting sites have a likelihood ratio larger than the median for interacting sites. These empirical distributions, allow us to approximate of the log odds of the ``interacting" vs ``non-interacting" amino acids distributions as $log_{odds}(x) = log[P(LL(M_{SA}|Q2) \ge x) / P(LL(M_{SA}|Q) \ge x]$, which fits well an exponential function $e^{\alpha x}- \beta$, where $\alpha = 0.195$ and $\beta = 1.018$ (in order to avoid bias, the log odds value is capped to $4.0$).

\fig{gwas_figure_3}{gwas3a}{12cm}{
Histogram of log-likelihood values of pairs of amino acids in contact (red) and not in contact (blue) for amino acids in different proteins (co-crystallized entries from PDB). Log-odds of contacting vs non-contacting pairs (black) and smoothed log-odds (dotted grey). Roughly 40\% of interacting records have log-likelyhood $> 1$}{Histogram of log-likelihood values of pairs of amino acids in contact vs not in contact (co-crystalized proteins).}

Figure \ref{fig:} shows the example of a predicted contact $LL(M_{SA}) = 7.7$ between \textit{Senp1} and \textit{Sumo1} proteins detected by our method. The co-crystallized structure from PDB highlights the interacting amino acids (less than $3 \AA$ apart) and the corresponding multiple alignment columns.

Although our approach aims at identifying contacting residues from different proteins, it can also be used to predict the presence or absence of interactions between proteins as a whole. We extracted from BioGrid \cite{stark2006biogrid} a set of $\sim3,000$ pairs of human proteins with evidence of interaction, and further required that both proteins belong to the same pathway (MsigDb, C2 groups \cite{subramanian2005gene}), and their corresponding genes are expressed in the same tissue (GTex \cite{lonsdale2013genotype}, expression of 1 FPKM or more, tissues $\in$ \{skeletal muscle, adipose tissue, pancreatic Islets\}). We randomly selected as ``non-interacting" pairs the same number of pairs amongst those that do not fulfil any of the three conditions.

\fig{gwas_figure_jmol_epistasis}{gwas_jmol}{14cm}{Example of interaction between amino acid \#441 of \textit{Senp1} and \#60 of \textit{Sumo1} proteins detected by our method with $LL(M_{SA}) = 7.7$. A) PDB structure 2G4D, shows that the amino acids are in close proximity. B) Multiple sequence alignment and phylogenetic tree showing the putative compensatory amino acid substitution pair ``D-N" replaced by ``H-S".}{Example of amino acid interaction}

Let the two proteins considered have amino acid sequences $A = a_1...a_m$ and $B = b_1...b_n$. To obtain the prediction score for this pair of proteins, we identify the pair of length-$k$ substrings $a_i, a_{i+1}, …, a_{i+k-1}$ and $b_j, b_{j+1}, …, b_{j+k-1}$ that exhibit the strongest support for parallel or anti-parallel interactions, i.e. for which $ \max \{ \sum_{l=0}^{k-1} LL(a_{i+l},b_{j+l}),  \sum_{l=0}^{k-1} LL(a_{i+l},b_{j+k-1-l})$ is maximized. Empirically, the value of $k$ that seems to provide the best predictive power if $k=3$. 
%We calculate the average likelihood ratio of three consecutive pairs of amino acids ($avg_3[LL(M_{SA})]$), either in the forward or reverse directions. For a given pair of genes, we calculate the highest $avg_3[LL(M_{SA})]$ and pick the highest number as representative for that pair. 
As shown in Figure \ref{S5}), prediction accuracy is quite good (p-value $< 2 \cdot 10^{-42}$), considering the modest amount of information considered.

\subsection{Epistatic GWAS analysis}

We applied our methods to a cohort of $\sim 13,000$ individuals in a case-control study of type II diabetes \cite{mccarthy2015T2D}.
This multi-ethnic study covers exons of unrelated individuals from five major ancestral groups (European descent, South Asian, East Asian, Hispanic and African American descent) using an average sequencing coverage over $80 \times$, yielding $1.7$ million coding variants. The filters described in Methods section resulted in a number of variant pairs being analyzed less than $50$ million. By means of the z-score relationship between Bayes Factor and p-values shown in \cite{goodman1999toward}, we can set the GWAS significance threshold for $50$ million pairs at $log_{10}[BF] =  8.0$.

\paragraph{Results} Variant annotated and filtered according to the previous paragraphs lead to $\sim 50$ million pairs of variants having high log likelihood in our logistic regression model  ($LL_{LogReg} > 6$, in equation \ref{eq:gwasLogLikLogReg}) that were further analysed under co-evolutionary and Bayesian models. The complete analysis took less than $2$ days using a $1,000$ CPU-cluster, thus showing that an epistatic GWAS analysis is feasible using current computational resources. Table \ref{tab:gwas_13k_results_1} shows the main results from our GWAS epistatic analysis, genes highlighted in red belong to a hand curated set of genes either associated with diabetes or known to be in diabetes related pathway. It should be noted that some of the top results include amino acid modification sites such as Phosphoserine (or Glycosylation, not shown), which are likely to b interaction loci.

\figtab{gwas_13k_results_1}{gwas_13k_results_1}{14cm}{Results from epistatic GWAS analysis of type II diabetes sequencing data. First column shows total $log_{10}(BF)$; second and third columns show p-value and (raw) Bayes factor for logistic regression model. For each variant in the putative interaction pair: genomic coordinate, gene and functional annotation are shown. Genes marked in red are manually curated gene sets form diabetes related pathways}{Results from epistatic GWAS analysis of type II diabetes sequencing data}.

%---
\section{Discussion}
%---

In this paper, we propose a novel methodology for genome wide association studies of pairs of variants under putative epistatic interaction. Due to the large number of statistical tests required in epistatic analysis, and the corresponding reduction of statistical power, this type of analysis is meant to be applied to datasets consisting of large number of samples, but our highly optimized algorithms are suitable for large scale sequencing genomic studies.

Finally, we show the application of our methods to a large scale exome sequencing study for type II diabetes consisting of $\sim 13,000$ samples and $\sim 1,7M$ variants. First, this shows that it is feasible to apply our methods GWAS-scale datasets. Second we show that some association of pairs of putatively interacting variants with type II diabetes have Bayes Factors showing suggestive association. Although larger cohorts are needed in order to find risk alleles that may be hidden by lower frequencies, thus not captured by this study.

The co-evolutionary model we propose in section \ref{sec:gwasQ2} requires multiple sequence alignment and the corresponding phylogenetic tree. Both the tree and the number of sequences in the $M_{SA}$ should remain constant throughout the genome in order to take advantage of computational optimizations (matrix exponential precalculation and ``constant tree caching") that allow the algorithm to be applied at genome-wide scale. Some multiple sequence alignments (such as Pfam) usually have different number of sequences for each protein (thus different phylogenetic trees). This poses two main disadvantages for our methodology: i) we cannot benefit from the previously mentioned optimizations, since they require a constant phylogenetic tree throughout the whole genome; and ii) we would add the problem of reconciling different phylogenetic trees from two proteins, which may lead to inconsistencies. For this reasons, we selected UCSC's multi-100way \cite{karolchik2014ucsc}, a genome wide multiple sequence alignment of 100 organisms, which has single genome wide phylogenetic tree.

We tested whether our co-evolutionary model can separate clinically relevant variants from ClinVar database \cite{landrum2013clinvar} according to their clinical significance attribute (CLNSIG). Interestingly, variants categorized as ``benign" or ``druggable" have higher scores (mean $LL(M_{SA})$ within protein) than variants categorized as pathogenic (Supplementary Tables \ref{tab:S4A}, \ref{tab:S4B} and Figure \ref{fig:S4}). We speculate that this might be because amino acids that can be compensated would be characterized as ``benign" whereas deleterious amino acids changes cannot be compensated by mutation. 

As future work, we plan to extend our method to analyze domain specific transition pairs, this would allow obtaining better estimates for known interaction domains. Another line of work is to perform GWAS using kernel based statistics of multiple variants \cite{wu2011rare} thus allowing simultaneous analysis of nearby variants in a putative interaction hotspot. In this case the epistatic information would be used as a function modifying the kernel, instead of a bayesian prior.

	%---
\section{Supplementary material}
%---

% Meeting 2015-06-30: Remove matrice plots, they only confuse people
%
% \fig{gwas_Qhat_vs_PAM1}{gwas_Qhat_vs_PAM1}{14cm}{Comparison (log ratio) between $\hat{P}(t)$ estimated from $\hat{Q}$ and PAM1}{Comparison (log ratio) between $\hat{P}(t)$ estimated from $\hat{Q}$ and PAM1}
%
% \fig{gwas_Q2}{gwas_Q2}{14cm}{$\hat{Q}_2$ matrix structure}{$\hat{Q}_2$ matrix structure}

\fig{gwas_genegene}{gwas_genegene}{14cm}{Distribution of $\ell_C$ for interacting genes (red) and non-interacting genes (green) showing a small but statistically significant difference.}{Distribution of $\ell_C$ for interacting genes and non-interacting genes}

\fig{gwas_clinvar}{gwas_clinvar}{14cm}{Distribution of $\ell_C$ across different clinical categories from ClinVar database showing a clear separation on ``\textit{Benign}" variants}{Distribution of $\ell_C$ across different clinical categories from ClinVar database}

\begin{table}[ht]
\begin{tabular}{|c|c|c|c|c|c|}
\hline 
Significance (CLNSIG)   &   Count  & $Mean[\ell_C]$  &  $Median[\ell_C]$ \\
\hline 
Benign (2)              &    272   &      34.1       &    26.2     \\
Likely benign (3)       &    258   &      31.5       &    25.5     \\
Likely pathogenic (4)   &    562   &      17.5       &    12.0     \\
Pathogenic (5)          &   4206   &      16.9       &    11.7     \\
Drug response (6)       &     18   &      32.6       &    22.1     \\
Other (255)             &     10   &      20.0       &    11.3     \\
\hline 
\end{tabular}
\\
\caption{ClinVar categories have different distributions. Columns 3 and 4 show the mean and median $\ell_C$ values calculated for each variant using the best (highest) within protein $\ell_C$} 
\label{tab:gwas_clinvar}
\end{table}

\begin{table}[ht]
\begin{tabular}{|c|c|c|}
\hline 
Project      & $Mean[\ell_C]$  &  $Median[\ell_C]$ \\
\hline 
1000 Genomes &            23.2 &          14.8  \\
HGMD         &            19.8 &          12.1  \\
ClinVar      &            18.6 &          11.9  \\
\hline 
\end{tabular} 
\caption{Overall distributions for 1000Genomes, HGMD and ClinVar. Columns 2 and 2 show the mean and median $\ell_C$ values calculated for each variant using the best (highest) within protein $\ell_C$. Values are calculated using a random sub-sample of all variants in each project.}
\label{tab:gwas_1Kg}
\end{table}

	
%-----------------------------------------------------------------------------
\chapter{Conclusions \label{ch:concl}}
%-----------------------------------------------------------------------------

%---
\section{Contributions}
%---

In this report we showed the three steps involved in the analysis of sequencing data and identifying the links to disease. Each step is characterized by very different problems that need to be addressed.
					
\begin{itemize}
\item[i)] The first step is to reduce large amounts of information generated by high throughput experiments into a manageable subset. In our case, it involves reducing the raw sequencing information to a variant call set, but it could be any other features to be analyzed (RNA expression, transcript structure, enrichment peaks, genome reference assembly, etc.). This is mainly done by mapping reads into a reference genome and then using variant call algorithms. This step is characterized by requiring fast parallel algorithms and usually, due to the amount of data involved, I/O can be one of the bottlenecks. Algorithm that work on ``chunks of data" instead of the whole data-set are preferred, and in many cases exist, because it makes the problem trivial to parallelize. Usually several stages of these highly specialized algorithms are combined into a ``data analysis pipeline". Programming data analysis pipelines is not trivial since it requires process coordinations, robustness, scalability and flexibility (data pipelines, particularly in research environments, tend to change often). Although many solutions are available (usually in the form of libraries), these tend to make pipeline programming cumbersome or create new programming paradigms thus introducing steep learning curves. In Chapter 2, we solved the problems related to pipeline programming in a novel way by creating a new programming language, BDS, that simplifies the creation of robust, scalable and flexible data pipelines. Although the main goal was managing our sequencing data pipelines, BDS is a flexible datacenter-scale programming language that can be applied to many large data pipelines (a.k.a. Big Data problems).

\item[ii)] The second step in our data analysis, consists of functional annotations, prioritization and filtering. The main concern in the annotation step performing an adequate filtering of what should be considered relevant variants for our experiment from irrelevant ones. Functional annotation of genomic variants was until not long ago an unsolved problem and shortly after created SnpEff \& SnpSift, they quickly became widely adopted by the research community. In Chapter 3 we described the challenges of variant annotations and some of the solutions we implemented in our algorithms.

\item[iii)] Finally, in Chapter 4, we analyzed the problem of  finding genetic links to complex disease. This is known to be a difficult problem affected by several hidden co-factors that bias the results (e.g. population structure). Furthermore there are unsolved problems, such as missing heritability, implying that genomic links to complex disease may not be found using traditional GWAS methodologies. We believe that alternative models that combine higher level information, may help to boost statistical significance. 

	\begin{itemize}
	\item[iii.a)] We were involved in two major projects on GWAS of type II diabetes using: a) cohorts of multi-ethnic unrelated individuals and b) family pedigrees. Results uncovered new genes linked to diabetes. Also, the studies indicate that one of the main hypothesis in the field, the ``Rare variant hypothesis", might not hold strong.
	
	\item[iii.b)] We proposed a new methodology for addressing a difficult problem: detection of two interacting genomic loci that affect disease risk. Our models combine genotype information and co-evolutionary methods. We show that efficient algorithms make these studies computationally feasible, albeit using large computational resources, and we apply them to real data form type II diabetes sequencing study of over 26,000 individuals.
	\end{itemize}
\end{itemize}

These three Chapters (three steps) complete our journey from ``raw data" to ``biological insight" trying to find the genetic causes of complex disease.

%---
\section{Future work}
%---

Here we propose several improvements, extensions or future lines of work for each of the methods developed in this thesis (some of them are currently being developed / explored): \\

\begin{enumerate}
\item[BDS]
	\begin{enumerate}
	\item Native support for new clusters and frameworks (that now supported via ``Generic cluster"): LSF, Mesos, Kubertes.
	\item Functional constructs: map, apply, filter. This allows for more compact and readable code.
	\item Richer data structures: BDS currently supports maps and list but does not support user defined structures.
	\end{enumerate}

\item[SnpEff]
	\begin{enumerate}
	\item Creation of a new VCF annotation standard coordinated with the developer of other annotations tools (mainly ENSEMBL’s VEP and ANNOVAR).
	\item GA4GH variant annotation specification \& API definition.
	\item Haplotype effect predictions: Using phased (or ``read phasing") to calculate compound variant effects (e.g. consecutive phased SNPs forming an MNP or two compensating frame shifts).
	\item Improved loss of functions predictions.
	\item Improved splice predictions using information theoretic analysis of splice sites from several species.
	\end{enumerate}

\item[GWAS Epistasis]
	\begin{enumerate}
	\item Further optimization in logistic regression analysis:faster computations boosts program performance significantly.
	\item Analysis of context dependent Q2 matrices based on protein domains.
	\item Improved calculation of Bayesian priors.
	\end{enumerate}
\end{enumerate}

%---
\section{Perspectives}
%---

Genomic research for complex disease is trending towards larger and larger cohorts in order to improve statistical power. Some years ago, projects involving hundreds to a thousand individuals were common. To put this in perspective, that’s the population of a village, or a small town. Nowadays, projects like the T2D consortia, sequence in the order of 20,000 people (i.e. the population of a large town). I am aware, through personal communications with other researchers, that projects being drafted for sequencing over 100,000 individuals (i.e. the population of a whole city). This quest for ever bigger sample sizes shows how elusive the genetic causes of complex diseases are. 

The methods developed here aim to help in the processing of these huge datasets (BDS), annotate and prioritize the variants (SnpEff) before testing for significance. But also help in looking at these variants from another perspective (epistatic GWAS) than the traditional ``single variant association" approach. It might be true that huge sample sizes are needed to uncover risk loci, but perhaps one of the reasons why traditional GWAS studies are not finding as many associations as expected is just that we they are looking in the wrong place. In science we must explore all possibilities.



\else
	%-----------------------------------------------------------------------------
\chapter{Introduction \label{ch:intro}}
%-----------------------------------------------------------------------------

%---
\section{Motivation}
%---

How does your DNA influence your risk of getting a disease? Contrary to popular belief, your future health is not ``hard wired" in your DNA. Only in a few diseases, referred as ``Mendelian diseases", there are well known, almost certain, links between genetic mutations and disease susceptibility. For the majority of what are known as ``complex traits", such as cancer or diabetes, genomic predisposition is subtle and, so far, not fully understood.

With the rapid decrease in the cost of DNA sequencing, the complete genome sequence of large cohorts of individuals can now be routinely obtained. This wealth of sequencing information is expected allow the identification of genetic variations linked to complex traits. In this work, I investigate the analysis of genomic data in relation to complex diseases, which offers a number of important computational and statistical challenges. We tackle several steps necessary for the analysis of sequencing data and identifying the links to disease. Each step, which will correspond to a chapter in my thesis, is characterized by very different problems that need to be addressed.

\begin{itemize}

\item[i] The first step is to analyze large amounts of information generated by sequencers to obtain a set of ``genomic variants" that distinguish each individual. To address these big data processing problems, Chapter 2 shows how we designed a programming language (BigDataScript or BDS), that simplifies the creation robust, scalable data pipelines.

\item[ii] Once genomic variants are obtained, we need to prioritize and filter them to discern which variants should be considered ``important" and which ones are likely to be less relevant. In this process, known as ``functional variant annotation" or simply ``variant annotation", we calculate how the protein product would be affected and add information from relevant genomic databases (such as protein structure, deleteriousness scores or how often the variant is present in a population). We created SnpEff \& SnpSift \cite{cingolani2012program, cingolani2012using} packages that, using optimized algorithms, solve several annotation problems: a) standardize the annotation process, b) calculate putative genetic effects, c) estimate genetic impact, d) add several sources of genetic information, and e) facilitate variants filtering. We applied our methods in two large Genome Wide Association Studies (GWAS) for type II diabetes projects, in order to prioritize variants for statistical analysis. As a result of these studies, novel genes associated with diabetes and glycemic traits were found.
					
\item[iii] Finally, we address the problem of finding associations between ``interacting genetic loci" and disease. One of the main problems in GWAS, known as ``missing heritability", is that most of the phenotypic variance attributed to genetic causes remains unexplained. Since interacting genetic loci have been pointed out as one of the possible causes of missing heritability, finding links between such interactions and disease has great significance in the field. We propose a methodology to increase the statistical power of this type of approaches by combining population-level genetic information with evolutionary information. 

\end{itemize}

In the rest of this introduction we give the background required to understand the material shown in Chapters 2 to 5 while providing motivations for our research. The transformation of raw sequencing data into biological insight in the aetiology of complex disease poses a series of computational, analytical, algorithmic and methodological challenges that we address in the rest of this thesis.

\subsection{Reference genome and genetic variants}

DNA is composed of four basic building blocks, called ``bases'' or ``nucleotides''. These four nucleotides, usually abbreviated $\{A, C, G, T\}$, are Adenine, Cytosine, Guanine, and Thymine. Bases form pairs, either as $A-T$ or $C-G$, that pile-up forming two long polymers, with backbones that run in opposite directions giving rise to a double-helix structure. Arbitrarily, one of the polymers is called the positive strand and the other is called the negative strand.  

The human genome has a total of 3 Giga-base-pairs (Gb), and those bases are divided into 23 chromosomes. We have two copies of each ``autosomal'' chromosomes, one inherited from our mother and one from our father. There are 22 autosomal chromosomes. The longest, being roughly 250 Mega-bases (Mb), is called ``Chromosome 1'' and the shortest, being ~50 Mb is called ``chromosome 22''. We also have two sex chromosomes, called 'X' and 'Y'.

In order to be able to compare different object’s length, we need some reference measure, such as the reference meter. Similarly, in order to compare DNA from different individuals (or samples), we need a ``reference genome". The human reference genome (e.g. GRCh37) does not correspond to the DNA of any particular person, but to a ``mosaic" of thirteen anonymous volunteers from Buffalo, New York \cite{REF}.

When samples are sequenced, the DNA is compared to the ``reference genome". Most of the DNA is the same, but there are differences. These differences, generically known as ``genomic variants" (or ``variants", for short), describe the particular genetic makeup of each individual. There are several different ways a sample can differ from a reference genome. These are known as ``variant types" and can be roughly categorized in the following way:

\begin{description}

\item[Single nucleotide variants (SNV)] or Single nucleotide polymorphism (SNP) are the simplest and more common variants produced by single base difference (e.g. a base in the reference genome, at a given coordinate,  is an ‘A’, whereas the sample is ‘C’). There are several biological mechanisms responsible for this type of variants: i) replication errors, ii) errors introduced by DNA repair mechanism, iii) deamination (a base is changed by hydrolysis which may not be corrected by DNA repair mechanisms), iv) tautomerism (and alteration on the hydrogen bond that results in an incorrect pairing).

\item[Multiple nucleotide polymorphism (MNP)] are differences of more than one base (e.g. reference is ‘ACG’ whereas the sample is ‘TGC’).

\item[Insertions (INS)] refer to a sample having extra base(s) compared to the reference genome (e.g. reference is ‘AT’ and sample is ‘ACT’). Some small insertion are usually attributed to DNA polymerase slipping and replicating the same base/s (this produces a type of insertion known as duplication). Large insertions are can be caused by unequal cross-over event (during meiosis) or transposable elements.

\item[Deletions (DEL)] are the opposite of insertions, the sample has some base(s) removed respect to the reference genome (e.g. reference is ‘ACT’ and sample is ‘AT’). As in the case of insertions, deletions can also be caused by ribosomal slippage, cross-over events during meiosis and transposable elements. 

\item[Mixed variants] can happen as a more complex combinations of combining SNV/MNP + Ins/Del.

\item[Copy number variations (CNVs)] arise when the sample has two or more copies of the same genomic region (e.g. a whole gene that has been duplicated or triplicated) or conversely, when the sample has less copies than the reference genome. Copy number variations can be attributed to problem during homologous recombination events.

\item[Rearrangements] are some complex variants that involve joining different regions (e.g. a translocation between chromosomes). Inversions, a type of rearrangement, result from a whole genomic region being inverted. These types of mutations are often attributed to cross-over events during meiosis.

\end{description}

As humans have two copies of each chromosome, variants could affect zero, one or two of the chromosomes and are called ``homozygous reference", ``heterozygous", and ``homozygous alternative" respectively. Variants are also be classified on how common they are within the population: common, low frequency, or rare (see sections \ref{sec:}). How these types of genetic variants influence traits or risk of disease is a topic of intense research that will be discussed throughout this thesis.

Proteins are composed by chains of amino acids and, as explained by the central dogma of biology,  DNA is the template that instructs cellular machinery how to produce proteins. There are 4 bases in the DNA. There are 20 amino acids, which are the building blocks of all proteins. Each of the twenty amino acids is encoded by a group of three DNA bases called ``codon''. More than one codon can code for the same amino acid (i.e. $4^3=64$ codons $ > 20 $ amino acids) allowing for code redundancy. Additionally, there are codons that mark the end of the protein, these are called ``STOP" codons and signal molecular machinery to end the transcription process. So variations in DNA may sometimes have direct effects on the protein product. We will talk about this in section \ref{sec:} and Chapter 3 where we cover the topic of ``functional annotations".

\subsection{DNA and disease}

It would be fair to say that the Garrod family was fascinated by urine. As a physician at King’s College, Alfred Baring Garrod, discovered gout related abnormalities in uric acid \cite{kennedy2001}. His son, Sir Archibald Garrod, was interested in a condition known as alkaptonuria, in which children are mostly asymptomatic except for producing brown or black urine, but by the age of 30 individuals develop pain in joints of the spine, hips and knees. In 1902, Archibald observed that the family inheritance pattern of alkaptonuria resembled Mendel’s recessive pattern and postulated that a mutation in a metabolic gene was responsible for the disease. Publishing his finding he gave birth to a new field of study known as ``Human biochemical genetics" \cite{REF}.

Diseases having simple inheritance patterns, such as Cystic fibrosis, Phenylketonuria and Huntington's are also known as Mendelian diseases \cite{REF}. The genetic components of several Mendelian diseases have been discovered since the mechanism was first elucidated by Garrod in 1902 and the process has been accelerated in recent years, thanks to the application of DNA sequencing techniques \cite{REF}.

In complex diseases (or complex traits), such as diabetes, cancer or Alzheimer’s, affected individuals cannot be segregated within pedigrees (i.e. no patterns of inheritance can be identified). As opposed to Mendelian diseases the aetiologically of complex traits is complicated due to factors such as: incomplete penetrance (symptoms are not always present in individuals who have the disease-causing mutation), oligogenic inheritance (characterized by more than one gene) and genetic heterogeneity (caused by any of a large number of alleles). This makes  it difficult to pinpoint the genetic variants that increase risk of complex disease.

\subsection{Type II diabetes}

Although this thesis focusses on the development of computational approaches that could be applied to the study of a number of complex diseases, our focus has been on type II diabetes mellitus (T2D), a complex disease first described by the Egyptians in 1500 BCE. Later the Greeks in 230 BCE used the term ``diabetes" meaning ``pass through" (or ``siphon") denoting the constant thirst and frequent urination of the patients. In the 1700s the term ``mellitus" (from honey) was added to denote that the urine was sweet and would ``attracts ants".

Diabetes symptoms include frequent urination, thirst, and constant hunger, high blood sugar (hyperglycemia) and insulin resistance. Long term complication from T2D may include eyesight problems, heart disease, strokes and kidney failure. Type II diabetes, is highly correlated with obesity and disease rate has increased dramatically during the last 50 years. According to the World Health Organisation the prevalence of diabetes is 9\% in adults and an estimated 1.5 millions deaths were caused by diabetes in 2012 \cite{REF}, which is predicted to be the 7th leading cause of death by 2030. The costs associated to treating diabetes patients only in the U.S. are estimated around \$245 billion dollars.

In recent years, over 80 genetic loci related to T2D have been identified \cite{REF}. Nevertheless, the overall effect sizes of these loci account for less than 10\% of the overall disease predisposition \cite{REF}. This poses the question of why, given that so much efforts has been directed at finding the genetic components of this disease, the loci found so far have such modest effects. This lack of large genetic effects do not only arise in T2D but also in almost all complex traits and could be explained by what is known as the ``missing heritability" problem.

\subsection{Missing heritability}

We all know that ``tall parents tend to have tall children", which is an informal way to say that height is a highly heritable trait. It is said that there are 30 cm from the tallest 5\% to the shortest 5\% of the population and genetics are accountable for 80\% to 90\% of this variation, which means that 27cm of variance are assumed to be ``carried" by DNA variants from parents to offspring. Since 2010 the GIANT consortia has been investigating the genetic component of complex traits like height, body mass index (BMI) and waist to hip ratio (WHR). Even though they found many variants associated those traits, their findings only explain 10\% of the phenotypic variance which corresponds to only a few centimeters in height \cite{REF:GIANT}.

In order to calculate heritability, we need to be able to measure it, so we need a formal definition. Heritability is defined as the proportion of phenotypic variance that is attributed to genetic variations. The total phenotypic variation is assumed to be caused by a combination of ``environmental" and genetic variations $Var[P] = Var[G] + Var[E] + 2 Cov[G, E]$ \cite{Emerson}.

The environmental variance $Var[E]$ is the phenotypic variance attributable only to environment, that is the variance for individuals having the same genome $Var[E] = Var[P|G]$. Since cloning humans to calculate this term may be an overkill, we resort to approximate it based on phenotypic differences observed in monozygotic and dizygotic twins.

If the covariance factor $Cov[G, E]$ is assumed to be zero, we can define heritability as $H^2 = \frac{Var[G] }{ Var[P]}$. This is called ``broad sense heritability" because $Var[G]$ takes into account all possible forms of genetic variance: $Var[G] = Var[G_A] + Var[G_D] + Var[G_I]$, where $Var[G_A]$ is the additive variance, $Var[G_D]$ is the variance form dominant alleles, and $Var[G_I]$ is the variance form interacting alleles (epistasis). Non-additive terms are difficult to estimate, so a simpler form of heritability called ``narrow sense heritability" that only takes into account additive variance is defined as $h^2 = \frac{ Var[G_A] }{ Var[P] }$ \cite{zuk2012mystery}.

Focusing on narrow sense heritability, the concept of ``explained heritability" is defined as the part of heritability due to known variants with respect to all phenotypic variation ($\pi_{explained} = h^2_{known} / h^2_{all}$). Similarly, missing heritability is defined as $\pi_{missing} = 1 - \pi_{explained} = 1 - h^2_{known} / h^2_{all}$. When all variants associated with traits are known, then $\pi_{missing} = 0$.

Until recently, it was widely assumed by the research community that the problem of missing heritability lied in finding the appropriate genetic variants to account for the numerator of the equation ($h^2_{known}$) \cite{zuk2012mystery}. However, in a series of theorems published recently, it has been proposed that there is a problem in the way the denominator is estimated \cite{zuk2012mystery}. The authors created a limiting pathway model ($LP(k)$) that accounts for epistasis (gene-gene interactions) in $k$ biological pathways. They showed that a severe inflation of $h^2_{all}$ estimators occurs even for small values of $k$ (e.g. $k \in [2,10]$). As a result, genetic variants estimated to account only for $20\%$ of heritability, could actually account for as much as $80\%$ using an appropriate model \cite{zuk2012mystery}.

Even though this result is encouraging, the problem is now shifted to detecting epistatic interactions, a problem that we analyze in section \ref{sec:} and Chapter 4. In the same work \cite{zuk2012mystery}, the authors show an example of power calculation assuming relatively large genetic effect that would require sequencing roughly $5,000$ individuals to detect links to genetic variants, which is a large but nowadays not uncommon, sample size. Nevertheless other estimates place the sample size requirements as high as  $500,000$ individuals \cite{zuk2012mystery}. Even though this sounds as an extremely large number of samples, it is quickly becoming possible thanks to large technological advances and cost reductions in sequencing and genotyping technologies.

\subsection{Conclusions}

Although some genetic causes of complex traits, such as type II diabetes, have been found, only a small portion of the phenotypic variance can be explained. This might indicate that many risk variants are yet to be discovered. Recent studies on the topic of missing heritability report that these ``difficult to find genetic variants" might be in epistatic interaction (analyzed in section \ref{sec:}) or rare variants (see section \ref{sec:}), analysis of either them requires more complex statistical models and larger sample sizes. In Chapter 4 of this thesis, we focus on methods for finding epistatic interactions related to complex disease and develop computationally tractable algorithms that can process data from sequencing experiments involving large number of samples in a reasonable amount of time.

%---
\section{Identification of genetic variants}
%---

Two of the main milestones in genetics were the discovery of the DNA structure in 1953 \cite{watson1953molecular}, followed by the first draft of the human genome in 2004 \cite{collins2004finishing}. The cost of sequencing the first human reference genome was around \$3 billion (unadjusted US dollars) and it was an endeavor that took around 10 years. Since that time, sequencing technology has evolved substantially so that a human genome can now be sequenced in a three days for a price of less than \$1,000, according to prices estimated by Illumina, one of the main genome sequencer manufacturers.

Having a standard reference sequence facilitates comparisons and analysis. For most well known organisms, ``reference genome" sequences are available and current large scale sequencing projects are extending significantly the number of genomes known, e.g. one project seeks to sequence 10,000 mammalian genomes \cite{REF}, another is targeting all microbes that live within human’s guts \cite{REF}.

The amount of information delivered by sequencing devices is growing much faster than computer speed (Moore's law) and data storage capacity. Having to process huge amounts of sequencing information poses several challenges, a problem informally known as ``data deluge''. In the following sections, we explain how sequencing data is generated and how the huge amount of information delivered by a sequencer can be handled in order to make the problem tractable. Just as a crude example, a leading edge sequencing system is advertized to be capable of delivering 18,000 human genomes at $30x$ coverage per year, yielding over 3.2 PB of information. We want to transform this raw data into knowledge of genomic variants that contribute to disease risk with the ultimate goal to translate these risk variants into biological knowledge that can help to design drugs to treat or prevent disease. As expected, processing huge datasets consisting of thousands of sample is a complex problem. In Chapter 2 we show how mitigate or solve some of these issues, by designing a computer language specially tailored to tackle what are know as ``Big data" problems.

\subsection{Sequencing data}

Different technologies for sequencing machines (or sequencers) exists. In a nutshell, a sequencer detects polymers (or chains) of DNA nucleotides and outputs a string of A, C, G, and Ts. Unfortunately, current technological limitations make it impossible to ``read" a full chromosome as one long DNA sequence. Instead, modern sequencers produce a large number of ``short reads", which range 100 bases to 20 Kilo-bases (Kb) in length, depending on the technology. Since sequencers are unable to read long DNA chains, preparing the DNA for sequencing involves fragmenting it into small pieces. These DNA fragments are a random sub-samples of the original chromosomes. Reading each part of the genome several times allows to increase accuracy and ensure that the sequencer reads as much as possible of the original chromosomes. The coverage of a sequencing experiment is defined as the number of times each base of the genome is read on average. For instance, if the sequencing experiment is designed to produce one billion reads, and each read is 150 bases long, then the total number of bases read is 150Gb. Since the human genome is 3Gb, the coverage is said to be 50.

After sequencing a sample, we have millions of reads but we do not know where these reads originate from in the genome. This is resolved by aligning (also called mapping) reads to the reference genome, which is assumed to be very similar to the genome being sequenced. Once the reads are mapped, we can infer if the sample’s DNA has any differences with respect to the reference genome, a problem is known as ``variant calling''. 

Using current technologies and computational methods for variant calling, detection accuracy varies significantly for different variant types. SNV are by far the most accurately detected. Insertions and deletions, collectively referred as InDels, can be detected less efficiently depending on their sizes. Small InDels consisting of ten bases or less are easier to detect than large InDels consisting of 200 bases or more. The reason being that the most commonly used sequencers reads DNA in stretches roughly 200 bases long. Due to this technological limitations, detection is less reliable for more complex variant types.

Although sequencing costs are dropping fast, it is still relatively expensive to sequence thousands of samples and in some cases it makes sense to focus on specific areas of the genome. A popular experimental setup is to focus on coding regions (exons). A technique called ``exome sequencing" consists of capturing exons using a DNA chip and then sequencing the captured DNA fragments only. Exons are roughly 3\% of the genome, thus this technique reduces sequencing costs significantly, for which it has been widely used by many research groups.

\subsection{Sequence alignment}

Given two sequences $s_1$ and $s_2$ from an alphabet (e.g. $\Sigma = \{A,C,G,T\}$), the alignment problem is to add gap characters (`-') to both sequences, so that a distance, such as Levenshtein distance, $d(s_1,s_2)$ is minimized.

This problem has a well known solution, the Smith-Waterman algorithm \cite{REF}, which is a variation of the global sequence alignment solution from Needleman-Wunsch \cite{REF}. The main problem is that the algorithm is $O(l_1 . l_2)$ where $l_1$ and $l_2$ are the length of the sequences. So, Smith-Waterman algorithm is slow for very long sequences, such as the human genome.

In order to speed up sequence alignments, several heuristic approaches emerged. Most notably, BLAST \cite{altschul1990basic}, which is used for mapping sequences several thousand nucleotides long (i.e. longer than a typical sequencer read) to a reference genome. BLAST uses an index to map parts of the query sequence, called seeds, to the reference genome. Once these seeds have been positioned against the reference, BLAST joins the seeds performing an alignment. Since the alignment is performed only using a small part of the reference, the algorithm is much faster.  

\subsection{Read mapping}

Sequence alignment has an exact algorithm solution and several faster heuristic solutions. But even the fastest solutions are too slow to be used with the millions of reads generated in a typical sequencing experiment. Faster algorithms can be used if we relax our requirements in two ways: i) we allow for sub-optimal results, and ii) instead of requiring information of where each base of the read maps to the reference genome, we just want to know where the first base maps. This relaxed version of the alignment algorithm is called ``read mapping'' and the reduced complexity is enough to speed up the computations significantly. An implicit assumption in this formulation, is that the read will be very similar to the reference and that there will be no big gaps.  Once the mapping is performed, the read is locally aligned, a strategy similar to BLAST algorithm \cite{REF}.

Reformulating the problem this way, allows us to use other methods, such as suffix array \cite{durbin1998biological}. Suffix arrays algorithms are fast, but memory requirements are $O[ n \; log(n) ]$ and this becomes the limiting factor. In order to reduce memory footprint of suffix arrays, Ferragina and Manzini \cite{ferragina2000opportunistic} created a data structure based on the Burrows-Wheeler transform.  This structure, known as an FM-Index, is memory efficient yet fast enough to allow mapping high number of reads.  An FM-index for the human genome can be built in only 1Gb of memory, compared to 12Gb required for an equivalent suffix array \cite{li2010fast}.  Given a genome $G$ and a read $R$, an FM-index search can find the $N_{occ}$ occurrences of $R$ in $G$ in $O(|R| + N_{occ} )$ time, where $|R|$ is the length of $R$ \cite{li2010fast}.

Efficient indexing and heuristic algorithms can decrease mapping time considerably.  Nevertheless, these algorithms are not guaranteed to find an optimal mapping.  Several parameters, such as read length, sequencing error profile, and genome complexity profile can affect performance.  The most commonly used implementation of the FM-index mapping algorithms are BWA \cite{li2010fast, li2010fastlong} and Bowtie \cite{langmead2009ultrafast, langmead2012fast}.  Each of them provide optimized versions for the two most common sequencing types: i) short reads with high accuracy \cite{li2010fast,langmead2009ultrafast} or ii) longer reads with lower accuracy \cite{li2010fastlong, langmead2012fast}.

It is worth noting that the mapping problem appears as a consequence of the technological limitations of sequencers.  Having long, highly accurate reads, the problem becomes much easier to solve.  As an extreme example, having only one read which is as long as a chromosome and has no errors requires no mapping processing.  

\subsection{Mapping quality}

Sequencers not only provide sequence information, but also provide an error estimate for each base \cite{li2011statistical}.  This is often referred as a quality ($Q$) value, which is the probability of an error, measured in negative decibels $Q = -10 \; log_{10}(p)$.

Mapping quality is an estimation of the probability that a read is incorrectly mapped to the reference genome. Mapping algorithms provide estimates of mapping errors. In the MAQ model \cite{li2008mapping}, which is one of the earliest models for calculating mapping quality, three main sources of error are explored: i) the probability that a read does not originate from the reference genome (e.g. sample contamination); ii) the probability that the true position is missed by the algorithm (e.g. mapping error); and iii) the probability that the mapping position is not the true one (e.g. if we have several possible mapping positions). It is assumed that the total error probability can be approximated as $\epsilon \approx max(\epsilon_1,\epsilon_2, \epsilon_3)$.

\subsection{Variant calling}

Once the sequencing reads have been mapped to the reference genome, we can try to find the differences between a sequenced sample and the reference genome.  This is referred as  ``variant calling".  Several factors complicate this task, the two main ones being sequencing errors and mapping errors, described in \ref{sec:mapq}.  Using sequencing and mapping error estimates, a maximum likelihood model can infer when there is a mismatch between a sample and the reference genome \cite{li2008mapping}.  This method works best for differences of a single base (SNV), but it can also work with different degrees of success for short insertions or deletions (InDels) usually consisting of less than 10 bases.

Due to the nature of short reads, this family of methods does not work for structural genomic variants, such as large insertions, deletions, copy number variations, inversions, or translocations.  A different family of algorithms are used to identify structural variants, but their accuracy so far has been low compared to SNV calling algorithm \cite{REF}.

Aligning sequences that contain InDels (gaps) is more difficult than ungapped alignments since finding optimal gap boundary depends on the scoring method being used. This biases variant calling algorithms towards detecting false SNVs near InDels \cite{depristo2011framework}.  An approach to reduce this problem is to look for candidate InDels and perform a local realignment in those regions.  This local re-alignment process reduces significantly the number of false positive SNVs \cite{depristo2011framework}. Another approach to reduce the number of false positive SNVs calls near InDels involves the ``Base Alignment Quality" (BAQ) \cite{li2011improving}, which is the probability of misalignment for each base.  It can be shown that replacing the original base quality with the minimum between base quality and BAQ produces an improvement in SNV calling accuracy.  The BAQ can be calculated using a special type of ``Hidden Markov Model" (HMM) designed for sequence alignment \cite{li2011improving, durbin1998biological}. A more sophisticated option for reducing errors consist of performing a local genome re-assembly on each polymorphic region (e.g. HaplotypeCaller algorithm \cite{REF:web_GATK}).

Finally, the error probabilities inferred by the sequencers are far from perfect.  Once the variants have been called, empirical error probabilities can be easily calculated \cite{mckenna2010genome} by comparing sequenced variants to a set of ``gold standard variants" (i.e. variants that have been extensively validated).  This allows to re-calibrate or re-estimate the error profile of the reads.  This is know as a re-calibration step, and usually improves the number of false positives calls \cite{depristo2011framework}.

%---
\section{Functional annotations of genomic variants}
%---

Once DNA is sequenced, reads are mapped and variants are called as described in previous sections, variants identified are annotated in order to gain biological insight. For instance, we would like to know if it is located in a gene and if so whether the variant could be deleterious to the functionality of the protein encoded by the gene. This is the focus of Chapter 3 of my thesis.

The simplest case of a genetic annotation would be to know whether a genetic variant lies onto a gene or not. This would be trivial to calculate, since it only requires comparing the genomic location of the variant with the genomic location of known genes. However, in a sequencing experiment there are usually millions of variants and hundreds of thousands of genomic ``features'' such as genes, transcripts, exons, introns, splice regions, promoters, etc. The sheer volume of data requires time and memory efficient algorithms and data structures.

\subsection{Functional annotations of coding variant}

Genetic variants that are located within the coding region of a protein-coding gene are called coding variants. Although they form a small subset of all variants, they are the ones whose function can best be predicted.  As explained by the central dogma of biology, genetic information flows from DNA molecules to mRNA molecules, which are used as a template to produce proteins. 

A coding variant that produces a codon change is called ``synonymous" or ``non-synonymous" depending on whether the resulting amino acid remains the same of changes. If the variant is synonymous, we can be reasonably confident that there will be almost no effect in protein function, conversely if a non-synonymous variant creates a new STOP codon, it might be a strong indicator that protein function will be  disrupted thus the variant is deleterious.

Estimating the putative effect of large coding variants (duplications, inversions or fusions) is much more challenging than in the cases of simple variants (SNVs, MNVs and small InDels) since there is still not enough studies to determine what effects large coding variants have in protein expression or function.

\subsection{Non-coding annotations}

For variants in non-coding regions of the genome, annotations are more difficult than in coding regions, mostly due to the fact that not only the location of most non-coding features (such as transcription factor binding sites, chromatin modifications, methylation) are not exactly known, but also non-coding features tend to be tissue specific. How DNA variants affect non-coding features is mostly speculative, and even for the few non-coding feature that have predictive models, these are riddled with false positives.

Assuming that non-coding features, or parts of them, have selective pressure to keep their functionality, conservation scores can be used as a proxy on how ``important" these regions are. Nevertheless, this might not apply for certain classes of non-coding features, such as some transcription factors, where there is evidence of negative selection (meaning that transcription factors binding sites might not only not be conserved, but also change more rapidly than other genomic regions).

\subsection{Conclusions}

In Chapter 3 we show two software packages we designed for efficiently performing functional annotations of sequencing variants. These packages, SnpEff \& SnpSift, allow to annotate, prioritize, filter and manipulate variant annotations as well as combine several public or custom-created databases. It should be noted SnpEff was one of the first annotation packages and has become one of the most widely used annotation software in both research and clinical environments. 

%---
\section{Genome wide association studies}
%---

A genome wide association study aims at identifying genetic variants associated to a particular phenotype. First, the genomes (or exome, depending on the study design) of affected individuals (cases) and healthy individuals (controls) need to be sequenced, variants called, annotated and filtered. Then, the goal is to find variants that exhibit some statistical association with the trait or phenotype of interest, which could be a disease status (e.g. diabetes vs healthy), a biomedical measurement (e.g. cholesterol level), or any measurable characteristic (e.g. height). Since the genome is so large, patterns of mutations that suggest correlation may be encountered by chance, so we need to establish statistical significance in order to distinguish true association from spurious ones. Like most studies, we will focus on SNVs, but most methods can be extended to other genomic variants.

\subsection{Single variant tests and models}

Let's imagine that there is only one variant in the whole genome for the cohort we are analyzing. Since each individual has two sets of chromosomes, the variant can be present in one, both, or neither chromosomes. When a variant is in both chromosomes is said to be ``homozygous'', whereas if present in only one of the chromosomes, it is said to be ``heterozygous". So the number of times a non-reference allele is present in an individual, is $ N_{nr} = \{0, 1,2\}$.

When the trait of interest is binary (e.g healthy vs disease), a cohort can be divided into cases and controls and we can build a 3 by 2 contingency table:

\[
\begin{array}{l|c|c|c|}
	& Homozygous Reference & Heterozygous & Homozygous non-reference\\
	& (N_{variant} = 0) & (N_{nr} = 1) & (N_{nr} = 2) \\
    \hline 
    Cases & N_{ca,ref} & N_{ca,het} & N_{ca,hom} \\ 
    \hline 
    Controls & N_{co,ref} & N_{co,het} & N_{co,hom} \\
    \hline 
\end{array} 
\]

Further assumptions about how many variants are required to increase disease risk can reduce this $3 \times 2$ table to a $2 \times 2$ table. In the ``dominant model'', the effect of a mutated gene dominates over the healthy one, so one variant is enough to increase risk. The opposite, called ``recessive model", is when both chromosomes have to be mutated in order to increase risk \cite{balding2006tutorial, clarke2011basic}. In these models, we can count how many cases and controls have at least one variant (dominant model) or two variants (recessive model). This simplifies the previous table, yielding a $2 \times 2$ contingency table, than can be tested using either a $\chi^2$ test or a Fisher exact test \cite{balding2006tutorial}.

Two other commonly used models, are the ``multiplicative" and the ``additive" models \cite{balding2006tutorial,clarke2011basic}. In these models, a disease risk is assumed to be multiplied (or increased) by a factor $\gamma$ with every variant present. We cannot simplify the contingency table, so we assess significance using a Cochran-Armitrage test \cite{clarke2011basic}.

\subsection{Multiple variant tests}

In a real case scenario there are thousands or millions of variants. We can extend the concept shown in the previous section by performing individual tests for each variant present in the cohort. Multiple testing can be addressed either by performing a correction, such as False Discovery Rate \cite{balding2006tutorial, clarke2011basic}, or using a stricter genome wide significance level. There are $3 \times 10^9$ bases in the genome, but taking into account the correlation between nearby variants (linkage disequilibrium), the genome wide significance level is generally accepted to be $p_{value} \leq 10^{-8}$.

In order to check if the null hypothesis of a significance tests is adequate, a QQ-plot is used (i.e. plotting the $y = -log(p_{value})$ vs $x = -log[ rank(p_{value}) / (N+1) ]$, where $N$ is the total number of variants). Adherence of the p-values to a 45 degree line on most of the range implies few systematic sources of association \cite{balding2006tutorial, clarke2011basic}. If the p-values have a higher slope than the $y=x$ line, there might be ``inflation", possibly due to co-factors, such as population structure (see section \ref{sec:popStruct}). If the inflation is not too high (e.g. less than $5\%$), this bias can be corrected by shifting the p-values towards the 45 degree slope. More sophisticated methods are explained in section \ref{sec:popStruct}.

\subsection{Continuous traits and correcting for co-factors}

Methods analyzed so far are suitable for binary ``traits" or ``phenotypes" (e.g. disease vs. healthy individuals). Statistical methods that link genetic information to traits can also be used on continuous or ``quantitative" traits (e.g. weight, height, cholesterol level, etc.). A linear regression can be used assuming the traits are approximately normally distributed \cite{balding2006tutorial, clarke2011basic}. A significance test ($p_{value}$) for linear models can be calculated using an $F$ statistic, but more sophisticated methods are also available \cite{balding2006tutorial, clarke2011basic}.

Using linear models, it is easy to include known co-factors to correct for biases or inflation. For instance, if it is known that a risk increases with age or that males are more susceptible than females, age and sex can be added to the linear equation in order to correct for these effects \cite{balding2006tutorial,clarke2011basic}. In a similar manner, we can add co-factors to binary traits using logistic regression.

\subsection{Population structure}

It is widely accepted that humans started in Africa and migrated to Europe, then to Asia and later to America \cite{hartl1997principles}. Out of an initial population, a few individuals migrate and colonize a new territory. This implies that the genetic variety of the new colony is significantly reduced, compared to the previous population, since the genetic pool is only a small ``founder population". The ``Out of Africa" hypothesis implies that each new migration produced a reduction in genetic variety, also known as a ``population bottleneck'' \cite{hartl1997principles}.

As we previously mentioned, each individual inherits two chromosome sets, a maternal and a paternal one. In a process known as recombination, a chromosome that is formed by part of the maternal chromosome and part of the paternal one, is inherited to the offspring. As a result of recombination, a child has two sets of chromosomes that are one from each parent and, on average, half of a chromosome from each grandparent. This breaking and shuffling of chromosomes every generation, increases genetic diversity. Nevertheless if variants are located nearby in the chromosome, the chances that they are broken apart by recombination event are smaller than if they are further away from each other. This produces a correlation of close variants or ``linkage disequilibrium" (LD). Nearby highly correlated variants are said to be in the same ``LD-block" \cite{hartl1997principles}. If a population has low genetic variety, the LD-blocks are large. So African population has more variety (smallest LD-blocks) and conversely, European, Asian and Amerindian populations have less variety (larger LD-blocks) \cite{hartl1997principles}.

\subsection{Population as confounding variable }

Imagine that we have a cohort of individuals drawn from two populations ($P_A$ and $P_B$) and that individuals in $P_A$ have much higher risk of diabetes than individuals from $P_B$. Now imagine that individuals from $P_A$ have a variant $v_A$ more often, but $v_A$ is actually neutral and has no health effects whatsoever. If we do not take into account population factors, our study would conclude that $variant_A$ is the cause of diabetes, just because we see $variant_A$ more often in affected individuals. In this case is clear that population structure is being a confounding variable. We could avoid this problem by analyzing each population separately \cite{patterson2006population}, but this would cause a loss of statistical power since we have fewer samples.

A population that is a mixture of two or more populations, is known as an ``admixed population''. For instance the ``African-American'' population is a mixture of, roughly, $80\%$ African and $20\%$ European genomes \cite{hartl1997principles,balding2006tutorial}. This means that analyzing a cohort of African-American individuals, we would get population structure as a confounding variable because of population admixture \cite{hartl1997principles}. Obviously, in this case we cannot analyze each population separately, because each individual in the sample is a mixture of two populations.

The admixed population problem can be studied by performing a correction using the eigen-structure of the sample covariance matrix \cite{patterson2006population}. Samples can be arranged as a matrix $C$ where each row is a sample and each column represents a position in the genome where there is a variant. The numbers $C_{i,j}$ in the matrix indicate whether a sample (row $i$) has a non-reference allele at a genomic position (column $j$). Since the allele can be present in zero, one, or two chromosomes in each individual, the possible values for $C_{i,j}$ are $\{0, 1, 2\}$. The covariance matrix is calculated as $M= \hat{C}^T . \hat{C}$, where $\hat{C}$ is the matrix $C$ corrected to have zero mean columns. Usually, the first two to ten principal components of $M$ are used as factors in linear models (see section \ref{sec:lin}) to correct for population structure \cite{patterson2006population}.

Whether a cohort has any population structure and needs correction or not, can be tested using two methods: a) plotting the projections of the first two principal components and empirically observing the number of clusters in the chart, or b) using a statistic of the eigenvalues of $M$ based on Tracy-Widom's distribution \cite{patterson2006population}.

\subsection{Common and Rare variants}

The ``allele frequency" (AF) is defined as the frequency a variant appears in a population. Variants are usually categorized according to AF into three groups: i) Common variants ($AF \geq 5\%$), ``low frequency" ($1\% < AF < 5\%$), and iii) ``rare variants" ($AF < 1\%$). Common variants originated earlier in the population while rare variants are either relatively recent or selected against.

There are three main models for disease susceptibility  \cite{hartl1997principles, gibson2012rare}:i) the Common-Disease-Common-Variant hypothesis (CDCV) assumes that if disease is common, it must be caused by a common variant; ii) the ``infinitesimal hypothesis" proposes that there are many common variants each having small risk effects; and iii) the Common-Disease-Rare-Variant hypothesis proposes that there exists many rare variants, each one having large risk effects.

\subsection{Rare variants test}

The ``rare variant model'' assumes that multiple rare variants have large effects on a trait. The problem is that, since these variants are rare, huge sample sizes are required for tests to identify statistically significant associations. To overcome this problem, methods known as ``burden tests", collapse several rare variants and perform statistical significance tests on grouped variants \cite{li2008methods}. An example of collapsing technique is to count the number or rare variant in a given window and apply a Fisher exact test, as shown in section \ref{sec:single}. A limitation of some burden tests is that they implicitly assume that all rare variants have the same direction of effect, although rare variants might have no effect, be deleterious, or protective \cite{li2008methods,wu2011rare}.

Several techniques allow weighting rare variants by collapsing them using a kernel matrix. This allows to incorporate other information, such as allele frequency and functional annotations. It can be shown that the statistic induced by kernel weighting functions follows a mixture of $\chi^2$ distributions and there is an efficient way to approximate it \cite{li2008methods,wu2011rare}, avoiding computationally expensive permutations tests.

\subsection{Conclusions}

In this section we introduced the basic concepts and methodologies used in GWAS. Although fairly mature, there is still heavy research and continuous improvement on GWAS statistical methods. Not only it is well known that traditional (i.e. single marker) GWAS methods fail under non-additive models \cite{culverhouse2002perspective}, but also variants so far discovered using these methods do not account for all the expected phenotypic variance attributed to genetic causes (i.e. missing heritability). As other authors pointed out, this might be because we need to look for epistatic variants which are not taken into account using these methods. In the next section, and in Chapter 4, we cover the topic of epistatic GWAS analysis.

%---
\section{Epistasis}
%---

Proteins are the most important part of the cell composing up to 50\% of a cell’s dry weight compared to 3\% of the DNA \cite{REF}. Proteins perform their functions mainly by interacting with other proteins, forming complex pathways that lead to a vast array of cellular functions including catalysis of chemical reactions, cell signaling, and structural conformation of the cell. The 3-dimensional structure of the protein, also called ``tertiary structure", is tailored to bind to other proteins in a specific manner to accomplish a functionality. 

Genome wide association studies focus on single variants or nearby groups of variants. An often cited reason for the lack of discovery of high impact risk factors in complex disease is that these models ignore loci interactions \cite{cordell2009detecting} and recently they have been pointed out as a potential solution for the ``missing heritability" problem \cite{REF}. With interactions being so ubiquitous in cell function, one may wonder why they have been so neglected by GWAS. We should point out that there are several reasons: i) models using interactions are much more complex \cite{REF} and by definition non-linear, ii) information on which proteins interacts with which other proteins is incomplete \cite{REF}, iii) in the cases where there protein-protein interaction information is available, precise interacting sites are unknown \cite{REF}. Taking into account the last two items, we need to explore all possible loci combinations, thus the number of Nth order interactions grows as $O(M^N)$ where $M$ is the number of variants \cite{REF}. This requires exponentially more computational power than single loci models. This also severely reduces statistical power, which translates into requiring larger cohort, thus increasing sample collection and sequencing costs \cite{REF}.

In Chapter 4 we develop a computationally tractable model for analyzing putative interaction of pairs of variants from sequencing experiments involving large case / control cohorts of complex disease. Our model is based on combining multiple sequence alignments using a coevolutionary model in order to perform GWAS analysis of pairs of non-synonymous variants that may interact.

5.1 Detection of interacting sites in proteins using co-evolution

Proteins interactions and interaction loci are expensive to identify reliably experimentally and difficult to predict computationally. Some computational prediction methods are based on the assumption that protein interactions sites are under evolutionary pressure to avoid mutations \cite{marks2012protein}, because such mutations reduce the efficiency or even disrupt pathways. Assuming that evolutionary pressure maintains favorable interaction between loci, compensatory mutations can happen more often than non-compensatory ones. The underlying idea is that fitness is higher for compensatory mutations and higher fitness co-occurring mutations would be fixed in the population. Since several organisms have been sequenced, we can try compare orthologous protein sites occurring in all these organisms and seek evidence of coevolution. It should be noted that this approach can be used to detect interacting sites within two different proteins or two interacting sites within the same protein. Detecting interacting sites within the protein can be valuable for determining protein structure. The most widely used method for inferring co-evolution starts from protein multiple sequence alignments ($\mathcal{M}_{sa}$), and identifies a pair of sites (e.g. one site from each interacting protein) that maximizes mutual information ($MI$) \cite{marks2012protein}. It is known that $MI$ has some limitations  \cite{dunn2008mutual} and is biased due to the fact that the multiple sequence alignment are related by an evolutionary process \cite{dunn2008mutual}. This means that some sequences will be very similar since they are evolutionarily close to each other (e.g. human and chimp), whereas other sequences will be very different (e.g. mouse and coelacanth). 

A proper albeit more complex statistical analysis takes into account MSA’s phylogenetic tree.  Sophisticated coevolutionary models are usually designed with the intent of aiding protein structure predictions, which require to pinpoint the exact loci in each protein. These complex models can take anywhere from minutes to days to run for each pair of proteins, thus making them unfit for GWAS-scale analysis. 

We propose to make use of co-evolutionary information to increase the interaction priors in a GWAS model. Since the goal is to increase GWAS priors instead of pinpointing the exact interaction loci, we can relax coevolutionary methods requirements to design computationally tractable models. In Chapter 4, we introduce an epistatic GWAS approach that while combining coevolutionary and sequencing information it is efficient enough to be applied to GWAS-scale, large cohort, datasets.

\subsection{Epistatic GWAS }

Arguably, the most common model linking binary phenotypes (disease vs. healthy) to genotypes is the logistic regression model which relates log odds probability of disease using multiple regression, $ln(\frac{p}{1-p}) = \hat{\beta}^T \hat{g}$, where $p$ is the probability of disease, $\hat{g}$ are the model’s input variables (usually including genotype, sex, age, population structure, etc.), and $\hat{\beta}$ are the logistic regression coefficients. 

Given a set of genotypes (typically genotype analysis includes ~2,000,000 variants and tens of thousands of samples), the simplest way to look for interactions is an exhaustive search of all combinations. This raises two issues: i) multiple testing, which is often resolved by stringent significance threshold, and ii) computational feasibility, which is solved by efficient algorithms, parallelization, and heuristic approaches to quickly discard uninformative loci combinations.

The definition of epistasis, form a statistical perspective, is a ``departure from a linear model" \cite{cordell2009detecting}. This means that in a logistic regression model the input includes terms with each of the genotypes ($g_i$ and $g_j$), as well as an ``interaction term" $gi . g_j$ \cite{cordell2002epistasis}. Although we mainly talk about interaction between two loci, higher order interactions (three or more loci combinations) can be analyzed, but these models require more parameters and extremely large samples are required to accurately fit them.

Although a comprehensive review is out of the scope of this thesis, it is worth mentioning that several other approaches for epistatic GWAS exist. Here we mention a few (shown in alphabetic order):

\begin{itemize}

\item Allele frequency: In \cite{ackermann2012systematic}, an analysis of imbalanced allele pair frequencies is performed under the assumptions that an implicit test for fitness can be achieved looking for over/under-represented allele pairs in a given population.

\item Bayesian model: In \cite{zhang2007bayesian}, a ``Bayesian partitioning model" is used by providing Dirichlet prior distributions for each partition and computing posterior probabilities using Markov chain Monte Carlo (MCMC) algorithms.  The methodology first test individual makers and picks only the top 10\% to further investigate for epistasis, because it is prohibitive to test all loci.

\item Linkage disequilibrium: Studying LD patterns in a population under two-loci model it was shown \cite{zhao2006test} that interactions creates LD in disease population. The authors show how LD-based p-values can uncover interaction and sometimes (in their simulations) outperform logistic regression tests.

\item Machine learning: From a machine learning point of view, finding interacting variants is simply an optimisation and attribute selection procedure \cite{mckinney2006machine}. Several approaches have emerged to tackle the ``interaction problem" and used a variety of different techniques \cite{koo2013review, mckinney2006machine} , such as neural networks, cellular automata, random forests, multifactor dimensionality reduction, support vector machines, etc.

\end{itemize}

Although all these models have advantages under some assumptions, none of them seems to be a ``clear winner" over the rest \cite{cordell2009detecting}, thus currently there are no de-facto standards in epistatic analysis. In light of this, there is need of different approaches to be explored. In Chapter 5 we combine coevolutionary models and GWAS epistasis of pairs of putatively interacting loci, by using Bayes Factors to combine information. 

%---
\section{Thesis roadmap and Contributions}
%---

The original research presented in this thesis covers topics related to the computational and statistical methodologies related to the analysis of sequencing variants to unveil genetic links to complex disease. Broadly speaking, we address three types of problems: (i) Data processing of large datasets from high throughput biological experiments such as resequencing in the context of a GWAS (Chapter 2); (ii) functional annotations, i.e. calculating variant’s impact at molecular, cellular or even clinical level (Chapter 3); (iii) identification of genetic risk factors for complex disease using models that combine population-level and evolutionary-level data to detect putative epistatic interactions (Chapter 4). It should be pointed out that the chapters are ordered similar to the analysis steps we used when analyzing our data for type II diabetes, starting from raw sequencing data and ending with GWAS analysis. When applicable, background material specific to each chapter is presented in a preface, together with an explanation of how that chapter ties in with the rest of the thesis.

This thesis comprises text and figures of scientific articles which have either been published, submitted for publication, or ready to be submitted (waiting upon data embargo restrictions):

\begin{description}
	
	\item[Chapter 2] For this paper, PC conceptualized the idea and performed the language design and implementation. RS \& MB helped in designing robustness testing procedures. PC, RS \& MB wrote the manuscript. 
	
	
		\begin{itemize}
		\item \textbf{Cingolani, Pablo}, Rob Sladek, and Mathieu Blanchette. ``BigDataScript: a scripting language for data pipelines." Bioinformatics 31.1 (2015): 10-16.
		\end{itemize}
	
	
	\item[Chapter 3] For this paper, PC designed, implemented and tested SnpEff \& SnpSift. RS \& MB suggested several extensions for common research use cases. PC, RS \& MB wrote the manuscript. The manuscript was submitted to Nature Protocols, and the editor suggested for it to be published after the main T2D paper is accepted for publication (see next paragraph).
	
		\begin{itemize}
		\item \textbf{Cingolani, Pablo}, Rob Sladek, and Mathieu Blanchette. ``Genomic variant annotation and prioritization" Ready for submission (waiting on consortia paper submission).
		\end{itemize}
	
	The following studies are T2D (type II diabetes) consortia projects which used SnpEff and SnpSift extensibly, several modules were designed with these projects in mind. This are part of a large consortia involving several institutions:
	
		\begin{itemize}
		
		\item McCarthy M., et al (T2D Genes Consortia). ``Variation in protein-coding sequence and predisposition to type 2 diabetes", Ready for submission.
		
		\item Mahajan, Anubha, et al. ``Identification and Functional Characterization of G6PC2 Coding Variants Influencing Glycemic Traits Define an Effector Transcript at the G6PC2-ABCB11 Locus." PLoS genetics 11.1 (2015): e1004876-e1004876.
		
		\end{itemize}
	
	The original SnpEff and SnpSift publications are provided in the appendices:
	
		\begin{itemize}
		
		\item \textbf{Cingolani, Pablo}, et al. ``A program for annotating and predicting the effects of single nucleotide polymorphisms, SnpEff: SNPs in the genome of Drosophila melanogaster strain w1118; iso-2; iso-3." Fly 6.2 (2012): 80-92.
		
		\item \textbf{Cingolani, Pablo}, et al. ``Using Drosophila melanogaster as a model for genotoxic chemical mutational studies with a new program, SnpSift." Toxicogenomics in non-mammalian species (2012): 92.
		
		\end{itemize}
	
	
	\item[Chapter 4] For this paper, PC designed the methodology under the supervision of MB and RS. PC implemented the algorithms. PC, RS \& MB wrote the manuscript. This work uses data from the T2D consortia, thus it cannot be published until the main T2D paper is accepted for publication (according to T2D data embargo).
	
		\begin{itemize}
		\item \textbf{Cingolani, Pablo}, Rob Sladek, and Mathieu Blanchette. ``A co-evolutionary approach for detecting epistatic interactions in genome-wide association studies" Ready for submission (data embargo restrictions).
		\end{itemize}
	
	
\end{description}

\subsection{Other contributions}

Other scientific articles (grouped by topic) published, submitted for publication, or ready to be submitted, not mentioned in this thesis:

\begin{description}

	\item Epigenetics 

	\begin{itemize}
		\item \textbf{Cingolani, Pablo}, et al. ``Intronic Non-CG DNA hydroxymethylation and alternative mRNA splicing in honey bees." BMC genomics 14.1 (2013): 666.
		\item Senut, Marie-Claude, et al. ``Lead exposure disrupts global DNA methylation in human embryonic stem cells and alters their neuronal differentiation." Toxicological Sciences (2014).
		\item Ruden D., ``Epigenetics as an answer to Darwin’s ‘special difficulty’ Part 2: Natural selection of metastable epialleles in honeybee castes", Submitted.
		\item Arko S, et al. ``Lead exposure induces changes in 5-hydroxymethylcytosine clusters in CpG islands in human embryonic stem cells and umbilical cord blood", Submitted.
		\item Senut, Marie-Claude, et al. ``Epigenetics of early-life lead exposure and effects on brain development." Epigenomics 4.6 (2012): 665-674.
	\end{itemize}	
	
	\item GWAS \& Disease 
	
	\begin{itemize}
		\item Oualkacha, Karim, et al. ``Adjusted sequence kernel association test for rare variants controlling for cryptic and family relatedness." Genetic epidemiology 37.4 (2013): 366-376.
		\item Bongfen, Silayuv E., et al. ``An N-ethyl-N-nitrosourea (ENU)-induced dominant negative mutation in the JAK3 kinase protects against cerebral malaria." PloS one 7.2 (2012): e31012.
		\item Hawn, Thomas R., et al. ``Host-directed therapeutics for tuberculosis: can we harness the host?." Microbiology and Molecular Biology Reviews 77.4 (2013): 608-627.
		\item Meunier, Charles, et al. ``Positional mapping and candidate gene analysis of the mouse Ccs3 locus that regulates differential susceptibility to carcinogen-induced colorectal cancer." PloS one 8.3 (2013): e58733.
		\item Caignard, Grégory, et al. ``Genome-wide mouse mutagenesis reveals CD45-mediated T cell function as critical in protective immunity to HSV-1." PLoS pathogens 9.9 (2013): e1003637.
		\item Bouttier M., et al. ``Genomics analysis reveals elevated LXRα signaling reduces M. tuberculosis viability", Submitted.
		\item Bouttier M., et al. ``Genomic analysis of enhancers engaged in M. tuberculosis-infected macrophages reveals that LXR signaling reduces mycobacterial burden", Submitted.
	\end{itemize}	
	
	\item Other 

	\begin{itemize}
		\item \textbf{Cingolani, Pablo}, and Jesus Alcala-Fdez. ``jFuzzyLogic: a robust and flexible Fuzzy-Logic inference system language implementation." FUZZ-IEEE. 2012.
		\item \textbf{Cingolani, Pablo}, and Jesús Alcalá-Fdez. ``jFuzzyLogic: a java library to design fuzzy logic controllers according to the standard for fuzzy control programming."International Journal of Computational Intelligence Systems 
	\end{itemize}	

\end{description}

	%---
\section{Genomes and genetic variants \label{sec:introRef}}
%---

DNA is composed of four basic building blocks, called ``bases'' or ``nucleotides'' \cite{alberts1995molecular}. These four nucleotides, usually abbreviated $\{A, C, G, T\}$, are Adenine, Cytosine, Guanine, and Thymine. Bases form pairs, either as $A-T$ or $C-G$, that pile-up forming two long polymers, with backbones that run in opposite directions giving rise to a double-helix structure \cite{watson1953molecular}. Arbitrarily, one of the polymers is called the positive strand and the other is called the negative strand. 

Proteins are composed by chains of amino acids and, as explained by the central dogma of biology \cite{alberts1995molecular},  DNA is the template that instructs cellular machinery how to produce proteins. There are 20 amino acids, which are the building blocks of all proteins. Each of the twenty amino acids is encoded by a group of three DNA bases called ``codon'' \cite{crick1961general}. More than one codon can code for the same amino acid (i.e. $4^3=64$ codons $ > 20 $ amino acids) allowing for code redundancy. Additionally, there are codons that mark the end of the protein, these are called ``STOP" and signal molecular machinery to end the translation process \cite{brenner1965genetic}.

Proteins compose up to 50\% of a cell's dry weight compared to 3\% of the DNA \cite{alberts1995molecular}. Proteins perform their functions mainly by interacting with other proteins, forming complex pathways that lead to a vast array of cellular functions including catalysis of chemical reactions, cell signaling, and structural conformation of the cell \cite{alberts1995molecular}. The 3-dimensional structure of the protein, also called ``tertiary structure", is tailored to bind to other proteins in a specific manner to accomplish a specific function. 

The human genome has a total of 3 Giga-base-pairs (Gb), and those bases are divided into 22 ``autosomal'' chromosome pairs (in each pair one chromsome is maternally inherited and the other paternally inherited) and ``sex" chromosomes. The longest of the autosomal chromosomes is roughly 250 Mega-bases (Mb) and the shortest one is ~50 Mb.

In order to compare DNA from different individuals (or samples), we need a ``reference genome". Having a standard reference sequence facilitates comparisons and analysis. For most well studied organisms, ``reference genome" sequences are available and current large scale sequencing projects are extending significantly the number of genomes known, e.g. one project seeks to sequence 10,000 mammalian genomes \cite{haussler2009genome}, another is targeting all microbes that live within the human gut \cite{turnbaugh2007human}. The human reference genome (e.g. GRCh37) does not correspond to the DNA of any particular person, but to a ``mosaic" of the genomes of thirteen anonymous volunteers from Buffalo, New York \cite{schneider2013genome}.

When the genome of an individual is sequenced, the DNA is compared to the ``reference genome". Most of the DNA is the same, but there are differences. These differences, generically known as ``genetic variants" (or ``variants", for short), describe the particular genetic make-up of each individual. There are several different ways a sample can differ from a reference genome. Each variant is the result of a mutations that happened at some point in the evolutionary history of the individual (or that of the reference genome). Variant types can be roughly categorized in the following way:

\begin{description}

	\item[Single nucleotide variants (SNV)] or Single nucleotide polymorphism (SNP) are the simplest and more common variants produced by single base difference (e.g. a base in the reference genome, at a given coordinate,  is an `A', whereas the sample is `C'). Depending on whether the variant was identified in an individual or in a population, it is called a Single Nucleotide Variant (SNV) or Single Nucleotide Polymorphism (SNP). It is estimated that there are roughly $3.6M$ SNPs per individual \cite{10002012integrated}. There are several biological mechanisms responsible for this type of variants: i) replication errors, ii) errors introduced by DNA repair mechanism, iii) deamination (a base is changed by hydrolysis which may not be corrected by DNA repair mechanisms), iv) tautomerism (and alteration on the hydrogen bond that results in an incorrect pairing) \cite{griffiths2005introduction}.

	\item[Multiple nucleotide polymorphism (MNP)] are sequence differences affecting several consecutive nucleotides and are typically treated as a single variant locus if they are in perfect linkage disequilibrium (e.g. reference is ‘ACG' whereas the sample is ‘TGC'). .

	\item[Insertions (INS)] refer to a sample having one or more extra base(s) compared to the reference genome (e.g. the reference sequence is ‘AT' and the sample is ‘ACT'). Short insertions and deletions (indels) of a chromosome region range from 1 to 20 bases in length are reported to be 10 to 30 times less frequent than SNV \cite{10002012integrated}. Small insertions are usually attributed to DNA polymerase slipping and replicating the same bases (this produces a type of insertion known as duplication). Large insertions can be caused by unequal cross-over event (during meiosis) or transposable elements.

	\item[Deletions (DEL)] are the opposite of insertions, the sample has some base(s) removed with respect to the reference genome (e.g. reference is ‘ACT' and sample is ‘AT'). As in the case of insertions, deletions can also be caused by ribosomal slippage, cross-over events during meiosis. Those include large deletions, which can result in the loss of an exon or one or more whole genes \cite{alberts1995molecular}. Short deletions are 10 to 30 times less frequent than SNV \cite{10002012integrated}.

	\item[Copy number variations (CNVs)] arise when the sample has two or more copies of the same genomic region (e.g. a whole gene that has been duplicated or triplicated) or conversely, when the sample has fewer copies than the reference genome. Copy number variations are often attributed to homologous recombination events \cite{alberts1995molecular}.

	\item[Rearrangements] such as inversions and translocations are events that involve two or more genomic breakpoints and a reorganization of genomic segments, possibly resulting in gene fusions or loss of critical regulatory elements. Inversions, a type of rearrangement, result from a whole genomic region being inverted.

\end{description}

\noindent As humans have two copies of each autosome, variants could affect zero, one or two of the chromosomes and are called ``homozygous reference", ``heterozygous", and ``homozygous alternative" respectively. Variants are also classified based on how common they are within the population: common ($\ge 5\%$), low frequency ($\le 5\%$), or rare ($\le 0.1\%$). How these types of genetic variants influence traits or disease risk is a topic of intense research that is discussed throughout this thesis.

\section{DNA and disease risk}

It would be fair to say that the Garrod family was fascinated by urine. As a physician at King's College, Alfred Baring Garrod, discovered gout related abnormalities in uric acid \cite{kennedy2001}. His son, Sir Archibald Garrod, was interested in a condition known as alkaptonuria, in which children are mostly asymptomatic except for producing brown or black urine, but by the age of 30 individuals develop pain in joints of the spine, hips and knees. In 1902, Archibald observed that the family inheritance pattern of alkaptonuria resembled Mendel's recessive pattern and postulated that a mutation in a metabolic gene was responsible for the disease. Publishing his finding he gave birth to a new field of study known as ``Human biochemical genetics" \cite{kennedy2001}.

Diseases having simple inheritance patterns, such as alkaptonuria, cystic fibrosis, phenylketonuria and Huntington's are also known as Mendelian diseases \cite{kennedy2001}. The genetic components of several Mendelian diseases have been discovered since the mechanism was first elucidated by Garrod in 1902 and the process has been accelerated in recent years, thanks to the application of DNA sequencing techniques \cite{bamshad2011exome}.

In complex diseases (or complex traits), such as diabetes or Alzheimer's disease, affected individuals cannot be segregated within pedigrees (i.e. no simple pattern of inheritance can be identified). In contrast to Mendelian diseases the aetiology of complex traits is complicated due to factors such as: incomplete penetrance (symptoms are not always present in individuals who have the disease-causing mutation) and genetic heterogeneity (caused by any of a large number of alleles). This makes it more difficult to pinpoint the genetic variants that increase risk of complex disease as demonstrated by the failure of linkage analysis methods and later on GWAS \cite{botstein2003discovering}.

\subsection{Heritability and Missing heritability}

We all know that ``tall parents tend to have tall children", which is an informal way to say that height is a highly heritable trait. It is said that there are 30 cm from the tallest 5\% to the shortest 5\% of the population and genetics account for 80\% to 90\% of this variation \cite{wood2014defining}, which means that 27cm of variance are assumed to be ``carried" by DNA variants from parents to offspring. Since 2010 the GIANT consortia has been investigating the genetic component of complex traits like height, body mass index (BMI) and waist to hip ratio (WHR). Even though they found many variants associated those traits, their findings only explain 10\% of the phenotypic variance which corresponds to only a few centimeters in height \cite{wood2014defining}.

In order to measure heritability we need a formal definition. Heritability is defined as the proportion of phenotypic variance that is attributed to genetic variations. The total phenotypic variation is assumed to be caused by a combination of ``environmental" and genetic variations $Var[P] = Var[G] + Var[E] + 2 Cov[G, E]$ \cite{zuk2012mystery}
\iffinal
\footnote{Although the referenced paper's notation does not seem absolutely consistent, we quote Emerson \textit{``A foolish consistency is the hobgoblin of little minds"} and proceed...}
\fi
.

The environmental variance $Var[E]$ is the phenotypic variance attributable only to environment, that is the variance for individuals having the same genome $Var[E] = Var[P|G]$. This can be estimated by studying monozygotic and dizygotic twins.

If the covariance factor $Cov[G, E]$ is assumed to be zero, we can define heritability as $H^2 = \frac{Var[G] }{ Var[P]}$. This is called ``broad sense heritability" because $Var[G]$ takes into account all possible forms of genetic variance: $Var[G] = Var[G_A] + Var[G_D] + Var[G_I]$, where $Var[G_A]$ is the additive variance, $Var[G_D]$ is the variance from dominant alleles, and $Var[G_I]$ is the variance from interacting alleles (epistasis). Non-additive terms are difficult to estimate, so a simpler form of heritability called ``narrow sense heritability" that only takes into account additive variance is defined as $h^2 = \frac{ Var[G_A] }{ Var[P] }$ \cite{zuk2012mystery}.

Focusing on narrow sense heritability, the concept of ``explained heritability" is defined as the part of heritability due to known variants with respect to phenotypic variation ($\pi_{explained} = h^2_{known} / h^2_{all}$). Similarly, missing heritability is defined as $\pi_{missing} = 1 \pi_{explained} = 1 h^2_{known} / h^2_{all}$. When all variants associated with traits are known, then $\pi_{missing} = 0$.

Until recently, it was widely assumed by the research community that the problem of missing heritability lay in finding the appropriate genetic variants to account for the numerator of the equation ($h^2_{known}$) \cite{zuk2012mystery}. However, in a series of theorems published recently, it has been proposed that there is a problem in the way the denominator is estimated \cite{zuk2012mystery}. The authors created a limiting pathway model ($LP(k)$) that accounts for epistasis (gene-gene interactions) in $k$ biological pathways. They showed that a severe inflation of $h^2_{all}$ estimators occurs even for small values of $k$ (e.g. $k \in [2,10]$). As a result, genetic variants estimated to account only for $20\%$ of heritability, could actually account for as much as $80\%$ using an appropriate model \cite{zuk2012mystery}.

Even though this result is encouraging, the problem is now shifted to detecting epistatic interactions, a problem that we discuss in section \ref{sec:epi} and Chapter \ref{ch:gwas}. In the same work \cite{zuk2012mystery}, the authors show an example of power calculation assuming relatively large genetic effect that would require sequencing roughly $5,000$ individuals to detect links to genetic variants, which is a large but nowadays not uncommon, sample size. Nevertheless other estimates place the sample size requirements as high as  $500,000$ individuals \cite{zuk2012mystery}. Even though this represents an extremely large number of samples, it is quickly becoming possible thanks to large technological advances and cost reductions in sequencing and genotyping technologies.

\subsection{Conclusions}

Although some genetic causes for complex traits, such as type II diabetes, have been found, only a small portion of the phenotypic variance can be explained. This might indicate that many risk variants are yet to be discovered. Recent studies on the topic of missing heritability suggest that the root of these ``difficult to find genetic variants" might be found in epistatic interactions (analyzed in section \ref{sec:epigwas}) or rare variants (see section \ref{sec:comonrare}). Analysis of either requires more complex statistical models and larger sample sizes with the corresponding increase in computational requirements. In Chapter \ref{ch:gwas} of this thesis, we focus on methods for finding epistatic interactions related to complex disease and develop computationally tractable algorithms that can process data from sequencing experiments involving large number of samples in a reasonable amount of time.

%---
\section{Identification of genetic variants}
%---

Two of the main milestones in genetics were the discovery of the DNA structure in 1953 \cite{watson1953molecular}, followed by the first draft of the human genome in 2004 \cite{collins2004finishing}. The cost of sequencing the first human reference genome was around \$3 billion (unadjusted US dollars) and it was an endeavor that took around 10 years. Since that time, DNA sequencing technology has evolved substantially so that a human genome can now be sequenced in three days for a price of less than \$1,000, according to prices estimated by Illumina, one of the main genome sequencer manufacturers \cite{hayden2015is}.

The amount of information delivered by sequencing devices is growing faster than computer speed (Moore's law) and data storage capacity \cite{schatz2010cloud}. Just as a crude example, a leading edge sequencing system is advertized to be capable of delivering 18,000 human genomes at $30 \times$ coverage per year, yielding over 3.2 PB of information. Having to process huge amounts of sequencing information poses several challenges, a problem informally known as ``data deluge''.
% In this section, we explain how sequencing data is generated and how the huge amount of information delivered by a sequencer can be handled in order to make the problem tractable. 
From this raw data we want to obtain a set of candidate genomic variants that contribute to disease risk with the ultimate goal to translate these risk variants into biological knowledge. As expected, processing huge datasets consisting of thousands of sample is a complex problem. In Chapter \ref{ch:bds} we show how mitigate or solve some of these issues, by designing a computer language specially tailored to tackle what are know as ``Big data" problems.

\subsection{Sequencing data}

DNA sequencing machines are based on different technologies, in a nutshell all these technologies detect a set of polymers (or chains) of DNA nucleotides and outputs a set of strings of A, C, G, and Ts. Unfortunately, current technological limitations make it impossible to ``read" a full chromosome as one long DNA sequence. Instead, modern sequencers produce a large number of ``short reads", which range from 100 bases to 20 Kilo-bases (Kb) in length, depending on the technology \cite{quail2012tale}. Since sequencers are unable to read long DNA chains, preparing the DNA for sequencing involves fragmenting it into small pieces. These DNA fragments are a random sub-samples of the original chromosomes \cite{shendure2008next}. Reading each part of the genome several times allows to increase accuracy and ensure that the sequencer reads as much as possible of the original chromosomes. The coverage of a sequencing experiment is defined as the number of times each base of the genome is read on average \cite{shendure2008next,quail2012tale}. For instance, if the sequencing experiment is designed to produce one billion reads, and each read is 150 bases long, then the total number of bases read is 150Gb. Since the human genome is 3Gb, the coverage is said to be $50 \times$.

After sequencing a sample, we have millions of reads but we do not know where these reads originate from in the genome. This is solved by aligning (also called mapping) reads to the reference genome, which is assumed to be very similar to the genome being sequenced. Once the reads are mapped, we can infer if the sample's DNA has any differences with respect to the reference genome, a problem is known as ``variant calling''. 

Although sequencing costs are dropping fast, it is still expensive to sequence thousands of samples and in some cases it makes sense to focus on specific areas of the genome. A popular experimental setup is to focus on coding regions (exons). A technique called ``exome sequencing" \cite{clark2011performance} consists of capturing exons using a DNA chip and then sequencing the captured DNA fragments only. Exons are roughly 1.2\% of the genome, thus this technique reduces sequencing costs significantly, for which it has been widely used by many research groups although it has the disadvantage of only analysing coding genomic variation.

\subsection{Read mapping}

Once the samples have been sequenced, we have a set of reads from the sequencer. The first step in the analysis is finding the location in the reference genome where each read is supposed to originate from, a process that is complicated by a several factors: i) there are differences between the reference genome and the sample genome, ii) sequencing reads may contain errors, iii) several parts of the reference genome are quite similar making reads from those regions indistinguishable, and iv) a typical sequencing experiment generates millions of reads \cite{shendure2008next}.

\paragraph{Local sequence alignment} We introduce a problem known as \textit{local sequence alignment}: Given two sequences $s_1$ and $s_2$ from an alphabet (e.g. $\Sigma = \{A,C,G,T\}$), the alignment problem is to add gap characters (`-') to both sequences, so that a distance, such as Levenshtein distance, $d(s_1,s_2)$ is minimized. This problem has a well known solution, the Smith-Waterman algorithm \cite{smith1981identification}, which is a variation of the global sequence alignment solution from Needleman-Wunsch \cite{needleman1970general}, having an algorithm complexity $O(l_1 . l_2)$ where $l_1$ and $l_2$ are the length of the sequences. So, Smith-Waterman algorithm is slow since in this case one of the sequences is the entire genome.

In order to speed up sequence alignments, several heuristic approaches emerged. Most notably, BLAST \cite{altschul1990basic}, which could be for mapping sequences to a reference genome. BLAST uses an index of the genome to map parts of the query sequence, called seeds, to the reference genome. Once these seeds have been positioned against the reference, BLAST joins the seeds performing an alignment only using a small part of the reference.

\paragraph{Read mapping} Sequence alignment has an exact algorithm solution and several faster heuristic solutions. But even the fastest solutions are too slow to be used with the millions of reads generated in a typical sequencing experiment. Faster algorithms can be used if we relax our requirements in two ways: i) we allow for sub-optimal results, and ii) instead of requiring the output to be a complete local alignment between a read and the genome, we just want to know the region in the reference genome where the read sequence is from. This relaxed version of the alignment algorithm is called ``read mapping'' and the reduced complexity is enough to speed up the computations significantly. In a nutshell, a read mapping is regarded as correct if it overlaps the true reference genome region where the read originated. Once the mapping is performed, the read is locally aligned, a strategy similar to BLAST algorithm \cite{li2010fast, langmead2009ultrafast}.

Reformulating the alignment problem as a \textit{mapping} problem allows us to use data structures such as suffix trees to index the reference genome. Using suffix trees we can query for a substring (read) \cite{durbin1998biological} of the indexed string in $O(m)$ time, where $m$ is the length of the query. Alternatively, we can use suffix arrays which are a space optimized alternative to suffix trees \cite{durbin1998biological}. An implicit assumption in this solution, is that the read is very similar to the reference and that there are no gaps. Suffix arrays algorithms are fast but, even though they are memory optimized versions of suffix trees, memory requirements are still high ($O[ n \; log(n) ]$, where $n$ is the length of the indexed sequence, in this case the reference genome) and this becomes the limiting factor. In order to reduce the memory footprint of suffix arrays, Ferragina and Manzini \cite{ferragina2000opportunistic} created a data structure based on the Burrows-Wheeler transform.  This structure, known as an FM-Index, is memory efficient yet fast enough to allow mapping high number of reads.  An FM-index for the human genome can be built in only 1Gb of memory, compared to 12Gb required for an equivalent suffix array \cite{li2010fast}.  Given a genome $G$ and a read $R$, an FM-index search can find the $N_{occ}$ exact occurrences of $R$ in $G$ in $O(|R| + N_{occ} )$ time, where $|R|$ is the length of $R$ \cite{li2010fast}. 

We should keep in mind that suffix trees, suffix arrays and FM-indexes are guaranteed to find all matching substring occurrences, nevertheless a sequencing read may not be an exact substring of the reference genome (due to sample's genome differences with the reference genome, read errors, etc.). So, even if efficient indexing and heuristic algorithms can decrease mapping time considerably, these algorithms are not guaranteed to find an optimal mapping. 
Several parameters, such as read length, sequencing error profile, and genome complexity profile can affect performance.  The most commonly used implementation of the FM-index mapping algorithms are BWA \cite{li2010fast, li2010fastlong} and Bowtie \cite{langmead2009ultrafast, langmead2012fast}.  Each provides optimized versions for the two most common sequencing types: i) short reads with high accuracy \cite{li2010fast,langmead2009ultrafast} or ii) longer reads with lower accuracy \cite{li2010fastlong, langmead2012fast}. 
It should also be taken into account that read-mapping algorithms implement heuristics to map reads having differences respect to the reference genome, obviously these heuristics are implementation dependent, thus two mapping algorithms can (and often do) lead to different mappings for the same read set which in turn can lead to different variants being called (see section \ref{sec:varcall}).


\paragraph{Mapping quality\label{sec:mapq}} Sequencers not only provide sequence information, but also provide an error estimate for each base \cite{li2011statistical}.  This is often referred as a quality ($Q$) value, which is the probability of an error, measured in negative decibels $Q = -10 \; log_{10}(\epsilon)$, where $\epsilon$ is the error probability. Mapping quality is an estimate of the probability that a read is incorrectly mapped to the reference genome. 

Mapping algorithms provide estimates of mapping quality. In the MAQ model \cite{li2008mapping}, which is one of the earliest models for calculating mapping quality, three main sources of error are explored: i) the probability that a read does not originate from the reference genome (e.g. sample contamination); ii) the probability that the true position is missed by the algorithm (e.g. mapping error); and iii) the probability that the mapping position is not the true one (e.g. if we have several possible mapping positions). It is assumed that the total error probability can be approximated as $\epsilon \approx max(\epsilon_1,\epsilon_2, \epsilon_3)$.

\subsection{Variant calling \label{sec:varcall}}

Genome-wide variant calling has until recently largely been done using genotyping arrays (for SNVs) or Comparative Genomic Hybridization arrays (for CNVs). The inherent limitations of these technologies, particularly their ability to only assay genotypes at sites that are known in advance to be polymorphic, combined with the declining cost of sequencing, have now made approaches based on high-throughput resequencing the tool of choice for variant calling in clinical studies. 

Once the sequencing reads have been mapped to the reference genome, we can try to find the differences between a sequenced sample and the reference genome. This process is called ``variant calling" \cite{nielsen2011genotype}.  Several factors complicate this task, the two main ones being sequencing errors and mapping errors, described in \ref{sec:mapq}. Based on sequencing data and mapping error estimates, tools such as GATK \cite{mckenna2010genome} and SamTools/BcfTools \cite{li2008mapping} use maximum likelihood models can infer when there is a mismatch between a sample and the reference genome and whether the sample is homozygous or heterozygous for the variant. This method works best for differences of a single base (SNV), but it can also work with different degrees of success for short insertions or deletions (InDels) usually consisting of less than 10 bases. 

Aligning sequences that contain InDels (gaps) is more difficult than ungapped alignments since finding optimal gap boundary depends on the scoring method being used. This biases variant calling algorithms towards detecting false SNVs near InDels \cite{depristo2011framework}.  An approach to reduce this problem is to look for candidate InDels and perform a local realignment in those regions.  This local re-alignment process reduces significantly the number of false positive SNVs \cite{depristo2011framework}. Another approach to reduce the number of false positive SNVs calls near InDels involves the ``Base Alignment Quality" (BAQ) \cite{li2011improving}, which is the probability of misalignment for each base.  It can be shown that replacing the original base quality with the minimum between base quality and BAQ produces an improvement in SNV calling accuracy.  The BAQ can be calculated using a special type of ``Hidden Markov Model" (HMM) designed for sequence alignment \cite{li2011improving, durbin1998biological}. A more sophisticated option for reducing errors consist of performing a local genome re-assembly on each polymorphic region (e.g. HaplotypeCaller algorithm \cite{GATK}).

Finally, one should note that the error probabilities inferred by the sequencers are far from perfect.  Once the variants have been called, empirical error probabilities can be easily calculated \cite{mckenna2010genome} by comparing sequenced variants to a set of ``gold standard variants" (i.e. variants that have been extensively validated).  This allows to re-calibrate or re-estimate the error profile of the reads.  This is know as a re-calibration step, and usually improves the number of false positives calls \cite{depristo2011framework}.

Due to the nature of short reads, this family of methods does not work for structural genomic variants, such as large insertions, deletions, copy number variations, inversions, or translocations.  A different family of algorithms are used to identify structural variants generally making use of pair end reads or split reads, but their accuracy so far has been low compared to SNV calling algorithm \cite{o2013low}.

One of the caveats of current sequencing technologies and computational methods for variant calling, detection accuracy varies significantly for different variant types. SNV are by far the most accurately detected. Insertions and deletions, collectively referred as InDels, can be detected less efficiently depending on their sizes. Small InDels \cite{durbin2010map} consisting of ten bases or less are easier to detect than large InDels consisting of 200 bases or more. The reason being that the most commonly used sequencers reads DNA in stretches roughly 200 bases long. Due to this technological limitations, detection is less reliable for more complex variant types.

%---
\section{Functional annotations of genomic variants \label{sec:funann}}
%---

The development of cost-effective, high-throughput next generation sequencing (NGS) technologies have had a profound impact on our ability to study the effects of individual genetic variants on the pathogenesis and progression of both monogenic and common polygenic diseases. As sequencing costs decrease and throughput increases, it has now become possible to quickly identify a large number of sequence polymorphisms (SNVs, indels, structural) using samples from affected and unaffected subjects and investigate these in epidemiologic studies to identify genomic regions where mutations increase disease risk. However, translating this information into biological or clinical insights is challenging as it is often difficult to determine which specific polymorphisms are the main pathogenetic drivers of disease across a population; and more importantly, how they affect the activity of disease-related molecular pathways in tissues and organism a specific patient. In part, this difficulty results from the large number of genetic variants that are observed in individual genomes (the human population is believed to contain approximately 3.5 million polymorphic sites with minor allele frequency above 5\%) combined with the limited ability of computational approaches to distinguish variants with no impact on genome function (the vast majority) from variants affecting gene function or expression that may be associated with disease risk or drug response (the minority). The development of algorithms for automated variant annotation,which link each variant with information that may help predict its molecular and phenotypic impact, is a critical step towards prioritizing variants that may have a functional impact from those that are harmless or have irrelevant functional effects. In this section we review the key concepts and existing approaches in this important field. In Chapter \ref{ch:snpeff} we introduce an approach to collect relevant information that will help answer questions about genetic variants discovered in next-generation sequencing studies, including: (i) will a given coding variant affect the ability of a protein to carry its functions; (ii) will a given non-coding variant affect the expression or processing of a given gene; and ultimately (iii) will a given coding or non-coding variant have any impact on phenotypes of interest?

Answering these questions is essential for many types of analyses that use large-scale genomics datasets to study quantitative traits and diseases, particularly when only a small number of individuals is studied comprehensively at a genome-wide level. For example, most genome-wide association studies (GWAS) or exome sequencing studies lack the statistical power to identify rare variants or variants with small effects associated with a disease, in part due to the large number of variants assayed. This limitation can be addressed by directing both statistical analysis and subsequent experimental steps to focus on smaller sets of genetic variants that have been prioritized based on external evidence of their putative impact. The common impairment of DNA repair mechanisms and chromatin stability in malignant cells leads to a similar challenge in cancer genomics, where the hundreds or thousands of mutations that distinguish an individual's tumor and germline genomes need to be classified on the basis of their putative phenotypic effects and potential roles in carcinogenesis.

The large number of databases containing potentially helpful information about a given variant make the process of gathering and presenting relevant data challenging, despite excellent tools that already exist to analyze large genomics datasets (including GATK \cite{mckenna2010genome} and Galaxy \cite{goecks2010galaxy}) and visualize the results (such as the UCSC \cite{karolchik2014ucsc} or Ensembl \cite{flicek2012ensembl} genome browsers). Each of these databases uses its own format and is updated asynchronously, which makes it difficult for any analysis to remain up to date. In addition, the lack of comprehensive and computationally efficient models that allow integrative analyses using these resources, makes the task of comprehensive variant annotation overwhelming. By efficiently combining information from tens or hundreds of genome-wide databases, the tools described here are designed to greatly facilitate the process of variant annotation, and make it accessible to groups with limited bioinformatics expertise or resources.

%---
\subsection{Variant types}
%---

Although variant calling is a challenging task and remains an important area of research, many high-quality tools exist for calling SNVs and indels.
We discuss here the problem of annotating the variants identified by some of these tools.
The most common type of variant identified by current technologies and analysis approaches is a single base difference with respect to the reference genome (SNV) followed by multiple base differences (MNP), as well as small insertions and deletions (InDels). Here, we focus on annotating these three types of variants which comprise most of the variants in a typical sequencing experiment. We do not address the annotation of large rearrangements due to the challenges involved in their identification and functional characterization and their relative rarity in the germ line.

\subsection{Types of genetic annotations}

The process of genetic variant annotation consists of the collection, integration, and presentation of experimental and computational evidence that may shed light on the impact of each variant on gene or protein activity and ultimately on disease risk or other phenotypes. Variant annotation has traditionally been divided in two apparently independent but actually interrelated tasks based on the variant's location with respect to known protein-coding genes. Coding variant annotation focuses on variants that are located within coding regions of annotated protein-coding genes and attempts to assess their impact on the function of the encoded protein. In contrast, non-coding variant annotation focuses on variants located outside the coding portion of genes (i.e. in intergenic regions, UTRs, introns, or non-protein-coding genes) and aims to assess their potential impact on transcriptional and post-transcriptional gene regulation. These two categories of variant annotations are not mutually exclusive, as variants located within exons can often have an impact on the gene transcript's processing (splicing). In addition, some transcripts can have both protein-coding and non-coding functions \cite{alberts1995molecular}. Despite the intermingling of the notion of coding and non-coding variants, we will consider each type of annotation separately as assessing their impact requires different sources of data and algorithms.

The ultimate goal of variant annotation is to predict the impact of a sequence variant, although this is an ill-defined term. One the one hand, one may be interested in the molecular impact of a variant on the activity of a protein. On the other, others may be interested in a variant's impact on much higher-level phenotypes such as disease risk. Mutations that are predicted to completely abrogate a gene's activity are called loss-of-function (LOF) mutations. Those that are predicted to have less severe consequences are called moderate or low impact mutations. In practice, a variant will be predicted to cause LOF if it has two properties: (i) its molecular impact is reliably predictable by existing computational approaches (e.g. gain of stop-codon); and (ii) its functional impact, reflected by altered protein activity or expression levels, is expected to be large. Many types of variants, including most non-coding variants, may have a large functional impact but lack predictability, and as a consequence are typically not predicted to be LOF variants.

\subsection{Coding variant annotation}

Coding variants occur in translated exons. When a reliable gene annotation is available, their main impact can be classified by determining their effect on the translated amino acid sequence (if any). A synonymous coding variant (also called silent) does not change the sequence of amino acids encoded by the gene, although it may impact aspects of post-transcriptional regulation such as splicing and translation efficiency and can affect the total protein activity through changes in the amount of translated protein that is made in the cell. In contrast, a non-synonymous coding variant changes one or more amino acids encoded by the gene and can directly alter the protein's activity, localization or stability. Non-synonymous variants include missense substitutions that change a single amino acid, nonsense substitutions that lead to the gain of a stop codon, frame-preserving indels that insert or delete one or more amino acids, and frame-shifting indels that may completely alter the protein's amino acid sequence. Primary annotation and assessment of impact, determines whether a variant falls in any of these categories.

\textbf{Caveats}
	\begin{enumerate}[label=\roman*]
	
	\item \textit{Gene misannotation.} Genomic variants that have a significant effect on a protein's expression or function represent a very small fraction of all variants. Assembly and gene annotation errors or genomic oddities that break classical computational models are also rare, but lead to false positives. This implies that one is likely to find a non-negligible fraction of false-positive high-impact variants among the list of what appear to be the strongest candidates for variants with severe effects. Tools such as SnpEff can anticipate some of the most common causes of misannotation, but the number and diversity of the type of events that can lead to false-positives makes the task very challenging. As a consequence, one should always manually inspect the top candidates to ensure that they have been assigned to the correct genes and transcripts.
	
	\item \textit{Gene isoforms.} In higher eukaryotes, most genes have more than one transcript (or isoform), due to alternative promoters, splicing, or polyadenylation sites. For example, a human gene has an average of 8.8 annotated messenger RNA (mRNA) isoforms and some genes are believed to have over 4,000 isoforms resulting from complex splicing programs. For these genes, a variant may be coding with respect to one mRNA isoform and non-coding with respect to another. There are two frequent approaches to address this situation: (i) annotate a variant using the most severe functional effect predicted for at least one mRNA isoform; or (ii) use only a single canonical transcript per gene to perform primary annotation. 
	
	\item \textit{Variant calling for indels.} Variant annotation relies on knowing the exact genomic coordinates of the variant: this is rarely a problem for isolated SNVs; however, insertions and deletions often cannot be located unambiguously. Consider for example the variant $AA \rightarrow A$. This mutation results in the loss of a single base, but was it the first or second A that was deleted? From the standpoint of the cell, this question is irrelevant and deletion of any A will have the same effect. In contrast, from the standpoint of most variant annotation software, deleting the first A is different from deleting the second. Consider the scenario of a previously annotated transcript where the first A is part of the 5' UTR and the second is the first base of a start codon. If the missing base is assigned to the leftmost position in the motif (as is the current convention), the deletion would be annotated as a low impact 5'UTR variant. However, assigning it to the rightmost A would make it appear (incorrectly) to be a high-impact start-codon deletion. Similar issues may arise when considering conservation scores or transcription factor binding site (TFBS) predictions.
	
		\end{enumerate}

\subsection{Loss of function variants}

True LOF variants are difficult to predict computationally, but specific types of genetic changes will frequently lead to severely impaired protein activity. These include i) stop-gains, also known as nonsense mutations; ii) start-loss mutations which change or remove the transcript's start codon; iii) indels causing frameshifts; iv) large deletions that remove either the first exon or at least 50\% of the protein coding sequence; and v) loss of splice acceptor or donor sites that alter the protein-coding sequence. Variants that introduce premature in-frame stop codons (nonsense mutations and most frameshift indels) are expected to abolish protein function, unless the variant is very near the C-terminus of the coding region \cite{yamaguchi2008distribution} (effectively, downstream of the last functional domain in the protein). Such mutations may have severe consequences in affected cells, tissues or organism, as is seen for mutations that cause monogenic diseases \cite{scheper2007translation}. In addition, a new stop codon that lies upstream of the last exon will likely trigger nonsense mediated decay (NMD), a process that degrades mRNA before protein synthesis occurs \cite{nagy1998rule}. NMD predictions are not exact and many factors can affect mRNA degradation, including the variant's distance from the last exon-exon junction or poly-A tail, and the possibility that transcription may re-initiate downstream of the LOF variant \cite{brogna2009nonsense}.

A variant that leads to the loss of a stop codon, sometimes called a read-through mutation, will result in an elongated protein-coding transcript that terminates at the next in-frame stop codon. While there are no general models that predict how deleterious this may be, such variants can also result in aberrant folding and degradation of the nascent proteins, leading to activation of cellular stress response pathways, in addition to their direct effects on protein activity and expression levels \cite{scheper2007translation}.

The effect of the loss of a start codon depends on the location of a replacement start codon with respect to the translation start site and reading frame of the native protein. If the new start codon maintains the reading frame, the only consequence may be the loss of a few amino acids in the protein transcript; however, in many cases, the new start codon will not be in-frame, thus producing a frame-shifted protein that is later degraded. In addition, the new start codon may lack an appropriate regulatory context (for example, if there is no Kozak sequence nearby or if it disrupts 5' UTR folding) leading to reduced expression of an N-terminally truncated protein. Consequently, losing a start codon is thought to be highly deleterious in most cases, due to the potential that it may reduce both protein production and activity.

\textbf{Caveats}
	\begin{enumerate}[label=\roman*]
	
	\item \textit{Rare amino acids.} Through a process called translational recoding, a UGA ``Stop" codon located in the appropriate mRNA context (determined by both primary mRNA sequence and secondary structure) may be translated to incorporate a selenocysteine amino acid (Sec / U) \cite{alberts1995molecular}. In humans, it is known to occur 100 codons located in mRNAs whose 3' UTR contains a Selenocysteine insertion sequence element (SECIS). Since the translation machinery goes so far to encode these special rare amino acids, the expectation is that mutations at those sites would be highly deleterious. This is supported by evidence that reduced efficiency of selenocysteine incorporation is linked to severe clinical outcomes, such as early onset myopathy  \cite{maiti2009mutation} and progressive cerebral atrophy  \cite{agamy2010mutations}.
	
	\item \textit{False-positives in LOF predictions.} Variants predicted to result in a LOF sometimes actually produce proteins that are partially functional  \cite{macarthur2012systematic}. In fact, an apparently healthy individual is typically heterozygous for around 100 predicted LOF variants, and homozygous for roughly 10, but many of those are unlikely to completely abolish the protein function. Indeed, these variants are enriched toward the 3' end of the gene, where they are likely to be less deleterious. 
	
	\end{enumerate}

\subsection{Variants with low or moderate impact}

Compared to the high impact variants discussed above, where extensive prior biological evidence strongly suggests that a specific type of variant will severely impair protein activity, there are few guidelines that can reliably predict how the majority of nonsynonymous (missense) variants will alter protein function or expression. As a result, the primary annotation performed by SnpEff and most related software packages will broadly categorize missense substitutions and their accompanying amino acid changes (e.g. $K154 \rightarrow L154$) as moderate impact variants. Short indels whose length is a multiple of three are treated similarly, unless they introduce a stop codon, as their effect will usually be localized.

Once missense and frame-preserving indel variants are identified, a more detailed estimation of their impact on protein function can be performed using heuristic and statistical models. The most common approaches are based on sequence conservation, either amongst orthologous or homologous proteins, or protein domains, sometimes adding information of the physio-chemical properties of the reference and variant amino acids (e.g. differences in side chain charge, hydrophobicity, or size). The SIFT algorithm \cite{kumar2009predicting} assesses the degree of selection against specific amino acid changes at a given position of a protein sequence by analyzing the substitution process at that site throughout a collection of predicted homologous proteins identified by PSI-BLAST \cite{altschul1997gapped}. Based on this multiple sequence alignment and the highly conserved regions it contains, SIFT calculates a normalized probability of amino acid replacement (called the SIFT score), which estimates the mutation's effect on protein function. Polyphen \cite{adzhubei2010method}, another commonly used tool, takes the process one step further by searching UniProtKB/Swiss-Prot \cite{uniprot2013update} and the DSSP database of secondary structure assignments \cite{joosten2011series} to determine if the variant is located in a known active site in the protein. In contrast to other methods that categorize each variant individually, VAAST \cite{rope2011using}, a commercially available package, computes scores for groups of variants located within a given gene and ``collapses" them into a single category, a concept similar to burden testing performed for rare variants identified in exome sequencing studies. For human proteins, SnpEff makes use of the Database for Nonsynonymous SNVs' Functional Predictions \cite{liu2011dbnsfp} (dbNSFP), which collects scores produced by several impact assessment algorithms in a single database. Individually, impact assessment methods usually have an estimated accuracy of 60\% to 80\% when compared to manually curated databases of human variants, but predictions from several algorithms can be combined to provide a stringent, but more accurate estimate of impact \cite{choi2012predicting}.

In most cases these algorithms apply best to SNVs since these are common in populations and there is more genomic sequence and experimental data available to refine the statistical methods. However, some recently developed algorithms are capable of assessing variants other than SNVs, including PROVEAN \cite{choi2012predicting}, which extends SIFT to assess the functional impact of indels.

\textbf{Caveats}
	\begin{enumerate}[label=\roman*]
	
	\item \textit{Imprecise models of protein function.} Accurate impact assessment of coding variants remains an open problem and most computational predictions are riddled with both false positives and false negatives. While both missense variants and frame-preserving indels are broadly cataloged as having moderate effects, this is mostly due to lack of a comprehensive model and the extremely complex computations that would be required for an in-depth analysis (such as protein structure predictions). In these cases, proteomic information can be revealing. SnpEff adds annotations from curated proteomic databases, such as NextProt  \cite{lane2012nextprot}, which can help to elucidate if a mutation alters a critical protein amino acid or domain (such as amino acids that are post-translationally modified as part of a signaling cascade or that are form the active site of an enzyme) resulting in a protein may no longer function.
	
	\item \textit{Gain of deleterious function.} Computational variant annotation may eventually be able to fairly accurately predict the molecular impact of a variant in terms of the degree to which it translates in a loss of function for the encoded protein. However, gains of function, including the acquired ability to interact with new partners and disrupt their function, remain vastly more difficult to tackle, although several such variants have been linked to disease \cite{whitcomb1996hereditary}.
	
	\item \textit{Unanticipated effects of synonymous variants.} In most cases, synonymous variants are regarded as non-deleterious (or low impact); however, one needs to seriously consider the possibility that they may have greater functional effects by altering mRNA splicing  \cite{coulombe2009fine} or secondary structure  \cite{sabarinathan2013rnasnp}. Synonymous SNVs may also alter translation efficiency, by changing a frequently used to a rarely used codon and have been linked to changes in protein expression  \cite{sauna2011understanding}.
	
	\end{enumerate}

\subsection{Non-coding variant annotation}

Although coding variants represent less than 2\% of variants in the human genome, they make up the vast majority of confirmed disease-related variants that have been validated at a functional level. This may result from ascertainment bias (since variants in coding regions are straightforward to discover and characterize at a basic level and many studies have largely ignored non-coding variants); or may be explained by the increased complexity of computational approaches and lab assays required to predict and validate the impact of non-coding variants; or by their potentially more subtle impact on gene expression or cell function. Nonetheless, in a compendium of current GWAS studies, roughly 40\% of the variants are intergenic and 30\% intronic. Functional studies of these variants are increasingly emphasizing the importance of non-coding genetic variation at risk loci for complex genetic diseases and traits \cite{hindorff2009potential}.

Functional non-coding regions of the genome encompass a wide variety of regulatory elements contained in DNA and RNA molecules that are involved in transcriptional and post-transcriptional regulation. Cis-regulatory elements include (i) binding sites for DNA-binding proteins such as transcription factors and chromatin remodelers; (ii) binding sites for RNA-binding proteins involved in splicing, mRNA localization, or translational regulation; (iii) micro RNA (miRNA) target sites; and (iv) long non-coding RNA (lncRNA) targets on DNA, RNA and proteins. Non-coding transcripts include well-characterized regulatory RNAs (e.g. miRNA, snoRNA, snRNA, piRNA and lncRNAs) as well as RNAs involved directly in protein synthesis (e.g. tRNA and rRNA).  The annotation and impact assessment of non-coding variants presents a significant challenge for several reasons: (i) reliable technologies to study transcriptional regulatory regions on a genome-wide basis are only just reaching maturity and provide limited resolution of binding sites for individual transcription factors and regulatory RNA molecules; (ii) non-coding functional regions of most genomes remain incompletely mapped as they vary widely among different cell types and cell states (for example, in diseased and healthy tissues); (iii) non-coding regulatory elements often are part of complex transcriptional programs that are time-dependent \cite{mattick2001non}, contain many redundant linkages or reciprocal connections between genes and respond to a wide range of intraand extracellular signals; and (iv) genomic regulatory elements rarely have a strict consensus sequence (for example, compare the position weight matrices used to identify transcription factor or miRNA binding sites with the amino acid triplet code) making the effect of a mutation on gene regulatory programs difficult to predict. As a result, high-quality annotation of non-coding variants relies more heavily on experimental data than is the case for coding variants: since many of these experimental techniques did not study the effects of SNVs on gene regulatory programs, they can only be used to annotate variants and not to predict their effects on gene transcription. In the few cases where the effects of SNVs have been studied (for example, the effects of SNVs that are common in a population and located in genetic loci associated with complex diseases), experimental approaches provide highly accurate functional assessment at a cost of reduced coverage compared to computational approaches.

Large-scale projects such as ENCODE \cite{encode2012integrated} and modENCODE \cite{celniker2009unlocking} have made major steps toward mapping gene transcription and transcriptional regulatory regions in many tissues and cell types, but similar studies in diseased tissues remain at an early stage (for example, the growing collection of disease-related epigenomes from the Epigenome Roadmap \cite{bernstein2010nih}). The base-by-base resolution and number of cell states studied for different types of regulatory elements and non-coding transcripts varies widely among datasets; in part due to the lack of sensitive, comprehensive and high-resolution technologies to study the different molecular species and modes of interaction that can be altered by non-coding variants. Efficient technologies for genome-wide, high-throughput mapping of binding sites for RNA-binding proteins (PAR-CLiP \cite{ascano2012identification}), miRNAs (PAR-CLiP \cite{hafner2012genome} and CLASH \cite{helwak2013mapping}) are starting to be applied on a broad scale as are protocols to map transcription factor binding sites (TFBS) which can improve resolution to a single base (Chip-exo \cite{rhee2012chip}). However, in most cases, DNA and RNA binding sites are only imprecisely located within Chip-Seq peaks that span genomic regions hundreds of base pairs in length, with computational approaches being used to pinpoint the bases most likely mediating the interaction. In the absence of more precise localization data, \textit{de novo} computational prediction of binding sites for DNA and RNA binding proteins remains insufficiently accurate to be of much use in annotating single noncoding variants.

This limitation is particularly critical for functional predictions of putative target sites for microRNAs and other regulatory RNA species. MicroRNAs are short RNA molecules that regulate gene expression post-transcriptionally by binding the messenger RNA of a gene through complementary, usually in the 3' region of the transcript, which leads to mRNA degradation or inhibits translation. Sequence variants that cause the loss or gain of a miRNA target site would lead to dysregulation of the gene, with likely deleterious effects. Although miRNAs are relatively well documented in most model organisms including human, their binding sites are only starting to be mapped experimentally, and computational predictions has very low specificity. Meaningful information regarding the possible role of a variant in disrupting a miRNA target site is starting to emerge \cite{liu2012mirsnp}, although variants that create new miRNA binding sites remain under the radar.

Even if the position of a functional element could be perfectly determined, predicting a variant's impact on chromatin conformation, promoter activity, gene expression, or transcript processing remains challenging. For transcription factors, this involves predicting whether the protein will still be able to recognize its mutated site (and with what affinity), as well as predicting the impact of these changes on gene expression levels. The latter is particularly hard to predict as a result of interactions, competition, and redundancy contained in regulatory networks of transcription factors or RNA binding proteins. As a consequence, computational prediction of the functional impact of non-coding variants remains a very active area of research and there is no broad consensus on the best methodology to use \cite{ward2012interpreting}. One significant exception is the identification of variants affecting canonical splice sites, defined as two bases on the 3' end on the intron (splice site acceptor) and 5' end of the intron (splice site donor). Variants that affect canonical splice sites are easily detected and typically lead to abnormal mRNA processing, involving exon loss or extension that leads to loss of function of the encoded protein.

\subsection{Impact assessment of non-coding variants}

Two broad classes of publicly available genome-wide datasets are commonly combined to assess the functional impact of non-coding genetic variants: (i) computational predictions of sequence conservation and sites involved in molecular interactions such as transcription factor and RBP binding, as well as miRNA-mRNA target interactions; and (ii) experimental genome-wide localization assays for DNA binding proteins, histone modifications, and chromatin accessibility.

\paragraph{Computational sources of evidence:} Interspecies sequence conservation plays a key role in scoring and prioritizing non-coding variants. This is based on the assumption is that sites or regions that have been more conserved across species than expected under a neutral model of evolution are likely to be functional; suggesting that mutations contained in them are likely to be deleterious. In the absence of strong experimental data, sequence conservation measures calculated from whole genome multiple alignments, (for example using PhastCons  \cite{siepel2005evolutionarily}, SciPhy  \cite{garber2009identifying}, PhyloP  \cite{pollard2010detection} , and GERP  \cite{davydov2010identifying}), have been developed to provide a generic indicator of function for non-coding variants. Although high conservation scores generally mean that a genomic region may be functional, the converse is not true and many experimentally proven functional noncoding regions show only modest sequence conservation (for example due to binding site turnover events). Finally, some regulatory regions (e.g. specific elements regulating immune response  \cite{raj2013common}) are under positive selection and may thus show less conservation than surrounding neutral regions. 

In humans, genome-wide computational predictions of transcription factor binding sites based on matching to publicly available position weight matrices are available from variety of sources, including Ensembl \cite{flicek2012ensembl} and Jaspar  \cite{bryne2008jaspar}.  Because of the low information content of most binding affinity profiles, the specificity of the predictions is generally very low. Related approaches exist to predict splicing regulatory regions  \cite{fairbrother2002predictive} and miRNA target sites \cite{ziebarth2011polymirts}, some of which are precomputed for whole genomes and available from the UCSC or Ensembl genome browsers. Recent efforts to determine RNA-binding protein sequence affinities can also be used to identify putative binding sites for these proteins in mRNA  \cite{ray2013compendium}.

\paragraph{Experimental sources of evidence:} To investigate the potential impact of variants on transcriptional regulation, many published experimental data sets produced by large-scale projects such as ENCODE \cite{encode2012integrated}, modENCODE \cite{celniker2009unlocking} and Roadmap Epigenomics \cite{bernstein2010nih}, can be used directly by annotation packages. These include: (i) ChIP-seq or ChIP-exo experiments that identify TFBSs on a genome-wide basis; (ii) DNAseI hypersensitivity or Formaldehyde-Assisted Isolation of Regulatory Elements (FAIRE) assays that identify regions with open chromatin; and (iii) ChIP-seq studies to identify the presence of specific promoter or enhancer-associated histone post-translational modifications, which can be combined to identify active, poised, and inactive enhancers and promoters \cite{ray2013compendium}. Most of these data sets are easily available through Galaxy \cite{goecks2010galaxy} (as tracks from the UCSC Genome Browser) or through SnpEff (as downloadable pre-computed datasets). In parallel with the types of studies described above, expression quantitative trait loci (eQTLs) represent an agnostic way to map putative regulatory regions. An increasing number of such loci are available through the GTEX database  \cite{lonsdale2013genotype}. Experimental data that may support assessment of the impact of variants on post-transcriptional regulation remain sparser, although databases such as doRiNa  \cite{anders2011dorina} or starBase  \cite{yang2011starbase} contain genome-wide datasets obtained by CLIP-Seq and degradome sequencing. To our knowledge, these data have yet to be used in the context of variant annotation studies.

\paragraph{Combining sources of evidence:} Despite the variety of computational and experimental sources of evidence available, impact assessment for non-coding variants remains relatively crude, due to the fact that biological models of gene regulation remain fairly simple. Nonetheless, significant steps forward have been made recently, and two web-based tools, HaploReg  \cite{ward2012haploreg} and RegulomeDb  \cite{boyle2012annotation}, perform SNV and indel impact assessment for variants from dbSNV on the basis of a broad body of computational and experimental evidence. Both use pre-computed scores for variants from dbSnp and therefore cannot be used for rare variants, but they are extremely valuable for exploration by associating the variant of interest with a variant in dbSnp via linkage disequilibrium. 

\textbf{Caveats}
	\begin{enumerate}[label=\roman*]
	
	\item \textit{Sparseness of functional sites within ChIP-seq peaks.} Even if a noncoding variant is located in a region that contains a ChIP-seq peak for a given TF and has all the hallmark signatures of regulatory chromatin, the likelihood that it is deleterious remains low, because most DNA bases contained within a peak are non-functional. 
	
	\item \textit{Gain of function mutations.} While this section has focused on variants causing the loss of a functional regulatory element, genetic variants may also create new or more effective transcription factor binding sites. These are substantially harder to detect as they can occur in regions that show no evidence of function in individuals possessing the reference allele, and show little conservation across species. Furthermore, computational methods to predict gain of affinity for a given TF caused by a variant have insufficient specificity to be of much use on their own. 
	
	\end{enumerate}

%---
\subsection{Clinical effect of variants}
%---

One of the most revealing types of annotation of both coding and noncoding variants reports whether the variant has previously been implicated in a phenotype or disease. Although such information is available for only a small minority of all deleterious variants, their number is growing and should be the first type of annotation one seeks out. Clinical annotations, until recently, have been scattered in a large number of specialized databases of medical conditions with a genetic basis, including the comprehensive, manually curated collection of genetic loci, variants and phenotypes in the Online Mendelian Inheritance in Man database \cite{hamosh2005online} (OMIM, www.omim.org); web pages containing detailed clinical and genetic information about uncommon disorders in the Swedish National Board of Health and Welfare Database for Rare Diseases (www.socialstyrelsen.se/rarediseases) and the peer-reviewed NIH GeneReviews collection \cite{bryne2008jaspar} (www.ncbi.nlm.nih.gov/books/NBK1116); and a curated collection of over 140,000 mutations associated with common and rare genetic disorders in the commercial Human Gene Mutation Database \cite{stenson2003human} (HGMD, www.hgmd.org/). In most cases, these datasets do not use standardized data collection or reporting formats; are designed to primarily provide information to patients and health professionals through a web interface; and rely on heterogeneous criteria to describe disease phenotypes and clinical outcomes; pathological and other clinical laboratory data; as well as the genetic and biologic experiments that have been used to demonstrate disease mechanisms at a molecular or cellular level. These shortcomings are being addressed by initiatives that provide centralized, evidence-based, comprehensive collections of known relationships between human genetic variants and their phenotype that are suitable for computational analysis, such as the NIH effort to aggregate records from OMIM, GeneReviews \cite{pagon1993genereviews} and locus-specific databases in ClinVar \cite{landrum2013clinvar} (www.ncbi.nlm.nih.gov/clinvar). 

Another important application of variant detection and annotation is in the study of cancer genomes, which is occurring increasingly in clinical settings to support treatment decisions for advanced tumors. Annotation of variants detected in tumor sequences can be analyzed for clinical cohorts, using similar techniques as other complex traits, as well as for individual patients, using techniques to identify differences between somatic (tumor) and germline (healthy) tissues. In the latter case, one looks for cancer-associated mutations that distinguish the somatic genome of cancer cells of an individual from the germline genome in order to find the driving mutations that pinpoint the specific mechanisms underlying tumorigenesis or metastasis. Ideally, these mutations can be used to select a treatment for the patient, establish prognosis, or to identify causative mutations that have led to the cancer's progression. In such a setting, given that sequence differences between the cancer and germline genomes are of greater interest than the background genetic changes between the germline and a reference genome, variant calling is performed using specialized algorithms, such as MuTect  \cite{cibulskis2013sensitive} and SomaticSniper  \cite{larson2012somaticsniper}.

One of the main problems in these databases is annotation accuracy. Biological knowledge, as well as molecular and phenotypic evidence supports the identification of certain groups of high impact variants based on simple criteria (such as premature stops, frameshifts, start lost and rare amino acid mutations); however, it is often hard to predict whether non-synonymous variants will have equally large effects on an organism's health. Even when the accepted ``rules of thumb" used in the primary annotation indicate that protein function is impaired, we should consider that these predictions may be based on a small number of model genes and will require appropriate wet-lab validation or confirmatory studies in cohorts. In addition, as more human genomes are sequenced, it is likely that some genetic variants that have been linked to Mendelian diseases will be found in healthy individuals  \cite{riggs2013towards}; and in many cases, may not actually be disease-causing mutations  \cite{bell2011carrier}.

\subsection{Data structures and computational efficiency}

Most computational pipelines for genomic variant annotation and primary impact assessment are relatively efficient and can annotate variants obtained from large resequencing projects involving thousands of samples within a few minutes or hours even using a moderately powered laptop. This is typically achieved through two key optimizations: (i) creation of reference annotation databases and (ii) implementation of efficient search algorithms. Reference database creation refers to the process of creating and storing precomputed genomic data from the reference genome, which can be searched quickly to extract information relevant to each variant. This process needs to be performed only once per reference genome and most annotation tools have pre-computed databases for many organisms available for users to download.

Since these databases are typically quite large, efficient search algorithms are used together with appropriate data structures to optimize the search process. In ANNOVAR \cite{wang2010annovar}, each chromosome is subdivided in a set of intervals of size $k$ and genomic features for a given chromosome are stored in a hash table of size $L/k$, where $L$ is the length of the chromosome. Another approach, used by SnpEff, is to use an ``interval forest", which is a hash of interval trees  \cite{cormen2001introduction} indexed by chromosome. Querying an interval tree requires $O[log(n) + m]$ time, where $n$ is the number of features in the tree and m is the number of features in the result. 

\subsection{Conclusions}

In Chapter \ref{ch:snpeff} we introduce SnpEff \cite{cingolani2012program} \& SnpSift \cite{cingolani2012using}, two approaches we designed for efficiently performing functional annotations of sequencing variants. These packages allow annotating, prioritizing, filtering and manipulating variant annotations as well as combining several public or custom-created databases. It should be noted SnpEff was one of the first annotation packages and has become one of the most widely used annotation software in both research and clinical environments. 

%---
\section{Genome wide association studies}
%---

A genome wide association study aims at identifying genetic variants associated to a particular phenotype. First, the genomes (or exome, depending on the study design) of affected individuals (cases) and healthy individuals (controls) need to be sequenced, variants called, annotated and filtered. Then, the goal is to find variants that exhibit some statistical association with the trait or phenotype of interest, which could be a disease status (e.g. diabetic vs healthy), a biomedical measurement (e.g. cholesterol level), or any measurable characteristic (e.g. height). Since the genome is so large, patterns of mutations that suggest correlation may be encountered by chance, so we need to establish statistical significance in order to distinguish true association from spurious ones. Like most studies, we will focus our discussion on SNVs, but most methods can be extended to other genomic variants.

\subsection{Single variant tests and models \label{sec:single}}

Consider a simple situation where there is only one variant in the whole genome for the cohort we are analysing. Since each individual has two sets of chromosomes, the variant can be present in one, both, or neither chromosomes, so the number of times a non-reference allele is present in an individual, is $ N_{nr} = \{0, 1,2\}$.

When the trait of interest is binary (e.g healthy vs disease), a cohort can be divided into cases and controls and we can build a 3 by 2 contingency table:

\[
\begin{array}{l|c|c|c|}
	& Homozygous Reference & Heterozygous & Homozygous non-reference\\
	& (N_{variant} = 0) & (N_{nr} = 1) & (N_{nr} = 2) \\
    \hline 
    Cases & N_{ca,ref} & N_{ca,het} & N_{ca,hom} \\ 
    \hline 
    Controls & N_{co,ref} & N_{co,het} & N_{co,hom} \\
    \hline 
\end{array} 
\]

Further assumptions about how many variants are required to increase disease risk can reduce this $3 \times 2$ table to a $2 \times 2$ table. In the ``dominant model'', the effect of a mutated gene dominates over the healthy one, so one variant is enough to increase risk. The opposite, called ``recessive model", is when both chromosomes have to be mutated in order to increase risk \cite{balding2006tutorial, clarke2011basic}. In these models, we can count how many cases and controls have at least one variant (dominant model) or two variants (recessive model). This simplifies the previous table, yielding a $2 \times 2$ contingency table, than can be tested using either a $\chi^2$ test or a Fisher exact test \cite{balding2006tutorial}.

Two other commonly used models, are the ``multiplicative" and the ``additive" models \cite{balding2006tutorial,clarke2011basic}. In these models, a disease risk is assumed to be multiplied (or increased) by a factor $\gamma$ with every variant present. We cannot simplify the contingency table, so we assess significance using a Cochran-Armitrage test \cite{clarke2011basic}.

\subsection{Multiple variant tests}

In a real case scenario there are thousands or millions of variants in a typical resequencing or genotyping study. We can extend the concept shown in the previous section by performing individual tests for each variant present in the cohort. Multiple testing can be addressed either by performing a correction, such as False Discovery Rate \cite{balding2006tutorial, clarke2011basic}, or using a stricter genome wide significance level. There are $3 \times 10^9$ bases in the genome, but taking into account the correlation between nearby variants (linkage disequilibrium), the genome wide significance level is generally accepted to be $p_{value} \leq 10^{-8}$.

In order to check if the null hypothesis of a significance tests is adequate, a QQ-plot is used \cite{clarke2011basic} (i.e. plotting the $y = -log(p_{value})$ vs $x = -log[ rank(p_{value}) / (N+1) ]$, where $N$ is the total number of variants). Adherence of the p-values to a 45 degree line on most of the range implies few systematic sources of association \cite{balding2006tutorial, clarke2011basic}. If the p-values have a higher slope than the $y=x$ line, there might be ``inflation", possibly due to co-factors, such as population structure (see section \ref{sec:popStruct}). If the inflation is not too high (e.g. less than $5\%$), this bias can be corrected by shifting the p-values towards the 45 degree slope. More sophisticated methods are explained in section \ref{sec:popStruct}.

\subsection{Continuous traits and correcting for co-factors \label{sec:cofactors}}

The methods described so far are suitable for binary ``traits" or ``phenotypes". Statistical methods that link genetic information to traits can also be used for continuous or ``quantitative" traits (such as weight, height and measurements of cholesterol level). A linear regression can be used assuming the traits are approximately normally distributed \cite{balding2006tutorial, clarke2011basic}. A significance test ($p_{value}$) for linear models can be calculated using an $F$ statistic, but more sophisticated methods are also available \cite{balding2006tutorial, clarke2011basic}.

Using linear models, it is easy to include known co-factors to correct for biases. For instance, if it is known that a risk increases with age or that males are more susceptible than females, age and sex can be added to the linear equation in order to correct for these effects \cite{balding2006tutorial,clarke2011basic}. In a similar manner, we can add co-factors to binary traits using logistic regression.

\subsection{Population structure \label{sec:popStruct}}

It is widely accepted that humans started in Africa and migrated to Europe, then to Asia and later to America \cite{hartl1997principles}. Out of an initial population, a few individuals migrate and colonize a new territory. This implies that the genetic variety of the new colony is significantly reduced, compared to the previous population, since the genetic pool is only a small ``founder population". The ``Out of Africa" hypothesis implies that each new migration produced a reduction in genetic variety, also known as a ``population bottleneck'' \cite{hartl1997principles}.

As we previously mentioned, each individual inherits two chromosome sets, a maternal and a paternal one. Through recombination a chromosome is formed by a crossover combining maternal and paternal chromosomes and then passed down, thus the offspring has two sets of chromosomes, one from each parent. This breaking and shuffling of chromosomes every generation, increases genetic diversity. Nevertheless if variants are located nearby in the chromosome, the chances that they are separated by recombination event are smaller than if they are further away from each other. This produces a correlation of close variants or ``linkage disequilibrium" (LD). Nearby highly correlated variants are said to be in the same ``LD-block" \cite{hartl1997principles}. If a population has low genetic variety, the LD-blocks are large. So African population has more variety (smaller LD-blocks) and conversely, European, Asian and Amerindian populations have less variety (larger LD-blocks) \cite{hartl1997principles}.

\subsection{Population structure as confounding variable }

Imagine that we have a cohort of individuals drawn from two populations ($P_A$ and $P_B$) and that individuals in $P_A$ have much higher risk of diabetes than individuals from $P_B$. Now imagine that individuals from $P_A$ have a variant $v_A$ more often, but $v_A$ is actually neutral and has no health effects whatsoever. If we do not take population factors into account , our study would conclude that $v_A$ is associated with increased susceptibility to diabetes, just because we see $v_A$ more often in affected individuals. In this case it is clear that population structure is a confounding variable. We could avoid this problem by analyzing each population separately \cite{patterson2006population}, but this would cause a loss of statistical power since we have fewer samples.

A population that is a mixture of two or more populations is known as an ``admixed population''. For instance the African-American population is a mixture of, roughly, $80\%$ African and $20\%$ European genomes \cite{hartl1997principles,balding2006tutorial}. In the case that structure is confounding an analysis of an admixed population, such as an African-American cohort, it is not possible to perform a separate analysis of each sub-population simply because each individual in the sample genetic background from both populations \cite{hartl1997principles}.

The admixed population problem can be studied by performing a correction using the eigen-structure of the sample covariance matrix \cite{patterson2006population}. Samples can be arranged as a matrix $C$ where each row is a sample and each column represents a position in the genome where there is a variant. The numbers $C_{i,j}$ in the matrix indicate the number of non-reference alleles in a sample (row) at a genomic position (column $j$). Since the allele can be present in zero, one, or two chromosomes in each individual, the possible values for $C_{i,j}$ are $\{0, 1, 2\}$. The covariance matrix is calculated as $M= \hat{C}^T . \hat{C}$, where $\hat{C}$ is the matrix $C$ corrected to have zero mean columns. Usually, the first two to ten principal components of $M$ are used as factors in linear models (see section \ref{sec:cofactors}) to correct for population structure \cite{patterson2006population}.

Whether a cohort has any population structure and needs correction or not can be tested using two methods: a) plotting the projections of the first two principal components and empirically observing the number of clusters in the chart, or b) using a statistic of the eigenvalues of $M$ based on Tracy-Widom's distribution \cite{patterson2006population}.

\subsection{Common and Rare variants\label{sec:comonrare}}

The ``allele frequency" (AF) is defined as the frequency a variant appears in a population. Variants are usually categorized according to AF into three groups: i) Common variants ($AF \geq 5\%$), ``low frequency" ($1\% < AF < 5\%$), and iii) ``rare variants" ($AF < 1\%$). Common variants originated earlier in the population while rare variants are either relatively recent or selected against.

There are three main models for disease susceptibility  \cite{hartl1997principles, gibson2012rare}:i) the Common-Disease-Common-Variant hypothesis (CDCV) assumes that if disease is common, it must be caused by a common variant; ii) the ``infinitesimal hypothesis" proposes that there are many common variants each having small risk effects; and iii) the Common-Disease-Rare-Variant hypothesis proposes that there exists many rare variants, each one having large risk effects.

\subsection{Rare variants test}

The ``rare variant model'' assumes that multiple rare variants have large effects on a trait. The problem is that, since these variants are rare, huge sample sizes are required for tests to identify statistically significant associations. To overcome this problem, methods known as ``burden tests" collapse groups of rare variants that are hypothesised to have  similar effect on gene or protein activity and perform statistical significance tests on the group \cite{li2008methods}. An example of collapsing technique is to count the number or rare variant in a genomic region surrounding an exon or a gene and apply a Fisher exact test, as shown in section \ref{sec:single}. A limitation of some burden tests is that they implicitly assume that all rare variants have the same direction of effect, although in reality they might have no effect, be deleterious, or protective \cite{li2008methods,wu2011rare}.

Several techniques allow weighting rare variants by collapsing them using a kernel matrix. This allows to incorporate other information, such as allele frequency and functional annotations. It can be shown that the statistic induced by kernel weighting functions follows a mixture of $\chi^2$ distributions and there is an efficient way to approximate it \cite{li2008methods,wu2011rare}, avoiding computationally expensive permutations tests.

	% Only show the chapter we are currently working on
\fi

\ifthesis
	% Show all appendices & bibliography in 'thesis' mode

	%-----------------------------------------------------------------------------
	% Appendices
	%-----------------------------------------------------------------------------
	%\ETDAppendix{Appendix A}{ 
\renewcommand{\theequation}{A-\arabic{equation}}
\setcounter{section}{0}
 \setcounter{equation}{0}
\def\thesection       {A.\arabic{section}}

%-----------------------------------------------------------------------------
\section{Algorithm details}
\label{sec:algoDetails}
%-----------------------------------------------------------------------------


Here we show the details on the algorithmic implementations...

 }%
\else
	% Skip all appendices and bibliography
\fi
	
%-----------------------------------------------------------------------------
% Bibliography. You need to run the following 
% command: 
%       bibtex thesis_mcgill
%-----------------------------------------------------------------------------
\bibHeading{References}
\bibliographystyle{plain}
\bibliography{mcgilletd}		% File: mcgilletd.bib

\end{document}
